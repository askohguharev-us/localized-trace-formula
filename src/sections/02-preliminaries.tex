\section{Preliminaries}\label{sec:prelim}

\subsection{Geometry and notation.}
Let $X=\Gamma\backslash\HH$ be a finite-area hyperbolic surface with $m$ cusps.
We write $z=x+iy$ on $\HH$, use the hyperbolic measure $d\mu=y^{-2}\,dx\,dy$, and take the (positive) Laplacian to be $-\Lap$.
Denote normalized cuspidal eigenpairs by $(-\Lap)\psi_j=\lambda_j\psi_j$ with $\lambda_j=\tfrac14+r_j^2$ and $\|\psi_j\|_{L^2(X)}=1$.
The continuous spectrum is treated via Eisenstein series but will be \emph{suppressed in Block~0} (only cuspidal contributions are kept).

We set
\[
X_{\mathrm{core}}:=X\setminus\{\text{cuspidal ends}\},\qquad
\injrad(X_{\mathrm{core}}):=\inf_{z\in X_{\mathrm{core}}}\injrad(z),
\]
and use the geometric size parameter
\[
C_{\mathrm{geo}}(X):=m+\injrad(X_{\mathrm{core}})^{-1},
\]
which controls polynomially all implicit constants below.

\subsection{Cutoffs, windows, and parameter regime.}
Fix $\chi\in C_c^\infty([0,\infty))$ such that $\chi\equiv1$ on $[0,1]$ and $\supp\chi\subset[0,4)$.
For a height scale $Y>0$ define the spatial cutoff
\[
\chi_Y(z):=\chi\!\big(y(z)/Y\big).
\]
Throughout Block~0 we couple $Y$ to the spectral scale via
\[
Y=R^\beta,\qquad R\gg1,\quad 0<\beta<\tfrac12.
\]

Let $h\in\mathcal{S}(\RR)$ be even with compactly supported Fourier transform.
We fix a constant $c_0\in\big(0,\tfrac{\log 2}{2}\big)$ and assume
\[
\supp \widehat{h}\subset[-c_0,c_0].
\]
For a window exponent $0<\theta<1$ we localize in frequency by
\[
h_R(t):=h\!\left(\frac{t-R}{R^\theta}\right),
\]
so $h_R$ selects the spectral window $[R-R^\theta,R+R^\theta]$ centered at $R$ with width $R^\theta$.
In the main estimates we will work in the admissible range
\[
0<\beta<\tfrac12,\qquad 0<\theta<\tfrac{1+\beta}{2},
\]
which will be seen to be optimal at the level of the error exponent.

\begin{definition}[Localized trace]\label{def:TR}
The \emph{localized trace distribution} is
\[
  \TR := \sum_j h_R(r_j)\,\|\chi_Y\psi_j\|_{L^2(X)}^2.
\]
\emph{Remark.} In later sections we prove a decomposition of the form
\[
\TR \;=\; \mathcal{I}_R(\chi_Y,h)\;+\;\mathcal{G}_R(\chi_Y,h)\;+\;O\!\big(R^{1-\varepsilon(\theta,\beta)}\big),
\]
where $\mathcal{I}_R$ is the identity contribution, $\mathcal{G}_R$ is a geometric sum over short closed geodesics (with the \emph{same} $c_0$ as above), and $\varepsilon(\theta,\beta)>0$ on the stated admissible region.
\end{definition}

\begin{lemma}[Windowed Plancherel: skeleton]\label{lem:planch}
Let $h\in\mathcal{S}(\RR)$ be even with $\supp\widehat{h}\subset[-c_0,c_0]$.
Then
\[
  \sum_j h_R(r_j)\,\|\chi_Y\psi_j\|_{L^2(X)}^2
  \;=\; \int_X \chi_Y(z)\,K_R(z,z)\,d\mu(z) \;+\; O_N(R^{-N})\quad\forall N,
\]
where $K_R$ is the Schwartz kernel of the spectral multiplier $h_R(\sqrt{-\Lap})$.
The remainder $O_N(R^{-N})$ depends polynomially on $C_{\mathrm{geo}}(X)$ and on finitely many seminorms of $h$ and $\chi$.
\end{lemma}

\begin{remark}[Effective volume and normalizations]
The effective volume is defined by
\[
\vol_{\mathrm{eff}}(Y):=\int_X \chi_Y^2\,d\mu,
\]
and satisfies (see Appendix~\ref{app:effvol})
\[
\vol_{\mathrm{eff}}(Y)=\vol(X)-\frac{m}{Y}\,\kappa_\chi+O(mY^{-2}),
\qquad
\kappa_\chi:=\int_1^\infty (1-\chi(u)^2)\,u^{-2}\,du\in(0,\tfrac12].
\]
Our Fourier transform and Plancherel conventions match those used in \S\ref{sec:kernel} and \S\ref{sec:proj-comm}; in particular the Plancherel measure is $\tfrac{1}{2\pi}\,r\tanh(\pi r)\,dr$.
\end{remark}
```0
