\section{Preliminaries}\label{sec:prelim}

This section develops the analytic, geometric, and spectral background
necessary for the construction of our localized Selberg trace formula.
The exposition is intentionally detailed: we introduce the geometry of
finite-area hyperbolic surfaces, the structure of cusps, the thick--thin
decomposition, the role of the injectivity radius, and the global area
formulas. We then recall the spectral decomposition of $L^2(X)$ into
cuspidal and continuous parts, the definition and properties of Eisenstein
series, and the Selberg transform that diagonalizes convolution kernels.
Finally, we review the microlocal framework, the wave kernel, and
small-time asymptotics needed for the analysis of localized projectors.

Throughout this section, we emphasize two guiding principles:

\begin{enumerate}
  \item \textbf{Explicit control of constants.}
        Every analytic estimate is formulated with constants depending at most
        polynomially on the geometric invariants of the surface:
        the volume $\vol(X)$, the injectivity radius $\inj(X)$, and the number
        of cusps $n$. This uniformity is indispensable for arithmetic
        applications across families of hyperbolic surfaces.
  \item \textbf{Compatibility with localization.}
        Both the geometric cutoffs (cusp truncation) and the spectral
        cutoffs (microlocal projectors at scale $R^\theta$) must be designed
        to interact smoothly with the hyperbolic geometry. The analytic tools
        presented here are selected with this requirement in mind.
\end{enumerate}

\paragraph{Organization of the section.}
The preliminaries are organized as follows:

\begin{itemize}
  \item \S\ref{subsec:surfaces} introduces the hyperbolic plane, its metric,
        volume form, and Laplace--Beltrami operator.
  \item \S\ref{subsec:cusps} describes cusp regions, their structure, and their
        influence on spectral theory.
  \item \S\ref{subsec:thickthin} presents the thick--thin decomposition and
        consequences for injectivity radius.
  \item \S\ref{subsec:gaussbonnet} recalls the Gauss--Bonnet theorem and area formulas.
  \item \S\ref{subsec:spectral} develops the spectral decomposition into cusp forms
        and Eisenstein series.
  \item \S\ref{subsec:selberg} introduces the Selberg transform and convolution kernels.
  \item \S\ref{subsec:cutoff} defines height truncations and effective volume.
  \item \S\ref{subsec:geodesics} recalls the length spectrum and closed geodesics.
  \item \S\ref{subsec:microlocal} and \S\ref{subsec:wave} present the microlocal
        framework and wave kernel analysis.
  \item \S\ref{subsec:constants} discusses effective constants and polynomial dependence.
  \item \S\ref{subsec:flow} recalls the dynamics of the geodesic flow.
  \item \S\ref{subsec:asymptotics} fixes asymptotic notation and conventions.
  \item \S\ref{subsec:prelim-summary} summarizes the analytic framework for later sections.
\end{itemize}

This systematic preparation ensures that all subsequent arguments in the
localized trace formula rest on a foundation of rigorously controlled
geometric and spectral estimates.

\subsection{Hyperbolic surfaces and their geometry}\label{subsec:surfaces}

We recall the basic structure of hyperbolic geometry which forms the analytic background for the localized trace formula.

\paragraph{The hyperbolic plane.}
Let 
\[
\HH = \{ z = x+iy \in \CC : y > 0 \}
\]
denote the upper half-plane. Equipped with the metric of constant curvature $-1$,
\[
ds^2 = \frac{dx^2+dy^2}{y^2},
\]
the associated volume element is
\[
d\mu(z) = \frac{dx\,dy}{y^2},
\]
and the Laplace--Beltrami operator takes the form
\[
\Delta = -y^2\left( \frac{\partial^2}{\partial x^2} + \frac{\partial^2}{\partial y^2}\right).
\]

\paragraph{Discrete groups and quotients.}
A discrete subgroup $\Gamma \subset \PSL(2,\RR)$ of finite covolume acts properly on $\HH$ by Möbius transformations. The quotient
\[
X = \Gamma\backslash\HH
\]
is a hyperbolic surface of finite area. Throughout we assume $\Gamma$ torsion-free to avoid orbifold singularities; the general orbifold case requires only cosmetic changes. The hyperbolic area is finite and given by Gauss--Bonnet (see Lemma~\ref{lem:GB} below).

\begin{lemma}[Hyperbolic area]\label{lem:GB}
Let $X = \Gamma\backslash\HH$ be a finite-area hyperbolic surface of genus $g$ with $n$ cusps. Then
\[
\vol(X) = 2\pi(2g-2+n) = -2\pi\chi(X),
\]
where $\chi(X)$ is the Euler characteristic of $X$.
\end{lemma}

\paragraph{Remarks.}
\begin{itemize}
  \item The metric is homogeneous and isotropic, making $\HH$ the universal covering of all hyperbolic surfaces.
  \item The Laplacian $\Delta$ is essentially self-adjoint on $C_c^\infty(X)$ and positive-definite.
  \item The hyperbolic measure $d\mu$ is invariant under $\PSL(2,\RR)$, ensuring that $L^2(X)$ carries a unitary representation of $\Gamma$.
\end{itemize}

\paragraph{Spectral relevance.}
The geometry of $\HH$ determines the structure of eigenfunctions on $X$. Negative curvature induces exponential growth of geodesics and governs wave propagation. These features will be exploited in later sections via microlocal and stationary phase methods.

\subsection{Cusps and their structure}\label{subsec:cusps}

Noncompact hyperbolic surfaces $X=\Gamma\backslash\HH$ possess \emph{cusps}, arising from parabolic fixed points of $\Gamma$ on $\partial\HH$. After conjugation, each cusp is equivalent to the standard cusp at infinity.

\paragraph{The standard cusp.}
Consider the parabolic subgroup
\[
\Gamma_\infty = \left\{ \pm \begin{pmatrix} 1 & n \\ 0 & 1 \end{pmatrix} : n \in \ZZ \right\} \subset \PSL(2,\RR).
\]
A fundamental domain for $\Gamma_\infty$ is
\[
\mathcal{F}_\infty = \{ z = x+iy \in \HH : |x|\le \tfrac12, \ y \ge 1 \}.
\]
The quotient $\Gamma_\infty\backslash \HH$ identifies horizontal translates and yields a cusp region modeled by
\[
\{ (x,y) : |x| \le \tfrac12, \ y \ge 1 \} \;\simeq\; [1,\infty)_y \times S^1_x,
\]
with hyperbolic metric $ds^2=(dx^2+dy^2)/y^2$.

\paragraph{Geometry of cusps.}
The injectivity radius in a cusp decays like
\[
\inj(z) \asymp \frac{1}{y}, \qquad y\to\infty.
\]
Thus neighborhoods of cusps become thinner at higher $y$.

\begin{lemma}[Volume of truncated cusps]\label{lem:cusp-vol}
Let $\mathcal{C}(Y) = \{ (x,y): |x|\le \tfrac12, \ 1\le y \le Y\}$. Then
\[
\vol(\mathcal{C}(Y)) = 1 - \frac{1}{Y}.
\]
\end{lemma}

\begin{proof}
Compute directly:
\[
\vol(\mathcal{C}(Y)) = \int_1^Y \int_{-1/2}^{1/2} \frac{dx\,dy}{y^2} 
= \int_1^Y \frac{dy}{y^2}
= 1 - \frac{1}{Y}.
\]
\end{proof}

\paragraph{Spectral consequences.}
Cusps are responsible for the continuous spectrum: Eisenstein series concentrate their mass in the regions $y\gg 1$. Because $\inj(z)\asymp y^{-1}$, no $L^2$ cusp eigenfunction can extend constantly into cusps. This motivates height truncations $\chi_Y$ and effective volumes $\vol_{\mathrm{eff}}(X;Y)$, introduced in later subsections.

\paragraph{Remark.}
For a surface with $n$ cusps, each cusp contributes a region isometric to the standard model above. The global truncated volume then satisfies
\[
\vol_{\mathrm{eff}}(X;Y) = \vol(X) - \frac{n}{Y} + O(Y^{-2}),
\]
which will appear in the identity contribution of the localized trace formula.

\subsection{Thick--thin decomposition}\label{subsec:thickthin}

A fundamental geometric tool is the \emph{thick--thin decomposition}.  
Fix a Margulis constant $\varepsilon_0 > 0$, which depends only on the dimension (for surfaces, one may take $\varepsilon_0 \approx 0.29$).  

\paragraph{Definition.}
The \emph{thin part} is
\[
X_{\mathrm{thin}}(\varepsilon_0) = \{ z\in X : \inj(z) < \varepsilon_0 \}.
\]
It consists of:
\begin{enumerate}
  \item \emph{Cuspidal regions}, isometric to
        $\{ (x,y)\in \HH : |x|\le \tfrac12,\ y > Y \}$ for some $Y$.
  \item \emph{Collar neighborhoods} of short closed geodesics $\gamma$ with 
        $\ell(\gamma)<\varepsilon_0$, modeled by cylinders
        \[
        \{ (\rho,\theta) : |\rho|<w(\gamma), \ \theta\in S^1\}, \quad
        ds^2 = d\rho^2 + \ell(\gamma)^2 \cosh^2\rho\, d\theta^2,
        \]
        where $w(\gamma)$ is the collar width determined by $\sinh w(\gamma)\sinh(\ell(\gamma)/2)=1$.
\end{enumerate}

The complement
\[
X_{\mathrm{thick}}(\varepsilon_0) = X \setminus X_{\mathrm{thin}}(\varepsilon_0)
\]
is compact modulo cusps and has injectivity radius $\ge \varepsilon_0$ everywhere.

\paragraph{Consequences.}
\begin{itemize}
  \item On $X_{\mathrm{thick}}$, all Sobolev and elliptic estimates hold with constants depending polynomially on $\varepsilon_0^{-1}$.
  \item On $X_{\mathrm{thin}}$, analysis splits into two canonical models: cusps and collars. This dichotomy guides the design of microlocal cutoffs and cusp truncations.
\end{itemize}

\subsection{Injectivity radius and its consequences}\label{subsec:injrad}

The injectivity radius $\inj(z)$ at $z\in X$ is
\[
\inj(z) = \sup\{ r>0 : B(z,r)\ \text{embeds isometrically into}\ X \}.
\]
The global invariant
\[
\inj(X) = \inf_{z\in X} \inj(z)
\]
governs the minimal scale of embedded disks.

\paragraph{Behavior.}
\begin{itemize}
  \item On $X_{\mathrm{thick}}$, one has $\inj(z)\ge \varepsilon_0$ uniformly.
  \item In cusps, $\inj(z) \asymp y^{-1}$ as $y\to\infty$.
  \item Near a short geodesic $\gamma$, $\inj(z)\asymp \ell(\gamma)$ inside its collar.
\end{itemize}

\paragraph{Spectral relevance.}
\begin{itemize}
  \item Heat kernel bounds: $|K_t(z,z)| \ll t^{-1} e^{-d^2/4t}$ with constants $\ll \inj(X)^{-D}$.
  \item Resolvent estimates: $\|(\Delta-(1/4+s^2))^{-1}\|\ll \inj(X)^{-D} \vol(X)^C$.
  \item Wave kernel: uniform bounds degrade polynomially in $\inj(X)^{-1}$.
\end{itemize}

\begin{lemma}[Injectivity radius bound]\label{lem:inj-bound}
Let $X=\Gamma\backslash\HH$ be a finite-area hyperbolic surface. Then for all $z\in X$,
\[
\inj(z) \ge c \cdot \min\!\left\{ \frac{1}{y}, \, \ell_{\min}(X) \right\},
\]
where $\ell_{\min}(X)$ is the length of the shortest closed geodesic and $c>0$ is absolute.
\end{lemma}

\begin{proof}[Sketch of proof]
In the cusp, $\inj(z)\asymp 1/y$ by geometry of the horoball. In the compact part, $\inj(z)\ge \ell_{\min}(X)/2$, where $\ell_{\min}(X)$ is realized by the shortest closed geodesic. Combine the two estimates.
\end{proof}

\paragraph{Implication for localized trace formula.}
Because Eisenstein series fail to decay in cusps, and $\inj(z)$ becomes arbitrarily small, explicit dependence of all error terms on $\inj(X)$ is required. Our construction enforces only polynomial dependence in $\inj(X)^{-1}$, ensuring effectiveness across families.

\subsection{Gauss--Bonnet and area formulas}\label{subsec:gaussbonnet}

The Gauss--Bonnet theorem gives an exact relation between topology and hyperbolic area.  
For a complete finite-area hyperbolic surface $X=\Gamma\backslash\HH$ of genus $g$ with $n$ cusps,
\begin{equation}\label{eq:GB}
  \vol(X) \;=\; -2\pi\,\chi(X) \;=\; 2\pi\,(2g-2+n).
\end{equation}
In particular, $\vol(X)$ is quantized in multiples of $2\pi$ and grows linearly with $n$ along congruence towers with fixed genus.

\paragraph{Finite covers.}
If $p:\widetilde X\to X$ is a finite cover of degree $d$, then
\[
  \vol(\widetilde X) = d\,\vol(X), \qquad \chi(\widetilde X) = d\,\chi(X),
\]
and cusps lift to cusps with total multiplicity $d$. This scaling is crucial for uniformity across arithmetic families.

\paragraph{Collar contribution and cusp truncation.}
In a cusp modeled by $\{(x,y): |x|\le \tfrac12,\, y\ge 1\}$ with metric $ds^2=(dx^2+dy^2)/y^2$,
\[
  \vol\big(\{y\ge Y\}\big) \;=\; \int_Y^\infty\!\!\int_{-1/2}^{1/2} \frac{dx\,dy}{y^2} \;=\; \frac{1}{Y}.
\]
Hence for $n$ cusps and a sharp height cutoff at $Y$,
\begin{equation}\label{eq:veff-sharp}
  \vol_{\mathrm{eff}}(X;Y) \;=\; \vol(X) - \frac{n}{Y}.
\end{equation}
With a smooth cutoff $\chi_Y$ supported in $y\le 2Y$ and $\chi_Y\equiv 1$ on $y\le Y$, one has the refined asymptotic
\begin{equation}\label{eq:veff-smooth}
  \vol_{\mathrm{eff}}(X;Y) \;=\; \int_X \chi_Y\,d\mu \;=\; \vol(X) - \frac{n}{Y} + O\!\left(\frac{1}{Y^2}\right),
\end{equation}
where the implied constant depends polynomially on geometric data (notably on $\inj(X)^{-1}$) and on finitely many $\|\nabla^k\chi_Y\|_\infty \ll Y^{-k}$.

\paragraph{Consequences for spectral asymptotics.}
The coefficient of the principal term in the localized Weyl law is proportional to $\vol(X)$ (or $\vol_{\mathrm{eff}}(X;Y)$ under truncation), cf. \eqref{eq:veff-smooth}. Thus \eqref{eq:GB} dictates the leading growth of the spectral counting function in both global and windowed regimes.

\medskip

\subsection{Analytic consequences of geometry}\label{subsec:geometry-analytic}

The constant curvature $-1$ and the thick--thin structure enforce quantitative analytic estimates with explicit geometric dependence.

\paragraph{Heat and wave kernels.}
Let $K_t^{\mathrm{heat}}(z,w)$ and $K_t^{\mathrm{wave}}(z,w)$ denote the heat and wave kernels of $\Delta$ (with spectral shift when convenient). Then for $0<t\le 1$,
\begin{equation}\label{eq:heat-smalltime}
  |K_t^{\mathrm{heat}}(z,z)| \;\ll\; t^{-1} \quad\text{uniformly on } X_{\mathrm{thick}}(\varepsilon_0),
\end{equation}
with implied constant $\ll \varepsilon_0^{-D}$ for some absolute $D>0$. For the wave kernel,
\begin{equation}\label{eq:wave-smalltime}
  |K_t^{\mathrm{wave}}(z,w)| \;\ll\; |t|^{-1/2} \quad\text{whenever } d(z,w)=|t|\le \tfrac12\,\inj(z),
\end{equation}
again with polynomial dependence on $\inj(X)^{-1}$. These are consequences of the Hadamard parametrix and finite propagation speed on a negatively curved surface.

\paragraph{Resolvent bounds.}
For $s\in\RR$ and $\lambda=1/4+s^2$ away from the spectrum,
\begin{equation}\label{eq:resolvent}
  \big\|(\Delta-\lambda)^{-1}\big\|_{L^2\to L^2} \;\ll\; \vol(X)^C\,\inj(X)^{-D}\,(1+|s|)^E,
\end{equation}
for fixed exponents $C,D,E>0$. Polynomial dependence is sufficient for all subsequent error budgets.

\paragraph{Length spectrum growth.}
Let $\pi_X(L)$ denote the number of primitive closed geodesics of length $\le L$. Then
\begin{equation}\label{eq:pgthm}
  \pi_X(L) \;=\; \operatorname{li}(e^L) + O\!\left(\frac{e^{\vartheta L}}{L}\right)
  \qquad (0<\vartheta<1),
\end{equation}
with constants depending polynomially on $\vol(X)$ and $\inj(X)^{-1}$. This exponential growth informs the truncation of the geometric side in localized windows.

\paragraph{Sobolev and elliptic estimates on the thick part.}
For $s\ge 0$ and $u\in C^\infty_c(X_{\mathrm{thick}}(\varepsilon_0))$,
\begin{equation}\label{eq:sobolev}
  \|u\|_{H^{s+2}(X)} \;\ll\; \|(\Delta+1)u\|_{H^{s}(X)} + \|u\|_{H^{s}(X)},
\end{equation}
with implied constant $\ll \varepsilon_0^{-D}$ for some $D>0$. This uniform control is used repeatedly in symbol calculus and stationary phase with parameters.

\begin{proposition}[Uniform Sobolev on the thick core]\label{prop:uniform-sobolev}
Fix $\varepsilon_0>0$. There exists $D=D(\varepsilon_0)$ with $D\ll \varepsilon_0^{-M}$ for some absolute $M$ such that \eqref{eq:sobolev} holds for all $s\ge 0$ and $u$ supported in $X_{\mathrm{thick}}(\varepsilon_0)$. The same bounds hold componentwise for vector-valued kernels appearing in the microlocal construction.
\end{proposition}

\paragraph{Implication for localized projectors.}
Since our windowed kernels only probe the wave group for $|t|\le R^{-\theta}$, the small-time bounds \eqref{eq:wave-smalltime} and the parametrix error control—together with \eqref{eq:veff-smooth}—yield operator-norm estimates with error terms polynomial in $\vol(X)$ and $\inj(X)^{-1}$, as required by arithmetic applications.

\medskip

\subsection{Summary of the geometric framework}\label{subsec:geometry-summary}

The following points will be used as standing inputs in later sections:
\begin{enumerate}
  \item \emph{Area/topology:} $\vol(X)=2\pi(2g-2+n)$ \eqref{eq:GB}; under finite covers, $\vol$ scales by degree.
  \item \emph{Cusps and truncation:} effective volume satisfies \eqref{eq:veff-smooth}; derivatives of $\chi_Y$ obey $\|\nabla^k\chi_Y\|_\infty\ll Y^{-k}$.
  \item \emph{Thick--thin control:} uniform injectivity on $X_{\mathrm{thick}}(\varepsilon_0)$; model geometry in cusps and collars guides microlocal cutoffs.
  \item \emph{Analytic kernels:} small-time heat/wave kernel bounds \eqref{eq:heat-smalltime}–\eqref{eq:wave-smalltime}; resolvent bound \eqref{eq:resolvent}.
  \item \emph{Length spectrum:} exponential growth \eqref{eq:pgthm} enabling geometric truncations matched to the spectral window.
  \item \emph{Uniform constants:} all implied constants in $O(\cdot)$ and $\ll$ are polynomial in $\vol(X)$, $\inj(X)^{-1}$, and cusp data.
\end{enumerate}
These inputs ensure that every step of the localized trace formula—construction of kernels, microlocal estimates, and geometric expansions—carries explicit and uniform dependence on the underlying geometry.

\subsection{Spectral decomposition and automorphic forms}\label{subsec:spectral}

The Hilbert space $L^2(X)$ decomposes orthogonally as
\begin{equation}\label{eq:L2dec}
  L^2(X) = L^2_{\mathrm{cusp}}(X) \;\oplus\; L^2_{\mathrm{cont}}(X),
\end{equation}
where the cuspidal part is spanned by Maass cusp forms and the continuous part is generated by Eisenstein series attached to the cusps. This is Selberg’s fundamental decomposition, later refined by Hejhal, Müller, and others.

\paragraph{Cuspidal spectrum.}
There exists an orthonormal basis $\{\phi_j\}$ of $L^2_{\mathrm{cusp}}(X)$ with
\begin{equation}\label{eq:Laplace-cusp}
  \Delta \phi_j = \lambda_j \phi_j, \qquad \lambda_j = \tfrac14 + r_j^2,\; r_j\ge 0.
\end{equation}
These eigenvalues have finite multiplicity and accumulate only at infinity. The spectral parameters $\{r_j\}$ satisfy Weyl’s law
\begin{equation}\label{eq:global-weyl}
  N(R) := \#\{ j : r_j \le R\} \;=\; \frac{\vol(X)}{4\pi}\,R^2 \;+\; O(R\log R).
\end{equation}
The error term is uniform with constants polynomial in $\vol(X)$ and $\inj(X)^{-1}$.

\paragraph{Continuous spectrum.}
Each cusp $\mathfrak{a}$ yields an Eisenstein series
\begin{equation}\label{eq:Eisenstein}
  E_\mathfrak{a}(z,s) \;=\; \sum_{\gamma\in \Gamma_\mathfrak{a}\backslash\Gamma}
  \Im\!\big(\sigma_\mathfrak{a}^{-1}\gamma z\big)^s, \qquad \Re(s)>1,
\end{equation}
where $\Gamma_\mathfrak{a}$ is the stabilizer of $\mathfrak{a}$ and $\sigma_\mathfrak{a}$ a scaling matrix. These extend meromorphically to $s\in\CC$ and obey the functional relation
\begin{equation}\label{eq:Eisenstein-functional}
  E_\mathfrak{a}(z,s) \;=\; \sum_{\mathfrak{b}} \phi_{\mathfrak{a}\mathfrak{b}}(s)\,
  E_\mathfrak{b}(z,1-s),
\end{equation}
where $\Phi(s)=(\phi_{\mathfrak{a}\mathfrak{b}}(s))$ is the scattering matrix, unitary on $\Re(s)=1/2$.

\paragraph{Spectral theorem.}
Every $f\in L^2(X)$ admits the expansion
\begin{equation}\label{eq:spectral-expansion}
\begin{aligned}
  f(z) &= \sum_j \langle f,\phi_j\rangle\,\phi_j(z) \\
  &\quad+ \frac{1}{4\pi}\sum_{\mathfrak{a}}\int_{-\infty}^{\infty}
     \langle f, E_\mathfrak{a}(\cdot,\tfrac12+it)\rangle\,
     E_\mathfrak{a}(z,\tfrac12+it)\,dt,
\end{aligned}
\end{equation}
valid in $L^2$ and pointwise almost everywhere.

\paragraph{Behavior in cusps.}
Cuspidal forms $\phi_j$ decay rapidly as $y\to\infty$ in cusp coordinates, reflecting square-integrability. By contrast, Eisenstein series remain of size $\asymp 1$ in cusp regions, carrying non-decaying $L^2$-mass. This dichotomy motivates the truncations $\chi_Y$ used in localized projectors.

\begin{lemma}[Eisenstein $L^2$-mass in cusps]\label{lem:eisenstein-mass}
For $E_\mathfrak{a}(z,1/2+it)$ normalized Eisenstein series and $Y\ge 1$,
\begin{equation}\label{eq:eisenstein-mass}
  \int_{y\ge Y} |E_\mathfrak{a}(z,1/2+it)|^2 \, d\mu(z)
  \;\ll\; Y^{-1+o(1)}\quad \text{as } Y\to\infty,
\end{equation}
uniformly in $t$. The implied constant is polynomial in $\vol(X)$ and $\inj(X)^{-1}$.
\end{lemma}

\paragraph{Implications.}
Lemma~\ref{lem:eisenstein-mass} ensures that truncating at $Y=R^\beta$ suppresses continuous spectrum contributions to order $O(R^{-\beta+o(1)})$. This mechanism underlies the elimination of Eisenstein terms in the localized trace formula.

\medskip

\subsection{Selberg transform and convolution kernels}\label{subsec:selberg}

Let $k(\rho)$ be a radial function of hyperbolic distance $\rho=d(z,w)$. Define
\begin{equation}\label{eq:radial-operator}
  (Kf)(z) = \int_X k(d(z,w)) f(w)\,d\mu(w).
\end{equation}
Then $K$ commutes with $\Delta$ and diagonalizes in the spectral decomposition:
\begin{align}
  K\phi_j &= \widehat{k}(r_j)\,\phi_j, \label{eq:selberg-cusp}\\
  K E_\mathfrak{a}(\cdot,\tfrac12+it) &= \widehat{k}(t)\,E_\mathfrak{a}(\cdot,\tfrac12+it). \label{eq:selberg-eisenstein}
\end{align}

\paragraph{Selberg transform.}
Given $k(\rho)$ smooth and compactly supported, its Selberg transform is
\begin{equation}\label{eq:selberg-transform}
  \widehat{k}(r) = \int_0^\infty k(\rho)\,\Phi_r(\rho)\,\sinh\rho\,d\rho,
\end{equation}
where $\Phi_r$ is the spherical function satisfying $\Delta \Phi_r = (1/4+r^2)\Phi_r$, $\Phi_r(0)=1$.

Conversely,
\begin{equation}\label{eq:selberg-inversion}
  k(\rho) = \frac{1}{2\pi}\int_{-\infty}^\infty
  \widehat{k}(r)\,\Phi_r(\rho)\,r\tanh(\pi r)\,dr.
\end{equation}

\paragraph{Properties.}
$\widehat{k}(r)$ is an even entire function, rapidly decaying as $|r|\to\infty$. This Fourier duality between $k(\rho)$ and $\widehat{k}(r)$ underlies the trace formula.

\paragraph{Localized test functions.}
To localize near spectral parameter $R$, we select
\begin{equation}\label{eq:localized-test}
  h_R(r) = \eta\!\left(\frac{r-R}{R^\theta}\right),
\end{equation}
for $\eta\in\mathcal{S}(\RR)$, even, $\eta(0)=1$. Then $\widehat{h}_R$ has Fourier support at scale $R^\theta$. Its inverse transform $k_R$ yields kernels $K_R$ microlocalized to distances $\lesssim R^{-\theta}$, crucial for constructing $\mathsf{T}_R$.

\subsection{Height cutoff and effective volume}\label{subsec:cutoff}

Cuspidal regions introduce continuous spectrum that must be suppressed.  
To achieve this we truncate at height $Y=R^\beta$, $\beta>0$.

\paragraph{Sharp cutoff.}
Define
\begin{equation}\label{eq:sharp-cutoff}
  \chi_Y(z) = \mathbf{1}_{\{ y\le Y\}}(z),
\end{equation}
with effective volume
\begin{equation}\label{eq:effective-volume}
  \vol_{\mathrm{eff}}(X;Y) = \int_X \chi_Y(z)\,d\mu(z).
\end{equation}

\begin{lemma}[Effective volume asymptotics]\label{lem:veff}
For a surface with $n$ cusps,
\begin{equation}\label{eq:veff-asymp}
  \vol_{\mathrm{eff}}(X;Y) = \vol(X) - \frac{n}{Y} + O(Y^{-2}), \qquad Y\to\infty,
\end{equation}
with constants polynomial in $\vol(X)$ and $\inj(X)^{-1}$.
\end{lemma}

\paragraph{Smooth cutoff.}
For analytic control we replace $\chi_Y$ with smooth $\chi_Y\in C^\infty(X)$,  
$\chi_Y=1$ on $y\le Y$, $\chi_Y=0$ for $y\ge 2Y$, satisfying
\begin{equation}\label{eq:cutoff-derivatives}
  |\nabla^k \chi_Y| \;\ll\; Y^{-k}, \qquad \forall k\ge 0.
\end{equation}
This ensures compatibility with microlocal analysis while preserving~\eqref{eq:veff-asymp}.

\medskip

\subsection{Length spectrum and closed geodesics}\label{subsec:geodesics}

Each primitive hyperbolic $\gamma_0\in\Gamma$ corresponds to a closed geodesic of length $\ell(\gamma_0)>0$, with
\begin{equation}\label{eq:trace-length}
  |\mathrm{tr}(\gamma_0)| = 2\cosh(\ell(\gamma_0)/2).
\end{equation}

\paragraph{Prime geodesic theorem.}
As $L\to\infty$,
\begin{equation}\label{eq:PGT}
  \pi_X(L) := \#\{\gamma_0:\ell(\gamma_0)\le L\} \;\sim\; \frac{e^L}{L}.
\end{equation}

\paragraph{Geodesic contributions.}
For radial test function $h$, Selberg’s trace formula yields
\begin{equation}\label{eq:geodesic-contribution}
  \int_X k(d(z,\gamma z))\,d\mu(z)
  = \frac{\ell(\gamma_0)}{2\sinh(\ell(\gamma)/2)}\,\widehat{h}(\ell(\gamma)).
\end{equation}
In the localized setting, $\widehat{h}_R$ is concentrated at scale $R^{-\theta}$, so only geodesics with $\ell(\gamma)\lesssim R^{-\theta}$ contribute significantly.

\paragraph{Error control.}
Truncating the geodesic sum at $\ell(\gamma)\le L_0$ yields
\begin{equation}\label{eq:geodesic-error}
  \sum_{\ell(\gamma)>L_0}
  \left|\frac{\ell(\gamma_0)}{2\sinh(\ell(\gamma)/2)}\,\widehat{h}(\ell(\gamma))\right|
  \;\ll\;\int_{L_0}^\infty e^L\,|\widehat{h}(L)|\,dL,
\end{equation}
which is rapidly decaying for Schwartz-class $h$.

\subsection{Microlocal analysis framework}\label{subsec:microlocal}

Microlocal analysis provides the core machinery for localized spectral projectors.  
We adopt semiclassical notation with parameter $h=R^{-1}$.

\paragraph{Pseudodifferential operators.}
Symbols $a(x,\xi;h)\in S^m(T^*X)$ satisfy
\begin{equation}\label{eq:symbol}
  |\partial_x^\alpha\partial_\xi^\beta a(x,\xi;h)|
  \;\le\; C_{\alpha\beta}(1+|\xi|)^{m-|\beta|}.
\end{equation}
The associated operators act by
\begin{equation}\label{eq:pdo}
  (\Op_h(a)f)(x) = \frac{1}{(2\pi h)^d}
  \int_{\RR^d}\int_{\RR^d} e^{i\langle x-y,\xi\rangle/h}a(x,\xi;h)f(y)\,dy\,d\xi.
\end{equation}

\paragraph{Propagation under the geodesic flow.}
Let $U(t) = e^{-it\sqrt{\Delta-1/4}}$ be the wave group.  
Egorov’s theorem implies
\begin{equation}\label{eq:egorov}
  U(-t)\,\Op_h(a)\,U(t) = \Op_h(a\circ \varphi_t) + O(h),
\end{equation}
where $\varphi_t$ is the geodesic flow on $S^*X$.

\paragraph{Microlocal projectors.}
Choose $h_R$ supported in $[R-R^\theta,R+R^\theta]$, and let $k_R$ be its inverse Selberg transform.  
The periodized kernel is
\begin{equation}\label{eq:kernel-KR}
  K_R(z,w) = \sum_{\gamma\in\Gamma} k_R(d(z,\gamma w)).
\end{equation}
With cusp cutoff $\chi_Y$, we set
\begin{equation}\label{eq:kernel-KRY}
  K_R^Y(z,w) = \chi_Y(z)\,K_R(z,w)\,\chi_Y(w).
\end{equation}
The operator
\begin{equation}\label{eq:TR}
  (\mathsf{T}_R f)(z) = \int_X K_R^Y(z,w)\,f(w)\,d\mu(w)
\end{equation}
is self-adjoint, bounded, and serves as an approximate spectral projector.

\paragraph{Spectral action.}
For cusp forms $\varphi_j$,
\begin{equation}\label{eq:TR-action}
  \mathsf{T}_R\varphi_j = \big(h_R(r_j)+O(R^{-A})\big)\varphi_j,
\end{equation}
valid for all $A>0$, with constants polynomial in $\vol(X)$, $\inj(X)^{-1}$, and cusp parameters.  
The continuous spectrum is annihilated up to negligible error.

\paragraph{Idempotence and orthogonality.}
We establish:
\begin{align}
  \|\mathsf{T}_R^2 - \mathsf{T}_R\|_{L^2\to L^2} &\;\ll\; R^{-\theta}, \label{eq:approx-idempotence}\\
  \|\mathsf{T}_{R_1}\mathsf{T}_{R_2}\| &\;\ll\; R^{-M}, \quad \forall M>0, \qquad |R_1-R_2|\gg R^\theta.
  \label{eq:orthogonality}
\end{align}
Thus $\mathsf{T}_R$ are nearly orthogonal projectors across disjoint spectral windows.

\subsection{Wave kernel and small-time analysis}\label{subsec:wave}

The wave kernel
\begin{equation}\label{eq:wave-kernel}
  K_t(z,w) = \cos\!\left(t\sqrt{\Delta-\tfrac14}\,\right)(z,w)
\end{equation}
encodes propagation of singularities along geodesics.  
It is singular when $d(z,w)=|t|$ and smooth otherwise.

\paragraph{Small-time asymptotics.}
For $|t|\ll 1$, the Hadamard parametrix yields
\begin{equation}\label{eq:parametrix}
  K_t(z,w) \;=\; \sum_{k=0}^N a_k(z,w)\,t^{-2k}
    \;+\; R_N(t,z,w),
\end{equation}
with $a_k$ smooth coefficients and $R_N$ a smooth error term satisfying
\[
  |R_N(t,z,w)| \;\ll\; |t|^{N-d},
\]
where $d=\dim(X)=2$.  
This expansion is valid up to the injectivity radius $\inj(X)$.

\paragraph{Smoothed projectors.}
For a spectral cutoff $h_R$ with Fourier transform supported in 
$[-R^{-\theta},R^{-\theta}]$, one has
\begin{equation}\label{eq:smoothed-proj}
  h_R(\sqrt{\Delta-1/4})
  = \frac{1}{2\pi}\int_\RR \widehat{h}_R(t)\, e^{it\sqrt{\Delta-1/4}}\,dt.
\end{equation}
Thus only small-time wave propagation contributes.  
Stationary phase applied to \eqref{eq:parametrix} produces precise asymptotics for kernels $K_R^Y$.

\paragraph{Stationary phase estimate.}
For $|t|\le R^{-\theta}$ and $d(z,w)\asymp |t|$, we obtain
\begin{equation}\label{eq:stationary-phase}
  K_t(z,w) \;\ll\; |t|^{-1/2},
\end{equation}
uniformly in $z,w$ with constants polynomial in $\vol(X)$, $\inj(X)^{-1}$, and cusp data.  
This estimate is the foundation of error bounds in the localized trace formula.

\subsection{Effective constants and polynomial dependence}\label{subsec:constants}

A central objective of this work is to ensure that all implicit constants 
in our analysis depend at most \emph{polynomially} on the basic geometric invariants of $X$.
This requirement is crucial for arithmetic applications, where one studies 
families of congruence covers with growing volume and potentially shrinking injectivity radius.  

\paragraph{Geometric invariants.}  
We emphasize dependence on:
\begin{itemize}
  \item $\vol(X)$: the hyperbolic area,
  \item $\inj(X)$: the global injectivity radius,
  \item $n$: the number of cusps,
  \item cusp widths and scattering matrix parameters.
\end{itemize}

\paragraph{Polynomial control.}
In our estimates, whenever we write
\[
  f(R) \;=\; O(g(R)),
\]
the implied constant is of the form
\[
  C(X) \;\ll\; \vol(X)^A \cdot \inj(X)^{-B} \cdot (1+n)^C,
\]
for some fixed nonnegative integers $A,B,C$ independent of $R$.  
This structure guarantees that the analysis remains uniform across towers of coverings
and under arithmetic deformations.

\paragraph{Examples.}
\begin{enumerate}
  \item For the effective volume asymptotics,
  \[
    \vol_{\mathrm{eff}}(X;Y) \;=\; \vol(X) - \frac{n}{Y} + O\!\left(\frac{1}{Y^2}\right),
  \]
  the implicit constant in the error term is polynomial in $n$ but independent of $R$.
  \item For the wave kernel bound \eqref{eq:stationary-phase},  
  \[
    |K_t(z,w)| \;\ll\; |t|^{-1/2}\,\inj(X)^{-D},
  \]
  for some absolute $D>0$, showing explicit polynomial dependence on $\inj(X)^{-1}$.
\end{enumerate}

\paragraph{Comparison with prior work.}
In classical treatments of the Selberg trace formula, error terms often involve
constants that grow exponentially in $1/\inj(X)$, making them ineffective
for arithmetic applications.  
Our framework systematically avoids such losses by incorporating the injectivity radius 
into microlocal cutoffs and by using smooth truncations $\chi_Y$ with derivative bounds
\[
  \|\nabla^k \chi_Y\|_{L^\infty} \;\ll\; Y^{-k}.
\]
This enforces polynomial control at every stage of the analysis.

\subsection{Microlocal cutoffs in phase space}\label{subsec:cutoffs}

Microlocal analysis allows us to localize simultaneously in both position and frequency.  
This dual localization is indispensable for constructing projectors that isolate spectral windows 
of size $R^\theta$ while maintaining control of cusp contributions.

\paragraph{Pseudodifferential symbols.}
Let $a(x,\xi;R)$ be a semiclassical symbol supported in the phase space region
\[
  \{ (x,\xi) \in T^*X : |\xi| \in [R-R^\theta, R+R^\theta]\}.
\]
We quantize this symbol to obtain a pseudodifferential operator
\[
  A_R = \Op_{R^{-1}}(a).
\]
By construction, $A_R$ acts as a microlocal cutoff to spectral frequencies in $[R-R^\theta,R+R^\theta]$.

\paragraph{Coupling with cusp cutoffs.}
To control the continuous spectrum, we combine $A_R$ with the geometric cutoff $\chi_Y$,
where $Y=R^\beta$:
\[
  \widetilde{A}_R \;=\; A_R \chi_Y.
\]
This operator suppresses cusp regions while preserving microlocal localization in the thick part.

\paragraph{Definition of the projector.}
We define the microlocal projector $\mathsf{T}_R$ schematically as
\[
  \mathsf{T}_R \;=\; \widetilde{A}_R \, U \, \widetilde{A}_R^*,
\]
where $U$ denotes an evolution operator derived from the wave group
$e^{it\sqrt{\Delta-1/4}}$, integrated against the window function $\widehat{h}_R(t)$.
This construction guarantees that $\mathsf{T}_R$ localizes to spectral window $[R-R^\theta,R+R^\theta]$
and annihilates the continuous spectrum up to negligible error.

\paragraph{Properties.}
The operators $\mathsf{T}_R$ enjoy the following structural properties:
\begin{enumerate}
  \item \textbf{Uniform boundedness:}  
        $\|\mathsf{T}_R\|_{L^2 \to L^2} \ll 1$, with constants polynomial in $\vol(X)$, $\inj(X)^{-1}$, and $n$.
  \item \textbf{Approximate idempotence:}  
        \[
          \mathsf{T}_R^2 - \mathsf{T}_R \;=\; O(R^{-\theta})
        \]
        in operator norm. Thus $\mathsf{T}_R$ acts as a nearly perfect spectral projector.
  \item \textbf{Orthogonality of windows:}  
        For $R_1,R_2$ with $|R_1-R_2|\gg R^\theta$,
        \[
          \|\mathsf{T}_{R_1}\mathsf{T}_{R_2}\| \;\ll\; R^{-M}, \qquad \forall M>0,
        \]
        reflecting super-polynomial orthogonality between disjoint spectral windows.
\end{enumerate}

\paragraph{Significance.}
This microlocal cutoff mechanism forms the technical core of our localized trace formula.
It ensures that both geometric and spectral localizations are respected,
while all constants remain polynomially controlled in the invariants of $X$.

\subsection{Geodesic flow and dynamics}\label{subsec:flow}

The geodesic flow on the unit cotangent bundle $S^*X$ of a hyperbolic surface $X$ is 
an Anosov flow, exhibiting uniform hyperbolicity and exponential mixing.  
This dynamical structure underlies much of the spectral theory and plays a decisive role 
in the localized trace formula.

\paragraph{Anosov property.}
For each $z\in S^*X$, the tangent space splits as
\[
  T_z(S^*X) = E^u(z) \oplus E^s(z) \oplus \RR H,
\]
where $E^u$ and $E^s$ are the unstable and stable bundles, and $H$ generates the geodesic flow.  
Vectors in $E^u$ expand exponentially under forward flow, while those in $E^s$ contract.

\paragraph{Exponential mixing.}
For Hölder observables $f,g$ on $S^*X$, correlations decay exponentially:
\[
  \left|\int_{S^*X} f \cdot g\circ\varphi_t \, d\mu - \int f\,d\mu \int g\,d\mu \right|
  \;\ll\; e^{-\kappa t}, \qquad t\to\infty,
\]
for some $\kappa>0$, where $\varphi_t$ is the geodesic flow and $d\mu$ the Liouville measure.
This mixing property reflects the chaotic nature of geodesic trajectories.

\paragraph{Consequences for spectral theory.}
\begin{itemize}
  \item \textbf{Closed geodesics.}  
        The growth rate of primitive closed geodesics is exponential:
        \[
          \pi_X(L) := \#\{\gamma_0 : \ell(\gamma_0)\le L\} \sim \frac{e^L}{L}, \qquad L\to\infty.
        \]
        This asymptotic feeds directly into the geometric side of the trace formula.
  \item \textbf{Quantum ergodicity.}  
        The Anosov property implies equidistribution of eigenfunctions in the high-energy limit, 
        except possibly along a subsequence of density zero.  
        Localized projectors enable refined statements in shrinking spectral windows.
  \item \textbf{Decay of correlations.}  
        Exponential mixing ensures that the wave kernel disperses rapidly, 
        yielding power-saving error terms in the analysis of localized traces.
\end{itemize}

\paragraph{Implications for localization.}
The chaotic dynamics of geodesic flow imply that localized spectral windows 
interact primarily with short closed geodesics of length $\lesssim R^{-\theta}$.  
Longer geodesics contribute only to remainders, suppressed by hyperbolic dispersion.  
This interplay between chaotic dynamics and microlocal projectors is one of the 
conceptual foundations of our construction.

\subsection{Asymptotic conventions}\label{subsec:asymptotics}

Throughout the paper we employ standard asymptotic notation, always with the understanding 
that implied constants depend at most polynomially on the geometric invariants of $X$, 
namely $\vol(X)$, $\inj(X)^{-1}$, and the number of cusps $n$.  

\begin{itemize}
  \item $f(R) = O(g(R))$ or $f(R)\ll g(R)$: there exists a constant $C>0$ such that 
        $|f(R)| \le C g(R)$ for all sufficiently large $R$, with 
        $C \ll \vol(X)^A \inj(X)^{-B} (1+n)^C$ for some fixed exponents $A,B,C$.
  \item $f(R)\asymp g(R)$: both $f(R)\ll g(R)$ and $g(R)\ll f(R)$ hold with constants 
        depending polynomially on the invariants of $X$.
  \item $f(R)\sim g(R)$: $\displaystyle \lim_{R\to\infty}\frac{f(R)}{g(R)}=1$.
  \item $f(R) = O_\varepsilon(g(R))$: the bound holds with a constant depending polynomially 
        on $\vol(X), \inj(X)^{-1}, n$, and possibly on $\varepsilon>0$.
\end{itemize}

We emphasize that no implicit constant in this paper grows faster than polynomially 
with respect to $\vol(X)$, $\inj(X)^{-1}$, or cusp parameters.  
This uniformity is a central feature of the localized trace formula, 
ensuring its applicability to arithmetic families and congruence towers.

\subsection{Summary of preliminaries}\label{subsec:prelim-summary}

We conclude this section with a summary of the analytic foundations assembled thus far:

\begin{enumerate}
  \item \emph{Geometric framework:} finite-area hyperbolic surfaces, cusp geometry, 
        thick–thin decomposition, and injectivity radius estimates.
  \item \emph{Spectral decomposition:} Maass cusp forms, Eisenstein series, scattering matrices, 
        and Weyl’s law.
  \item \emph{Selberg transform:} convolution kernels, diagonalization, inversion, 
        and localized test functions.
  \item \emph{Height cutoff and effective volume:} explicit asymptotics 
        $\vol_{\mathrm{eff}}(X;Y) = \vol(X)-n/Y+O(Y^{-2})$.
  \item \emph{Length spectrum:} primitive closed geodesics, the prime geodesic theorem, 
        and truncated contributions.
  \item \emph{Microlocal tools:} pseudodifferential calculus, Egorov’s theorem, 
        wave kernel analysis, and propagation of singularities.
  \item \emph{Localized projectors:} construction of $K_R^Y$, approximate idempotence, 
        and orthogonality across windows.
  \item \emph{Error control:} explicit polynomial dependence of all constants 
        on $\vol(X)$, $\inj(X)$, and cusp parameters.
\end{enumerate}

This comprehensive framework supports the construction of the localized kernel $K_R^Y$, 
its trace analysis, and ultimately the establishment of the localized trace formula 
with explicit error terms and broad applications.

\subsection{Spectral counting and localized Weyl laws}\label{subsec:weyl}

A cornerstone of spectral geometry is Weyl’s law, which in its global form asserts
\begin{equation}\label{eq:weyl-global}
  N(R) := \#\{ j : r_j \le R \} 
  = \frac{\vol(X)}{4\pi}R^2 + O(R \log R),
\end{equation}
as $R \to \infty$.  
Here $r_j$ are the spectral parameters of Maass cusp forms.  
For arithmetic surfaces, sharper remainders are available using analytic number theory, 
but for localized analysis this global form is insufficient.

\paragraph{Localized counting function.}
Define the localized counting function
\[
  N(R,\theta) := \#\{ j : r_j \in [R-R^\theta,\, R+R^\theta]\}.
\]
Formally differentiating \eqref{eq:weyl-global} suggests the asymptotic
\[
  N(R,\theta) \;\sim\; \frac{\vol(X)}{2\pi}\,R^{1+\theta}, \qquad R\to\infty.
\]
This heuristic indicates that the number of eigenvalues in a short window 
grows like $R^{1+\theta}$, with the proportionality constant depending only on $\vol(X)$.  

\paragraph{Localized Weyl law.}
Our localized trace formula rigorously establishes this prediction and quantifies the error term:
\[
  N(R,\theta) = \frac{\vol(X)}{2\pi}\, R^{1+\theta} + O\!\big(R^{1-\varepsilon(\theta,\beta)}\big),
\]
for some $\varepsilon(\theta,\beta)>0$, with constants depending polynomially on $\vol(X)$, $\inj(X)^{-1}$, and the number of cusps.  
The exponent $\varepsilon(\theta,\beta)$ is explicitly computable, see Section~\ref{sec:results}.

\paragraph{Comparison with global law.}
Note that the error term $O(R^{1-\varepsilon})$ represents a power-saving refinement compared with 
the $O(R\log R)$ term in \eqref{eq:weyl-global}.  
This improvement arises from spectral localization and cusp suppression, 
allowing cancellation effects that are invisible in the global averaging.

\paragraph{Uniformity across families.}
The polynomial dependence of implicit constants guarantees that this localized Weyl law 
remains effective when $X$ varies across congruence subgroups or arithmetic families.  
This uniformity is essential for analytic number theory, particularly in the study of 
$L$-functions and eigenvalue statistics.

\subsection{Eisenstein series and continuous spectrum}\label{subsec:eisenstein}

The continuous spectrum of $\Delta$ is generated by Eisenstein series, 
which play a central role in the spectral decomposition of $L^2(X)$.  
For each cusp $\mathfrak{a}$ of $X$, one defines
\begin{equation}\label{eq:eisenstein}
  E_\mathfrak{a}(z,s) \;=\; \sum_{\gamma \in \Gamma_\mathfrak{a}\backslash \Gamma} 
  \Im\!\big(\sigma_\mathfrak{a}^{-1}\gamma z\big)^s,
  \qquad \Re(s)>1,
\end{equation}
where $\sigma_\mathfrak{a}$ is a scaling matrix mapping $\infty$ to $\mathfrak{a}$ 
and $\Gamma_\mathfrak{a}$ is the stabilizer of $\mathfrak{a}$.  
The series converges absolutely for $\Re(s)>1$ and admits meromorphic continuation 
to $s\in \mathbb{C}$.

\paragraph{Functional equations.}
The Eisenstein series satisfy the functional equation
\[
  E_\mathfrak{a}(z,s) \;=\; \sum_{\mathfrak{b}} 
    \phi_{\mathfrak{a}\mathfrak{b}}(s)\, E_\mathfrak{b}(z,1-s),
\]
where $\Phi(s)=(\phi_{\mathfrak{a}\mathfrak{b}}(s))$ is the scattering matrix.  
On the critical line $\Re(s)=1/2$, $\Phi(1/2+it)$ is unitary, reflecting the 
self-adjointness of $\Delta$.

\paragraph{Spectral expansion with continuous part.}
The spectral theorem yields the decomposition
\begin{equation}\label{eq:spectral-expansion}
  f(z) = \sum_j \langle f, \phi_j \rangle \phi_j(z) 
  \;+\; \frac{1}{4\pi}\sum_{\mathfrak{a}} \int_{-\infty}^{\infty} 
      \langle f, E_\mathfrak{a}(\cdot,\tfrac12+it)\rangle 
      E_\mathfrak{a}(z,\tfrac12+it)\, dt,
\end{equation}
valid for all $f\in L^2(X)$.

\paragraph{Growth in cusps.}
Unlike cusp forms, Eisenstein series do not decay in the cusp regions.  
Indeed, for $y\to\infty$,
\[
  E_\mathfrak{a}(x+iy, 1/2+it) \;\asymp\; y^{1/2+it} + \phi_{\mathfrak{a}\mathfrak{a}}(1/2+it)\, y^{1/2-it}.
\]
As a result, their $L^2$-mass in $\{y>Y\}$ decays only as $O(Y^{-1+o(1)})$, 
not exponentially.  

\paragraph{Suppression by cusp truncation.}
To isolate the discrete spectrum, we truncate at $y \leq R^\beta$.  
One obtains
\[
  \int_{y>R^\beta} |E_\mathfrak{a}(z,1/2+it)|^2 \, d\mu(z) \;\ll\; R^{-\beta+o(1)}.
\]
Thus Eisenstein contributions are effectively suppressed in the localized trace formula 
for any $\beta>0$, while the main volume contribution remains intact.  

\paragraph{Conclusion.}
The combination of spectral windowing and cusp truncation eliminates continuous spectrum 
contamination up to controllable polynomial error terms.  
This ensures that the localized trace formula genuinely isolates the cuspidal spectrum, 
which is the primary object of interest for arithmetic and dynamical applications.

\subsection{Wave kernel and small-time analysis}\label{subsec:wave}

Let
\begin{equation}\label{eq:wave-def}
  U(t) := e^{\,it\sqrt{\Delta-1/4}}, 
  \qquad 
  W(t) := \cos\!\big(t\sqrt{\Delta-1/4}\big) = \tfrac12\big(U(t)+U(-t)\big).
\end{equation}
Denote by $K_t(z,w)$ the Schwartz kernel of $W(t)$ on $X$. On the universal cover $\HH$,
the corresponding kernel depends only on the hyperbolic distance $\rho=d(z,w)$ and admits
the Hadamard parametrix for $|t|$ small:
\begin{equation}\label{eq:hadamard}
  K_t^{\HH}(z,w) \;=\; (2\pi)^{-1}\,\partial_t\!\Big( \mathbf{1}_{\{|t|>\rho\}}\,(t^2-\rho^2)^{-1/2} \Big)
  \,+\, \sum_{j=0}^{N} a_j(z,w)\,|t|^{2j-1} \,+\, R_N(t;z,w),
\end{equation}
where $a_j$ are smooth coefficients determined by the geometry and
$R_N$ is smooth with bounds uniform for $|t|\le t_0\ll 1$ on compact sets. On the quotient,
\begin{equation}\label{eq:periodization}
  K_t(z,w) \;=\; \sum_{\gamma\in\Gamma} K_t^{\HH}(z, \gamma w),
\end{equation}
which converges rapidly for $|t|$ in a fixed compact set (cf.\ finite propagation on compacta,
together with exponential decay of the tail of the $\Gamma$-sum for fixed $t$).

\paragraph{Fourier representation of localized projectors.}
For a real, even $h_R\in \mathcal{S}(\R)$ with
\begin{equation}\label{eq:window-support}
  h_R(t)\approx 1 \ \text{on}\ |t-R|\le R^\theta, 
  \qquad \supp \widehat{h}_R \subset [-c\,R^{-\theta},\,c\,R^{-\theta}],
\end{equation}
the spectral multiplier admits
\begin{equation}\label{eq:hR-functional}
  h_R\!\big(\sqrt{\Delta-1/4}\big)
  \;=\; \frac{1}{2\pi}\int_{\R} \widehat{h}_R(t)\,U(t)\,dt 
  \;=\; \frac{1}{\pi}\int_{0}^{\infty} \widehat{h}_R(t)\,W(t)\,dt.
\end{equation}
Let $\chi_Y$ be a smooth height cutoff (cf.\ \S\ref{subsec:cutoff}).
Define the truncated kernel and operator
\begin{equation}\label{eq:KR-trunc}
  K_R^Y(z,w) \;:=\; \chi_Y(z)\,\Big(\frac{1}{\pi}\int_0^\infty \widehat{h}_R(t)\,K_t(z,w)\,dt\Big)\,\chi_Y(w),
  \qquad 
  \mathsf{T}_R f(z)\;=\;\int_X K_R^Y(z,w)\,f(w)\,d\mu(w).
\end{equation}
By \eqref{eq:window-support}, only times $|t|\lesssim R^{-\theta}$ contribute.

\paragraph{Small-time structure on the diagonal.}
Combining \eqref{eq:hadamard}–\eqref{eq:KR-trunc} with stationary phase at $t=0$ and $\rho=0$,
one obtains the diagonal asymptotic (uniform on compacta of $X$):
\begin{equation}\label{eq:diag-asymp}
  K_R^Y(z,z)
  \;=\; \frac{C_\eta}{2\pi}\,R^{1+\theta}\,\chi_Y(z) 
  \;+\; \sum_{m=1}^{M-1} c_m(z)\,R^{1+\theta-2m\theta}\,\chi_Y(z)
  \;+\; \mathcal{E}_M(z;R),
\end{equation}
where $C_\eta=\int_{\R}\eta(u)\,du$ for $h_R(t)=\eta\!\big((t-R)/R^\theta\big)$,
the $c_m(z)$ are smooth coefficients depending polynomially on curvature invariants
(here constant), and
\begin{equation}\label{eq:diag-error}
  \|\mathcal{E}_M(\cdot;R)\|_{L^\infty(X)} 
  \;\ll\; R^{1+\theta-2M\theta} \;+\; R^{\theta}\,Y^{-1}
\end{equation}
(the second term stems from derivatives of $\chi_Y$, cf.\ $|\nabla^k \chi_Y|\ll Y^{-k}$).

\paragraph{Pointwise and off-diagonal bounds.}
For $|t|\lesssim R^{-\theta}$ and any $N\ge 0$,
\begin{equation}\label{eq:Kt-pointwise}
  |K_t(z,w)| \;\ll_N\; (1+R\,d(z,w))^{-N} \,+\, \mathbf{1}_{\{||t|-d(z,w)|\le c/R\}}\,(R\,|t|)^{-1/2},
\end{equation}
uniformly on $X_{\mathrm{thick}}$ with constants polynomial in $\inj(X)^{-1}$. Integrating
\eqref{eq:Kt-pointwise} against $\widehat{h}_R(t)$ yields, for any $N$,
\begin{equation}\label{eq:KR-offdiag}
  |K_R^Y(z,w)| \;\ll_N\; \chi_Y(z)\chi_Y(w)\,(1+R\,d(z,w))^{-N} 
  \;+\; \chi_Y(z)\chi_Y(w)\,R^{1/2}\,\mathbf{1}_{\{d(z,w)\lesssim R^{-\theta}\}}.
\end{equation}
In particular, away from the $R^{-\theta}$-neighborhood of the diagonal,
\begin{equation}\label{eq:KR-rapid-decay}
  d(z,w)\gg R^{-\theta}\quad \Longrightarrow\quad |K_R^Y(z,w)| \;\ll_N\; (R\,d(z,w))^{-N}.
\end{equation}

\paragraph{Operator norm bounds.}
Schur’s test with \eqref{eq:KR-offdiag} implies
\begin{equation}\label{eq:L2-bound}
  \|\mathsf{T}_R\|_{L^2\to L^2} \;\ll\; 1,
\end{equation}
with implied constant polynomial in $\vol(X)$, $\inj(X)^{-1}$, and the number of cusps.
Moreover, for any $s\in\R$,
\begin{equation}\label{eq:Hs-bound}
  \|\mathsf{T}_R\|_{H^s\to H^s} \;\ll_s\; 1.
\end{equation}
Approximate idempotence follows from the support property \eqref{eq:window-support} and
the spectral theorem:
\begin{equation}\label{eq:idempotence}
  \|\mathsf{T}_R^2-\mathsf{T}_R\|_{L^2\to L^2}
  \;\le\; \sup_{t\ge 0} \big|h_R(t)^2-h_R(t)\big|
  \;\ll\; R^{-\theta}.
\end{equation}

\paragraph{Trace and the identity term.}
Integrating \eqref{eq:diag-asymp} over $X$ gives
\begin{equation}\label{eq:trace-identity}
  \Tr(\mathsf{T}_R) 
  \;=\; \int_X K_R^Y(z,z)\,d\mu(z)
  \;=\; \frac{C_\eta}{2\pi}\,R^{1+\theta}\,\vol_{\mathrm{eff}}(X;Y)
  \;+\; O\!\big(R^{1-\theta}\big) \;+\; O\!\big(R^{1+\theta}Y^{-1}\big),
\end{equation}
which matches the localized Weyl density and exhibits the explicit influence of cusp
truncation via $\vol_{\mathrm{eff}}(X;Y)$.

\paragraph{Localized Egorov for $|t|\le R^{-\theta}$.}
Let $A=\Op_h(a)$ with $h=R^{-1}$ and $a\in S^0$. Then for $|t|\le c\,R^{-\theta}$,
\begin{equation}\label{eq:egorov}
  U(-t)\,A\,U(t) \;=\; \Op_h(a\circ g^t) \;+\; O\!\big(h^{1-\theta}\big)_{L^2\to L^2},
\end{equation}
where $g^t$ is the geodesic flow on $S^*X$ and the remainder norm is polynomially
controlled by $\inj(X)^{-1}$. Inserting \eqref{eq:egorov} into \eqref{eq:hR-functional} yields
\begin{equation}\label{eq:commute-AR}
  \mathsf{T}_R\,A \;=\; \Op_h\!\Big( \big(\chi_Y\,a\,\chi_Y\big)\,\mathbf{1}_{||\xi|-R|\le c\,R^\theta} \Big)
  \;+\; O\!\big(R^{-\theta}\big)_{L^2\to L^2},
\end{equation}
which makes precise the phase-space localization of $\mathsf{T}_R$ and its stability under
short-time dynamics.

\paragraph{Summary.}
Equations \eqref{eq:KR-offdiag}–\eqref{eq:idempotence} show that $\mathsf{T}_R$ is a
self-adjoint, near-idempotent, microlocally localized operator with kernel supported at
spatio-temporal scales $d(z,w)\lesssim R^{-\theta}$, $|t|\lesssim R^{-\theta}$, and
\eqref{eq:trace-identity} identifies the main (identity) contribution to the trace.
These facts are the analytic backbone for the localized trace formula proved later.
