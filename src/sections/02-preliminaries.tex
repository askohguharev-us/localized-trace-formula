% File: src/sections/02-preliminaries.tex
\section{Preliminaries}\label{sec:prelim}

In this section we collect the analytic, geometric, and spectral preliminaries needed for the localized trace formula. The purpose of this exposition is twofold: first, to establish notation and conventions consistent throughout the paper; and second, to provide a careful survey of the background results which our arguments build upon. We emphasize explicit constants, uniformity with respect to the geometry of the surface, and compatibility with the microlocal framework developed in subsequent sections. The presentation is intentionally detailed, in order to make the later arguments as transparent and verifiable as possible.

\subsection{Hyperbolic surfaces and groups}\label{subsec:surfaces}

Let $\HH = \{ z = x + iy \in \mathbb{C} : y > 0\}$ denote the upper half–plane with the hyperbolic metric
\[
ds^2 = \frac{dx^2 + dy^2}{y^2}.
\]
The corresponding volume element is $d\mu(z) = y^{-2}\,dx\,dy$ and the Laplace–Beltrami operator is
\[
\Lap = -y^2(\partial_x^2 + \partial_y^2).
\]

Let $\Gamma \subset \PSL(2,\RR)$ be a discrete subgroup of finite covolume acting by Möbius transformations. We denote the quotient surface by
\[
X = \Gamma \backslash \HH.
\]
We assume throughout that $\Gamma$ is torsion–free to avoid orbifold complications; however, the majority of the arguments extend with minor modifications to the case of finite stabilizers. The surface $X$ has finite hyperbolic volume, may be compact or noncompact, and in the latter case contains finitely many cusps.

A cusp corresponds to a $\Gamma$–orbit of $\infty \in \partial \HH$. After conjugation, each cusp can be represented by the subgroup $\Gamma_\infty = \{ \pm \begin{pmatrix} 1 & n \\ 0 & 1 \end{pmatrix} : n \in \mathbb{Z}\}$ and admits a fundamental domain of the form
\[
\{ z = x+iy : |x| \leq 1/2, \ y \geq 1\}.
\]
The geometry near the cusp is thus modeled by $[1,\infty)_y \times S^1_x$ with the hyperbolic metric.

\subsection{Spectral decomposition}\label{subsec:spectral}

The Hilbert space $L^2(X)$ decomposes as
\[
L^2(X) = L^2_{\mathrm{cusp}}(X) \oplus L^2_{\mathrm{cont}}(X),
\]
where the cuspidal subspace $L^2_{\mathrm{cusp}}(X)$ is spanned by square–integrable eigenfunctions of $\Lap$, and the continuous part is generated by Eisenstein series associated to the cusps.

The discrete cuspidal spectrum consists of eigenvalues
\[
\Lap \phi_j = \left( \tfrac14 + r_j^2 \right) \phi_j, \qquad r_j \ge 0,
\]
with eigenfunctions $\phi_j \in L^2_{\mathrm{cusp}}(X)$ forming an orthonormal basis. The parameter $r_j$ is called the spectral parameter. The asymptotic distribution of $\{r_j\}$ is governed by Weyl’s law, which states that
\[
N(R) := \#\{ j : r_j \leq R\} \sim \frac{\vol(X)}{4\pi} R^2 \qquad (R\to\infty).
\]

The continuous spectrum is parametrized by $r \in \RR$ and consists of generalized eigenfunctions $E_\mathfrak{a}(z,1/2+ir)$ (Eisenstein series) attached to cusps $\mathfrak{a}$. The full spectral resolution of $\Lap$ is encoded by the spectral theorem, which provides the Plancherel identity and the expansion of $L^2$–functions in terms of $\{\phi_j\}$ and Eisenstein series.

\subsection{Selberg transform and test functions}\label{subsec:selberg}

Central to the trace formula is the Selberg transform, which associates to a compactly supported radial function $k \colon \HH \to \RR$ a spectral multiplier $\widehat{k}(r)$. Explicitly, if $k(z,w) = k(d(z,w))$ depends only on the hyperbolic distance $d(z,w)$, then the associated operator
\[
(Kf)(z) = \int_X k(z,w) f(w)\, d\mu(w)
\]
commutes with $\Lap$ and thus acts diagonally in the spectral decomposition. Its eigenvalue on $\phi_j$ is $\widehat{k}(r_j)$, and similarly on Eisenstein series. The transform $\widehat{k}(r)$ is given by the Harish–Chandra–Selberg formula involving Legendre functions.

In our localized setting, we choose families of test functions depending on the large parameter $R$, designed to restrict spectral contributions to the window $[R-R^\theta, R+R^\theta]$. The shape of these test functions, their decay, and their smoothness properties are crucial for obtaining effective error terms.

\subsection{Height cutoff and effective volume}\label{subsec:cutoff}

For noncompact surfaces, contributions from the cusp region must be carefully regulated. We introduce a cutoff in the height coordinate: for $Y > 0$, let
\[
\chi_Y(z) = \mathbf{1}_{\{ y \leq Y\}}(z),
\]
the indicator function restricting to points below height $Y$. This cutoff is used to define the \emph{effective volume}
\[
\vol_{\mathrm{eff}}(X;Y) = \int_{X} \chi_Y(z)\, d\mu(z).
\]
As $Y\to\infty$, $\vol_{\mathrm{eff}}(X;Y) \to \vol(X)$, but in applications we take $Y = R^\beta$ with $0<\beta<1$, balancing between truncation error and control of continuous spectrum.

The precise asymptotics of $\vol_{\mathrm{eff}}$ as a function of $Y$ are elementary to derive, but we record them carefully in Appendix~\ref{app:effvol}, as they play an essential role in isolating the discrete spectrum.

\subsection{Geometric side: closed geodesics}\label{subsec:geodesics}

The length spectrum of $X$ consists of lengths $\ell(\gamma)$ of closed geodesics corresponding to conjugacy classes of hyperbolic elements $\gamma \in \Gamma$. Each such class contributes an explicit term to the trace formula, involving $\ell(\gamma)$ and the Selberg transform of the test function.

The primitive length spectrum (lengths of primitive geodesics) forms the foundation, with powers of primitive elements producing repetitions. The contributions of short geodesics are especially significant in the localized regime, since the window size $R^\theta$ interacts with the exponential growth of geodesic classes.

We recall Buser’s inequality relating injectivity radius to the length of the shortest closed geodesic, and the general estimates bounding the growth of the length spectrum.

\subsection{Microlocal analysis framework}\label{subsec:microlocal}

Localization in spectral windows requires tools from microlocal analysis. In particular, we interpret spectral projectors as Fourier integral operators acting on $L^2(X)$, with canonical relations determined by geodesic flow on the unit cotangent bundle $S^*X$.

Key ingredients include:

\begin{itemize}
  \item The semiclassical pseudodifferential calculus adapted to the hyperbolic metric.
  \item Propagation of singularities for the wave equation on $X$.
  \item Egorov’s theorem, ensuring compatibility of projectors with dynamical evolution.
  \item Cutoffs in phase space designed to eliminate leakage across the spectral window.
\end{itemize}

The interaction between the microlocal framework and the cusp cutoff is delicate; we must verify that truncated projectors remain bounded and retain approximate orthogonality. This is the subject of later sections, but the present preliminaries record the basic operators and estimates.

\subsection{Notation and conventions}\label{subsec:notation}

Throughout the paper, constants implied by $O(\cdot)$ or $\ll$ notation are allowed to depend on $\varepsilon>0$ (arbitrarily small) but are otherwise uniform in the geometric parameters of $X$. We emphasize explicit dependence wherever possible.

The Fourier transform of a Schwartz function $f \in \mathcal{S}(\RR)$ is defined by
\[
\widehat{f}(\xi) = \int_{-\infty}^\infty f(x) e^{-2\pi i x \xi}\, dx.
\]
We use the convention $\Tr A$ for the operator trace when $A$ is trace–class. Indicator functions are denoted $\mathbf{1}_{\{\cdot\}}$.

We denote by $\injrad(X)$ the injectivity radius of $X$ and by $\ell(\gamma)$ the length of a closed geodesic. For an operator $T$, we write $\|T\|_{2\to2}$ for its norm on $L^2$.

\subsection{Summary}\label{subsec:summary}

The material presented in this section establishes the geometric, spectral, and analytic background for the localized trace formula. The main ingredients are:

\begin{enumerate}
  \item The structure of $X = \Gamma\backslash\HH$ and its cusp geometry.
  \item The decomposition of $L^2(X)$ into discrete and continuous spectrum.
  \item The Selberg transform and its role in diagonalizing convolution operators.
  \item The introduction of a height cutoff and the notion of effective volume.
  \item The role of closed geodesics and the length spectrum.
  \item The microlocal framework needed for spectral localization.
\end{enumerate}

These preliminaries prepare the ground for the detailed kernel constructions and microlocal projectors of the following sections.

% Continuation of File: src/sections/02-preliminaries.tex

\subsection{Spectral counting and Weyl law refinements}\label{subsec:weyl}

A central theme in spectral theory is the asymptotic distribution of eigenvalues of the Laplacian on $X$. As recalled above, the global Weyl law states
\[
N(R) = \#\{ j : r_j \leq R\} = \frac{\vol(X)}{4\pi} R^2 + O(R \log R).
\]
The error term has been sharpened in special cases, but in general the logarithmic factor remains the best available bound. For arithmetic surfaces, advanced techniques from analytic number theory yield slightly stronger remainder estimates.

In our localized framework, we require not only the global counting function $N(R)$ but also the statistics of eigenvalues in short intervals. Specifically, for a window of length $R^\theta$ centered at $R$, we study
\[
N(R,\theta) = \#\{ j : r_j \in [R-R^\theta, R+R^\theta]\}.
\]
The expected size of this quantity is approximately
\[
N(R,\theta) \sim \frac{\vol(X)}{2\pi} R \cdot R^\theta,
\]
up to acceptable error terms. The localized trace formula is designed precisely to capture this asymptotic, by filtering the spectral side and evaluating corresponding contributions on the geometric side.

\subsection{Eisenstein series and continuous spectrum}\label{subsec:eisenstein}

For each cusp $\mathfrak{a}$ of $X$, let $\Gamma_\mathfrak{a}$ denote the stabilizer subgroup in $\Gamma$. After conjugation we may assume $\Gamma_\mathfrak{a} = \Gamma_\infty$, and define the Eisenstein series
\[
E_\mathfrak{a}(z,s) = \sum_{\gamma \in \Gamma_\mathfrak{a}\backslash\Gamma} \Im(\sigma_\mathfrak{a}^{-1}\gamma z)^s,
\]
where $\sigma_\mathfrak{a} \in \PSL(2,\RR)$ is a scaling matrix sending $\infty$ to the cusp $\mathfrak{a}$. These series converge for $\Re(s)>1$, extend meromorphically to $s\in\CC$, and satisfy the functional equation
\[
E_\mathfrak{a}(z,s) = \sum_\mathfrak{b} \phi_{\mathfrak{a}\mathfrak{b}}(s) E_\mathfrak{b}(z,1-s),
\]
with scattering matrix $\Phi(s)=(\phi_{\mathfrak{a}\mathfrak{b}}(s))$. The analytic properties of $\Phi(s)$ play a key role in controlling continuous spectrum contributions.

The spectral expansion of $L^2(X)$ can be expressed as
\[
f(z) = \sum_j \langle f,\phi_j\rangle \phi_j(z) + \frac{1}{4\pi}\sum_\mathfrak{a} \int_{-\infty}^\infty \langle f, E_\mathfrak{a}(\cdot,\tfrac12+ir)\rangle E_\mathfrak{a}(z,\tfrac12+ir)\, dr.
\]
Here the measure $dr$ reflects the Plancherel theorem for the continuous spectrum. Our cutoff at height $y\le Y$ is designed to suppress large contributions from Eisenstein series, ensuring that the localized trace formula isolates the discrete spectrum.

\subsection{Wave kernel and propagation estimates}\label{subsec:wave}

The fundamental solution of the hyperbolic wave equation encodes dynamical propagation on $X$. Let
\[
K_t(z,w) = \cos\!\left(t \sqrt{\Lap - 1/4}\,\right)(z,w),
\]
the wave kernel at time $t$. Its singularities are supported on the geodesic flow: $K_t(z,w)$ is large when $d(z,w)=|t|$, and is smooth otherwise. Microlocal analysis of $K_t$ provides the basis for constructing projectors onto spectral windows.

By integrating $K_t$ against oscillatory weights, one obtains spectral projectors localized near frequency $R$ with window width $R^\theta$. Stationary phase analysis then yields estimates on their kernels, with amplitudes related to the geometry of closed geodesics and error terms controlled by propagation speed. This procedure is standard in semiclassical analysis, but on hyperbolic surfaces additional care is needed near cusps.

\subsection{Effective constants and geometric dependence}\label{subsec:constants}

A major obstacle in previous approaches to localized trace formulas has been the lack of explicit dependence of constants on geometric data. In number–theoretic applications, one requires bounds that are polynomial in invariants such as:

\begin{itemize}
  \item The injectivity radius $\injrad(X)$.
  \item The hyperbolic volume $\vol(X)$.
  \item Parameters describing the cusps (heights, widths).
\end{itemize}

Our formulation ensures that all constants in error terms are explicit and polynomially bounded in these invariants. This guarantees applicability to families of surfaces, such as congruence subgroups of $\PSL(2,\ZZ)$, where uniform estimates are crucial.

\subsection{Microlocal cutoffs and phase space}\label{subsec:cutoffs}

The microlocal analysis requires partitioning the unit cotangent bundle $S^*X$ into regions corresponding to spectral localization. We employ pseudodifferential operators $A_R$ whose symbols select frequencies in a narrow band around $R$. The composition $A_R \chi_Y$ combines spectral localization with cusp truncation.

Phase space cutoffs must satisfy three properties:

\begin{enumerate}
  \item \textbf{Sharpness:} Transition between inside and outside the window occurs on scale $R^\theta$.
  \item \textbf{Boundedness:} Operator norms remain uniformly bounded on $L^2(X)$.
  \item \textbf{Orthogonality:} Different windows produce approximately orthogonal projectors.
\end{enumerate}

These conditions are verified using semiclassical symbol calculus and energy estimates for the wave equation.

\subsection{Geodesic flow and dynamical background}\label{subsec:flow}

The geodesic flow on $S^*X$ is Anosov, exhibiting uniform hyperbolicity. This dynamical system underlies the chaotic features of eigenfunctions and the exponential growth of closed geodesics. The Selberg trace formula can be viewed as an equality between spectral data (eigenvalues) and dynamical data (periodic orbits of the flow).

In the localized setting, short spectral windows correspond to restricting attention to geodesics of length up to logarithmic order in $R$. Hyperbolicity ensures exponential mixing, which manifests as decay of correlations and allows power–saving estimates in error terms. These dynamical properties are encoded microlocally in the propagation of singularities and the distribution of closed geodesics.

\subsection{Notation for asymptotics}\label{subsec:asymptotics}

We adopt the following asymptotic notation, consistent with analytic number theory and semiclassical analysis:

\begin{itemize}
  \item $f(R) = O(g(R))$ means $|f(R)| \le C g(R)$ for some $C$ independent of $R$.
  \item $f(R) \ll g(R)$ is synonymous with $f(R)=O(g(R))$.
  \item $f(R) \asymp g(R)$ means both $f(R) \ll g(R)$ and $g(R) \ll f(R)$.
  \item $f(R) \sim g(R)$ means $\lim_{R\to\infty} f(R)/g(R)=1$.
\end{itemize}

Dependence of constants on $\varepsilon>0$ is always permitted unless explicitly stated otherwise. We also occasionally use Vinogradov notation with subscripts, such as $f(R) \ll_\delta g(R)$, to indicate dependence on a parameter $\delta$.

\subsection{Summary of preliminaries}\label{subsec:prelim-summary}

This completes the background material. We have established:

\begin{enumerate}
  \item The decomposition of $L^2(X)$ into cuspidal and continuous spectrum, with explicit basis elements.
  \item The role of Eisenstein series and scattering matrices in the continuous spectrum.
  \item The construction of test functions and projectors using the Selberg transform.
  \item The introduction of height cutoffs leading to effective volume estimates.
  \item The dynamical background of the geodesic flow and its Anosov properties.
  \item The framework of microlocal analysis, pseudodifferential cutoffs, and propagation estimates.
\end{enumerate}

These preliminaries provide the essential tools for the kernel construction in Section~\ref{sec:kernel} and the analysis of localized projectors in Section~\ref{sec:projector}. They also ensure that all constants are explicit, polynomially controlled, and uniform across families of surfaces.
