\section{Introduction}\label{sec:intro}

This preprint develops a localized (microlocal) version of the Selberg trace
formula on a finite-area hyperbolic surface
$X=\Gamma\backslash\HH$, with an emphasis on \emph{windowed} counting of
spectral data. The localization is achieved by inserting smooth
cutoffs both in configuration and spectral variables, thus producing a
trace distribution $\TR$ adapted to a prescribed microlocal region.

\smallskip
\noindent\textbf{Motivation.}
In problems of spectral counting one often requires information restricted
to a phase-space window (e.g. near a geodesic segment or in a fixed
angular sector). Classical trace formulae \cite{selberg1956,hejhal1976}
capture global information; a localized variant fits naturally with ideas
from semiclassical and microlocal analysis
\cite{zworski2012,dyatlovzworski2019}.

\smallskip
\noindent\textbf{Informal statement.}
Let $-\Lap$ be the Laplacian on $X$ with cuspidal eigenvalues
$\lambda_j=\tfrac14+r_j^2$ and eigenfunctions $\psi_j$.
Let $\chi_Y$ be a smooth spatial cutoff (removing cuspidal regions above
height $Y$) and let $h\in\mathcal{S}(\RR)$ be an even test function with
compactly supported Fourier transform. The localized trace distribution is
\[
  \TR := \sum_j h_R(r_j)\,\|\chi_Y\psi_j\|_{L^2(X)}^2 \;+\; \text{(geometric side)},
\]
where $h_R(r):=h(r/R)$ controls the width of the spectral window. We show
that for admissible $(\theta,\beta)$ (precise ranges below) one obtains an
asymptotic expansion whose main term matches the expected (windowed) Weyl
law, together with power saving remainders.

\smallskip
\noindent\textbf{Contributions.}
\begin{itemize}
  \item A microlocally localized trace identity intertwining the spectral
        window $h_R$ with spatial truncation $\chi_Y$;
  \item Precise ranges of admissible parameters $(\theta,\beta)$ dictated
        by stationary phase and decay of the Selberg transform;
  \item A windowed Weyl law with error $O(R^{1-\varepsilon(\theta,\beta)})$,
        where $\varepsilon(\theta,\beta)=\min\{\theta,1-\tfrac12\beta\}$.
\end{itemize}

\smallskip
\noindent\textbf{Organization.}
Section~\ref{sec:prelim} records geometric and analytic preliminaries,
cutoff conventions and normalizations of transforms.
The main localized identity is proved in later blocks; here we state the
minimal skeleton needed for cross-referencing.
