\section{Introduction}\label{sec:intro}

The theory of trace formulas occupies a central position in spectral geometry and global analysis. Since Selberg’s pioneering work \cite{selberg1956}, the Selberg trace formula has provided a bridge between spectral data of the Laplace--Beltrami operator on negatively curved surfaces and the geometry of closed geodesics. In its classical form, the trace formula is global: it couples the entire spectrum—discrete and continuous—to geometric conjugacy classes, yielding a powerful but intrinsically nonlocal identity. While this global perspective underlies many foundational developments, numerous analytic questions demand \emph{localized} control, in particular an ability to interrogate the spectrum inside short frequency windows and to extract effective, quantitatively stable information that scales polynomially with geometric invariants of the surface.

This article develops a \emph{localized trace formula} for finite-area hyperbolic surfaces that isolates the \emph{discrete cuspidal spectrum} in short windows and, crucially, \emph{dispenses with explicit Eisenstein-series analysis}. The construction proceeds by combining microlocal cutoffs in frequency with a geometric cutoff in height inside cusp regions. The resulting identity features (i) a purely spectral term supported on a window $I_R=[R-R^\theta,\,R+R^\theta]$ with $0<\theta<1$, (ii) an explicit geometric contribution from short closed geodesics, and (iii) a power-saving remainder $O\!\big(R^{1-\varepsilon(\theta,\beta)}\big)$, with constants that depend at most polynomially on the basic geometric data of the surface. In particular, we obtain a \emph{windowed Weyl law} for the cuspidal spectrum on noncompact finite-volume surfaces with quantitative error control.

\subsection{Historical context and motivation}

Selberg’s original identity \cite{selberg1956} and its analytic elaboration in Hejhal’s monographs \cite{hejhal1976,hejhal1983} established the trace formula as a primary tool for relating spectra and length spectra on hyperbolic surfaces, including finite-volume noncompact cases with cusps. The spectral theory on manifolds with cusps was put on a modern footing by M\"uller \cite{mueller1983}, who developed the analytic framework required to handle continuous spectrum and scattering phenomena. On the side of applications to eigenfunctions, Iwaniec and Sarnak \cite{iwaniec1995} used trace methods to obtain landmark bounds for sup norms, while Buser’s geometric exposition \cite{buser1992} clarified the interplay between geometry and spectral structure. At the semiclassical and microlocal interface, Zworski \cite{zworski2012} and Dyatlov--Zworski \cite{dyatlovzworski2019} developed tools that illuminate propagation, resonances, and the phase-space viewpoint crucial for localization; classical insights of Chazarain \cite{chazarain1974} on Poisson-type relations remain inspirational for transforming global identities into statements with microlocal content.

Despite these advances, a localized identity for finite-volume noncompact surfaces that cleanly isolates short windows of the \emph{cuspidal} spectrum—without entanglement with Eisenstein series—has remained elusive. The present work fills this gap. Our construction is designed for quantitative analysis: the constants are controlled in terms of basic geometric invariants (volume, injectivity radius, systole), and the remainder exhibits a genuine power saving in $R$ that is uniform across a natural range of window exponents $\theta$ and cusp-height exponents $\beta$.

\subsection{Main difficulties and strategy}

Three intertwined obstacles must be overcome to localize the trace formula in noncompact finite-volume settings:

\begin{itemize}
  \item \textbf{Continuous spectrum and scattering.} On finite-area surfaces, Eisenstein series contribute a continuous part whose analytic behavior is delicate in short windows. Classical global identities accommodate this contribution; a localized identity must either capture or avoid it without sacrificing quantitative control.
  \item \textbf{Microlocal windowing compatible with cusp geometry.} Constructing sharp frequency projectors that respect the hyperbolic dynamics and remain compatible with cusp neighborhoods requires microlocal cutoffs that are stable under geodesic flow and that can be paired with height cutoffs without introducing spurious boundary terms.
  \item \textbf{Effectivity of constants.} For applications, bounds must exhibit polynomial dependence on geometric data. This precludes reliance on crude truncations that produce losses exponential in height or length, and it disfavors global decompositions whose constants degrade when restricting to short windows.
\end{itemize}

Our approach addresses these issues simultaneously by introducing a two-parameter localization scheme. Let $0<\theta<1$ and $\beta>0$. For a large parameter $R\gg1$ we consider a frequency window $I_R=[R-R^\theta,\,R+R^\theta]$ and enforce a height cutoff $y\le Y=R^\beta$ in cusp coordinates. We build a microlocal operator $\TR$ adapted to $I_R$ and supported below height $Y$, and we analyze $\Tr(\TR f)$ for admissible test functions $f$ whose Fourier transform $\widehat{f}$ is supported at controlled scales. The height cutoff suppresses Eisenstein contributions, while the microlocal construction preserves the principal spectral mass in $I_R$. The outcome is a localized identity with an explicit geometric side and a power-saving remainder.

\subsection{Statement of the main result}

We now present the principal theorem in a self-contained form; proofs and refinements appear throughout Sections~\ref{sec:kernel}–\ref{sec:geometric}.

\begin{theorem}[Localized trace formula for cuspidal windows]\label{thm:intro}
Let $X=\Gamma\backslash\HH$ be a finite-area hyperbolic surface. Fix parameters $0<\theta<1$ and $\beta>0$, and for $R\gg1$ define the window $I_R=[R-R^\theta,\,R+R^\theta]$ and the height cutoff $y\le Y=R^\beta$ in cusp coordinates. There exists a family of microlocal window operators $\TR$ and an admissible class of test functions $f$ such that
\begin{equation}\label{eq:intro-trace}
  \Tr(\TR f)
  \;=\;
  \vol_{\mathrm{eff}}(X;R,\beta)\,\widehat{f}(0)
  \;+\;
  \sum_{\substack{\gamma\ \text{primitive}\\ \ell(\gamma)\le R^\theta}}
  \mathcal{A}_\gamma(R,\theta)\,\widehat{f}\!\big(\ell(\gamma)\big)
  \;+\;
  O\!\big(R^{1-\varepsilon(\theta,\beta)}\big),
\end{equation}
where:
\begin{itemize}
  \item $\vol_{\mathrm{eff}}(X;R,\beta)$ is the \emph{effective volume} induced by the height cutoff $y\le Y$ (defined explicitly in Appendix~\ref{app:effvol});
  \item the sum ranges over primitive closed geodesics $\gamma$ with length $\ell(\gamma)\le R^\theta$;
  \item $\mathcal{A}_\gamma(R,\theta)$ are explicit amplitudes depending on $\gamma$ and the window parameters (see \S\ref{sec:geometric});
  \item the remainder exponent $\varepsilon(\theta,\beta)>0$ is uniform on compact subsets of $(0,1)\times(0,\infty)$, and all implicit constants depend at most polynomially on the geometric invariants of $X$.
\end{itemize}
In particular, \eqref{eq:intro-trace} yields a windowed Weyl law for the cuspidal spectrum with power-saving remainder under the height cutoff.
\end{theorem}

The operator $\TR$ is constructed by composing a frequency-localized semiclassical quantization with a geometric cutoff in cusp neighborhoods. The frequency localization is implemented at scale $R^\theta$, while the geometric cutoff at height $Y=R^\beta$ suppresses continuous-spectrum artifacts without degrading the principal spectral mass in $I_R$.

\subsection{Effective volume and geometric amplitudes}

The identity term in \eqref{eq:intro-trace} equals the Fourier mass at zero times an \emph{effective volume} that reflects the loss of mass from truncation in the cusps. Writing the cusp neighborhoods in standard coordinates $(x,y)$ with hyperbolic metric $ds^2=(dx^2+dy^2)/y^2$, denote by $\mathcal{C}(Y)$ the union of regions $\{y>Y\}$ across cusps. Then
\[
  \vol_{\mathrm{eff}}(X;R,\beta)
  \;=\;
  \vol\!\big(X\setminus\mathcal{C}(Y)\big)
  \;+\;
  \mathrm{Err}_{\mathrm{vol}}(R,\beta),
\]
where $\mathrm{Err}_{\mathrm{vol}}(R,\beta)$ captures the microlocal leakage induced by the windowing and is shown to be $O\!\big(R^{-\sigma(\beta)}\big)$ for some $\sigma(\beta)>0$ (see Appendix~\ref{app:effvol}). This makes precise the heuristic that, under the height cutoff, the identity term behaves like the “truncated volume’’ times $\widehat{f}(0)$, up to a quantitatively controlled error.

For the geometric side with $\ell(\gamma)\le R^\theta$, we obtain amplitudes of the form
\begin{equation}\label{eq:intro-amp}
  \mathcal{A}_\gamma(R,\theta)
  \;=\;
  \frac{\ell(\gamma_0)}{2\sinh\!\big(\ell(\gamma)/2\big)}
  \,\Phi_{R,\theta}\!\big(\ell(\gamma)\big)
  \,+\, \mathrm{Err}_\gamma(R,\theta),
\end{equation}
where $\gamma_0$ is the primitive element underlying $\gamma$, $\Phi_{R,\theta}$ is an explicit window response derived from the frequency cutoff, and $\mathrm{Err}_\gamma(R,\theta)$ collects microlocal remainders summable over $\ell(\gamma)\le R^\theta$. Formula \eqref{eq:intro-amp} is a windowed refinement of the usual hyperbolic-term weight in the global trace, adapted to the frequency aperture $R^\theta$; see \S\ref{sec:geometric} for a detailed derivation and summation bounds.

\subsection{Comparison with prior approaches}

In the compact case, microlocal and semiclassical methods can localize spectral information by exploiting global ellipticity and the absence of continuous spectrum; see, e.g., \cite{zworski2012} for general tools and numerous applications. For finite-area noncompact surfaces, however, cusp geometry introduces continuous spectrum and scattering, and naive truncations create large, often uncontrollable constants. Our method avoids explicit Eisenstein analysis by coupling phase-space localization with a geometric height cutoff tuned to the window scale. This pairing is robust: it preserves the principal contribution of the cuspidal spectrum in $I_R$ while ensuring that the truncated identity term and the short-geodesic sum admit effective, polynomially controlled constants. The power-saving remainder further distinguishes the result from coarse truncation schemes and brings the noncompact case closer—in a localized sense—to the precision historically associated with compact settings.

\subsection{Novelty and contributions}

We summarize the main contributions:

\begin{enumerate}
  \item \textbf{Localized identity for noncompact finite-volume surfaces.} We construct a trace identity that isolates the cuspidal spectrum in short frequency windows, with no explicit Eisenstein contribution.
  \item \textbf{Microlocal window operators compatible with cusps.} The operators $\TR$ simultaneously implement frequency and geometric cutoffs in a way stable under the hyperbolic dynamics and amenable to sharp estimates.
  \item \textbf{Effective constants and power-saving remainder.} All constants depend at most polynomially on geometric invariants of $X$, and the remainder gains a uniform power $\varepsilon(\theta,\beta)>0$.
  \item \textbf{Windowed Weyl law and short-geodesic control.} The identity yields a Weyl law in short intervals and identifies the precise geometric content supported on lengths $\ell(\gamma)\le R^\theta$.
  \item \textbf{Modularity of the construction.} The framework is designed to integrate with standard microlocal tools (propagation, Egorov-type estimates) and admits extensions to related settings in constant negative curvature.
\end{enumerate}

\subsection{Applications and future directions}

The localized identity opens several avenues:

\begin{itemize}
  \item \textbf{Fine statistics of eigenvalues in short windows.} The ability to isolate $I_R$ with power-saving control supports refined counting and spacing statistics compatible with semiclassical heuristics.
  \item \textbf{Quantitative bounds for eigenfunctions.} Combining \eqref{eq:intro-trace} with known microlocal propagation yields improved $L^\infty$ and $L^p$ estimates for cuspidal eigenfunctions subject to window constraints, in the spirit of \cite{iwaniec1995} but with windowed inputs.
  \item \textbf{Short-length geodesic sums.} The geometric side identifies the short-length regime $\ell(\gamma)\le R^\theta$ as the relevant one for the chosen aperture, suggesting new variants of short geodesic summation with explicit weights $\Phi_{R,\theta}$.
  \item \textbf{Truncation geometry and effective volume.} The effective volume $\vol_{\mathrm{eff}}(X;R,\beta)$ quantifies the spectral impact of cusp truncation at height $Y=R^\beta$, offering a calibrated parameter that may be optimized in diverse applications.
  \item \textbf{Extensions.} The methodology is adaptable to related rank-one settings in constant negative curvature and, with additional technical work, to situations with controlled variable curvature where suitable microlocal windowing persists.
\end{itemize}

\subsection{Organization of the paper}

Section~\ref{sec:prelim} recalls standard preliminaries on hyperbolic surfaces, cusp geometry, the Laplace--Beltrami operator, and the global Selberg trace formula in the finite-volume setting. Section~\ref{sec:kernel} constructs the localized kernel and records its phase-space properties under the geodesic flow. Section~\ref{sec:projector} builds the microlocal window operators $\TR$ that implement the frequency aperture $R^\theta$ together with the cusp height cutoff $y\le Y=R^\beta$. Section~\ref{sec:microlocal} establishes the microlocal decomposition underlying the spectral side of \eqref{eq:intro-trace} and proves the power-saving remainder. Section~\ref{sec:geometric} computes the geometric contributions, including $\vol_{\mathrm{eff}}(X;R,\beta)$ and the short-geodesic amplitudes $\mathcal{A}_\gamma(R,\theta)$, and carries out the necessary summations. Appendix~\ref{app:effvol} provides the effective-volume calculation and ancillary estimates required for the truncation analysis.

This concludes the introduction. The localized trace formula \eqref{eq:intro-trace} furnishes a quantitatively robust, windowed bridge between cuspidal spectral data and hyperbolic geometry on finite-area surfaces, aligning microlocal analysis with classical trace identities while maintaining effectivity throughout.
