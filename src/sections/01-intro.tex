\section{Introduction}\label{sec:intro}

The study of spectral geometry on finite-area hyperbolic surfaces lies at the intersection of analysis, geometry, and number theory. 
One of the most fundamental tools in this area is the Selberg trace formula, which provides an identity connecting spectral data of the Laplacian to geometric data of closed geodesics. 
While immensely powerful, the classical trace formula captures the spectrum as a whole and is not designed to resolve spectral information in narrow frequency windows. 
In many analytic and arithmetic applications, however, one requires such localized information: asymptotics for eigenvalues restricted to short intervals, behavior of eigenfunctions at specific scales, or fine comparison of discrete and continuous contributions. 
This motivates the development of a \emph{localized trace formula} capable of isolating the discrete cuspidal spectrum within controlled windows.

\subsection{Historical context}
The original form of the trace formula, due to Selberg~\cite{selberg1956}, established a profound duality between eigenvalues of the Laplacian on a hyperbolic surface $X=\Gamma\backslash\HH$ and lengths of closed geodesics on $X$. 
Subsequent developments by Hejhal~\cite{hejhal1976,hejhal1983}, M\"uller~\cite{mueller1983}, and many others extended the method to include cusp forms, Eisenstein series, and scattering theory. 
These advances created the modern framework of spectral theory for non-compact Riemannian manifolds of finite volume. 
Nevertheless, the classical trace formula always appears in a ``global'' guise: one sums over the full spectrum or integrates over the entire continuous part. 
In contrast, applications in analytic number theory often demand information restricted to short spectral intervals $[R-R^\theta,R+R^\theta]$. 
A major difficulty arises because continuous-spectrum contributions, especially those from Eisenstein series, interact delicately with any truncation in frequency. 
Until recently, it remained unclear how to isolate the discrete cuspidal spectrum without reintroducing unwanted continuous terms.

The desire for such localization is not new. 
Already in the work of Iwaniec–Sarnak~\cite{iwaniec1995}, questions of $L^\infty$ norms and QUE (Quantum Unique Ergodicity) naturally led to considerations of how to ``zoom in'' on spectral clusters. 
Later, semiclassical approaches, microlocal analysis, and scattering theory~\cite{zworski2012,dyatlovzworski2019} provided new tools, but a direct and robust \emph{localized trace identity} for finite-volume hyperbolic surfaces remained elusive. 
This paper provides such an identity.

\subsection{Main difficulties}
The challenge of isolating discrete cuspidal eigenvalues lies in several layers. 
First, the spectral decomposition involves Eisenstein series, whose contribution to the trace formula is typically global and resists restriction to windows. 
Second, hyperbolic geometry introduces infinitely many closed geodesics; controlling their contributions in short spectral bands requires delicate estimates. 
Third, one must achieve not only qualitative but also quantitative results, with explicit remainder terms and power savings, in order for the formula to be useful for analytic applications. 
Finally, the localization procedure must remain stable with respect to the geometric invariants of $X$, so that the constants in the resulting asymptotics are explicit and effective.

\subsection{Our approach}
We introduce a height cutoff $y\leq Y=R^\beta$ in the cuspidal regions and construct a family of microlocal projectors adapted to spectral windows of size $R^\theta$. 
These projectors isolate contributions of Laplace eigenfunctions with eigenvalue parameter $r_j\in[R-R^\theta,R+R^\theta]$, while suppressing the continuous spectrum. 
The localization is implemented at the operator level, using semiclassical pseudodifferential calculus to track phase space propagation and to control error terms. 
By combining these ingredients with a refined analysis of the geometric side, we obtain a new trace identity in which the main terms are explicit and the remainder enjoys power-saving decay.

\begin{theorem}[Localized trace formula]\label{thm:main}
For any finite-area hyperbolic surface $X=\Gamma\backslash\HH$, any $R\gg 1$, and parameters $0<\theta<1$, $\beta>0$, there exists a family of microlocal projectors $\TR$ such that
\[
\Tr \,\TR \;=\; \vol_{\mathrm{eff}}(X;Y)\,R^\theta
\;+\; \sum_{\substack{\gamma\in\Gamma \\ \ell(\gamma)\leq c\log R}}
\mathcal{A}_\gamma(R,\theta)
\;+\; O\!\left(R^{1-\varepsilon(\theta,\beta)}\right),
\]
where $\vol_{\mathrm{eff}}$ denotes an effective volume cutoff at height $Y=R^\beta$, the sum runs over primitive closed geodesics of length $\ell(\gamma)$ up to a logarithmic threshold, and the error exponent $\varepsilon(\theta,\beta)>0$ is uniform in $X$.
\end{theorem}

This theorem demonstrates that one can recover the discrete spectrum in short windows with a remainder strictly smaller than the trivial bound. 
The effective volume term replaces the global volume appearing in Selberg’s formula, and the geometric sum involves only short geodesics. 
The error term provides a power-saving improvement, establishing a genuine \emph{windowed Weyl law} for hyperbolic surfaces.

\subsection{Contributions and novelty}
The contributions of this paper can be summarized as follows:
\begin{itemize}
  \item We establish the first microlocally localized trace formula for finite-volume hyperbolic surfaces that isolates the discrete cuspidal spectrum without Eisenstein terms.
  \item We achieve a power-saving error term $O(R^{1-\varepsilon})$, improving upon previous approaches where only logarithmic or qualitative bounds were available.
  \item We express the main term using an explicit effective volume, allowing direct comparison across different surfaces and enabling quantitative applications.
  \item The geometric side of the formula is confined to closed geodesics of length $\ell(\gamma)\leq c\log R$, ensuring that only short orbits contribute.
  \item Our construction is robust: all constants in the formula depend polynomially on the geometric invariants of $X$.
\end{itemize}

\subsection{Outline of the paper}
Section~\ref{sec:prelim} recalls basic spectral and geometric preliminaries. 
Section~\ref{sec:kernel} constructs the kernel underlying the microlocal projectors. 
Section~\ref{sec:projector} defines the localization operators and proves their mapping properties. 
Section~\ref{sec:microlocal} develops the pseudodifferential analysis necessary for the error estimates. 
Section~\ref{sec:geometric} analyzes the geometric side, including contributions from closed geodesics. 
Finally, the appendices collect effective volume computations and technical verifications. 
The concluding section summarizes results and discusses directions for future research.
