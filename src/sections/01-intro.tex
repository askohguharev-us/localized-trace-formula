\section{Introduction}
\label{sec:intro}

The trace formula of Selberg has long been recognized as a central tool in spectral geometry and analytic number theory. 
On compact hyperbolic surfaces, it provides an exact relation between the discrete Laplace spectrum and closed geodesics. 
On finite-volume non-compact surfaces, however, the continuous spectrum enters through Eisenstein series, complicating any attempt to isolate the cuspidal spectrum in short spectral windows. 
This difficulty has prevented the derivation of localized results analogous to windowed Weyl laws, despite their importance for the fine structure of eigenvalues.

\subsection{Motivation}
The global Weyl law describes the asymptotic distribution of Laplace eigenvalues, but in applications one often requires finer control on short intervals. 
In compact settings, microlocal techniques allow such refinements. 
For non-compact finite-volume surfaces, the lack of localization has left a significant gap in the theory. 
Our work addresses this by developing a \emph{microlocally localized trace formula} that completely avoids the contribution of Eisenstein series. 

\subsection{Main result}
Let $X = \Gamma \backslash \HH$ be a finite-area hyperbolic surface, with Laplacian eigenvalues $1/4 + r_j^2$. 
Fix parameters $\theta \in (0,1)$ and $\beta > 0$, and consider frequency windows of the form
\[
  I_R := [R - R^\theta, \, R + R^\theta], 
  \qquad R \to \infty.
\]
Introduce a cutoff at cusp height $y \leq Y = R^\beta$. 
We construct a family of microlocal projectors $\TR$ adapted to this cutoff, and show that they satisfy the following:

\begin{theorem}[Localized trace formula]\label{thm:main}
For any finite-area hyperbolic surface $X = \Gamma \backslash \HH$, the localized trace operator $\TR$ obeys the asymptotic expansion
\[
  \Tr \, \TR = \vol_{\mathrm{eff}}(X;Y,R,\theta) 
  \;+\; \sum_{\substack{\gamma \in \Gamma \\ \ell(\gamma) \leq R^\theta}} 
  \mathcal{A}_\gamma(R,\theta) 
  \;+\; O\!\left( R^{1-\varepsilon(\theta,\beta)} \right),
\]
where $\vol_{\mathrm{eff}}$ denotes an effective volume term depending polynomially on the geometric data of $X$, 
the second sum runs over closed geodesics of length at most $R^\theta$, 
and the error exponent $\varepsilon(\theta,\beta) > 0$ depends explicitly on the chosen parameters.
\end{theorem}

This expansion yields, in particular, a \emph{windowed Weyl law} for the cuspidal spectrum with effective constants. 

\subsection{Historical context}
Selberg’s original formula~\cite{selberg1956} and its extensions by Hejhal~\cite{hejhal1976,hejhal1983} and M\"uller~\cite{mueller1983} established a powerful global framework, but did not address localization in short spectral intervals. 
Iwaniec–Sarnak~\cite{iwaniec1995} and Buser~\cite{buser1992} emphasized the importance of such fine-scale control for analytic and geometric applications. 
Recent semiclassical approaches, such as those of Zworski~\cite{zworski2012} and Dyatlov–Zworski~\cite{dyatlovzworski2019}, introduced microlocal tools that inspired our method. 
Our contribution synthesizes these directions into a trace formula that is both localized and effective on finite-volume hyperbolic surfaces. 

\subsection{Outline of the paper}
In Section~\ref{sec:prelim} we recall background on hyperbolic geometry and spectral decomposition. 
Section~\ref{sec:kernel} constructs the microlocal kernel underlying our projector. 
Section~\ref{sec:projector} defines the localized trace operator and proves its basic properties. 
Section~\ref{sec:microlocal} develops the microlocal analysis needed to estimate the spectral side. 
Section~\ref{sec:geometric} evaluates the geometric contributions from short closed geodesics. 
Finally, the appendices collect effective volume computations and technical lemmas. 
The concluding section summarizes results and discusses further directions.
