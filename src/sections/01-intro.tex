\section{Introduction}\label{sec:intro}

This preprint develops a localized (microlocal) version of the Selberg trace
formula on a finite-area hyperbolic surface $X=\Gamma\backslash\HH$, with an
emphasis on \emph{windowed} counting of spectral data. The localization is
achieved by inserting smooth cutoffs both in configuration and spectral variables,
thus producing a trace distribution $\TR$ adapted to a prescribed microlocal region.

\smallskip
\noindent\textbf{Motivation.}
In spectral counting problems one often seeks information restricted to a
phase-space window (e.g.\ near a geodesic segment or within a fixed angular sector).
Classical trace formulae \cite{selberg1956,hejhal1976} capture global information;
a localized variant aligns naturally with tools of semiclassical and microlocal
analysis \cite{zworski2012,dyatlovzworski2019}, while avoiding the need to invoke
the full scattering theory in the cusps.

\smallskip
\noindent\textbf{Informal statement.}
Let $-\Lap$ be the Laplacian on $X$, with cuspidal eigenvalues
$\lambda_j=\tfrac14+r_j^2$ and $L^2$-normalized eigenfunctions $\psi_j$.
Fix a smooth spatial cutoff $\chi_Y(z):=\chi\!\big(y(z)/Y\big)$ with $Y=R^\beta$,
where $0<\beta<\tfrac12$, and let $h\in\mathcal{S}(\RR)$ be even with
$\supp\widehat{h}\subset[-c_0,c_0]$ for some $c_0>0$. For a spectral center
$R\gg1$ and a window exponent $0<\theta<1$, set
\[
  h_R(t):=h\!\left(\frac{t-R}{R^\theta}\right),
  \qquad
  \TR := \sum_j h_R(r_j)\,\|\chi_Y\psi_j\|_{L^2(X)}^2.
\]
Then for the admissible range
\[
  0<\beta<\tfrac12,
  \qquad
  0<\theta<\tfrac{1+\beta}{2},
\]
we establish a localized trace identity of the schematic form
\[
  \TR \;=\; \underbrace{\mathcal{I}_R(\chi_Y,h)}_{\text{identity / effective volume}}
  \;+\;
  \underbrace{\mathcal{G}_R(\chi_Y,h)}_{\substack{\text{sum over primitive closed}\\\text{geodesics }\ell(\gamma)\le c_0}}
  \;+\;
  O\!\big(R^{\,1-\varepsilon(\theta,\beta)}\big),
\]
with an explicit positive exponent
\[
  \varepsilon(\theta,\beta)=\EpsDef \;>\; 0,
\]
and with constants depending polynomially on the geometric complexity
$C_{\mathrm{geo}}(X):=m+\injrad(X_{\mathrm{core}})^{-1}$ and on finitely many
$C^k$-seminorms of $\chi$ and $h$.
The identity term $\mathcal{I}_R$ features the \emph{effective volume}
$\int_X \chi_Y^2\,d\mu$, which admits a cusp-expansion with a shape-constant
$\kappa_\chi$, while $\mathcal{G}_R$ is a geometric sum over primitive closed
geodesics with $\ell(\gamma)\le c_0$ determined by $\supp\widehat{h}$.
As a consequence, one obtains a windowed Weyl law with power-saving error.

\smallskip
\noindent\textbf{Contributions (Block 0 skeleton).}
\begin{itemize}
  \item A microlocally localized trace identity intertwining the spectral window
        $h_R$ and the spatial truncation $\chi_Y$, stated with explicit admissible
        ranges $0<\beta<\tfrac12$ and $0<\theta<\tfrac{1+\beta}{2}$;
  \item An identity term controlled by the effective volume
        $\vol_{\mathrm{eff}}(Y)=\int_X \chi_Y^2\,d\mu$ and its cusp asymptotics
        (with a shape-constant $\kappa_\chi$);
  \item A geometric term $\mathcal{G}_R$ summing over \emph{primitive} closed
        geodesics with $\ell(\gamma)\le c_0$ (the support bound for $\widehat{h}$);
  \item A power-saving remainder $O(R^{1-\EpsDef})$, with constants polynomial in
        $C_{\mathrm{geo}}(X)$ and in finitely many seminorms of $\chi,h$;
  \item A windowed Weyl law as an immediate corollary.
\end{itemize}

\smallskip
\noindent\textbf{Organization.}
\Cref{sec:preliminaries} records geometric and analytic preliminaries,
cutoff conventions and transform normalizations. The kernel and projector
localization appear in \S\S\ref{sec:kernel}--\ref{sec:projector}, while
microlocal and geometric contributions are outlined in
\S\ref{sec:microlocal}--\ref{sec:geometric}. Full proofs are deferred to later
blocks; Block~0 contains the minimal skeleton needed for cross-referencing.
