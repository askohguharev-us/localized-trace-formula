\section{Introduction}
\label{sec:intro}

The trace formula, introduced by Selberg in the 1950s, has become one of the central analytic tools in modern spectral theory. 
It provides a bridge between two worlds: the spectral side, consisting of eigenvalues and resonances of the Laplace operator, 
and the geometric side, built from closed geodesics on a hyperbolic surface. 
Over the last decades, refinements of the trace formula have played a decisive role in problems ranging from 
Weyl laws for automorphic spectra to bounds on eigenfunctions, scattering theory, and quantum chaos.

Despite its broad applicability, the classical trace formula has a major limitation: 
its global character. It encodes the full spectrum of the Laplace operator at once, 
but offers little direct access to the distribution of eigenvalues in short intervals. 
For many applications in spectral geometry and analytic number theory, however, 
one needs precisely such \emph{local} information: asymptotics for eigenvalues 
in narrow spectral windows, with uniform control on error terms depending on the geometry of the surface. 

The goal of this paper is to provide such a tool. 
We establish a \emph{localized trace formula} that isolates the discrete cuspidal spectrum 
of a finite-area hyperbolic surface in frequency windows of size $R^\theta$ around a large parameter $R$, 
under a geometric cutoff $y \leq Y = R^\beta$. 
Our formula has three main contributions:
\begin{enumerate}
  \item an \emph{identity term}, involving an effective renormalized volume of the truncated surface,
  \item a \emph{geometric term}, expressed in terms of short closed geodesics of length $\ll R^\theta$, and
  \item a remainder term, which saves a fixed power $R^{-\varepsilon(\theta,\beta)}$ compared to the trivial bound. 
\end{enumerate}
This refinement avoids continuous-spectrum contributions entirely and gives a sharp, windowed Weyl law with explicit constants depending polynomially on the geometric data of the surface.

\subsection*{Historical context}
Selberg’s original trace formula \cite{selberg1956} laid the foundation for harmonic analysis on hyperbolic surfaces, 
and subsequent work by Hejhal \cite{hejhal1976, hejhal1983} developed it into a versatile analytic tool. 
Further advances by Müller \cite{mueller1983}, Iwaniec–Sarnak \cite{iwaniec1995}, and others demonstrated 
its central role in bounding eigenfunctions and understanding cusp forms. 
Microlocal refinements, such as those in the works of Buser \cite{buser1992} and 
Dyatlov–Zworski \cite{dyatlovzworski2019}, highlighted the possibility of isolating contributions 
from specific regions in phase space. 
Our approach builds on this lineage but introduces a crucial localization in both frequency and geometry, 
achieving uniform power savings in spectral windows.

\subsection*{Main results}
Our principal theorem (\cref{thm:main}) states that for a finite-area hyperbolic surface 
$X = \Gamma \backslash \HH$, the trace of an appropriately microlocalized spectral projector 
onto eigenvalues in the window $[R-R^\theta, R+R^\theta]$ admits the expansion
\[
  \TR(f) = \text{Identity}(R,\theta,\beta) + \text{Geometric}(R,\theta,\beta) 
  + O\!\left(R^{1-\varepsilon(\theta,\beta)}\right),
\]
where the first two terms are given explicitly and the remainder term exhibits power savings. 
Precise formulations are given in \cref{sec:kernel,sec:projector,sec:microlocal,sec:geometric}. 

\subsection*{Structure of the paper}
The organization is as follows. 
In \cref{sec:prelim}, we recall background on hyperbolic surfaces and spectral theory. 
In \cref{sec:kernel}, we introduce the localized kernel and establish its basic analytic properties. 
\cref{sec:projector} describes the microlocal spectral projector, and \cref{sec:microlocal} 
derives the main localization estimates. 
\cref{sec:geometric} contains the analysis of geometric contributions from closed geodesics. 
Finally, in the appendices we record auxiliary computations, including the effective volume term and technical lemmas.

\subsection*{Contributions}
To summarize, the key novelties of this work are:
\begin{itemize}
  \item a trace formula localized simultaneously in frequency and geometry, 
  \item complete removal of continuous-spectrum contributions,
  \item a power-saving remainder term uniform in the geometric data of the surface, and
  \item an explicit windowed Weyl law with effective constants. 
\end{itemize}

This localized trace formula provides a new analytic tool for the study of automorphic spectra, 
with potential applications to eigenvalue spacing, bounds on cusp forms, 
and the analysis of quantum chaos on hyperbolic surfaces.
