\section{Introduction}\label{sec:intro}

The theory of trace formulas stands at the very core of modern spectral geometry and global analysis. 
Since the pioneering work of Selberg in the 1950s, the Selberg trace formula has served as a central tool for connecting spectral data of Laplace--Beltrami operators on hyperbolic surfaces with the geometry of closed geodesics. 
The standard trace formula provides a global identity involving the full spectrum, continuous and discrete alike, together with the geometric side consisting of identity, hyperbolic, and parabolic contributions. 
It has become indispensable in the analysis of automorphic forms, the distribution of Laplace eigenvalues, and applications to analytic number theory.

Despite its fundamental power, the classical trace formula is not adapted to situations in which one wishes to probe spectral data only in \emph{short frequency windows}. 
In many analytic and arithmetic applications, one is not interested in the entire spectrum but rather in localized portions, such as eigenvalues in a window $[R-R^{\theta},R+R^{\theta}]$ with $0<\theta<1$. 
Such localization arises naturally in the study of the fine structure of eigenvalue distributions, the equidistribution of eigenfunctions, quantum unique ergodicity, and problems involving sup-norm bounds. 
For compact hyperbolic surfaces, windowed versions of the Weyl law have been studied using microlocal analysis, semiclassical techniques, and the thermodynamic formalism. 
For finite-volume non-compact surfaces, however, the situation is substantially more delicate due to the presence of continuous spectrum and Eisenstein series.

The present work develops a \emph{localized trace formula} that isolates the discrete cuspidal spectrum of finite-volume hyperbolic surfaces in short windows while avoiding explicit use of Eisenstein series. 
This represents a substantial technical advance: the continuous spectrum, which has traditionally complicated the analysis, is completely sidestepped by our method. 
Instead, we introduce microlocal projectors adapted to frequency cutoffs and height restrictions, yielding an identity in which the spectral side is concentrated on the desired window and the geometric side involves only effective, explicitly controlled contributions.

\subsection{Historical context}

Selberg's original formula \cite{selberg1956} remains the archetype of trace identities in spectral geometry. 
Hejhal's two-volume monograph \cite{hejhal1976,hejhal1983} systematized its analytic theory and provided applications to the distribution of zeros of $L$-functions and the study of Maass forms. 
M\"uller's seminal paper \cite{mueller1983} extended the spectral theory to manifolds with cusps, developing analytic tools that have since become standard in the field. 
Later, Iwaniec and Sarnak \cite{iwaniec1995} employed the trace formula to bound $L^\infty$ norms of eigenfunctions, while Buser \cite{buser1992} offered a geometric viewpoint linking the spectrum to the length spectrum of closed geodesics.

At the semiclassical interface, Zworski \cite{zworski2012} and Dyatlov--Zworski \cite{dyatlovzworski2019} developed scattering and microlocal frameworks that clarified the role of resonances and provided refined propagation estimates. 
Chazarain's Poisson formula \cite{chazarain1974} introduced microlocal ideas that later inspired many developments in spectral asymptotics. 
More recent works on arithmetic surfaces (e.g. Booker--Str\"ombergsson) highlighted the importance of effective constants and the need for windowed trace identities suitable for explicit estimates. 
However, a full-fledged localized trace formula for finite-volume non-compact hyperbolic surfaces had not yet been achieved.

\subsection{Main difficulties}

Several fundamental obstacles have prevented the localization of the Selberg trace formula in non-compact settings:

\begin{itemize}
  \item \textbf{Continuous spectrum:} Finite-volume hyperbolic surfaces possess Eisenstein series and scattering poles, whose contributions are difficult to truncate in short frequency windows.
  \item \textbf{Microlocal control:} Localization requires sharp projectors on frequency bands, but such projectors must be constructed in a way compatible with the geometry of cusps and the hyperbolic Laplacian.
  \item \textbf{Effective constants:} For applications, it is essential to obtain bounds whose constants depend at most polynomially on the geometric invariants of the surface, such as volume, systole, and injectivity radius. Achieving such effective control has remained a longstanding challenge.
\end{itemize}

Our approach introduces a novel microlocal cutoff depending both on frequency and on height in the cusp regions. 
This cutoff simultaneously restricts to a window $[R-R^{\theta},R+R^{\theta}]$ and enforces $y\le Y=R^\beta$, thereby eliminating contributions of Eisenstein series and yielding an identity involving only the cuspidal spectrum. 
The geometric side of the formula then consists of three parts: an identity term depending on an \emph{effective volume}, a contribution from closed geodesics of length $\le R^{\theta}$, and an error term with a power saving.

\subsection{Statement of the main result}

We now state the principal theorem of this article in simplified form. 
A complete version with all technical conditions will be presented in Section~\ref{sec:mainresults}.

\begin{theorem}[Localized trace formula]\label{thm:intro}
Let $X=\Gamma\backslash\HH$ be a finite-area hyperbolic surface. 
Fix parameters $0<\theta<1$ and $\beta>0$, and define the spectral window $I_R=[R-R^{\theta},R+R^{\theta}]$ with height cutoff $y\le Y=R^\beta$. 
Then, for suitable test functions $f$, one has the identity
\[
\Tr(\mathsf{T}_R f) \;=\; \vol_{\mathrm{eff}}(X,R,\beta)\, \hat{f}(0) \;+\; \sum_{\ell(\gamma)\le R^{\theta}} \mathcal{A}_\gamma(R,\theta) \, \hat{f}(\ell(\gamma)) \;+\; O(R^{1-\varepsilon(\theta,\beta)}),
\]
where:
\begin{itemize}
  \item $\vol_{\mathrm{eff}}(X,R,\beta)$ denotes the effective volume under the cusp cutoff $y\le Y$,
  \item the sum runs over primitive closed geodesics $\gamma$ with length $\ell(\gamma)\le R^{\theta}$,
  \item $\mathcal{A}_\gamma(R,\theta)$ are explicit amplitude factors depending on $\gamma$ and the window parameters,
  \item the error term admits a power saving $R^{1-\varepsilon(\theta,\beta)}$ with $\varepsilon(\theta,\beta)>0$.
\end{itemize}
\end{theorem}

This theorem encapsulates the essence of our contribution: a localized identity that isolates the cuspidal spectrum, has explicit and effective geometric terms, and achieves a power-saving error bound.

\subsection{Novelty and contributions}

The contributions of this paper can be summarized as follows:

\begin{enumerate}
  \item \textbf{First localized trace formula for non-compact hyperbolic surfaces:} To our knowledge, this is the first construction that completely avoids Eisenstein series while isolating short spectral windows.
  \item \textbf{Microlocal projectors:} We introduce projectors adapted simultaneously to frequency and cusp geometry, a new tool that may find further applications.
  \item \textbf{Effective constants:} All constants are polynomially bounded in geometric invariants, ensuring applicability to quantitative problems.
  \item \textbf{Power-saving error:} The remainder term achieves a power saving uniform in $\theta,\beta$, extending known results from the compact case.
  \item \textbf{Windowed Weyl law:} As a corollary, we obtain a Weyl law in short intervals with effective error bounds.
\end{enumerate}

These contributions address long-standing gaps in the literature and provide a framework for further developments in spectral theory, automorphic forms, and mathematical physics.

\subsection{Structure of the paper}

The remainder of this article is organized as follows:

\begin{itemize}
  \item Section~\ref{sec:prelim} recalls the necessary preliminaries on hyperbolic surfaces, the Laplacian, and the Selberg trace formula.
  \item Section~\ref{sec:kernel} introduces the localized kernel and its microlocal properties.
  \item Section~\ref{sec:projector} develops the microlocal projectors adapted to frequency and height cutoffs.
  \item Section~\ref{sec:microlocal} proves the microlocal decomposition underlying the main theorem.
  \item Section~\ref{sec:geometric} analyzes the geometric side, including effective volume and short closed geodesics.
  \item Section~\ref{sec:mainresults} collects the full statements of results and discusses corollaries.
  \item Appendices provide technical estimates and supporting computations.
\end{itemize}

This completes the introduction. The localized trace formula developed herein provides a new paradigm for analyzing spectral windows on non-compact hyperbolic surfaces, bridging microlocal analysis and classical trace identities while ensuring effective, polynomially controlled constants.
