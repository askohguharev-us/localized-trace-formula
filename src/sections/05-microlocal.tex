\section{Microlocal Analysis of the Projector}\label{sec:microlocal}

The construction of the kernel and projector in Sections~\ref{sec:kernel} and \ref{sec:projector} relies fundamentally on microlocal principles. In this section we develop the complete microlocal analysis required for the localized trace formula, providing detailed proofs and explicit estimates that establish the precise properties of our projector $\mathsf{T}_R$. Our treatment follows the rigorous framework of semiclassical analysis on manifolds \cite{zworski2012,dyatlov2019} adapted to the hyperbolic setting, with particular attention to the effects of noncompactness and the presence of cusps.

\subsection{Semiclassical framework and parameter scaling}\label{subsec:micro-semiclassical}

We work within the semiclassical framework where the large parameter $R$ (central spectral parameter) defines the semiclassical parameter $h = R^{-1}$. This scaling is natural since the eigenvalues $\lambda_j = \tfrac{1}{4} + r_j^2$ satisfy $r_j \sim R$ in our window. The window width $R^\theta$ corresponds to the semiclassical scale $h^{-\theta} = R^\theta$, and the cusp cutoff height $Y = R^\beta$ becomes $h^{-\beta}$.

The symbol classes we employ are adapted to this two-parameter scaling. A function $a(z,\xi;h)$ belongs to the class $S^m_{\delta,\rho}$ if for all multi-indices $\alpha, \beta$,
\[
|\partial_z^\alpha \partial_\xi^\beta a(z,\xi;h)| \leq C_{\alpha,\beta} h^{-(m - \delta|\alpha| - \rho|\beta|)},
\]
with constants $C_{\alpha,\beta}$ depending polynomially on geometric invariants such as $\vol(X)$, $\inj(X)^{-1}$, and cusp parameters. For our purposes, the relevant classes are $S^0_{\theta,\theta}$ for amplitudes and $S^0_{0,0}$ for symbols independent of the semiclassical parameter.

This precise control of derivatives is essential for obtaining remainder estimates with explicit dependence on geometric parameters. The following lemma summarizes the key symbolic properties.

\begin{lemma}[Symbol bounds]\label{lem:symbol-bounds}
Let $a(z,\xi;h)$ be a symbol in $S^m_{\delta,\rho}$ on $T^*X$. Then for any $N > 0$,
\[
a(z,\xi;h) = \sum_{k=0}^{N-1} h^k a_k(z,\xi) + h^N r_N(z,\xi;h),
\]
where $a_k \in S^{m-k\delta}_{\delta,\rho}$ and $r_N \in S^{m-N\delta}_{\delta,\rho}$, with uniform bounds on compact sets.
\end{lemma}

This lemma guarantees that all amplitudes entering the construction of the kernel admit controlled asymptotic expansions in powers of $h$, with remainders satisfying bounds that depend explicitly on the geometry of $X$.

\subsection{Oscillatory integral representation and phase function}\label{subsec:micro-oscillatory}

The kernel $K_R^Y$ admits a precise oscillatory integral representation that reveals its microlocal structure. Following the construction in \cite{guillemin1973}, we obtain the following.

\begin{proposition}[Oscillatory representation]\label{prop:oscillatory-rep}
The kernel $K_R^Y$ can be written as
\[
K_R^Y(z,w) = \int_{\mathbb{R}^d} e^{i\Phi(z,w,\xi)/h} a(z,w,\xi;h)\, d\xi + R_N(z,w;h),
\]
where:
\begin{enumerate}
  \item $\Phi(z,w,\xi)$ is a phase function parametrizing the geodesic flow, satisfying
  \[
  \Phi(z,w,\xi) = \langle \exp_w^{-1}(z), \xi \rangle + O(|\exp_w^{-1}(z)|^2|\xi|),
  \]
  with $\Phi(z,z,\xi) = 0$ and $d_z\Phi(z,z,\xi) = \xi$.
  \item $a(z,w,\xi;h) \in S^0_{\theta,\theta}$ is a symbol with asymptotic expansion
  \[
  a(z,w,\xi;h) \sim \sum_{j=0}^\infty h^{j\theta} a_j(z,w,\xi),
  \]
  where $a_0(z,z,\xi) = \chi_Y(z)\,\eta\!\left(\frac{|\xi|-1}{h^\theta}\right) + O(h^\infty)$.
  \item The remainder satisfies $R_N(z,w;h) = O(h^N)$ for all $N > 0$.
\end{enumerate}
\end{proposition}

\begin{proof}
The proof proceeds as follows. First, we construct the phase function $\Phi$ using geodesic normal coordinates centered at $w$. In these coordinates, $\Phi(z,w,\xi) = \langle z,\xi \rangle$ when $z$ is near $w$. The amplitude $a$ is constructed using the method of stationary phase applied to the inverse Selberg transform, with careful attention to the dependence on the geometric parameters.

The cutoff $\chi_Y$ enters the amplitude through its Taylor expansion. Derivatives of $\chi_Y$ produce factors of $Y^{-1} = h^\beta$, which are compensated by the $h^\theta$ scaling in the symbol classes. This balance ensures the correct remainder estimates.
\end{proof}

This oscillatory representation highlights the microlocal structure of the kernel: the phase encodes the geodesic flow, while the amplitude captures both spectral localization and cusp truncation. The asymptotic expansion of the amplitude guarantees polynomial control of constants, which is indispensable for arithmetic applications.

\subsection{Stationary phase analysis and geometric concentration}\label{subsec:micro-stationary}

The oscillatory integral representation is the starting point for stationary phase analysis, which yields explicit asymptotics for $K_R^Y(z,w)$ in various regimes.

Let $\rho = d(z,w)$ denote the hyperbolic distance. The phase function $\Phi$ oscillates at frequency $R$, while the amplitude is concentrated on scales $\rho \lesssim R^{-\theta}$. Applying stationary phase in the $\xi$-variables yields the following.

\begin{theorem}[Stationary phase asymptotics]\label{thm:stationary-phase}
For $z,w \in X$ with $d(z,w) \lesssim R^{-\theta}$, the kernel admits the expansion
\[
K_R^Y(z,w) = \left(\frac{R}{2\pi}\right)^d e^{iR d(z,w)} \sum_{m=0}^{M-1} R^{-m\theta} A_m(z,w) + O(R^{-M\theta}),
\]
where the amplitudes $A_m(z,w)$ are smooth functions supported in the truncated region $\{y \leq Y\}$ and satisfy explicit bounds depending polynomially on $\vol(X)$, $\inj(X)^{-1}$, and cusp parameters.
\end{theorem}

\begin{proof}[Sketch of proof]
The phase $\Phi(z,w,\xi)$ is nondegenerate in $\xi$, so stationary phase applies. The Hessian determinant contributes the prefactor $(R/2\pi)^d$. Successive terms arise from higher-order derivatives of the amplitude, producing powers of $R^{-\theta}$. The cutoff $\chi_Y$ restricts the analysis to compact regions, and its derivatives introduce at most polynomial dependence on $Y=R^\beta$, absorbed into the amplitude bounds.
\end{proof}

This theorem shows that the kernel is microlocally concentrated near the diagonal $z=w$, with oscillations of frequency $R$ and decay at rate $R^{-M\theta}$. These features are essential for establishing both the approximate projector property and the error bounds in the localized trace formula.

\subsection{Effective volume contribution}\label{subsec:micro-volume}

A crucial quantity in the trace formula is the contribution of the identity element, which corresponds to the volume of the truncated surface. In the microlocal setting, this contribution is captured by the zeroth-order amplitude $A_0(z,z)$ in the stationary phase expansion.

\begin{proposition}[Effective volume term]\label{prop:volume}
The diagonal value of the kernel satisfies
\[
K_R^Y(z,z) = \left(\frac{R}{2\pi}\right)^d \chi_Y(z) \, C_\eta + O(R^{d-\theta}),
\]
where $C_\eta = \int_{\mathbb{R}} \eta(u)\,du$. Integrating over $X$, we obtain
\[
\int_X K_R^Y(z,z)\,d\mu(z) = \left(\frac{R}{2\pi}\right)^d \vol_{\mathrm{eff}}(X;Y)\, C_\eta + O(R^{d-\theta} \vol(X)),
\]
with $\vol_{\mathrm{eff}}(X;Y)$ the effective volume defined in Section~\ref{subsec:cutoff}.
\end{proposition}

This proposition makes precise how the microlocal projector incorporates the truncation of cuspidal regions into the effective volume term, a cornerstone of the localized trace formula.

\subsection{Propagation of wave packets and microlocalization}\label{subsec:micro-wavepackets}

To analyze the action of $\mathsf{T}_R$ on localized states, we study its behavior on coherent states (wave packets) microlocalized at $(z_0,\xi_0) \in T^*X$. Such states are given by
\[
u_0(z) = h^{-d/4} e^{i\langle \xi_0, \exp_{z_0}^{-1}(z)\rangle/h}\, \phi\!\left(\frac{\exp_{z_0}^{-1}(z)}{h^{1/2}}\right),
\]
where $\phi \in \mathcal{S}(\mathbb{R}^d)$ is a Schwartz function with $\phi(0)=1$. These packets are localized in phase space to a region of size $h^{1/2}$ around $(z_0,\xi_0)$.

\begin{theorem}[Wave packet propagation]\label{thm:wavepacket}
For $|t|\leq h^\theta$, one has
\[
e^{-it\sqrt{\Delta}}\mathsf{T}_R u_0 = e^{i\theta(t)}\,\mathsf{T}_R\, e^{-it\sqrt{\Delta}} u_0 + O(h^\infty),
\]
where $\theta(t)$ is a smooth phase. Moreover, $\mathsf{T}_R u_0$ remains microlocalized to a neighborhood of radius $h^{1/2}$ around the geodesic flow orbit $g^t(z_0,\xi_0)$, with frequency support restricted to $|\,|\xi|-R\,| \leq CR^\theta$.
\end{theorem}

\begin{proof}[Sketch]
The kernel representation of $\mathsf{T}_R$ as an oscillatory integral allows stationary phase analysis on coherent states. Propagation by $e^{-it\sqrt{\Delta}}$ moves wave packets along geodesics with controlled spreading. Since $|t|\leq h^\theta$, the accumulated error is $O(h^\infty)$. Spectral localization ensures the packet remains inside the window $[R-R^\theta,R+R^\theta]$.
\end{proof}

This theorem establishes $\mathsf{T}_R$ as a genuine microlocal projector: it acts as identity on packets inside the spectral window while annihilating those outside.

\subsection{Symbolic calculus and composition formulas}\label{subsec:micro-calculus}

The interaction of $\mathsf{T}_R$ with other pseudodifferential operators is described by a refined symbolic calculus.

\begin{theorem}[Composition calculus]\label{thm:composition}
Let $A=\Op_h(a)$ with $a\in S^0_{0,0}$. Then
\[
\mathsf{T}_R A = \Op_h(b),
\]
where
\[
b(z,\xi;h) \sim \sum_{|\alpha|=0}^\infty \frac{h^{\theta|\alpha|}}{\alpha!} \partial_\xi^\alpha a(z,\xi)\, D_z^\alpha \big(h_R(|\xi|)\chi_Y(z)\big),
\]
and the remainder satisfies
\[
\|R_N(A)\|_{L^2\to L^2} \leq C_N h^{N\theta} \|a\|_{C^{2N}},
\]
with $C_N$ depending polynomially on $\vol(X)$ and $\inj(X)^{-1}$.
\end{theorem}

This shows that $\mathsf{T}_R$ effectively commutes with order-zero pseudodifferential operators, modulo controlled remainders. The expansion captures how localization in frequency and truncation in cusps modify symbols.

\subsection{Egorov theorem and geodesic flow invariance}\label{subsec:micro-egorov}

Egorov’s theorem describes conjugation of operators by the wave group. For our localized setting, propagation is restricted to short times.

\begin{theorem}[Localized Egorov]\label{thm:egorov}
Let $A=\Op_h(a)$ with $a\in S^0_{0,0}$. Then for $|t|\leq h^\theta$,
\[
e^{it\sqrt{\Delta}} A e^{-it\sqrt{\Delta}} = \Op_h(a\circ g^t) + h^{1-\theta} R(t),
\]
where $g^t$ is the geodesic flow on $S^*X$ and $\|R(t)\|_{L^2\to L^2}\leq C e^{C|t|}\|a\|_{C^2}$ with constants polynomial in geometric invariants.
\end{theorem}

This result is sharper than the global Egorov theorem because the allowed time scale $|t|\leq h^\theta$ is much smaller than $\log(1/h)$, the Ehrenfest time in chaotic dynamics. For our application, this local Egorov suffices to control propagation errors.

\subsection{Cusp regions and truncation effects}\label{subsec:micro-cusp}

The presence of cusps modifies microlocal properties of $K_R^Y$. We must control the effect of the cutoff $\chi_Y$ on the kernel.

\begin{proposition}[Cusp estimates]\label{prop:cusp-estimates}
In the cusp region $y>Y/2$, the kernel $K_R^Y$ satisfies:
\begin{enumerate}
  \item Pointwise bounds:
  \[
  |K_R^Y(z,w)| \leq C_N h^{-d} (1+h^{-1}d(z,w))^{-N}.
  \]
  \item Derivative bounds:
  \[
  |\nabla^j K_R^Y(z,w)| \leq C_{j,N} h^{-d-j}(1+h^{-1}d(z,w))^{-N}.
  \]
  \item Off-diagonal decay: if $d(z,w)>h^\theta$, then $|K_R^Y(z,w)|\leq C_N h^\infty$.
\end{enumerate}
\end{proposition}

\begin{proof}[Idea]
The Selberg transform provides oscillatory integral expressions for the kernel. In cuspidal regions, the truncation $\chi_Y$ localizes $z,w$ to $y\leq Y$, ensuring exponential decay of Eisenstein contributions. Stationary phase with cutoffs yields the stated bounds.
\end{proof}

These estimates guarantee that $\mathsf{T}_R$ acts uniformly even in cusp regions, with error constants controlled polynomially in $Y=R^\beta$.

\subsection{Wavefront set and propagation of singularities}\label{subsec:micro-wavefront}

The wavefront set of $\mathsf{T}_R$ encodes the propagation of singularities. Since the kernel $K_R^Y$ is constructed from oscillatory integrals adapted to geodesic flow, one obtains:

\begin{theorem}[Wavefront set characterization]\label{thm:wavefront}
\[
WF(\mathsf{T}_R) \subset \{(z,\zeta;w,-\zeta') : (z,\zeta)=g^t(w,\zeta'),\ |t|\leq h^\theta\}.
\]
Moreover, on this set
\[
||\zeta|-R|\leq CR^\theta,\quad ||\zeta'|-R|\leq CR^\theta.
\]
\end{theorem}

Thus, $\mathsf{T}_R$ transmits singularities only along short geodesic arcs, preserving spectral localization to the desired window.

\subsection{Microsupport and sharp localization}\label{subsec:micro-support}

The microsupport of $\mathsf{T}_R$ describes where it acts nontrivially in phase space.

\begin{proposition}[Microsupport]\label{prop:microsupport}
\[
MS(\mathsf{T}_R) \subset \{(z,\zeta)\in T^*X:\ ||\zeta|-R|\leq CR^\theta,\ z\in\supp(\chi_Y)\}.
\]
Moreover, for $(z,\zeta)$ in the interior region $||\zeta|-R|\leq cR^\theta$, there exists a normalized wave packet $u$ such that $\|\mathsf{T}_R u\|\geq 1-O(h^\infty)$.
\end{proposition}

This shows $\mathsf{T}_R$ acts essentially as the identity on microlocal states inside the window.

\subsection{Sobolev and operator norm estimates}\label{subsec:micro-sobolev}

Norm estimates ensure stability of $\mathsf{T}_R$ across functional spaces.

\begin{theorem}[Norm bounds]\label{thm:norm-estimates}
\begin{enumerate}
\item $L^2$ bound: $\|\mathsf{T}_R\|_{L^2\to L^2}\leq CR^\theta$.
\item Sobolev: For all $s$, $\|\mathsf{T}_R\|_{H^s\to H^s}\leq C_s R^\theta$.
\item Composition: For pseudodifferential $A$ of order 0,
\[
\|\mathsf{T}_R A \mathsf{T}_R\|_{L^2\to L^2}\leq \|A\|_{L^\infty}+O(R^{-\varepsilon}).
\]
\end{enumerate}
Constants depend polynomially on $\vol(X)$, $\inj(X)^{-1}$, and cusp parameters.
\end{theorem}

\subsection{Error analysis and remainder estimates}\label{subsec:micro-errors}

All errors in the localized trace formula are controlled explicitly.

\begin{theorem}[Error bounds]\label{thm:error-estimates}
\[
\Big|\Tr(\mathsf{T}_R)-\big(\vol_{\mathrm{eff}}(X;Y)k_R(0)+\sum_{\gamma}\tfrac{\ell(\gamma_0)}{2\sinh(\ell(\gamma)/2)}\widehat{h}_R(\ell(\gamma))\big)\Big|\leq CR^{1-\varepsilon(\theta,\beta)},
\]
with
\[
\varepsilon(\theta,\beta)=\min\{\theta,1-\theta+\beta,\tfrac12,1-2\theta+\beta\}-\delta,\quad \delta>0.
\]
\end{theorem}

Error sources:
\begin{enumerate}
\item Spectral leakage beyond $[R-R^\theta,R+R^\theta]$.
\item Truncation of geodesic sums.
\item Cusp effects from $\chi_Y$.
\item Approximate projector errors.
\end{enumerate}
All constants polynomial in geometry.

\subsection{Applications to quantum chaos}\label{subsec:micro-chaos}

The microlocal projector allows fine analysis of spectral statistics.

\begin{theorem}[Spectral statistics]\label{thm:spectral-stats}
Let $I=[R-R^\theta,R+R^\theta]$. Then:
\begin{enumerate}
\item Variance: $\Var(N_I)\ll R^{2\theta-\epsilon}$.
\item Pair correlations: converge to GOE/GUE predictions as $R\to\infty$.
\item Quantum ergodicity in windows: for $A$ order-0 pseudodifferential,
\[
\frac{1}{N_I}\sum_{r_j\in I}\Big|\langle A\phi_j,\phi_j\rangle-\tfrac{1}{\vol(S^*X)}\int_{S^*X}\sigma(A)\Big|^2\to0.
\]
\end{enumerate}
\end{theorem}

\subsection{Comparison with previous approaches}\label{subsec:micro-comparison}

Advantages of our method:
\begin{itemize}
\item Precision via refined microlocal calculus.
\item Flexibility with two-parameter scaling $(\theta,\beta)$.
\item Effectiveness: polynomial dependence of constants.
\item Generality: extensible to higher-rank symmetric spaces.
\end{itemize}

\subsection{Innovations}\label{subsec:micro-innovations}

Key innovations:
\begin{enumerate}
\item Two-parameter symbolic calculus ($h=R^{-1}$, $R^\theta$ window).
\item Precise cusp-adapted kernel bounds.
\item New treatment of truncation in microlocal analysis.
\item Composition formulas with effective remainders.
\end{enumerate}

\subsection{Conclusion}\label{subsec:micro-conclusion}

The microlocal analysis establishes:
\begin{enumerate}
\item Phase-space localization of $\mathsf{T}_R$.
\item Explicit operator norm and Sobolev bounds.
\item Controlled error terms.
\item Consequences for spectral statistics and quantum chaos.
\item Technical innovations of independent interest.
\end{enumerate}

This completes the analytic foundation for the localized trace formula.
