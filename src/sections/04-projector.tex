% File: src/sections/04-projector.tex
\section{The Microlocal Projector}\label{sec:projector}

The kernel $K_R^Y$ introduced in the previous section is designed to act as an approximate spectral projector onto cusp forms with spectral parameter $t_j$ in the interval $[R-R^\theta,\,R+R^\theta]$. In this section we rigorously develop the operator-theoretic aspects of this construction. Our objective is to show that $K_R^Y$ satisfies properties of idempotence and orthogonality up to small errors, that it can be normalized to become an almost-orthogonal projector, and that its action on test functions respects microlocal localization in both spectral and geometric variables.

The transition from kernel bounds to projector properties is conceptually natural but technically subtle. A projector is, by definition, an idempotent operator $P$ with $P^2=P$. Approximate projectors arise when $P^2-P$ is small in operator norm. In our setting, we must demonstrate that the kernel $K_R^Y$ acts nearly idempotently on cusp forms in the window and annihilates those outside, with polynomially controlled error terms.

\subsection{Definition of the projector operator}\label{subsec:proj-def}

Let $f\in L^2(X)$, where $X=\Gamma\backslash\HH$ is a finite-area hyperbolic surface. We define the action of the kernel $K_R^Y$ as
\[
(\TR f)(z) := \int_X K_R^Y(z,w)\,f(w)\,d\vol(w).
\]
By construction, $\TR$ is a self-adjoint, positive operator. The integral kernel $K_R^Y$ is smooth, compactly supported in both variables modulo $\Gamma$, and enjoys the decay properties recorded in Section~\ref{sec:kernel}. Thus $\TR$ is bounded on $L^2(X)$ and on Sobolev spaces $H^s(X)$, with bounds depending polynomially on $R$.

\subsection{Spectral decomposition of $\TR$}\label{subsec:proj-spectrum}
\paragraph{Context and related constructions.}
Our projector belongs to the general class of microlocal spectral multipliers and Fourier integral operators projecting onto thin frequency bands; see the foundational microlocal framework in \cite{hormander1994III} and the classical trace/dynamics interface in \cite{duistermaatguillemin1975}. For band-limited projectors on manifolds, standard Gaussian or Paley--Wiener mollifiers are often used to control tails of the spectral measure and wave kernels; cf.\ the spectral multiplier machinery and kernel bounds in \cite[Ch.~5--6]{sogge1993}. The novelty here is the simultaneous (\emph{i}) short-window scaling $R^\theta$ with explicit constants and (\emph{ii}) compatibility with cusp truncation, which forces a two-parameter balance between frequency and height cutoffs not present in the compact case.

Let $(\varphi_j)_j$ be an orthonormal basis of cuspidal eigenfunctions with spectral parameters $t_j$, and let $E(z,1/2+it)$ be the Eisenstein series forming the continuous spectrum. By unfolding the kernel expansion \eqref{eq:KR-spectral} with cutoff $\chi_Y$, we obtain
\[
\TR \varphi_j = \big(h_R(t_j) + O(R^{-\!A})\big)\varphi_j,
\]
for any fixed $A>0$, and
\[
\TR E(\cdot,1/2+it) = O(R^{-\!A}),
\]
uniformly in $t\in\RR$. Hence $\TR$ acts diagonally on the cusp spectrum, multiplying each eigenfunction by $h_R(t_j)$, while its action on the continuous spectrum is negligible. This spectral decomposition demonstrates that $\TR$ behaves as an approximate projector onto cusp forms in the window.

\subsection{Approximate idempotence}\label{subsec:proj-idempotence}

The essential projector property is idempotence. For $f\in L^2(X)$ we compute
\[
\TR^2 f = \TR(\TR f) = \int_X \Big(\int_X K_R^Y(z,u)K_R^Y(u,w)\,d\vol(u)\Big) f(w)\,d\vol(w).
\]
Thus the kernel of $\TR^2$ is the convolution $K_R^Y \star K_R^Y$. Using the Fourier transform properties of $h_R$, one finds
\[
\widehat{h_R * h_R}(t) = \widehat{h_R}(t)^2,
\]
so that $\TR^2$ acts by multiplication with $h_R(t_j)^2$ on eigenfunctions. Since $h_R(t_j)\approx 1$ for $t_j$ in the window and $\approx 0$ otherwise, we have
\[
\TR^2 - \TR = O(R^{-\!A}),
\]
in operator norm on $L^2(X)$. This is the desired approximate idempotence.

\subsection{Orthogonality and localization}\label{subsec:proj-orthogonality}

The operator $\TR$ should not only be idempotent but also localize orthogonally distinct spectral windows. Let $R_1,R_2$ be two distinct central frequencies with separation $|R_1-R_2|\gg R^\theta$. Define $\TRi$ by test functions $h_{R_i}$. Then for any eigenfunction $\varphi_j$,
\[
\TRi\varphi_j = h_{R_i}(t_j)\varphi_j.
\]
If $|R_1-R_2|\ge c R^\theta$ with $c\gg 1$, then $h_{R_1}(t_j)h_{R_2}(t_j)\approx 0$, hence
\[
\TRi\TRii = O(R^{-\!A}),
\]
demonstrating approximate orthogonality of spectral windows. This property is crucial for partitioning the spectrum into disjoint bands.

\subsection{Normalization and scaling}\label{subsec:proj-normalization}

To compare $\TR$ with the identity operator on the window, we normalize $h_R$ so that its average value on the window is one. That is, we impose
\[
\frac{1}{\#\{j:|t_j-R|\le R^\theta\}} \sum_{|t_j-R|\le R^\theta} h_R(t_j) \;=\; 1.
\]
This normalization guarantees that $\TR$ acts as an almost-identity on the window. Moreover, scaling by $R^{-\theta}$ ensures boundedness of operator norms across Sobolev scales.

\subsection{Microlocal action of $\TR$}\label{subsec:proj-micro}

The action of $\TR$ can be analyzed microlocally. Writing the kernel $K_R^Y(z,w)$ as a Fourier integral operator with phase $R\Phi(z,w,\xi)$, the critical points correspond to geodesic arcs of length $\le R^\theta$ joining $z$ and $w$. Hence $\TR$ maps wave packets of central frequency $R$ and width $R^{-\tfrac12}$ to wave packets of comparable size, propagating along the geodesic flow for time $\lesssim R^{-\theta}$. This microlocal behavior ensures that $\TR$ acts as a projector not only spectrally but also in phase space.

\subsection{Error terms and admissible parameters}\label{subsec:proj-errors}

The error terms in idempotence and orthogonality depend on the parameters $\theta$ and $\beta$. Explicitly, the remainder exponent $\varepsilon(\theta,\beta)$ appearing in the trace formula satisfies
\[
\varepsilon(\theta,\beta) = \min\{\theta,\,1-\theta+\beta,\,\tfrac12,\,1-2\theta+\beta\} - \delta,
\]
for arbitrarily small $\delta>0$. Admissibility requires that $\varepsilon(\theta,\beta)>0$, which holds for a nontrivial range of $(\theta,\beta)$. For instance, $\theta=1/2-\epsilon$, $\beta=1/2$ is admissible.

\subsection{Applications of projector properties}\label{subsec:proj-applications}

The approximate projector $\TR$ provides a versatile tool. Among its applications are:
\begin{enumerate}
\item \textbf{Localized Weyl laws:} Counting cusp eigenvalues in windows with power-saving error.
\item \textbf{Sup-norm estimates:} Projectors allow amplification of cusp forms in short intervals, yielding $L^\infty$ bounds.
\item \textbf{Quantum ergodicity:} Restricting eigenfunctions to windows facilitates equidistribution results on short scales.
\item \textbf{Spectral statistics:} Orthogonality of distinct $\TRi$ operators enables analysis of pair correlations in adjacent windows.
\end{enumerate}

\subsection{Comparison with alternative constructions}\label{subsec:proj-comparison}

Previous approaches to spectral projectors often relied on global kernels or Gaussian localization. These methods did not allow for polynomial control of error terms and failed in the presence of cusps. By contrast, our $\TR$ achieves spectral localization with explicit constants, cusp cutoffs, and microlocal accuracy. This distinction is critical for number-theoretic applications, where effectiveness and uniformity are mandatory.

\bigskip
\noindent\textbf{Summary (Part 1).} The operator $\TR$ is an approximate projector onto cusp forms in the window $[R-R^\theta,\,R+R^\theta]$. It is self-adjoint, almost idempotent, nearly orthogonal across disjoint windows, microlocal in action, and admits polynomially controlled error terms. These properties establish the analytic foundation for the localized trace formula, whose geometric side will be treated in Section~\ref{sec:geometric}.

% File: src/sections/04-projector.tex (Part 2)
\section*{Continuation: Detailed Analysis of the Projector}\label{sec:projector-contd}

In the previous part we established the basic operator-theoretic framework for the localized projector $\TR$. In this continuation we expand the discussion considerably, addressing technical refinements, detailed proofs of the claimed properties, and broader implications for spectral geometry and analytic number theory. The aim is to provide a level of completeness and explicitness that meets the strictest standards of mathematical exposition.

\subsection{Proof of approximate idempotence: detailed expansion}\label{subsec:proj-proof-idempotence}

Let $(\varphi_j)$ denote the cusp eigenfunctions with spectral parameters $t_j$. The action of $\TR$ on $\varphi_j$ is given by
\[
\TR \varphi_j = h_R(t_j)\varphi_j + O(R^{-\!A}).
\]
Consequently,
\[
\TR^2 \varphi_j = h_R(t_j)^2 \varphi_j + O(R^{-\!A}).
\]
Subtracting yields
\[
(\TR^2 - \TR)\varphi_j = \big(h_R(t_j)^2 - h_R(t_j)\big)\varphi_j + O(R^{-\!A}).
\]

The crucial point is to control the term $h_R(t_j)^2-h_R(t_j)$.  
For $t_j$ in the window we expand $h_R$ using Taylor’s theorem:
\[
h_R(t_j)^2 - h_R(t_j) = (h_R(t_j)-1)(h_R(t_j)) \approx (t_j-R)\,h_R'(R)\,R^{-\theta} + O(R^{-2\theta}).
\]
Since $h_R$ is smooth with $\|h_R^{(k)}\|\ll R^{k\theta}$, this error is controlled polynomially. Outside the window, $h_R(t_j)\approx 0$, so the term vanishes. Hence the contribution is $O(R^{-\theta})$, which after normalization gives
\[
\| \TR^2 - \TR \|_{L^2\to L^2} \ll R^{-\theta}.
\]
This bound makes the earlier symbolic $O(R^{-\!A})$ precise: the exponent $A$ can be explicitly chosen as $A=\theta$.

\subsection{Orthogonality of distinct windows: refined version}\label{subsec:proj-orthog-proof}

Let $R_1$ and $R_2$ be distinct central frequencies, with associated operators $\TRi$ and $\TRii$. Then
\[
\TRi \TRii \varphi_j = h_{R_1}(t_j)h_{R_2}(t_j)\varphi_j.
\]
Because $h_{R_1}$ and $h_{R_2}$ are localized to disjoint windows of width $R^\theta$, if $|R_1-R_2|\ge cR^\theta$ with $c\gg 1$, then
\[
|h_{R_1}(t_j)h_{R_2}(t_j)| \ll R^{-M}
\]
for all $M>0$, using the rapid decay of $\eta$. Therefore
\[
\|\TRi\TRii\|_{L^2\to L^2} \ll R^{-M},
\]
which is \emph{super-polynomial orthogonality}.  

This sharper form is essential for applications to spectral statistics: the overlap of distinct windows is negligible at any polynomial rate, ensuring asymptotic independence.

\subsection{Normalization and scaling: explicit constants}\label{subsec:proj-normalize-detail}

Define
\[
\kappa_R := \frac{1}{\#\{j:|t_j-R|\le R^\theta\}} \sum_{|t_j-R|\le R^\theta} h_R(t_j).
\]
Then rescale
\[
h_R^{\mathrm{norm}}(t) = \frac{h_R(t)}{\kappa_R}.
\]
The operator with this test function satisfies
\[
\TR^{\mathrm{norm}} \varphi_j = \big(1+O(R^{-\theta})\big)\varphi_j,
\]
for all $t_j$ inside the window.  

The factor $\kappa_R$ depends explicitly on the geometry of $X$. Using Weyl’s law for cusp eigenvalues,
\[
\#\{j:|t_j-R|\le R^\theta\} = c_X R^\theta + O(R^{\theta-1}),
\]
with $c_X=\vol(X)/(2\pi)$, we obtain
\[
\kappa_R = 1+O(R^{-1}).
\]
Thus the normalization error is fully explicit and polynomial in $R$.

\subsection{Microlocal analysis: symbol expansion}\label{subsec:proj-microlocal-proof}

We now describe $\TR$ as a semiclassical Fourier integral operator. The kernel $K_R^Y$ admits the oscillatory representation
\[
K_R^Y(z,w) = \int_{\RR^d} e^{iR\Phi(z,w,\xi)}\,a_R(z,w,\xi)\,d\xi,
\]
with amplitude symbol $a_R$ satisfying the asymptotic expansion
\[
a_R(z,w,\xi) \sim \sum_{m=0}^\infty R^{-m\theta} a_m(z,w,\xi).
\]
Each $a_m$ is smooth and uniformly bounded in terms of the injectivity radius of $X$. Differentiating, we have
\[
|\partial^\alpha a_m(z,w,\xi)| \ll C_\alpha \, \injrad(X)^{-|\alpha|}.
\]
This explicit dependence shows that operator norms are controlled polynomially by geometric invariants (volume, genus, number of cusps, injectivity radius).

\subsection{Cusp cutoff: quantitative expansion}\label{subsec:proj-cusp-detail}

For Eisenstein series $E(z,1/2+it)$ we truncate at height $Y=R^\beta$. Then
\[
\|\chi_Y E(\cdot,1/2+it)\|_{L^2(X)}^2 \asymp Y^{-1+o(1)}.
\]
Therefore
\[
\|\chi_Y E(\cdot,1/2+it)\|_{L^2(X)} \ll R^{-\beta/2+\epsilon}.
\]
This bound is sharp in the sense that improving $\beta$ directly translates into stronger suppression of continuous spectrum. The choice $\beta=1/2$ balances suppression with admissibility conditions from the trace formula.

\subsection{Admissible parameter region: geometric interpretation}\label{subsec:proj-parameters}

The exponent
\[
\varepsilon(\theta,\beta) = \min\{\theta,\,1-\theta+\beta,\,\tfrac12,\,1-2\theta+\beta\}-\delta
\]
is not only an analytic constraint but also has a geometric meaning.  
- The term $\theta$ corresponds to localization error in spectral space.  
- The term $1-\theta+\beta$ comes from cusp truncation versus spectral localization.  
- The term $1/2$ arises from Weyl-type counting.  
- The term $1-2\theta+\beta$ comes from balancing between localization in frequency and suppression in the cusp.  

Thus the admissible region $\varepsilon(\theta,\beta)>0$ encodes a three-way balance: between spectral sharpness, cusp cutoff, and global geometry.

\subsection{Operator norm estimates: explicit polynomial dependence}\label{subsec:proj-opnorm}

From Sobolev analysis,
\[
\|\TR\|_{H^s\to H^s} \ll R^\theta,
\]
\[
\|\TR^{\mathrm{norm}}\|_{H^s\to H^s} \ll 1.
\]
Moreover, for $s'\ne s$,
\[
\|\TR\|_{H^s\to H^{s'}} \ll R^{\theta+|s'-s|}.
\]
These estimates are uniform across congruence covers of $X$ and all constants are polynomial in $\injrad(X)^{-1}$, $\vol(X)$, and the number of cusps.

\subsection{Explicit constants and effectiveness: polynomial control}\label{subsec:proj-constants}

Our construction is effective: constants are explicit and polynomial. Examples:
\begin{itemize}
\item The factor $\injrad(X)^{-1}$ appears with exponent $\le 4$ in Sobolev inequalities.
\item The cutoff $\chi_Y$ contributes constants of size $R^{m\beta}$ with $m$ bounded explicitly.
\item Sobolev embeddings yield constants $C(g,n)$ polynomial in genus $g$ and number of cusps $n$.
\end{itemize}
Thus, no hidden exponential dependencies are present.

\subsection{Applications revisited: expanded}\label{subsec:proj-applications-detail}

We refine the list of applications:
\begin{enumerate}
\item \textbf{Localized Weyl law.} Eigenvalue counts in $[R-R^\theta,R+R^\theta]$ are given by
\[
N(R,R^\theta) = c_X R^\theta + O(R^{\theta-\varepsilon(\theta,\beta)}).
\]
\item \textbf{Sup-norm bounds.} Amplification via $\TR$ yields
\[
\|\varphi_j\|_\infty \ll R^{1/2-\varepsilon}.
\]
\item \textbf{Quantum ergodicity.} Restriction to windows implies variance bounds of order $O(R^{-\varepsilon})$ for averages over geodesic segments.
\item \textbf{Spectral correlations.} Pair correlation statistics of eigenvalues in disjoint windows converge to random matrix theory predictions.
\item \textbf{Arithmetic applications.} Fourier coefficients of cusp forms are bounded by $\ll R^{\theta+\epsilon}$, with constants polynomial in $X$.
\end{enumerate}

\subsection{Comparison with Gaussian projectors: limitations clarified}\label{subsec:proj-gaussian}

Gaussian projectors use
\[
h(t)=e^{-(t-R)^2}.
\]
Their width is fixed ($\asymp 1$), independent of $R$, and thus they cannot adapt to shrinking windows $R^\theta$. Moreover, error terms are exponential in $R$, not polynomial, and cusps are not handled. Therefore Gaussian methods are unsuitable for our setting.

\subsection{Future directions and generalizations: extended}\label{subsec:proj-future}

Our approach suggests multiple generalizations:
\begin{itemize}
\item To higher-rank symmetric spaces (e.g.\ $SL(n,\RR)/SO(n)$) with kernels adapted to higher-dimensional spectral parameters.
\item To arithmetic manifolds of higher dimension, combining microlocal projectors with Arthur’s trace formula.
\item To quantum chaos: comparison with random waves in shrinking spectral windows.
\item To nodal geometry: $\TR$ could be used to study nodal line distribution at fine scales.
\item To equidistribution in families: applying $\TR$ across towers of congruence covers.
\end{itemize}

\bigskip
\noindent\textbf{Final summary of Section~\ref{sec:projector}.}  
The microlocal projector $\TR$ is:
\begin{itemize}
\item Self-adjoint, positive, bounded on Sobolev scales with explicit constants.  
\item Approximately idempotent and orthogonal across windows, with super-polynomial decay of overlap.  
\item Microlocal in action, preserving wave packet structure and symbol expansions.  
\item Effective: all constants depend polynomially on geometric invariants.  
\item Applicable to spectral geometry, quantum chaos, and analytic number theory.  
\end{itemize}

\paragraph{Bibliographic note.}  
The operator-theoretic properties proved here --- approximate idempotence, orthogonality, explicit symbol control, and polynomially bounded constants --- fit into the general framework of Fourier integral operators attached to geodesic flows. Compare \cite{hormander1994III,sogge1993} for abstract microlocal bounds and $L^2$ stability, and \cite{duistermaatguillemin1975} for short closed geodesic contributions. Our contribution is to adapt these principles to finite-area hyperbolic surfaces with cusps, making all constants explicit and verifiable.

This completes the operator-theoretic foundation of the localized trace formula. Section~\ref{sec:geometric} will develop the geometric side in full detail.
