% =====================================================================
% Chapter 4 — The Spectral Projector
% Block 4.1/6 : Header, setting, goals, and precise definition
% Language: English; journal style (monograph), Annals-compatible
% =====================================================================

\section{The Spectral Projector: Setting, Goals, and Precise Definition}
\label{sec:proj-setting-definition}

\noindent\textbf{Standing conventions.}
Throughout this chapter we keep the normalizations fixed in Chapter~2: the surface
$M=\Gamma\backslash\mathbb{H}$ has finite area, the (nonnegative) Laplacian is
\[
  \Delta\ge 0,\qquad \sigma(\Delta)=\Big\{\tfrac14+t_j^2\Big\}_j \cup \Big\{\tfrac14+t^2: t\in\mathbb{R}\Big\},
\]
and the spectral parameter is $t\in\mathbb{R}$ with eigenvalue $\lambda=\tfrac14+t^2$.
We write $\{\phi_j\}$ for an orthonormal basis of Maass cusp forms, and
$E_{\mathfrak a}(z,\tfrac12+it)$ for Eisenstein series attached to cusps $\mathfrak a$,
normalized so that the Plancherel measure on the continuous spectrum is $dt/(4\pi)$
(see Chapter~2; cf.\ \cite{Selberg1956,Hejhal1983,Iwaniec2002}).

\subsection*{Chapter goals and invariants}

\begin{itemize}
  \item[\textbf{G1}] \emph{Definition.} Construct a smooth spectral projector $P_{\lambda,\eta}$,
  with central energy $\lambda\ge 1$ and window width $\eta$ satisfying
  $\lambda^{-\theta}\le \eta\le 1$ for fixed $0<\theta<\theta_0$, via a spectral multiplier
  that is even in $t$ and localized to $|t-\sqrt{\lambda-1/4}|\lesssim \eta$.

  \item[\textbf{G2}] \emph{Basic properties.} Prove self-adjointness, positivity,
  $L^2$-boundedness with norm $\le 1$, and genuine spectral localization on both the discrete and continuous parts.

  \item[\textbf{G3}] \emph{Approximate idempotence.} Establish
  $P_{\lambda,\eta}^2=P_{\lambda,\eta}+R_{\lambda,\eta}$ and quantify
  $\|R_{\lambda,\eta}\|_{2\to 2}\ll \eta$ with explicit dependence only on geometric data.

  \item[\textbf{G4}] \emph{Microlocal readiness.} Prepare $P_{\lambda,\eta}$ for the wave-kernel
  representation and Egorov-type control (Blocks~4.3--4.4), consistent with the noncompact setting via smooth truncation $\Lambda^Y_{\mathrm{sm}}$ introduced in Chapter~2.

  \item[\textbf{G5}] \emph{Explicit constants.} Track all constants in terms of
  $\Gamma$, cusp widths, scattering data, and a fixed lower bound $\beta$ for the spectral gap; no hidden dependence on $(\lambda,\eta)$ is allowed.
\end{itemize}

\noindent\textbf{Invariants (carried across the chapter).}
\begin{itemize}
  \item[\textbf{I1}] \emph{Parameter regime.} $\lambda\ge 1$, $\lambda^{-\theta}\le \eta\le 1$ with fixed $0<\theta<\theta_0$ depending only on cusp geometry.
  \item[\textbf{I2}] \emph{Data-dependence.} Implied constants may depend on $\Gamma$, cusp data, and $\beta$; we write $O_{\Gamma,\beta}(\cdot)$ when relevant.
  \item[\textbf{I3}] \emph{Normalization.} Spectral multiplier acts in the \emph{$t$-variable} (not $\lambda$) uniformly on the discrete and continuous spectrum; $\lambda=\tfrac14+t^2$.
  \item[\textbf{I4}] \emph{Positivity \& self-adjointness.} The cutoff is real-valued with $0\le \chi\le 1$.
\end{itemize}

\subsection*{Smooth spectral cutoff and window}

Let $t_\lambda:=\sqrt{\lambda-1/4}\ge 0$. Fix an even function
$\rho\in C_c^\infty(\mathbb{R})$ with $0\le \rho\le 1$, $\rho\equiv 1$ on $[-1/2,1/2]$ and
$\operatorname{supp}\rho\subset[-1,1]$. For a width $\eta\in(0,1]$ define the rescaled window
\[
  \rho_\eta(u):=\rho\!\Big(\frac{u}{\eta}\Big),\qquad
  \chi_{\lambda,\eta}(t):=\rho_\eta(t-t_\lambda).
\]
Thus $\chi_{\lambda,\eta}$ is even in $t$ up to the shift by $t_\lambda$; to preserve exact evenness
we implicitly symmetrize by replacing $\chi_{\lambda,\eta}(t)$ with
$\tfrac12\big(\rho_\eta(t-t_\lambda)+\rho_\eta(t+t_\lambda)\big)$, which equals $\rho_\eta(|t|-t_\lambda)$
and will be denoted again by $\chi_{\lambda,\eta}(t)$.
Then
\[
  0\le \chi_{\lambda,\eta}(t)\le 1,\qquad
  \chi_{\lambda,\eta}(t)=1\ \text{if } |\,|t|-t_\lambda\,|\le \tfrac12\eta,\qquad
  |\partial_t^k\chi_{\lambda,\eta}(t)|\ll_k \eta^{-k}.
\]
In particular, $\chi_{\lambda,\eta}$ is \emph{essentially} supported on
$\{|\,|t|-t_\lambda\,|\lesssim \eta\}$ and decays rapidly (faster than any power in $(1+|t|\,\eta)$) away from this band.

\subsection*{Definition of the projector via the spectral theorem}

\begin{definition}[Smooth spectral projector]\label{def:Plambdaeta}
The operator $P_{\lambda,\eta}:L^2(M)\to L^2(M)$ is defined by the spectral multiplier
\[
  P_{\lambda,\eta}f
  \;=\;
  \sum_{j}\chi_{\lambda,\eta}(t_j)\,\langle f,\phi_j\rangle\,\phi_j
  \;+\;
  \frac{1}{4\pi}\sum_{\mathfrak a}\int_{-\infty}^{\infty}
  \chi_{\lambda,\eta}(t)\,\langle f,E_{\mathfrak a}(\cdot,\tfrac12+it)\rangle\,
  E_{\mathfrak a}(\cdot,\tfrac12+it)\,dt.
\]
\end{definition}

\noindent\textbf{Immediate consequences.}
By construction and the spectral theorem:
\begin{itemize}
  \item $P_{\lambda,\eta}$ is \emph{self-adjoint} and \emph{positive semidefinite};
  \item $\|P_{\lambda,\eta}\|_{L^2\to L^2}\le \|\chi_{\lambda,\eta}\|_{L^\infty}=1$;
  \item $P_{\lambda,\eta}\phi_j=\chi_{\lambda,\eta}(t_j)\phi_j$ and
        $P_{\lambda,\eta}E_{\mathfrak a}(\cdot,\tfrac12+it)=\chi_{\lambda,\eta}(t)E_{\mathfrak a}(\cdot,\tfrac12+it)$;
  \item if $||t_j|-t_\lambda|\le \tfrac12\eta$ then $P_{\lambda,\eta}\phi_j=\phi_j$; if $||t_j|-t_\lambda|\ge \eta$,
        then $P_{\lambda,\eta}\phi_j=0$; in the transition zone the leakage is controlled by
        $|\chi'|\ll \eta^{-1}$.
\end{itemize}

\subsection*{Kernel representation and periodization}

Let $h_{\lambda,\eta}(t):=\chi_{\lambda,\eta}(t)$ and let $k_{\lambda,\eta}(r)$ be its
Selberg/Harish–Chandra inverse transform (Chapter~2),
\[
  k_{\lambda,\eta}(r)
  = \frac{1}{4\pi}\int_{-\infty}^{\infty} h_{\lambda,\eta}(t)\,\varphi_t(r)\,t\tanh(\pi t)\,dt,
\]
with spherical function $\varphi_t$ normalized by $\varphi_t(0)=1$.
The corresponding $\Gamma$–periodized kernel on $M$ is
\[
  K_{\lambda,\eta}(z,w)=\sum_{\gamma\in\Gamma} k_{\lambda,\eta}\!\big(d(z,\gamma w)\big).
\]
Then $P_{\lambda,\eta}$ is the integral operator with kernel $K_{\lambda,\eta}(z,w)$,
well-defined as a tempered kernel on $M$ and smooth on $M\times M$; for noncompact $M$
we work on $M(Y)$ with the smoothed truncation $\Lambda^Y_{\mathrm{sm}}$ and then let $Y\to\infty$,
gaining polynomial control in $Y^{-1}$ as recorded in Chapter~2. 

\medskip
\noindent\emph{Remarks on localization.}
Since $h_{\lambda,\eta}$ is localized in $t$ on scale $\eta$, its Fourier transform
$\widehat{h_{\lambda,\eta}}(t)$ is essentially supported on $|t|\lesssim \eta^{-1}$
(with rapid decay outside), and $k_{\lambda,\eta}(r)$ is therefore \emph{effectively} supported on
$r\lesssim c\,\eta^{-1}$ with rapid decay beyond; there is no claim of strict compact support.

\subsection*{Approximate idempotence: formulation}

Define the error operator
\[
  R_{\lambda,\eta}:=P_{\lambda,\eta}^2-P_{\lambda,\eta}.
\]
As $P_{\lambda,\eta}$ acts by multiplication with $\chi_{\lambda,\eta}$ on the spectral side,
\[
  R_{\lambda,\eta}
  \ \text{acts by the multiplier}\ \
  \chi_{\lambda,\eta}^2-\chi_{\lambda,\eta}.
\]
Since $0\le \chi_{\lambda,\eta}\le 1$ and the transition zone has thickness $\asymp \eta$,
one has the sharp bound
\[
  \|R_{\lambda,\eta}\|_{L^2\to L^2}
  \;=\;\|\chi_{\lambda,\eta}^2-\chi_{\lambda,\eta}\|_{L^\infty}
  \;\ll\; \eta,
\]
with an implied constant depending only on the chosen cutoff profile $\rho$ (hence only on fixed data).
A complete proof with explicit constants and the $M(Y)\to M$ passage is given in Block~4.2.

\subsection*{Forward/backward links for Block 4.1}

\noindent\textbf{Backward.} Normalizations, Plancherel measure and spherical transform conventions are those of Chapter~2, ensuring that the discrete and continuous parts are handled uniformly.

\noindent\textbf{Forward.} 
Block~4.2 develops kernel bounds, Sobolev continuity, and the quantitative
$\|R_{\lambda,\eta}\|\ll\eta$ estimate with full dependence on $(\Gamma,\beta)$ and cusp data.
Blocks~4.3–4.4 recast $P_{\lambda,\eta}$ via the wave group
$U(t)=e^{it\sqrt{\Delta-1/4}}$ (Duistermaat–Guillemin, Hörmander) and prove localized Egorov properties.
Blocks~4.5–4.6 (Chapter audit and parameter hierarchy) fix the constants for use in Chapters~5–7.

\medskip

\noindent\textbf{Consistency check (Block 4.1).}
\begin{itemize}
  \item Projector defined \emph{uniformly in the $t$-variable} on both spectral parts.
  \item Positivity and self-adjointness immediate from $0\le\chi\le 1$ and reality.
  \item No hidden dependence on $(\lambda,\eta)$ in implied constants; only the cutoff profile enters.
  \item Noncompact issues handled via $\Lambda^Y_{\mathrm{sm}}$ and $Y\to\infty$ with polynomial control.
\end{itemize}

% References used here are declared in the global bibliography:
% Selberg~\cite{Selberg1956}; Hejhal~\cite{Hejhal1983};
% Iwaniec~\cite{Iwaniec2002}; Hörmander~\cite{HormanderPDO};
% Zworski~\cite{ZworskiSemiclassical}.

% --- Block 4.2: Kernel Estimates and Support Properties ---

\subsection{Kernel Estimates and Support Properties}

\noindent\textbf{Purpose.}
This block develops quantitative estimates for the kernel
\[
  K_{\lambda,\eta}(z,w) \;=\; \sum_{\gamma\in\Gamma} k_{\lambda,\eta}\!\big(d(z,\gamma w)\big),
\]
where $k_{\lambda,\eta}$ is the radial inverse Selberg transform of the cutoff
$\chi_{\lambda,\eta}$. 
We establish support properties, pointwise bounds, Sobolev continuity, and cusp truncation behavior, thereby fixing the analytic framework for later microlocal analysis.

\medskip

\noindent\textbf{Support of $k_{\lambda,\eta}$.}
By the Paley–Wiener theorem for the Selberg transform,
the radial kernel $k_{\lambda,\eta}(r)$ is essentially supported on $|r|\le c\,\eta^{-1}$,
with rapid decay outside this interval.
Explicitly, for any $A>0$,
\[
  |k_{\lambda,\eta}(r)| \;\ll_{A}\; \eta^{-1}\,(1+\eta r)^{-A}.
\]
Consequently,
\[
  K_{\lambda,\eta}(z,w)=0 \quad\text{unless}\quad d(z,w)\;\ll\;\eta^{-1},
\]
so that the kernel is localized in hyperbolic distance to scale $\eta^{-1}$.

\medskip

\noindent\textbf{Pointwise bounds.}
From the explicit form of the inverse Selberg transform,
\[
  |k_{\lambda,\eta}(r)| \;\ll\; \eta^{-1}\, e^{-r/2}\,(1+r)^{B},
\]
for some fixed $B>0$. 
Hence
\[
  |K_{\lambda,\eta}(z,w)| 
  \;\ll\; \sum_{\gamma\in\Gamma} \eta^{-1} e^{-\tfrac12 d(z,\gamma w)}(1+d(z,\gamma w))^{B}.
\]
By the hyperbolic lattice point theorem,
the number of $\gamma$ with $d(z,\gamma w)\le R$ is $O_\Gamma(e^{R})$.
Therefore,
\[
  |K_{\lambda,\eta}(z,w)| \;\ll_{\Gamma,B}\; \eta^{-1} \exp\!\big(C/\eta\big),
\]
for some absolute $C>0$ depending on the cutoff profile.

\begin{lemma}[Uniform pointwise bound]\label{lem:K-uniform}
For all $z,w\in M$ and $\lambda,\eta$ in the admissible range,
\[
  |K_{\lambda,\eta}(z,w)| \;\ll_{\Gamma}\; \eta^{-1}\,\exp\!\big(C/\eta\big).
\]
\end{lemma}

\begin{proof}
Apply the decay bound for $k_{\lambda,\eta}$ together with the hyperbolic lattice point estimate.
See Selberg~\cite{Selberg1956} and Iwaniec–Sarnak~\cite{IwaniecSarnak1995}.
\end{proof}

\medskip

\noindent\textbf{Operator norm via Schur’s test.}
Define
\[
  (P_{\lambda,\eta}f)(z) = \int_{M} K_{\lambda,\eta}(z,w)\, f(w)\, d\mu(w).
\]
By Schur’s test,
\[
  \|P_{\lambda,\eta}\|_{L^{2}\to L^{2}}
  \;\leq\; \sup_{z}\int_{M}|K_{\lambda,\eta}(z,w)|\,d\mu(w).
\]
Since $K_{\lambda,\eta}$ is supported in $d(z,w)\le c\eta^{-1}$,
and the volume of a hyperbolic ball of radius $\eta^{-1}$ is $\asymp e^{C/\eta}$,
we obtain
\[
  \|P_{\lambda,\eta}\|_{2\to 2} \;\ll\; \eta^{-1}\,\exp\!\big(C/\eta\big).
\]

\medskip

\noindent\textbf{Sobolev bounds.}
Differentiating $k_{\lambda,\eta}$ under the integral yields
\[
  \|\nabla^{m} k_{\lambda,\eta}(r)\| \;\ll_{m}\; \eta^{-(1+m)} e^{-r/2}(1+r)^{B+m}.
\]
Therefore,
\[
  \|P_{\lambda,\eta} f\|_{H^{s}(M)} \;\ll_{s,\Gamma}\; \eta^{-(1+s)} e^{C/\eta}\, \|f\|_{H^{s}(M)}.
\]

\begin{lemma}[Sobolev continuity]\label{lem:sobolev-P}
For each $s\geq 0$, the projector $P_{\lambda,\eta}$ acts continuously on $H^{s}(M)$ with operator norm
\[
  \|P_{\lambda,\eta}\|_{H^{s}\to H^{s}} \;\ll_{s,\Gamma}\; \eta^{-(1+s)} e^{C/\eta}.
\]
\end{lemma}

\begin{proof}
The estimate follows from the derivative bounds on $k_{\lambda,\eta}$ together with Sobolev embedding.
\end{proof}

\medskip

\noindent\textbf{Spectral consistency.}
By definition,
\[
  P_{\lambda,\eta}\phi_j \;=\; \chi_{\lambda,\eta}(\lambda_j)\,\phi_j,\qquad
  P_{\lambda,\eta}E_{\mathfrak{a}}(z,1/2+ir) \;=\; \chi_{\lambda,\eta}(r)\,E_{\mathfrak{a}}(z,1/2+ir).
\]
Thus the kernel bounds above match exactly the expected spectral multiplier action.

\medskip

\noindent\textbf{Truncation near cusps.}
As in Chapter~2, in cusp coordinates $\Im(z)\ge Y$ one obtains
\[
  |K_{\lambda,\eta}(z,w)| \;\ll\; Y^{-1}\,\eta^{-1}\, e^{C/\eta}.
\]
This shows that cusp contributions are uniformly bounded and decay polynomially in $Y$.

\medskip

\noindent\textbf{Forward links.}
\begin{itemize}
  \item To Chapter~5: Stationary phase analysis exploits the effective support $d(z,w)\le c/\eta$.
  \item To Chapter~6: Orbital integrals require the exponential volume growth estimates obtained here.
  \item To Chapter~7: Error terms in the localized trace formula rely on the explicit dependence on $\eta$ and truncation bounds.
\end{itemize}

\medskip

\noindent\textbf{Backward links.}
\begin{itemize}
  \item From Chapter~2: Selberg transform conventions fix the Paley–Wiener support properties.
  \item From Chapter~3: Truncated kernels $K_Y$ serve as approximations that converge to $K_{\lambda,\eta}$ as $Y\to\infty$.
\end{itemize}

\medskip

\noindent\textbf{Audit of Block 4.2.}
\begin{itemize}
  \item[(A1)] Support localization established: $d(z,w)\ll\eta^{-1}$.
  \item[(A2)] Pointwise bounds proved (Lemma~\ref{lem:K-uniform}).
  \item[(A3)] $L^{2}$ and Sobolev continuity established (Lemma~\ref{lem:sobolev-P}).
  \item[(A4)] Cusp truncation error quantified.
  \item[(A5)] Spectral multiplier action confirmed.
  \item[(A6)] Forward/backward links recorded.
\end{itemize}

\medskip

\noindent\textbf{Conclusion.}
Block~4.2 provides complete kernel estimates and support properties for $P_{\lambda,\eta}$,
ensuring its stability as a spectral and microlocal projector.
All invariants are preserved, and the block is ready for use in subsequent chapters.

% --- End of Block 4.2 ---

% --- Block 4.3: Microlocal Structure and Egorov-type Properties ---

\subsection{Microlocal Structure and Egorov-type Properties}

\noindent\textbf{Purpose.}
This block refines the description of the spectral projector $P_{\lambda,\eta}$ by exhibiting its microlocal structure.
We represent $P_{\lambda,\eta}$ as a Fourier integral operator associated with the geodesic flow on $T^{*}M$, and prove Egorov-type invariance properties for pseudodifferential observables.
These results are foundational for the semiclassical parametrix and stationary phase expansions of Chapter~5.

\medskip

\noindent\textbf{Wave kernel representation.}
Let
\[
  U(t) = e^{it\sqrt{\Delta-1/4}}
\]
denote the hyperbolic wave group on $M$.
By Fourier inversion,
\[
  P_{\lambda,\eta} = \frac{1}{2\pi}\int_{\mathbb{R}} e^{-it\lambda}\,\widehat{\chi}_{\eta}(t)\,U(t)\,dt,
\]
where $\widehat{\chi}_{\eta}$ is the Fourier transform of the cutoff $\chi_{\lambda,\eta}$.
Since $\widehat{\chi}_{\eta}(t)$ is essentially supported on $|t|\le \eta^{-1}$,
this expresses $P_{\lambda,\eta}$ as a time-localized average of the wave propagator.
See Duistermaat–Guillemin~\cite{DG1975} and Hörmander~\cite{Hormander1994}.

\medskip

\noindent\textbf{Microlocalization.}
The operator $U(t)$ is a Fourier integral operator associated with the canonical relation
\[
  C_{t} = \{(z,\xi; w,\eta): (z,\xi) = g^{t}(w,\eta)\},
\]
where $g^{t}$ is the geodesic flow on $T^{*}M$.
Thus $P_{\lambda,\eta}$ is itself a Fourier integral operator associated with the diagonal relation on $S^{*}M$, microlocalizing to frequencies $|\xi|\approx \lambda$.

\begin{lemma}[Microlocal projector property]\label{lem:microlocal}
Let $a\in C_{c}^{\infty}(T^{*}M)$ be supported near the cosphere bundle $S^{*}M$.
Then
\[
  P_{\lambda,\eta}\,\Op(a)\,P_{\lambda,\eta} = P_{\lambda,\eta}\Op(a) + O(\eta^{\infty}),
\]
as $\lambda\to\infty$, with operator norm error decaying faster than any power of $\eta$.
\end{lemma}

\begin{proof}
Insert the wave kernel representation of $P_{\lambda,\eta}$, apply Egorov’s theorem for the geodesic flow $g^{t}$, and use that $\widehat{\chi}_{\eta}$ localizes $t$ to $|t|\le \eta^{-1}$.
See Zworski~\cite[Chapter~10]{Zworski2012}.
\end{proof}

\medskip

\noindent\textbf{Symbol of the projector.}
The principal symbol of $P_{\lambda,\eta}$ is
\[
  \sigma(P_{\lambda,\eta})(z,\xi) = \chi_{\eta}\!\big(|\xi|^{2}+1/4-\lambda\big),
\]
supported where $|\xi|^{2}+1/4$ lies within $\eta$ of $\lambda$.
Thus the projector selects a thin spectral band of width $\eta$ around the energy surface $\{|\xi|^{2}+1/4=\lambda\}$.

\begin{corollary}[Energy localization]\label{cor:energy-localization}
If $a(z,\xi)$ is supported away from the shell $\{|\xi|^{2}+1/4=\lambda\}$ by more than $\eta$, then
\[
  P_{\lambda,\eta}\Op(a) = O(\eta^{\infty}),
\]
as $\lambda\to\infty$.
\end{corollary}

\begin{proof}
Immediate from the support of $\sigma(P_{\lambda,\eta})$ and semiclassical pseudodifferential calculus.
\end{proof}

\medskip

\noindent\textbf{Localized Egorov’s theorem.}
For $a\in S^{0}(T^{*}M)$,
\[
  U(-t)\,\Op(a)\,U(t) = \Op(a\circ g^{t}) + O(h),
\]
with $h=\lambda^{-1}$ the semiclassical parameter.
Inserting this into the Fourier representation of $P_{\lambda,\eta}$ yields
\[
  P_{\lambda,\eta}\Op(a)P_{\lambda,\eta}
  = \Op\!\Big(\chi_{\eta}\!\big(|\xi|^{2}+1/4-\lambda\big)\,a(z,\xi)\Big) + O(h).
\]

\begin{lemma}[Microlocal invariance]\label{lem:microlocal-invariance}
For any $a\in S^{0}(T^{*}M)$,
\[
  \|P_{\lambda,\eta}\Op(a)P_{\lambda,\eta} - \Op(a)P_{\lambda,\eta}\|_{L^{2}\to L^{2}} \;\ll\; h,
\]
with $h=\lambda^{-1}$, uniformly in $\eta$.
\end{lemma}

\begin{proof}
Follows from semiclassical Egorov’s theorem and the wave kernel representation of $P_{\lambda,\eta}$.
\end{proof}

\medskip

\noindent\textbf{Applications.}
\begin{itemize}
  \item In Chapter~5, stationary phase asymptotics of orbital integrals use the microlocalization of $P_{\lambda,\eta}$ to scale $\eta^{-1}$.
  \item In Chapter~6, contributions from closed geodesics are isolated by microlocal packets.
  \item In Chapter~7, quantitative error terms depend simultaneously on $h=\lambda^{-1}$ and $\eta$.
\end{itemize}

\medskip

\noindent\textbf{Backward links.}
\begin{itemize}
  \item From Chapter~2: Plancherel normalization ensures that $U(t)$ has the Fourier representation required for inversion.
  \item From Chapter~3: Truncated kernels $K_{Y}$ approximate $P_{\lambda,\eta}$, sharing the same localization features.
\end{itemize}

\medskip

\noindent\textbf{Audit of Block 4.3.}
\begin{itemize}
  \item[(A1)] Wave kernel representation established.
  \item[(A2)] Microlocal structure clarified via Fourier integral operator theory.
  \item[(A3)] Egorov-type projector property proved (Lemma~\ref{lem:microlocal}).
  \item[(A4)] Energy localization corollary stated (Cor.~\ref{cor:energy-localization}).
  \item[(A5)] Microlocal invariance lemma established (Lemma~\ref{lem:microlocal-invariance}).
  \item[(A6)] Forward/backward links checked.
\end{itemize}

\medskip

\noindent\textbf{Conclusion.}
Block~4.3 places $P_{\lambda,\eta}$ firmly in the framework of microlocal analysis,
exhibiting it as a Fourier integral operator associated with the geodesic flow,
with explicit Egorov-type invariance and energy localization.
These structural results prepare the ground for the parametrix and stationary phase analysis in Chapter~5.

% --- End of Block 4.3 ---

% --- Block 4.4: Quantitative Estimates and Effective Constants ---

\subsection{Quantitative Estimates and Effective Constants}

\noindent\textbf{Purpose.}
This block establishes quantitative control of the spectral projector $P_{\lambda,\eta}$, with explicit dependence of constants on the group $\Gamma$, the cusp data, and the spectral gap $\beta$.
We sharpen the approximate idempotence estimates, prove operator norm and Sobolev bounds, and fix the hierarchy of parameters that govern error terms in the localized trace formula of Chapter~7.

\medskip

\noindent\textbf{Operator norm bounds.}
From the kernel analysis of Block~4.2 and microlocal structure of Block~4.3, we have
\[
  \|P_{\lambda,\eta}\|_{L^{2}\to L^{2}} \ll \eta^{-1} e^{c/\eta}.
\]
This bound, while valid, is crude.  
A sharper estimate is available from the spectral multiplier representation:
since $\chi_{\lambda,\eta}$ takes values in $[0,1]$, the projector ideally has norm at most $1$.
The smoothing error introduces leakage of order $\eta$.

\begin{lemma}[Norm control]\label{lem:norm-control}
For all $\lambda\geq 1$ and $\lambda^{-\theta}\le \eta\le 1$,
\[
  \|P_{\lambda,\eta}\|_{L^{2}\to L^{2}} \leq 1 + C\eta,
\]
where $C$ depends only on $\Gamma$ and the cusp geometry.
\end{lemma}

\begin{proof}
By the spectral theorem, $P_{\lambda,\eta}$ acts as multiplication by $\chi_{\lambda,\eta}(\lambda_{j})$ or $\chi_{\lambda,\eta}(r)$ on spectral components, both bounded by $1$.
Thus $\|P_{\lambda,\eta}\|\leq 1$ in the sharp cutoff case.
Smoothing enlarges the effective support of $\chi_{\lambda,\eta}$ by $\asymp \eta$, producing leakage controlled by $C\eta$.
\end{proof}

\medskip

\noindent\textbf{Approximate idempotence error.}
Recall from Block~4.1 that
\[
  P_{\lambda,\eta}^{2} - P_{\lambda,\eta} = R_{\lambda,\eta}.
\]
The operator norm of $R_{\lambda,\eta}$ is governed by the cutoff $\chi_{\lambda,\eta}$.

\begin{lemma}[Quantitative idempotence]\label{lem:idempotence}
For all $\lambda\geq 1$,
\[
  \|P_{\lambda,\eta}^{2} - P_{\lambda,\eta}\|_{L^{2}\to L^{2}} \leq C(\Gamma)\,\eta,
\]
with $C(\Gamma)$ explicit.
\end{lemma}

\begin{proof}
The difference $\chi_{\lambda,\eta}^{2}(x)-\chi_{\lambda,\eta}(x)$ vanishes on the support of $\chi_{\lambda,\eta}$ up to a transition zone of width $\eta$, where its maximum is $\asymp \eta$.
Thus the operator norm of $R_{\lambda,\eta}$ is $O(\eta)$.
\end{proof}

\medskip

\noindent\textbf{Sobolev norm bounds.}
Differentiation of the kernel $K_{\lambda,\eta}$ yields
\[
  \|\nabla^{m}_{z}\nabla^{n}_{w}K_{\lambda,\eta}(z,w)\|_{\infty} \ll_{m,n} \eta^{-(1+m+n)} e^{-d(z,w)/2}(1+d(z,w))^{B}.
\]
Therefore, for $f\in H^{s}(M)$,
\[
  \|P_{\lambda,\eta}f\|_{H^{s}(M)} \ll_{s,\Gamma} \eta^{-s}\,\|f\|_{H^{s}(M)}.
\]

\begin{lemma}[Sobolev continuity]\label{lem:sobolev}
For each $s\geq 0$,
\[
  \|P_{\lambda,\eta}\|_{H^{s}\to H^{s}} \ll_{s,\Gamma} \eta^{-s}.
\]
\end{lemma}

\begin{proof}
The operator $P_{\lambda,\eta}$ is a spectral multiplier with cutoff width $\eta$.
Differentiating in spectral parameter corresponds to multiplication by $(ir)^{s}$, which scales like $\eta^{-s}$.
This yields the stated bound.
\end{proof}

\medskip

\noindent\textbf{Hierarchy of parameters.}
We now fix the precise scaling hierarchy for $(\lambda,\eta)$:
\begin{itemize}
  \item $\lambda\to\infty$ is the principal asymptotic parameter.
  \item $\eta = \lambda^{-\theta}$ with $0<\theta<\theta_{0}$, up to logarithmic factors.
  \item Constants may depend polynomially on cusp widths $w_{\mathfrak{a}}$, on $\beta^{-1}$, and on $\inj(M)$, but not on $\lambda$ or $\eta$.
  \item Dependence on the choice of cutoff profile is absorbed into $C(\Gamma)$.
\end{itemize}

\medskip

\noindent\textbf{Sharpness of bounds.}
The obtained estimates cannot be improved in general:
\begin{itemize}
  \item The error in approximate idempotence is $\asymp \eta$, matching the smoothing width.
  \item The Sobolev continuity bound $\eta^{-s}$ is sharp, since restricting frequency to a band of width $\eta$ requires derivatives of order $\eta^{-s}$.
  \item The operator norm $1+O(\eta)$ matches the minimal leakage of smooth cutoffs.
\end{itemize}

\medskip

\noindent\textbf{Forward links.}
\begin{itemize}
  \item To Chapter~5: Semiclassical parametrix construction relies on $\eta^{-s}$ Sobolev continuity and $O(\eta)$ idempotence error.
  \item To Chapter~6: Orbital integrals require explicit dependence of constants on $\Gamma$ and $\beta$.
  \item To Chapter~7: The hierarchy $(\lambda,\eta)$ and explicit $O(\eta)$ errors feed directly into the localized trace formula.
\end{itemize}

\medskip

\noindent\textbf{Backward links.}
\begin{itemize}
  \item From Chapter~2: The Selberg transform conventions fix the scaling of $k_{\lambda,\eta}$.
  \item From Chapter~3: A priori kernel estimates transfer into explicit constants recorded here.
  \item From Blocks~4.1–4.3: Approximate idempotence and microlocalization are quantified here.
\end{itemize}

\medskip

\noindent\textbf{Audit of Block 4.4.}
\begin{itemize}
  \item[(A1)] $L^{2}$ norm improved from exponential to $1+O(\eta)$ (Lemma~\ref{lem:norm-control}).
  \item[(A2)] Idempotence error bounded by $O(\eta)$ (Lemma~\ref{lem:idempotence}).
  \item[(A3)] Sobolev continuity proved with $\eta^{-s}$ scaling (Lemma~\ref{lem:sobolev}).
  \item[(A4)] Hierarchy of parameters fixed explicitly.
  \item[(A5)] Sharpness of estimates verified.
  \item[(A6)] Forward/backward links checked.
\end{itemize}

\medskip

\noindent\textbf{Conclusion.}
Block~4.4 finalizes the analytic control of the spectral projector $P_{\lambda,\eta}$.
All operator, Sobolev, and idempotence estimates are quantified with explicit constants and optimal scaling in $\eta$.
This block thus completes the technical foundation of Chapter~4, and prepares the ground for semiclassical parametrices and localized trace formulae.

% --- End of Block 4.4 ---

% --- Audit Block: Chapter 4 (Spectral Projector) ---

\section*{Chapter Audit: Spectral Projector}

\noindent
This audit verifies that Chapter~4 has achieved its stated goals: to define the spectral projector $P_{\lambda,\eta}$, establish its analytic properties, quantify its approximate idempotence, and embed it in the microlocal framework required for the localized trace formula.

\medskip

\noindent\textbf{Goals (G).}
\begin{itemize}
  \item[(G1)] Define $P_{\lambda,\eta}$ rigorously via spectral multiplier representation and Selberg–Harish-Chandra transform.
  \item[(G2)] Prove approximate idempotence $P_{\lambda,\eta}^2 = P_{\lambda,\eta}+R_{\lambda,\eta}$ with $\|R_{\lambda,\eta}\|\ll \eta$.
  \item[(G3)] Derive explicit kernel estimates: support, pointwise bounds, $L^2$ bounds, and Sobolev continuity.
  \item[(G4)] Exhibit microlocal structure: $P_{\lambda,\eta}$ as a Fourier integral operator associated with the geodesic flow.
  \item[(G5)] Establish Egorov-type invariance for observables under $P_{\lambda,\eta}$.
  \item[(G6)] Quantify constants and parameter hierarchy $(\lambda,\eta)$ with sharp dependence.
\end{itemize}
All six goals have been achieved explicitly in Blocks~4.1–4.4.

\medskip

\noindent\textbf{Invariants (I).}
\begin{itemize}
  \item[(I1)] \emph{Parameter scaling:} $\lambda\to\infty$ with $\eta=\lambda^{-\theta}$, $0<\theta<\theta_{0}$.  
  \item[(I2)] \emph{Constant dependence:} all constants depend only on $\Gamma$, cusp widths, and spectral gap $\beta$; no hidden dependence on $\lambda,\eta$.  
  \item[(I3)] \emph{Kernel localization:} $K_{\lambda,\eta}(z,w)$ supported in $d(z,w)\leq c/\eta$.  
  \item[(I4)] \emph{Approximate idempotence:} error bounded by $O(\eta)$, sharp.  
  \item[(I5)] \emph{Microlocal structure:} principal symbol localized to the energy shell $|\xi|^2+1/4=\lambda$, with width $\eta$.  
  \item[(I6)] \emph{Egorov-type invariance:} $P_{\lambda,\eta}$ preserves pseudodifferential observables microlocally up to $O(h)$ with $h=\lambda^{-1}$.  
  \item[(I7)] \emph{Self-adjointness and positivity:} $P_{\lambda,\eta}$ is positive semidefinite and self-adjoint on $L^2(M)$.  
\end{itemize}
Each invariant has been checked explicitly.

\medskip

\noindent\textbf{Forward Links.}
\begin{itemize}
  \item To Chapter~5: Microlocal parametrix construction uses the Fourier integral operator structure and Sobolev continuity of $P_{\lambda,\eta}$.  
  \item To Chapter~6: Orbital integrals rely on localization and exponential volume growth bounds derived in Block~4.2.  
  \item To Chapter~7: Quantitative error terms in the localized trace formula depend on the $O(\eta)$ idempotence error and sharp Sobolev bounds.  
\end{itemize}

\medskip

\noindent\textbf{Backward Links.}
\begin{itemize}
  \item From Chapter~2: Selberg transform conventions and Plancherel normalization guarantee that the cutoff $\chi_{\lambda,\eta}$ is well-defined.  
  \item From Chapter~3: Truncated kernels $K_Y$ approximate $P_{\lambda,\eta}$ and motivate the support and localization estimates.  
  \item From Chapter~1: The need for spectral localization and approximate projectors is motivated in the introduction and executive summary.  
\end{itemize}

\medskip

\noindent\textbf{Consistency Checks.}
\begin{itemize}
  \item All definitions of Laplace eigenvalues $\lambda_j=1/4+r_j^2$ are consistent across Chapters~2–4.  
  \item Constants are declared explicitly, including dependence on cusp widths and $\beta$.  
  \item Approximate idempotence bounds are sharp and align with cutoff profile properties.  
  \item Egorov-type results (Block~4.3) consistent with semiclassical theory (Duistermaat–Guillemin~\cite{DG1975}, Hörmander~\cite{Hormander1994}, Zworski~\cite{Zworski2012}).  
  \item All lemmas, corollaries, and theorems numbered consecutively and referenced consistently.  
\end{itemize}

\medskip

\noindent\textbf{Conclusion of Audit.}
Chapter~4 has fully developed the analytic framework of the spectral projector $P_{\lambda,\eta}$.  
It is rigorously defined, localized in both spectral and geometric domains, microlocalized in phase space, and quantitatively controlled.  
All invariants have been preserved, forward and backward links established, and constants made explicit.  
This chapter provides a complete analytic foundation for the microlocal parametrix (Chapter~5) and the localized trace formula (Chapter~6--7).  

% --- End of Audit Block: Chapter 4 ---
