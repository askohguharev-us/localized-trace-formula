\section{The Microlocal Projector}\label{sec:projector}

The microlocal kernel $K_R^Y$ constructed in Section~\ref{sec:kernel} provides the analytic backbone for a localized spectral projector.
Our objectives in this section are to pass from kernel bounds to operator-theoretic statements, to quantify idempotence and orthogonality with explicit error exponents, to normalize the operator on the spectral window, and to record its phase-space localization and uniformity across families.
All constants are tracked polynomially in geometric data of $X=\Gamma\backslash\HH$ and in the parameters of the cutoff $(\theta,\beta)$.

\subsection{Set-up and notational conventions}\label{subsec:proj-setup}
We keep the notation of Section~\ref{sec:kernel}.
The Laplace–Beltrami operator is denoted by $\Lap$, and the spectral parameter is $r\in\RR$ with $\lambda=\tfrac14+r^2$.
The frequency window is centered at $R\to\infty$ with width $R^\theta$, where $0<\theta<1$.
The cusp cutoff height is $Y=R^\beta$ with $0<\beta<1$.
The spectral test function is
\[
h_R(r)=\eta\!\big((r-R)/R^\theta\big),
\]
with $\eta$ even, nonnegative, Schwartz, and $\eta(0)=1$, as in \eqref{eq:hR-def}.
Its Fourier transform satisfies
\[
\widehat{h}_R(t)=R^\theta\,\widehat{\eta}(tR^\theta)e^{itR},
\]
essentially supported on $|t|\lesssim R^{-\theta}$, cf.\ \eqref{eq:hhat}.
The geometric profile $k_R(\rho)$ is the inverse spherical transform of $h_R$ and admits the oscillatory form \eqref{eq:kR-asymp} with the short/intermediate/long range bounds \eqref{eq:short}–\eqref{eq:long}.
We write $\chi_Y$ for the smooth height cutoff, identically $1$ on $\{y\le Y\}$ and supported in $\{y\le 2Y\}$, with derivative bounds $\partial_y^m\chi_Y\ll Y^{-m}$.

\subsection{Definition of the operator \texorpdfstring{$\TR$}{TR}}\label{subsec:proj-def}
We define the integral operator
\begin{equation}\label{eq:TR-def}
(\TR f)(z):=\int_X K_R^Y(z,w)\,f(w)\,d\vol(w),
\qquad
K_R^Y:=\chi_Y K_R \chi_Y,
\end{equation}
where $K_R$ is given by the geometric sum \eqref{eq:geom-sum}.
By construction $K_R^Y(z,w)=\overline{K_R^Y(w,z)}$, hence $\TR$ is self-adjoint on $L^2(X)$.
Positivity of $h_R$ and the Harish–Chandra transform imply that $\TR$ is positive.
Using \eqref{eq:short}–\eqref{eq:long} and Schur–Plancherel estimates, we obtain the Sobolev mapping bounds
\begin{equation}\label{eq:TR-sobolev}
\|\TR\|_{H^s\to H^{s'}}\ll R^{\theta+|s'-s|},
\qquad
\|\TR\|_{L^2\to L^2}\ll R^\theta,
\end{equation}
uniformly in $Y=R^\beta$, with constants polynomial in $\injrad(X)^{-1}$, $\vol(X)$, and the number of cusps.

\subsection{Spectral multiplier identity and diagonal action}\label{subsec:proj-spectrum}
Let $\{\varphi_j\}$ be an orthonormal basis of cuspidal eigenfunctions with $\Lap\varphi_j=(\tfrac14+t_j^2)\varphi_j$.
Let $E(z,1/2+it)$ denote normalized Eisenstein series.
Unfolding \eqref{eq:KR-spectral} and inserting the cutoff $\chi_Y$ yields
\begin{equation}\label{eq:TR-diagonal}
\TR\varphi_j=h_R(t_j)\,\varphi_j+O(R^{-A}),
\qquad
\TR E(\cdot,1/2+it)=O(R^{-A}),
\end{equation}
for any $A>0$, uniformly in $j$ and $t\in\RR$.
The implied constants depend polynomially on geometric data and on $(\theta,\beta)$.
Equation \eqref{eq:TR-diagonal} expresses that $\TR$ is a spectral multiplier equal to $h_R$ on the cuspidal spectrum and negligible on the continuous spectrum after truncation.

\subsection{Approximate idempotence}\label{subsec:proj-idempotence}
A projector should satisfy $P^2=P$.
For $\TR$ we compute
\[
(\TR^2 f)(z)=\int_X\!\Big(\int_X K_R^Y(z,u)K_R^Y(u,w)\,d\vol(u)\Big)f(w)\,d\vol(w).
\]
Thus $\TR^2$ has kernel $K_R^Y\star K_R^Y$ and spectral multiplier $h_R^2$.
Therefore
\begin{equation}\label{eq:idemp-eig}
(\TR^2-\TR)\varphi_j=\big(h_R(t_j)^2-h_R(t_j)\big)\varphi_j+O(R^{-A}).
\end{equation}
Inside the window $|t_j-R|\le R^\theta$ we Taylor expand around $R$:
\[
h_R(t_j)^2-h_R(t_j)=(t_j-R)\,h_R'(R)\,R^{-\theta}+O(R^{-2\theta}).
\]
Outside the window, $h_R(t_j)\ll (1+|t_j-R|/R^\theta)^{-M}$ for all $M$.
Taking the $L^2$ operator norm we obtain
\begin{equation}\label{eq:TR-idemp-norm}
\|\,\TR^2-\TR\,\|_{L^2\to L^2}\ll R^{-\theta}.
\end{equation}
This is the basic idempotence estimate used repeatedly in Section~\ref{sec:geometric}.

\subsection{Orthogonality across disjoint windows}\label{subsec:proj-orth}
Let $R_1,R_2$ be two central frequencies with $|R_1-R_2|\ge c\,R^\theta$, where $c\gg1$ is fixed.
Define $\mathsf{T}_{R_i}$ using $h_{R_i}$.
On cusp eigenfunctions,
\[
\mathsf{T}_{R_1}\mathsf{T}_{R_2}\varphi_j=h_{R_1}(t_j)h_{R_2}(t_j)\,\varphi_j.
\]
Since $h_{R_1}$ and $h_{R_2}$ are supported on disjoint windows up to super-polynomial tails,
\[
|h_{R_1}(t_j)h_{R_2}(t_j)|\ll \Big(1+\frac{|R_1-R_2|}{R^\theta}\Big)^{-M}\ll R^{-M}
\]
for all $M>0$.
Hence
\begin{equation}\label{eq:orth-norm}
\|\mathsf{T}_{R_1}\mathsf{T}_{R_2}\|_{L^2\to L^2}\ll R^{-M},
\end{equation}
and the same estimate holds for compositions with Eisenstein series after truncation.
This super-polynomial orthogonality is crucial for spectral statistics on adjacent windows.

\subsection{Normalization on the window}\label{subsec:proj-normal}
Define the average
\[
\kappa_R:=\frac{1}{N(R,R^\theta)}\sum_{|t_j-R|\le R^\theta} h_R(t_j),
\qquad
N(R,R^\theta)=\#\{j:|t_j-R|\le R^\theta\}.
\]
By the windowed Weyl law and \eqref{eq:normalization},
\[
N(R,R^\theta)=\frac{\vol(X)}{2\pi}R^\theta+O(R^{\theta-1}),
\qquad
\kappa_R=1+O(R^{-1}).
\]
Set $h_R^{\mathrm{norm}}=h_R/\kappa_R$ and define $\TR^{\mathrm{norm}}$ accordingly.
Then
\begin{equation}\label{eq:norm-identity}
\TR^{\mathrm{norm}}\varphi_j=(1+O(R^{-\theta}))\,\varphi_j
\quad\text{for all }|t_j-R|\le R^\theta,
\end{equation}
and $\|\TR^{\mathrm{norm}}\|_{H^s\to H^s}\ll 1$, uniformly in $R$ and $s\in\RR$.

\subsection{Microlocal description and Egorov scale}\label{subsec:proj-micro}
Write the kernel in oscillatory form
\[
K_R^Y(z,w)=\int_{\RR^2} e^{iR\,\Phi(z,w,\xi)}\,a_R(z,w,\xi)\,d\xi,
\]
with phase $\Phi$ parametrizing geodesic distance and symbol $a_R$ admitting a full expansion
\[
a_R(z,w,\xi)\sim\sum_{m\ge0} R^{-m\theta} a_m(z,w,\xi).
\]
Stationary phase shows that critical points of $\Phi$ correspond to geodesic arcs from $w$ to $z$ of length $\lesssim R^\theta$.
Hence $\WF(\TR)$ lies on the canonical relation of the geodesic flow at times $|t|\lesssim R^{-\theta}$, and wave packets of central frequency $R$ propagate microlocally for time $t_R\sim R^{-\theta}$.
For any semiclassical pseudodifferential operator $A$ with symbol $\sigma_A$, Egorov’s theorem yields
\begin{equation}\label{eq:egorov}
\TR^* A \TR = \TR^*\TR\,(A\circ g^{t_R}) + O(R^{-\theta}),
\end{equation}
in the $L^2\to L^2$ operator norm, with constants polynomial in geometric data.

\subsection{Cusp truncation and continuous spectrum}\label{subsec:proj-cusp}
Let $E(z,1/2+it)$ be an Eisenstein series.
Truncation at height $Y=R^\beta$ implies
\[
\|\chi_Y E(\cdot,1/2+it)\|_{L^2(X)}\ll R^{-\beta/2+\epsilon},
\]
uniformly in $t$.
Combining this with \eqref{eq:TR-sobolev} and the short-time support of $\widehat{h}_R$ gives
\begin{equation}\label{eq:eisenstein-suppression}
\|\TR\,E(\cdot,1/2+it)\|_{L^2(X)}\ll R^{-\beta/2+\theta+\epsilon},
\end{equation}
and, by choosing admissible $(\theta,\beta)$, a clean $O(R^{-A})$ suppression for any fixed $A$.
This quantifies the negligible effect of the continuous spectrum in the localized projector.

\subsection{Sobolev bounds and off-diagonal decay}\label{subsec:proj-soboff}
The integral kernel obeys the pointwise bounds in distance, cf.\ \eqref{eq:short}–\eqref{eq:long}.
Combining Schur tests with the parametrix on the universal cover (Section~\ref{subsec:parametrix}) gives
\[
\|K_R^Y\|_{L^2\to L^2}\ll R^\theta,
\qquad
\|K_R^Y\|_{L^1\to L^\infty}\ll R^{\tfrac12+\theta}.
\]
Composing with fractional powers of $(1+\Lap)$ yields \eqref{eq:TR-sobolev}.
Moreover, for $d(z,w)\ge c>0$ fixed, the long-range bound \eqref{eq:long} implies
\[
|K_R^Y(z,w)|\ll R^\theta e^{-d(z,w)/2},
\]
which we will use on the geometric side in Section~\ref{sec:geometric} to separate the identity and closed geodesic contributions.

\subsection{Hilbert–Schmidt and trace ideal membership}\label{subsec:proj-hs}
Since $h_R$ is compactly supported on a window of measure $\asymp R^\theta$ and $\chi_Y$ cuts off the cusps, Plancherel shows
\[
\int_{X\times X} |K_R^Y(z,w)|^2\,d\vol(z)\,d\vol(w)\;\asymp\; R^\theta\,\vol_{\mathrm{eff}}(X;Y),
\]
cf.\ Section~\ref{subsec:cusp-cutoff}.
Hence $K_R^Y$ is Hilbert–Schmidt with norm $\ll R^{\theta/2}\,\vol_{\mathrm{eff}}^{1/2}$.
In particular, for fixed $R$ the operator is compact on $L^2(X)$.
We will not use trace class properties, but the HS bound is convenient for controlling error terms in the trace.

\subsection{Commutators and stability under pseudodifferential perturbations}\label{subsec:proj-comm}
Let $A\in\Psi^m(X)$ be a fixed pseudodifferential operator.
Using the microlocal representation and symbolic calculus one obtains
\[
\|[\TR,A]\|_{L^2\to L^2}\ll R^{-\theta},
\]
with constants depending polynomially on seminorms of $A$ and on geometric data.
In particular,
\[
\|[\TR,(1+\Lap)^{s/2}]\|_{L^2\to L^2}\ll R^{-\theta},
\]
for each fixed $s$, showing that $\TR$ almost commutes with elliptic weights on the Egorov scale $t_R\sim R^{-\theta}$.
These commutator estimates ensure that localization is compatible with standard functional spaces.

\subsection{Window calculus and stability under convolution}\label{subsec:proj-windowcalc}
For $\delta\in[1,2]$ define $h_{R,\delta}(r)=\eta((r-R)/(\delta R^\theta))$ and let $k_{R,\delta}$ be its inverse spherical transform.
Spectral convolution corresponds to geometric convolution:
\[
(h_{R,\delta_1}\!*\,h_{R,\delta_2})^\vee
\;\longleftrightarrow\;
k_{R,\delta_1}\star k_{R,\delta_2}.
\]
Since both multipliers have time support $\lesssim R^{-\theta}$, the convolution does not spread beyond that scale, and stationary phase yields the stability estimate
\begin{equation}\label{eq:window-stability}
\big\|\,k_{R,\delta_1}\star k_{R,\delta_2}-k_{R,\sqrt{\delta_1^2+\delta_2^2}}\,\big\|_{L^1\to L^\infty}\ll R^{-A},
\end{equation}
for any fixed $A>0$.
Applied to iterates of $\TR$, \eqref{eq:window-stability} shows that repeated application does not smear the spectral window by more than a negligible tail.

\subsection{Parameter region and the error exponent}\label{subsec:proj-params}
The remainder exponent appearing in the localized trace formula is
\[
\varepsilon(\theta,\beta)
=
\min\Big\{\theta,\;1-\theta+\beta,\;\tfrac12,\;1-2\theta+\beta\Big\}
-\delta,
\]
for arbitrarily small $\delta>0$.
Each component has a transparent origin:
\begin{itemize}
\item $\theta$ quantifies spectral localization error from the window width.
\item $1-\theta+\beta$ balances the cusp truncation loss against spectral shrinkage.
\item $\tfrac12$ reflects the intrinsic spectral density from Weyl’s law.
\item $1-2\theta+\beta$ arises from short-time propagation versus cusp derivatives.
\end{itemize}
Admissibility requires $\varepsilon(\theta,\beta)>0$.
Typical admissible choices include $\theta=\tfrac12-\epsilon$ and $\beta=\tfrac12$.

\subsection{Comparison with Gaussian and Paley–Wiener projectors}\label{subsec:proj-compare}
Gaussian multipliers $h(t)=e^{-(t-R)^2}$ have fixed width independent of $R$ and do not adapt to $R^\theta$.
Their time-side decay does not match the short propagation scale $t_R\sim R^{-\theta}$, and they offer no mechanism for cusp control.
Paley–Wiener cutoffs ensure compact spectral support but yield geometric kernels with inferior microlocal concentration on shrinking scales.
By contrast, the present choice \eqref{eq:hR-def}–\eqref{eq:hhat} delivers simultaneous spectral and geometric localization, compatible with cusp truncation and with polynomial control of constants.

\subsection{Uniformity in arithmetic families}\label{subsec:proj-families}
Let $X_\mathfrak{q}$ range over a family of congruence covers or arithmetic surfaces with controlled injectivity radius away from cusps.
The constants implicit in \eqref{eq:TR-sobolev}, \eqref{eq:TR-idemp-norm}, \eqref{eq:orth-norm}, and \eqref{eq:egorov} are polynomial in $\injrad(X_\mathfrak{q})^{-1}$, in the number of cusps, and in $\vol(X_\mathfrak{q})$.
This uniformity is essential for applications to averaged sup-norm bounds, windowed Weyl laws in families, and spectral statistics across towers.

\subsection{Auxiliary lemmas used later on the geometric side}\label{subsec:proj-aux}
We record two lemmas that will be invoked in Section~\ref{sec:geometric}.

\begin{lemma}[Short-time $L^1\to L^\infty$ gain]\label{lem:L1-Linf}
Let $\psi$ be the cutoff from \eqref{eq:kR-asymp}.
There exists $C>0$ such that
\[
\|K_R^Y\|_{L^1\to L^\infty}\le C\,R^{\tfrac12+\theta}.
\]
Moreover, if $d(z,w)\ge c>0$ then
\[
|K_R^Y(z,w)|\ll R^\theta e^{-d(z,w)/2}.
\]
\end{lemma}

\begin{proof}
Combine the oscillatory structure \eqref{eq:kR-asymp} with non-stationary phase when $\rho\gtrsim R^{-\theta}$ and with the exponential factor $\sinh(\rho/2)^{-1}$ for long range.
Summation over $\gamma\in\Gamma$ is controlled by orbit growth in hyperbolic balls.
The cusp cutoff multiplies by a bounded function with derivatives polynomially controlled in $Y^{-1}=R^{-\beta}$, which does not worsen the stated exponents.
\end{proof}

\begin{lemma}[Hilbert–Schmidt control]\label{lem:HS}
There is $C'>0$ such that
\[
\|K_R^Y\|_{\mathrm{HS}}^2
=
\int_{X\times X} |K_R^Y(z,w)|^2\,d\vol(z)\,d\vol(w)
\le C'\,R^\theta\,\vol_{\mathrm{eff}}(X;Y).
\]
\end{lemma}

\begin{proof}
Use Plancherel on the universal cover together with the spectral support of $h_R$ and the fact that multiplication by $\chi_Y$ is bounded on $L^2$ with norm $\le 1$.
\end{proof}

\subsection{Applications enabled by the projector}\label{subsec:proj-applications}
We highlight several consequences that will be developed later or are standard once \eqref{eq:TR-idemp-norm}–\eqref{eq:egorov} are available.
\begin{enumerate}
\item \textbf{Windowed Weyl law.}
Counting cusp eigenvalues in $[R-R^\theta,R+R^\theta]$ with a power-saving remainder $O(R^{1-\varepsilon(\theta,\beta)})$.
\item \textbf{Sup-norm amplification.}
Applying $\TR$ to $\varphi_j$ in the window and using $L^p$ bounds for kernels yields uniform $L^\infty$ improvements of the form $\|\varphi_j\|_\infty\ll R^{1/2-\varepsilon}$.
\item \textbf{Quantum ergodicity on fine scales.}
Variance bounds for quantum averages restricted to windows, exploiting \eqref{eq:egorov}.
\item \textbf{Spectral statistics.}
Orthogonality of different $\mathsf{T}_{R_i}$ as in \eqref{eq:orth-norm} underpins pair-correlation analysis between adjacent windows.
\item \textbf{Arithmetic consequences.}
Polynomial dependence of constants allows uniform statements across congruence families, relevant for Fourier coefficient bounds and short-interval prime geodesic phenomena.
\end{enumerate}

\subsection{Extended microlocal refinements}\label{subsec:proj-refinements}
For later technical steps we note three refinements.
\begin{itemize}
\item \emph{Wavefront localization.}
The set $\WF(K_R^Y)\subset T^*X\times T^*X$ is contained in an $R^{-\theta}$-neighborhood of the graph of the geodesic flow for times $|t|\lesssim R^{-\theta}$.
\item \emph{Symbolic expansion.}
Every derivative of the amplitude $a_R$ gains a factor $R^{-\theta}$ and multiplies by a polynomial in $\injrad(X)^{-1}$ and the cusp height derivatives.
\item \emph{Commutator stability.}
For $A\in\Psi^0(X)$ one has $\|[\TR,A]\|\ll R^{-\theta}$, with the same exponent as in \eqref{eq:TR-idemp-norm}, reflecting the Egorov time scale.
\end{itemize}

\subsection{Proof of the main operator properties}\label{subsec:proj-proofs}
For completeness we sketch the derivations that were quoted above.
\paragraph{Diagonal action.}
Insert the spectral resolution into \eqref{eq:TR-def}.
Because $h_R$ multiplies the spectral measure and $\chi_Y$ kills the continuous spectrum to $O(R^{-A})$, one gets \eqref{eq:TR-diagonal}.
\paragraph{Idempotence.}
The kernel of $\TR^2$ is the geometric convolution of kernels, corresponding to the spectral product $h_R^2$; estimate the difference by Taylor expansion near $R$ as in \eqref{eq:idemp-eig}.
\paragraph{Orthogonality.}
If the windows are disjoint at scale $R^\theta$, then $h_{R_1}h_{R_2}$ is $\ll R^{-M}$ uniformly, which gives \eqref{eq:orth-norm}.
\paragraph{Egorov scale.}
Use the oscillatory representation with phase $R\Phi$.
Stationary phase at time $t_R\sim R^{-\theta}$ and symbolic calculus imply \eqref{eq:egorov}.
\paragraph{Sobolev bounds.}
Combine Schur test, \eqref{eq:short}–\eqref{eq:long}, and the parametrix of Section~\ref{subsec:parametrix} to obtain \eqref{eq:TR-sobolev}.

\subsection{Robustness under geometric perturbations}\label{subsec:proj-stability}
Let $g_\varepsilon$ be a smooth family of hyperbolic metrics with bounded derivatives and controlled geometry.
Then the constructions above vary continuously with $\varepsilon$, and all constants retain polynomial bounds in geometric parameters.
In particular, $\TR$ depends stably on the metric in the operator norm topology, with
\[
\|\TR[g_\varepsilon]-\TR[g_0]\|_{L^2\to L^2}\ll \|g_\varepsilon-g_0\|_{C^k},
\]
for some $k$ depending only polynomially on the differentiation order needed in the symbolic calculus.

\subsection{A remark on alternative test functions}\label{subsec:proj-tests}
The particular choice $h_R(r)=\eta((r-R)/R^\theta)$ is convenient but not unique.
Any family of even, nonnegative, Schwartz multipliers with the same window and the same time-side support $|t|\lesssim R^{-\theta}$ gives identical conclusions.
Moreover, compactly supported Paley–Wiener multipliers at this scale still yield the same microlocal picture, although with slightly different constants for the $L^1\to L^\infty$ gain.

\subsection{Synopsis for later use}\label{subsec:proj-synopsis}
We collect the properties of $\TR$ used downstream:
\begin{itemize}
\item \textbf{Diagonal action:} \eqref{eq:TR-diagonal}.
\item \textbf{Idempotence:} \eqref{eq:TR-idemp-norm}.
\item \textbf{Orthogonality:} \eqref{eq:orth-norm}.
\item \textbf{Normalization:} \eqref{eq:norm-identity}.
\item \textbf{Microlocal Egorov:} \eqref{eq:egorov}.
\item \textbf{Sobolev bounds:} \eqref{eq:TR-sobolev}.
\item \textbf{Cusp suppression:} \eqref{eq:eisenstein-suppression}.
\end{itemize}
These seven items form the operator-theoretic toolkit feeding into Section~\ref{sec:geometric} and the final trace identity.

\subsection{Conclusion}\label{subsec:proj-conclusion}
The localized projector $\TR$ achieves simultaneous spectral localization to the window $[R-R^\theta,R+R^\theta]$, microlocal concentration along short geodesic arcs on the Egorov scale $t_R\sim R^{-\theta}$, and suppression of the continuous spectrum by cusp truncation with height $Y=R^\beta$.
Its idempotence and orthogonality hold with explicit error exponents polynomially controlled in geometric parameters.
Normalization on the window gives an operator close to the identity in the strong sense of \eqref{eq:norm-identity}.
All these features are indispensable for the geometric expansion in Section~\ref{sec:geometric} and for the sharp remainder estimates in the localized trace formula.

% End of Section 04
