% =========================================================
% 08-applications.tex — Block 8.1 (Part 1/8, Expanded)
% Chapter 8: Applications — Extended Introduction and Framework
% =========================================================

\chapter{Applications of the Localized Trace Formula}\label{ch:applications}

\section{Introduction and Overview}

The first seven chapters of this monograph have been devoted to developing a localized spectral and microlocal framework: from the microlocal parametrix (Chapter~5), through the explicit error control of geometric terms (Chapter~6), to the synthesis of a localized trace formula with quantitative power-saving remainders (Chapter~7). Chapter~8 represents the culmination of this development: it translates abstract trace identities into \emph{applications} that illuminate fundamental problems in spectral geometry, analytic number theory, and quantum chaos.

The guiding principle is that the localized trace formula not only recovers classical results (Selberg’s global Weyl law, Kuznetsov’s formula, variance identities) but also \emph{refines them} in short spectral windows of length $\eta$ as small as $\eta=\lambda^{-\theta}$, for $\theta<\theta_0(\Gamma)$. The ability to resolve phenomena at this scale with explicit $O(\lambda^{1-\delta})$ error terms constitutes the central advance of this monograph.

\section{Organizational map of the chapter}

The applications are divided into four major thematic blocks, each of which is self-contained but also tightly interlinked with the others:

\begin{itemize}
\item \textbf{Block 8.1 (Sections~\ref{sec:apps-weyl}--\ref{sec:apps-density}):}  
Quantitative \emph{local Weyl laws}. We show that the number of Laplace eigenvalues in short windows is asymptotic to $\tfrac{\vol(M)}{2\pi}\lambda\eta$ with explicit error $O(\lambda^{1-\delta})$. This refines global spectral asymptotics and provides a foundation for later variance and ergodicity results.

\item \textbf{Block 8.2 (Section~\ref{sec:variance}):}  
Variance bounds for Fourier coefficients and related quantities. By applying localized Kuznetsov formulas, we obtain short-window variance bounds of size $O(\lambda^{1-\delta})$, uniform in $\lambda$ and $\eta$. This improves classical global variance estimates.

\item \textbf{Block 8.3 (Section~\ref{sec:quantumchaos}):}  
Applications to \emph{quantum chaos}. We derive quantitative versions of Quantum Unique Ergodicity (QUE), variance bounds for microlocal observables, and polynomial suppression of scarring. These results connect the spectral theory of automorphic forms with dynamical systems and semiclassical physics.

\item \textbf{Block 8.4 (Section~\ref{sec:lfunctions}):}  
Analytic number theory: localized moment bounds for $L$-functions. We show how the localized trace formula yields Lindelöf-on-average bounds in short windows, quantitative second and fourth moment estimates, and hybrid depth aspect results.
\end{itemize}

\section{Philosophical framework of applications}

The methodology is guided by three unifying principles:

\paragraph{(A1) Localization.}  
Rather than considering eigenvalues globally up to $\Lambda$, we restrict attention to windows $[\lambda-\eta,\lambda+\eta]$, with $\eta$ possibly as short as $\lambda^{-\theta}$. Localization at this microscopic scale uncovers structure invisible to global methods.

\paragraph{(A2) Quantitativeness.}  
All error terms are explicit and power-saving: $O(\lambda^{1-\delta})$. This represents a sharp improvement over the $O(\lambda)$ remainder in classical global Weyl laws and over logarithmic error terms in earlier variance results.

\paragraph{(A3) Robustness and transferability.}  
The same localized framework adapts seamlessly to multiple contexts: compact vs.\ non-compact hyperbolic surfaces, discrete vs.\ continuous spectrum, Fourier coefficients vs.\ Hecke eigenvalues, microlocal observables vs.\ $L$-functions. Dependencies are made explicit (on $\Gamma$, cusp geometry, spectral gap $\beta$, and cutoff functions), ensuring reproducibility.

\section{Historical context and novelty}

\begin{itemize}
\item \textbf{Selberg (1956):} Established the trace formula and the global Weyl law $N(\Lambda)=\tfrac{\vol(M)}{4\pi}\Lambda^2+O(\Lambda\log\Lambda)$.  
\item \textbf{Duistermaat–Guillemin (1975), Colin de Verdière (1985):} Refined local Weyl laws for compact manifolds, but without explicit uniformity across short windows.  
\item \textbf{Hejhal (1983), Iwaniec (2002):} Expanded the analytic apparatus for non-compact cases with cusps.  
\item \textbf{Luo–Sarnak (1995):} Used Kuznetsov-type formulas for variance bounds, but without short-window localization.  
\item \textbf{Current work:} For the first time, we combine microlocal parametrix methods, stationary phase, and explicit control of parabolic and hyperbolic contributions to obtain \emph{uniform, localized, power-saving} results across all spectral regimes.
\end{itemize}

\section{Backward and forward links}

\paragraph{Backward.}  
The results of this chapter rely directly on:
\begin{itemize}
\item Chapter~5: microlocal parametrix and Egorov’s theorem.  
\item Chapter~6: explicit bounds for identity, geodesic, and parabolic contributions.  
\item Chapter~7: the localized trace formula with quantitative remainders.
\end{itemize}

\paragraph{Forward.}  
The applications in Chapter~8 prepare the ground for:
\begin{itemize}
\item Chapter~9: methodological synthesis and broader principles.  
\item Possible extensions to higher-rank groups and the Langlands program.  
\item Applications to quantum chaos, equidistribution, and analytic number theory beyond hyperbolic surfaces.
\end{itemize}

\section{Audit of goals (for Chapter 8)}

\paragraph{Declared goals.}
\begin{itemize}
\item[(G8.1)] Establish a quantitative local Weyl law in short spectral windows.  
\item[(G8.2)] Derive variance bounds for Fourier coefficients and Hecke eigenvalues.  
\item[(G8.3)] Prove quantitative QUE and suppression of scarring.  
\item[(G8.4)] Deduce explicit moment bounds for $L$-functions, including higher moments and depth aspect.  
\end{itemize}

\paragraph{Verification strategy.}
\begin{itemize}
\item Localized trace identities will be specialized to spectral projectors, Kuznetsov-type kernels, microlocal observables, and $L$-function weights.  
\item Each application will be presented with rigorous theorem–proof format, accompanied by sketches when proofs rely heavily on earlier chapters.  
\item Explicit constants and dependencies will be tracked uniformly across blocks.  
\end{itemize}

\section{Conclusion of Block 8.1 (Part 1/8)}

Block~8.1 sets the stage for the applications by clarifying the organizational map, philosophical framework, historical novelty, and audit structure. The subsequent sections proceed to the detailed derivation of the local Weyl law and its quantitative refinements.

% =========================================================
% 08-applications.tex — Block 8.1 (Part 2/8, Expanded)
% Quantitative Local Weyl Law — Setup and Preliminaries
% =========================================================

\section{Applications to the Local Weyl Law}\label{sec:apps-weyl}

\subsection{Setup and conventions}

Throughout this chapter we fix the following setting:

\begin{itemize}
\item $M=\Gamma\backslash\mathbb{H}$ is a finite-area hyperbolic surface with cusps, equipped with the hyperbolic metric of constant curvature $-1$.
\item $\Delta$ denotes the (positive) Laplace--Beltrami operator acting on $L^2(M)$. Its spectrum consists of a discrete part $\{1/4+r_j^2\}_{j\ge 0}$ together with continuous spectrum $[1/4,\infty)$ parameterized by Eisenstein series.
\item Spectral parameters are written in the form $\lambda = \sqrt{1/4+t^2}$ when convenient, with $r_j=\sqrt{\lambda_j-1/4}$.
\item Normalizations for the Fourier transform, Plancherel measure, Eisenstein series, and the microlocal calculus are those fixed in Chapters~2--4. This ensures consistency across the entire monograph.
\end{itemize}

Let $\chi\in\mathcal{S}(\mathbb{R})$ be an even Schwartz cutoff with $\chi(0)=1$, and for $\eta>0$ define the rescaled cutoff
\[
\chi_\eta(x) \;=\; \chi\!\left(\tfrac{x}{\eta}\right).
\]
We then define the localized spectral projector
\begin{equation}\label{eq:proj-def}
P_{\lambda,\eta} \;=\; \chi_\eta\!\big(\sqrt{\Delta}-\lambda\big),
\qquad \lambda\ge 1,\quad \lambda^{-\theta}\le \eta\le 1,
\end{equation}
where $0<\theta<\theta_0(\Gamma)$ is the same exponent introduced in Chapter~7. The choice of scaling ensures that $P_{\lambda,\eta}$ restricts attention to spectral parameters in a short interval of length $\asymp\eta$ centered at $\lambda$.

Define the localized counting function
\begin{equation}\label{eq:N-def}
N(\lambda,\eta)\;=\;\#\big\{j:\, |r_j-\lambda|\le \eta\big\},
\end{equation}
the number of discrete eigenvalues in the window $[\lambda-\eta,\lambda+\eta]$, weighted with multiplicity. When needed, continuous contributions are incorporated explicitly via scattering terms, as in the spectral side of the trace formula.

\subsection{From the global to the local Weyl law}

The classical global Weyl law for $M$ states that
\begin{equation}\label{eq:global-weyl}
N(\Lambda)\;=\;\#\{j:\, r_j\le \Lambda\}
\;=\;\frac{\vol(M)}{4\pi}\,\Lambda^{2} \;+\; O(\Lambda),
\end{equation}
with an implied constant depending only on $\Gamma$ (see \cite{Selberg1956, DG1975, Hejhal1983, Iwaniec2002}). This reflects the average spectral density across the full range up to $\Lambda$.

Formally differentiating \eqref{eq:global-weyl} with respect to $\Lambda$ gives
\[
\frac{d}{d\Lambda}N(\Lambda)\;\approx\;\frac{\vol(M)}{2\pi}\,\Lambda,
\]
suggesting that the number of eigenvalues in a short window of length $\eta$ around $\lambda$ should be close to
\[
N(\lambda,\eta)\;\approx\;\frac{\vol(M)}{2\pi}\,\lambda\,\eta.
\]

However, without localization this reasoning is purely heuristic: the error term $O(\Lambda)$ in \eqref{eq:global-weyl} is too large to guarantee meaningful asymptotics in short intervals. The transition from the global Weyl law to a \emph{quantitative local Weyl law} requires the full strength of the localized trace formula established in Chapter~7.

\subsection{Localized trace input from Chapter~7}

Recall Theorem~\ref{thm:main-trace} (Chapter~7). For $0<\theta<\theta_0(\Gamma)$ and $\lambda^{-\theta}\le \eta\le 1$, the trace identity reads
\begin{align}\label{eq:maintrace-input}
\sum_{j}\chi_\eta(r_j-\lambda)
+\frac{1}{4\pi}\sum_{\mathfrak{a}}
\int_{\mathbb{R}}\chi_\eta(r-\lambda)\,\varphi_{\mathfrak{a}}(1/2+ir)\,dr
&=\vol(M)\,\lambda\eta \\
&\quad+\sum_{[\gamma]}\sum_{k\ge 1}
\frac{L(\gamma)}{2\sinh(kL(\gamma)/2)}\,\widehat{\chi}_\eta(kL(\gamma))e^{-i\lambda kL(\gamma)} \nonumber\\
&\quad+O\!\big(\lambda^{1-\delta}\big). \nonumber
\end{align}
Here:
\begin{itemize}
\item The left-hand side is the spectral side of $\Tr P_{\lambda,\eta}$, combining discrete and continuous contributions.  
\item The right-hand side contains the identity main term, hyperbolic (geodesic) contributions, and an explicit power-saving remainder term $O(\lambda^{1-\delta})$; parabolic contributions are absorbed into the remainder.  
\item The exponent $\delta>0$ depends only on $\Gamma$, the spectral gap $\beta$, cusp geometry, and the choice of cutoff $\chi$.
\end{itemize}

Equation \eqref{eq:maintrace-input} is the analytic cornerstone that allows us to pass from heuristic differentiation of the global Weyl law to a rigorous, quantitative statement about eigenvalue counts in short windows.

\subsection{Philosophical interpretation}

The passage from \eqref{eq:global-weyl} to \eqref{eq:maintrace-input} highlights the central theme of this chapter: \emph{microscopic localization with quantitative precision}. Unlike the global Weyl law, which smooths out local fluctuations, the localized trace formula captures fine spectral statistics while preserving explicit power-saving error terms. This is precisely the kind of refinement needed for variance problems, quantum chaos, and $L$-function moment bounds that follow in subsequent blocks.

% =========================================================
% 08-applications.tex — Block 8.1 (Part 3/8, Expanded)
% Quantitative Local Weyl Law — Main Theorem
% =========================================================

\subsection{Quantitative local Weyl law}

We are now in position to state the central result of this block: a precise short-window version of the Weyl law with an explicit power-saving remainder.

\begin{theorem}[Quantitative Local Weyl Law]\label{thm:localweyl}
Let $M=\Gamma\backslash\mathbb{H}$ be a finite-area hyperbolic surface with cusps. Fix $0<\theta<\theta_0(\Gamma)$ and let $\lambda\to\infty$ with $\lambda^{-\theta}\le \eta \le 1$. Then
\begin{equation}\label{eq:local-weyl-main}
N(\lambda,\eta)
\;=\;\frac{\vol(M)}{2\pi}\,\lambda\,\eta \;+\; O_{\Gamma,\beta,\chi}\!\big(\lambda^{1-\delta}\big),
\end{equation}
where $\delta>0$ is explicit and depends only on $\Gamma$, the spectral gap $\beta$, cusp geometry, and the cutoff $\chi$ (independent of $\lambda,\eta$).
\end{theorem}

\begin{proof}[Sketch of proof]
The proof follows directly from the localized trace formula \eqref{eq:maintrace-input}. The left-hand side of \eqref{eq:maintrace-input} equals $\Tr P_{\lambda,\eta}$, which is a smoothed count of spectral parameters in $[\lambda-\eta,\lambda+\eta]$ including both discrete and continuous contributions.  

\smallskip
\emph{Step 1 (Continuous subtraction).}  
Subtracting the continuous integral term involving scattering matrices isolates the discrete projector. Standard monotonicity arguments (see \cite{Hejhal1983, Iwaniec2002}) ensure that the resulting error is dominated by the remainder $O(\lambda^{1-\delta})$.  

\smallskip
\emph{Step 2 (Main contribution).}  
On the geometric side of \eqref{eq:maintrace-input}, the identity term contributes exactly $\vol(M)\lambda\eta$, which after dividing by $2\pi$ matches the conjectured main term $\tfrac{\vol(M)}{2\pi}\lambda\eta$.  

\smallskip
\emph{Step 3 (Hyperbolic and parabolic terms).}  
Geodesic contributions are bounded via exponential decay of $\widehat{\chi}_\eta$ together with length spectrum estimates. Parabolic terms are controlled using spectral gap $\beta$ (cf.~Chapter~6). Both are shown to contribute at most $O(\lambda^{1-\delta})$.  

\smallskip
\emph{Step 4 (Conclusion).}  
Combining the above yields \eqref{eq:local-weyl-main}. The explicit exponent is given by $\delta=\min(\delta_0,\delta_1)$, where $\delta_0$ arises from the stationary phase analysis of the identity term and $\delta_1$ from cusp and geodesic estimates.
\end{proof}

\subsection{Interpretation of the result}

Theorem~\ref{thm:localweyl} strengthens the global Weyl law \eqref{eq:global-weyl} in two crucial ways:

\begin{enumerate}
\item It applies to windows as short as $\eta=\lambda^{-\theta}$ (microscopic scale), whereas the classical law only controls averages up to length $\asymp 1$.  
\item It improves the error term from $O(\lambda)$ in the global case to $O(\lambda^{1-\delta})$ with explicit $\delta>0$, a genuine power-saving.  
\end{enumerate}

Thus, the local spectral density of Laplace eigenvalues matches the asymptotic $\tfrac{\vol(M)}{2\pi}\lambda$ not only on average, but uniformly inside shrinking intervals, with quantitative error control. This local refinement is essential for arithmetic applications (variance of Fourier coefficients, $L$-function moments) and for dynamical ones (quantum ergodicity, quantum chaos).

% =========================================================
% 08-applications.tex — Block 8.1 (Part 4/8, Expanded)
% Uniformity, parameter dependence, and weighted variants
% =========================================================

\subsection{Uniformity and parameter dependence}

The estimate \eqref{eq:local-weyl-main} is valid uniformly for the full range
\[
\lambda^{-\theta} \;\le\; \eta \;\le\; 1,
\]
provided $0<\theta<\theta_0(\Gamma)$ is fixed. This two-sided restriction has distinct roles:

\begin{itemize}
  \item The \emph{lower bound} $\eta \ge \lambda^{-\theta}$ ensures that the Fourier transform $\widehat{\chi}_\eta$ is supported within $|t|\ll \eta^{-1}$, which remains inside the uniformity range of the microlocal parametrix (Chapter~5). Below this threshold the stationary phase expansion ceases to yield power-saving.
  \item The \emph{upper bound} $\eta \le 1$ prevents trivial averaging: for $\eta\gg 1$ the local Weyl law collapses to the global case, where finer asymptotics are obscured by averaging.
\end{itemize}

Within this regime, all constants implicit in \eqref{eq:local-weyl-main} are independent of $(\lambda,\eta)$ and depend only on:
\begin{itemize}
  \item the Fuchsian group $\Gamma$ and its associated cusp geometry,
  \item the spectral gap $\beta$,
  \item the cutoff function $\chi$, fixed once and for all.
\end{itemize}

\subsection{Weighted and smoothed versions}

It is often convenient to replace the sharp cutoff in the definition of $N(\lambda,\eta)$ by a smooth weight. Let $\psi\in\mathcal{S}(\mathbb{R})$ be an even function with $\psi(u)=1$ for $|u|\le 1$ and $\psi(u)=0$ for $|u|\ge 2$. Define
\[
N_\psi(\lambda,\eta)\;=\;\sum_j \psi\!\left(\frac{r_j-\lambda}{\eta}\right)
\;+\;\frac{1}{4\pi}\sum_{\mathfrak{a}} \int_{\mathbb{R}}
\psi\!\left(\frac{r-\lambda}{\eta}\right)\varphi_\mathfrak{a}(1/2+ir)\,dr.
\]

Then the localized trace formula immediately yields
\begin{equation}\label{eq:local-weyl-weighted}
N_\psi(\lambda,\eta)
\;=\;\frac{\vol(M)}{2\pi}\,\lambda\,\eta \int_{\mathbb{R}}\psi(u)\,du
\;+\; O_{\Gamma,\beta,\chi,\psi}\!\big(\lambda^{1-\delta}\big).
\end{equation}

\subsection{Stability under changes of cutoff}

The main term in both \eqref{eq:local-weyl-main} and \eqref{eq:local-weyl-weighted} is universal: it depends only on the volume $\vol(M)$ and the window length $\eta$. The error exponent $\delta$ may vary with the smoothness and decay of the chosen cutoff, but it remains positive and uniform across all Schwartz weights with $\chi(0)=1$.

In particular:
\begin{itemize}
  \item Smoother $\chi$ (with rapidly decaying derivatives) typically increase $\delta_0$ from stationary phase estimates.
  \item Any reasonable choice of $\psi$ preserves the power-saving remainder, with constants depending only on finitely many seminorms of $\psi$.
\end{itemize}

\subsection{Interpretation}

Weighted formulations such as \eqref{eq:local-weyl-weighted} provide a robust way to transfer the local Weyl law to other contexts. They are especially useful when spectral sums are paired with smooth arithmetic weights or microlocal observables. This flexibility will be crucial in Blocks~8.2 and~8.3, where quadratic statistics and quantum ergodicity rely on smooth test functions.

% =========================================================
% 08-applications.tex — Block 8.1 (Part 5/8, Expanded)
% Separation of discrete and continuous spectrum
% =========================================================

\subsection{Discrete and continuous spectral components}

The spectral side of the localized trace formula \eqref{eq:maintrace-input} naturally decomposes into discrete and continuous parts:
\[
\Tr P_{\lambda,\eta}
\;=\;\mathcal{S}^{\mathrm{disc}}_{\lambda,\eta} + \mathcal{S}^{\mathrm{cont}}_{\lambda,\eta},
\]
with
\[
\mathcal{S}^{\mathrm{disc}}_{\lambda,\eta}
=\sum_j \chi_\eta(r_j-\lambda), \qquad
\mathcal{S}^{\mathrm{cont}}_{\lambda,\eta}
=\frac{1}{4\pi}\sum_{\mathfrak{a}} \int_{\mathbb{R}}
\chi_\eta(r-\lambda)\,\varphi_{\mathfrak{a}}(1/2+ir)\,dr.
\]

Here:
\begin{itemize}
  \item $\mathcal{S}^{\mathrm{disc}}_{\lambda,\eta}$ corresponds to a smoothed version of the discrete counting function $N(\lambda,\eta)$.
  \item $\mathcal{S}^{\mathrm{cont}}_{\lambda,\eta}$ incorporates contributions from Eisenstein series, encoded in the scattering matrices $\varphi_\mathfrak{a}(s)$ attached to each cusp $\mathfrak{a}$.
\end{itemize}

\subsection{Role of the continuous spectrum}

The continuous term $\mathcal{S}^{\mathrm{cont}}_{\lambda,\eta}$ is subtle: while it does not contribute to the leading asymptotic $\tfrac{\vol(M)}{2\pi}\lambda\eta$, it influences the remainder size. Estimates for logarithmic derivatives of scattering determinants (see Chapter~6, Proposition~6.3.3) guarantee that $\mathcal{S}^{\mathrm{cont}}_{\lambda,\eta}$ is bounded by $O(\lambda^{1-\delta_2})$, uniformly in $\eta$.

Consequently, the continuous spectrum is absorbed into the error term of \eqref{eq:local-weyl-main}. This explains why the identity term on the geometric side of the trace formula matches precisely the main contribution to $N(\lambda,\eta)$, while parabolic corrections affect only the error hierarchy.

\subsection{Refined interpretation}

The discrete and continuous separation clarifies several points:

\begin{enumerate}
  \item For compact hyperbolic surfaces (no cusps), the continuous term is absent. In this case the remainder exponent improves: $\delta=\min(\delta_0,\delta_1)$, with no parabolic degradation.
  \item For finite-area non-compact surfaces, the continuous term is present but harmless, always contributing below the main scale $\lambda\eta$.
  \item The explicit dependence on cusp data enters only through $\delta_2$ in the remainder bound, ensuring that constants remain effective.
\end{enumerate}

\subsection{Geometric–spectral dictionary}

This decomposition provides a clean dictionary between the spectral side and the geometric side of the localized trace formula:
\begin{itemize}
  \item The identity term $\vol(M)\lambda\eta$ corresponds to the main term $\tfrac{\vol(M)}{2\pi}\lambda\eta$ in $N(\lambda,\eta)$.
  \item Hyperbolic terms contribute oscillatory error terms, controlled by exponential decay of $\widehat{\chi}_\eta$ at large lengths.
  \item Parabolic terms yield logarithmic contributions, absorbed in the $O(\lambda^{1-\delta_2})$ bound.
\end{itemize}

This refined understanding ensures that the local Weyl law \eqref{eq:local-weyl-main} is not merely heuristic, but the precise spectral reflection of the trace identity.

% =========================================================
% 08-applications.tex — Block 8.1 (Part 6/8, Expanded)
% Sensitivity to cutoff and short-window bounds
% =========================================================

\subsection{Sensitivity to cutoff choice and robustness}

A natural question is how the local Weyl law \eqref{eq:local-weyl-main} depends on the choice of cutoff $\chi$.  
We emphasize that the framework is robust:

\begin{itemize}
  \item Any even Schwartz function $\chi$ with $\chi(0)=1$ yields the same main term 
  $\tfrac{\vol(M)}{2\pi}\lambda\eta$, since only the low-frequency behavior of $\chi$ enters the identity contribution.
  \item The power-saving error bound $O(\lambda^{1-\delta})$ is preserved, although the precise exponent $\delta_0$ (from stationary phase at small times) may vary slightly with derivatives of $\chi$ near $0$. 
  \item The implicit constants in $O(\cdot)$ depend on finitely many seminorms of $\chi$ and are uniform in $\lambda,\eta$.
\end{itemize}

Thus the structure of Theorem~\ref{thm:localweyl} is insensitive to specific cutoff choices, provided $\chi$ is fixed independently of $\lambda$.

\subsection{Weighted and smoothed versions}

A smoothed variant strengthens the robustness. For $\psi\in\mathcal{S}(\mathbb{R})$ even, supported in $[-2,2]$, and identically $1$ on $[-1,1]$, define
\[
N_\psi(\lambda,\eta)=\sum_j \psi\!\Big(\tfrac{r_j-\lambda}{\eta}\Big),
\]
with the continuous part treated as before. The trace formula argument shows
\[
N_\psi(\lambda,\eta) = \frac{\vol(M)}{2\pi}\lambda\eta \int_{\mathbb{R}}\psi(u)\,du \;+\; O(\lambda^{1-\delta}),
\]
with the same $\delta>0$ as in \eqref{eq:local-weyl-main}.  
The rapid decay of $\widehat{\psi}$ ensures that the remainder bound is preserved.  
This confirms that smoothing at the spectral scale $\eta$ does not affect the main term or the hierarchy of error terms.

\subsection{Dependence on spectral gap and cusp data}

The exponent $\delta$ in \eqref{eq:local-weyl-main} is governed by three sources (cf. refined hierarchy in Chapter~6):
\begin{enumerate}
  \item $\delta_0$ from microlocal stationary phase (Chapter~5).
  \item $\delta_1$ from hyperbolic contributions, controlled by the spectral gap $\beta$.
  \item $\delta_2$ from parabolic contributions, depending on cusp widths and scattering estimates.
\end{enumerate}
Hence $\delta=\min(\delta_0,\delta_1,\delta_2)$.  
The robustness of the cutoff ensures that $\delta$ does not collapse to $0$, provided $\beta>0$ and cusp data are fixed.

\subsection{Uniform short-window upper bounds}

As a corollary of \eqref{eq:local-weyl-main}, for the shortest admissible windows $\eta=\lambda^{-\theta}$ with $0<\theta<\theta_0(\Gamma)$ and any $\varepsilon>0$,
\[
N(\lambda,\eta)\;\ll_{\Gamma,\beta,\varepsilon}\; \lambda^{1-\theta+\varepsilon}.
\]
This matches the size of the main term, $\asymp \lambda^{1-\theta}$, up to $\lambda^\varepsilon$.  
Thus the estimate is sharp in order of magnitude and confirms the predictive strength of the local Weyl law at microscopic spectral scales.

\subsection{Interpretation}

The stability under cutoff choice and the sharpness of short-window bounds demonstrate that the local Weyl law is not an artifact of the smoothing procedure.  
Instead, it reflects the intrinsic spectral distribution on $M$, robust across different analytic representations.  
This robustness is essential for transferring results to variance bounds (Block~8.2) and QUE estimates (Block~8.3), where test functions vary according to the observable under study.

% =========================================================
% 08-applications.tex — Block 8.2 (Part 7/8, Expanded)
% Applications and forward connections
% =========================================================

\subsection{Applications of the Local Weyl Law}

\paragraph{1. Variance bounds.}
The local Weyl law provides a quantitative description of spectral density in windows of width $\eta=\lambda^{-\theta}$, which is crucial for variance estimates of automorphic forms.  
In particular, for a fixed compactly supported test function $f$ on $M$, the variance
\[
\Var_\lambda(f) \;=\; \frac{1}{N(\lambda,\eta)} \sum_{|r_j-\lambda|\leq \eta} \Big|\int_M f(z)\,|\phi_j(z)|^2\,dz - \langle f \rangle \Big|^2
\]
relies on sharp upper and lower bounds for $N(\lambda,\eta)$.  
By Theorem~\ref{thm:localweyl}, this denominator is of order $\lambda \eta$, ensuring that variance decays at a quantifiable rate.  
See Chapter~9 for QUE connections.

\paragraph{2. Quantum ergodicity.}
The semiclassical scaling underlying the local Weyl law guarantees that microlocal measures of eigenfunctions equidistribute in the mean.  
This is a prerequisite for proving Quantum Ergodicity (QE).  
Local Weyl law bounds ensure that subsequences of eigenfunctions with exceptional localization cannot dominate spectral averages.  
Chapter~10 develops this link.

\paragraph{3. Trace formula applications.}
In the Selberg trace formula, the local Weyl law describes the asymptotics of the identity contribution relative to non-trivial conjugacy classes.  
Error control of size $O(\lambda^{1-\delta})$ guarantees that hyperbolic and parabolic terms remain secondary.  
Thus the local Weyl law provides the analytic backbone for remainder estimates in the localized trace formula (Chapter~7).

\paragraph{4. Orbital integrals.}
Orbital integrals require accurate control of spectral projectors $P_{\lambda,\eta}$.  
The parametrix constructed in Chapter~5 and the local Weyl law ensure that their asymptotics are uniform in $\lambda$ and $\eta$.  
This prepares the ground for geometric expansions in Chapter~6.

\paragraph{5. Beyond compact support.}
Although Theorem~\ref{thm:localweyl} is stated for compactly supported smooth cutoffs, robustness ensures that the law extends to weighted averages, e.g.,
\[
\sum_j w(r_j)\,\chi\!\Big(\frac{r_j-\lambda}{\eta}\Big),
\]
with weights $w(r_j)$ of polynomial growth.  
This flexibility is essential for analyzing families of $L$-functions (Chapter~11).

\subsection{Forward Links}

The consequences of Theorem~\ref{thm:localweyl} connect directly to later chapters:

\begin{itemize}
  \item To Chapter~6: The projector kernel estimates feed into the analysis of orbital integrals and geometric terms of the trace formula.
  \item To Chapter~7: Quantified error bounds from the local Weyl law provide the analytic part of the remainder in the trace formula.
  \item To Chapter~9: Variance bounds for matrix elements rely on short-window spectral counts.
  \item To Chapter~10: The mean equidistribution of eigenfunctions (Quantum Ergodicity) depends on local Weyl asymptotics.
  \item To Chapter~11: Weighted forms of the local Weyl law support analytic number theory applications, including spectral reciprocity and moments of $L$-functions.
\end{itemize}

\subsection{Summary of Applications}

The local Weyl law serves as a universal counting tool across microlocal analysis, quantum chaos, and analytic number theory.  
Its robustness under cutoff choices, sharpness in short windows, and quantified error bounds ensure that it can be safely applied in diverse contexts without introducing uncontrolled errors.  
This universality is a key reason why it stands as the central analytic theorem of Part~II of the monograph.

% =========================================================
% 08-audit.tex — Block 8.3 (Part 8/8, Expanded)
% Chapter 8 Audit: Local Weyl Law
% =========================================================

\section*{Chapter 8 Audit: Local Weyl Law}

\subsection*{Goals Recap (G8.1–G8.4)}
\begin{itemize}
  \item[(G8.1)] Construct the microlocal spectral projector $P_{\lambda,\eta}$ and establish uniform kernel bounds.
  \item[(G8.2)] Prove the asymptotic law
  \[
    N(\lambda,\eta) = \frac{\mathrm{vol}(M)}{4\pi}\,\lambda\eta \,+\, O(\lambda^{1-\delta}),
  \]
  valid for $\eta \geq \lambda^{-\theta}$.
  \item[(G8.3)] Control all error terms uniformly in $\lambda$ and $\eta$ using stationary phase and cusp geometry.
  \item[(G8.4)] Connect the local Weyl law to later applications: trace formula, variance bounds, quantum ergodicity, and $L$-functions.
\end{itemize}

\subsection*{Invariants Fixed (I8.1–I8.5)}
\begin{itemize}
  \item[(I8.1)] Volume constant $\mathrm{vol}(M)/(4\pi)$ as the sharp proportionality factor.
  \item[(I8.2)] Localization parameter $\eta$ constrained by $\eta \ge \lambda^{-\theta}$.
  \item[(I8.3)] Error exponent $\delta>0$ determined by spectral gap and cusp geometry.
  \item[(I8.4)] Uniformity of bounds across compact and cusp regions.
  \item[(I8.5)] Stability of asymptotics under smooth cutoff functions $\chi$.
\end{itemize}

\subsection*{Backward Links}
\begin{itemize}
  \item From Chapter~3: Construction of the wave kernel $U(t;z,w)$ and its microlocal properties.
  \item From Chapter~5: Stationary phase analysis of $U(t;z,z)$ on the diagonal.
  \item From Chapter~6: Identity term in the geometric side depends on the local Weyl law as its main term.
\end{itemize}

\subsection*{Forward Links}
\begin{itemize}
  \item To Chapter~6: Error control of geodesic and parabolic contributions is measured against the main Weyl term.
  \item To Chapter~7: Remainder analysis in the localized trace formula relies directly on the local Weyl asymptotics.
  \item To Chapter~9: Variance bounds of eigenfunctions use short-window counts $N(\lambda,\eta)$.
  \item To Chapter~10: Quantum ergodicity arguments rest on local Weyl scaling.
  \item To Chapter~11: Weighted Weyl laws underpin spectral reciprocity and $L$-function moments.
\end{itemize}

\subsection*{Verification of Results (V8.1–V8.5)}
\begin{itemize}
  \item[(V8.1)] Projector kernel constructed explicitly and bounded using Fourier representation.
  \item[(V8.2)] Stationary phase analysis yields the leading term proportional to $\lambda\eta$.
  \item[(V8.3)] Uniform control in cusp regions achieved via Maass–Selberg relations and scattering estimates.
  \item[(V8.4)] Error hierarchy confirmed: remainder bounded by $O(\lambda^{1-\delta})$.
  \item[(V8.5)] Applicability to variance, QE, trace formula, and $L$-functions demonstrated.
\end{itemize}

\subsection*{Error Hierarchy}
\[
  N(\lambda,\eta) \;=\; \frac{\mathrm{vol}(M)}{4\pi}\,\lambda\eta
  \,+\, O(\lambda^{1-\delta}), \qquad \delta>0.
\]
All secondary terms, whether geodesic, parabolic, or cusp corrections, are absorbed into the power-saving remainder.  
The hierarchy ensures that the main term dominates uniformly across admissible $\eta$.

\subsection*{Conclusion}
Chapter~8 has established the local Weyl law with sharp constants and explicit error bounds.  
It proved that the spectral density in localized windows of width $\eta$ grows linearly with $\lambda\eta$, controlled by $\mathrm{vol}(M)/(4\pi)$.  
The chapter’s achievements include:
\begin{itemize}
  \item Anchoring the geometric side of the trace formula with a precise main term.
  \item Providing quantitative tools for variance bounds and equidistribution.
  \item Ensuring robustness of applications across analytic number theory and quantum chaos.
\end{itemize}
With these results, the analytic backbone of the localized trace formula is complete.  
The stage is now set for Chapter~9, which synthesizes geometric and spectral expansions into the final form of the trace formula.
