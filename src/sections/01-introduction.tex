\section{Introduction}\label{sec:intro}

The Selberg trace formula stands as one of the cornerstones of modern spectral geometry
and the analytic theory of automorphic forms. It provides a striking and far–reaching
identity that connects the spectral decomposition of the Laplace–Beltrami operator on a
finite–area hyperbolic surface to the geometry of closed geodesics. Since Selberg’s
pioneering work in the 1950s, the formula has been generalized and refined in many
directions, and it continues to play a central role across mathematics: from analytic
number theory and the distribution of prime geodesics to quantum chaos and microlocal
analysis. 

Yet in its classical form the trace formula is intrinsically \emph{global}. It averages
over the full spectrum and does not, on its own, resolve fine information in
\emph{short spectral windows}. This global nature has historically limited its direct
applicability in problems where one seeks local spectral resolution—such as short–interval
eigenvalue statistics, uniform sup–norm bounds for automorphic eigenfunctions, and
quantitative analysis of small spectral gaps. Overcoming these limitations requires a
new level of refinement.

The present paper develops such a refinement. Our goal is to establish a
\emph{localized trace formula} that isolates the \emph{discrete cuspidal spectrum} of a
finite–area hyperbolic surface
\[
   X \;=\; \Gamma \backslash \HH,
\]
within spectral windows of the form $[R-R^{\theta},\,R+R^{\theta}]$, for
$0<\theta<1$, while at the same time controlling—and in fact suppressing—the contributions
of the continuous spectrum. Our construction blends microlocal analysis with a carefully
engineered cusp truncation at height $y \leq Y = R^\beta$. The outcome is a localized
identity whose error terms are fully \emph{effective}, with constants that depend
polynomially on the geometric invariants of $X$ such as volume, injectivity radius, and
number of cusps. 

This framework yields not only a refined Weyl law in short windows but also a general
microlocal toolkit suitable for delicate spectral problems on noncompact arithmetic
surfaces. It bridges spectral, geometric, and microlocal aspects in a unified analytic
structure.

\paragraph{Main result and novelty.}
We construct a family of microlocal window operators $\TR$, precisely defined in later
sections, that serve as approximate spectral projectors onto windows
$[R-R^{\theta},\,R+R^{\theta}]$ while incorporating a cusp cutoff $y \leq Y = R^\beta$.
The parameters $(\theta,\beta)$ are drawn from an admissible range that balances spectral
localization with cusp suppression. Unlike Gaussian projectors, our operators are
designed to operate effectively on the shrinking scale $R^\theta$, yielding explicit
polynomial control over all constants and ensuring separation from the continuous
spectrum. 

We prove a \emph{windowed Weyl law} with a leading term expressed through an
\emph{effective volume} $\vol_{\mathrm{eff}}(X;R,\theta,\beta)$ adapted to the truncation,
plus contributions from short closed geodesics, and a remainder term that enjoys a
power–saving bound $O(R^{1-\varepsilon(\theta,\beta)})$ with an explicit exponent
$\varepsilon(\theta,\beta)>0$. This represents the first fully effective localized trace
formula in the noncompact setting.

\paragraph{Principal contributions.}
\begin{itemize}
  \item A new localized trace formula isolating the cuspidal spectrum within
        $[R-R^\theta,R+R^\theta]$ under cusp cutoff $y \leq R^\beta$.
  \item Construction of microlocal spectral projectors $\TR$ adapted to short windows,
        preserving near–orthogonality and eliminating continuous spectrum artifacts.
  \item Identification of the \emph{effective volume}
        $\vol_{\mathrm{eff}}(X;R,\theta,\beta)$ and its asymptotics.
  \item A geometric side consisting of explicit amplitudes associated to short closed
        geodesics.
  \item An explicit power–saving remainder term with exponent
        $\varepsilon(\theta,\beta)>0$.
  \item Constants throughout depending polynomially on $\vol(X)$, $\inj(X)$, and cusp
        data, ensuring uniformity across families of arithmetic surfaces.
\end{itemize}

\subsection{Historical context and related work}\label{subsec:history}

The genesis of the trace formula lies in Selberg’s original work
\cite{selberg1956}, which established the spectral–geometric identity for compact
quotients of $\HH$. Hejhal’s two–volume monograph \cite{hejhal1976,hejhal1983} developed
the theory systematically, extending to noncompact finite–volume surfaces and placing the
analytic foundations on firm ground. M\"uller \cite{mueller1983} analyzed the delicate
continuous spectrum via Eisenstein series and scattering theory. On the geometric side,
Buser \cite{buser1992} highlighted the interplay between isoperimetric inequalities,
geometry, and spectral properties.

The microlocal and semiclassical perspective was pioneered by Duistermaat–Guillemin
\cite{duistermaatguillemin1975}, who linked spectral asymptotics to classical dynamics via
the wave trace. This view has been deepened in semiclassical analysis
\cite{zworski2012,dyatlov2019} and in the theory of Fourier integral operators
\cite{hormander1994III}. Chazarain \cite{chazarain1974} provided early microlocal tools.
On the automorphic side, sup–norm bounds of eigenfunctions by Iwaniec–Sarnak
\cite{iwaniec1995} demonstrated the power of spectral methods. 

More recent contributions include localized Weyl laws and sharp spectral asymptotics
\cite{canzani2019}, microlocal methods for hyperbolic dynamics \cite{dyatlov2018}, and
applications to quantum chaos and number theory \cite{sarnak2019}. Nevertheless, a
localized trace formula that unifies spectral localization with cusp suppression and
effective constants has not previously been available.

\subsection{Motivation and central difficulties}\label{subsec:difficulties}

Localizing a global identity like the Selberg trace formula to short spectral windows
raises several challenges:

\begin{enumerate}
  \item \textbf{Continuous spectrum.} Eisenstein series and scattering matrices contribute
        in a subtle way; one must suppress their influence without disturbing the cuspidal
        spectrum. This necessitates truncation by a cutoff $\chi_Y$ that is compatible
        with microlocal propagation.
  \item \textbf{Effective constants.} Many prior results involve constants growing
        exponentially with $\inj(X)^{-1}$ or other geometric parameters. For
        number–theoretic applications, only \emph{polynomially controlled} constants are
        viable.
  \item \textbf{Microlocalization at scale $R^\theta$.} Classical cutoffs like Gaussians
        do not adapt to shrinking windows. A new microlocal projector respecting the
        hyperbolic geometry and cusp structure must be engineered.
\end{enumerate}

Overcoming these obstacles is the central technical innovation of this paper.

\subsection{Overview of the method and key ideas}\label{subsec:method}

Our construction begins with a test function
\[
   h_R(t) = \eta\!\left(\frac{t-R}{R^\theta}\right),
\]
where $\eta$ is an even Schwartz function with $\eta(0)=1$. The inverse spherical
transform of $h_R$ gives a radial kernel $k_R(\rho)$ on $\HH$, localized near geodesic
segments of length $\lesssim R^\theta$ and oscillating at frequency $R$.

To suppress cuspidal effects, we truncate by $\chi_Y(y)$ with $Y=R^\beta$, defining
\[
   K_R^Y(z,w) \;=\; \chi_Y(y_z)\,K_R(z,w)\,\chi_Y(y_w).
\]
Here $K_R$ is the $\Gamma$–automorphic kernel obtained by summing $k_R$ over $\Gamma$.
The associated operator $\TR$ acts on $L^2(X)$ and is shown to approximate a spectral
projector: it preserves cusp eigenfunctions with $t_j \in [R-R^\theta,R+R^\theta]$,
annihilates those outside, and nearly vanishes on Eisenstein series due to cusp
suppression.

The spectral trace $\Tr\,\TR$ can then be evaluated spectrally (via eigenfunctions and
Eisenstein series) and geometrically (via the Selberg expansion). Comparing the two gives
the localized trace formula, with identity, geodesic, and remainder terms analyzed in
detail. The final remainder exponent $\varepsilon(\theta,\beta)$ reflects the balance
between spectral localization, cusp truncation, and geometric bounds.

\subsection{Informal statement of results}\label{subsec:informal}

For $X=\Gamma\backslash\HH$, $R\to\infty$, and admissible parameters $(\theta,\beta)$
with $\varepsilon(\theta,\beta)>0$, the trace satisfies
\[
   \Tr \TR \;=\;
   \vol_{\mathrm{eff}}(X;R,\theta,\beta)\,\frac{C_\eta}{2\pi}\,R^{1+\theta}
   \;+\;
   \sum_{\substack{\gamma \text{ primitive}\\ \ell(\gamma)\ll R^{-\theta}}}
   \mathcal{A}_\gamma(R,\theta)\,e^{iR\ell(\gamma)}
   \;+\;
   O\!\left(R^{1-\varepsilon(\theta,\beta)}\right).
\]

Here $C_\eta=\int_\RR \eta(u)\,du$, $\vol_{\mathrm{eff}}$ accounts for cusp truncation,
and $\mathcal{A}_\gamma$ are explicit amplitudes depending on $\ell(\gamma)$. This leads
to a localized Weyl law
\[
   N_{\mathrm{cusp}}(R,\theta) \;=\; \frac{\vol(X)}{2\pi}\,R^{1+\theta}
   \;+\; O(R^{1-\varepsilon(\theta,\beta)}),
\]
with constants polynomial in the geometry of $X$.

\subsection{Parameter choices and trade–offs}\label{subsec:params}

The parameters $(\theta,\beta)$ quantify spectral vs.~cuspidal localization:

\begin{itemize}
  \item $\theta$ controls window width $R^\theta$. Larger $\theta$ broadens the window,
        improving remainder exponents but reducing resolution. Smaller $\theta$ sharpens
        localization but requires stronger bounds.
  \item $\beta$ controls cusp cutoff $Y=R^\beta$. Larger $\beta$ suppresses Eisenstein
        series more effectively but alters the effective volume; smaller $\beta$ reduces
        truncation error but leaves cusp contributions.
\end{itemize}

The admissible region $\varepsilon(\theta,\beta)>0$ describes the feasible trade–offs.

\subsection{Comparison with Gaussian projectors}\label{subsec:gaussian}

Gaussian projectors $h(t)=e^{-(t-R)^2}$ are fixed–scale and do not adapt to shrinking
windows. They produce error terms poorly suited to arithmetic applications and cannot
suppress cusps. In contrast, our $\TR$ is tailored to $R^\theta$, integrates cusp
truncation from the outset, and provides polynomial control of constants.

\subsection{Applications and outlook}\label{subsec:applications}

The localized trace formula opens several directions:

\begin{itemize}
  \item \textbf{Windowed eigenvalue statistics:} analysis of short–interval spectral
        correlations, pair correlation functions, and comparison with random matrix
        predictions.
  \item \textbf{Quantum ergodicity in windows:} fine–scale equidistribution of cusp forms
        restricted to spectral bands.
  \item \textbf{Sup–norm bounds:} refined estimates $\|\varphi_j\|_\infty\ll R^{1/2-\delta}$
        for eigenfunctions in short windows.
  \item \textbf{Prime geodesic theorems:} short–interval analogues via inversion of the
        geometric side.
  \item \textbf{Extensions:} higher–rank groups, families of congruence surfaces, and
        links to analytic number theory.
\end{itemize}

\subsection{Notation and conventions}\label{subsec:notation}

\begin{itemize}
  \item $\HH=\{x+iy:y>0\}$ with metric $ds^2=(dx^2+dy^2)/y^2$.
  \item $X=\Gamma\backslash\HH$, $\Gamma\subset\PSL(2,\RR)$ discrete of cofinite volume.
  \item $\Delta$ is the Laplace–Beltrami operator, $\lambda_j=1/4+t_j^2$.
  \item $A\ll B$ means $|A|\le CB$ with $C$ polynomial in geometric data.
  \item $\mathcal{S}(\RR)$ denotes the Schwartz class.
  \item Fourier transform: $\widehat{f}(\xi)=\int_\RR f(x)e^{-2\pi i x\xi}\,dx$.
\end{itemize}

\subsection{Organization of the paper}\label{subsec:outline}

Section~\ref{sec:preliminaries} recalls spectral preliminaries.  
Section~\ref{sec:kernel} constructs the truncated kernel $K_R^Y$ and establishes
estimates.  
Section~\ref{sec:projector} develops the projector $\TR$ and its properties.  
Section~\ref{sec:microlocal} provides microlocal analysis and symbol calculus.  
Section~\ref{sec:geometric} establishes the geometric expansion and geodesic terms.  
Section~\ref{sec:results} states and proves the localized trace formula and Weyl law.  
Section~\ref{sec:conclusion} discusses extensions and applications.  
Appendix~A computes the effective volume; Appendix~B collects technical estimates.

