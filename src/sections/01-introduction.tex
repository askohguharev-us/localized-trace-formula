\section{Introduction}\label{sec:intro}

The Selberg trace formula is a cornerstone of modern spectral geometry and the theory
of automorphic forms. It relates the length spectrum of closed geodesics on a hyperbolic
surface to the spectral data of the Laplace--Beltrami operator. In its classical form,
however, the trace formula is \emph{global}: it averages over the full spectrum and
does not by itself resolve fine information in \emph{short spectral windows}. This
global character has traditionally limited its direct applicability to questions where
local spectral resolution is essential, e.g. short--interval eigenvalue statistics,
uniform bounds on automorphic eigenfunctions, and the quantitative analysis of spectral
gaps.

The goal of this paper is to develop a \emph{localized trace formula} that isolates the
\emph{discrete cuspidal spectrum} of a finite--area hyperbolic surface
\[
  X \;=\; \Gamma\backslash\HH
\]
inside short windows of the form $[R-R^\theta,\,R+R^\theta]$ with $0<\theta<1$, while at
the same time controlling---and in fact suppressing---contributions of the continuous
spectrum. Our construction combines microlocal analysis with a carefully designed height
cutoff near the cusp. This allows us to avoid spurious continuous--spectrum effects and
to obtain \emph{effective}, polynomially controlled constants throughout. The framework
yields a refined, windowed Weyl law and, more broadly, a versatile microlocal toolkit for
spectral analysis on noncompact hyperbolic surfaces.

\paragraph{Main result and novelty.}
We construct a family of microlocal window operators $\TR$ (defined precisely later) that
selects the spectral window $[R-R^\theta,\,R+R^\theta]$ while enforcing a cusp height
cutoff $y\le Y=R^\beta$, with parameters $0<\theta<1$ and $\beta\ge 0$. Unlike
\emph{Gaussian projector} approaches, our design is flexible and yields \emph{polynomial}
control of all geometric constants (in terms of volume, injectivity radius, and features
of the length spectrum), and it systematically avoids continuous--spectrum contamination.
In particular, we establish a \emph{windowed Weyl law} whose constants depend at most
polynomially on the geometric data of $X$. The identity term is expressed through an
\emph{effective volume} compatible with the cusp truncation, the geometric side collects
contributions of short closed geodesics, and the remainder admits a power saving
$O(R^{1-\varepsilon(\theta,\beta)})$ with $\varepsilon(\theta,\beta)>0$ explicit.

\paragraph{Contributions.}
\begin{itemize}
  \item A localized trace formula that isolates the \emph{discrete cuspidal spectrum} in
        the short window $[R-R^\theta,\,R+R^\theta]$ under a height cutoff $y\le R^\beta$.
  \item An identity term given by the \emph{effective volume} $\mathrm{vol}_{\mathrm{eff}}(X;R,\theta,\beta)$,
        compatible with truncation and stable under the parameter ranges we consider.
  \item A geometric term that sums contributions of short closed geodesics with amplitudes
        depending explicitly on $(\theta,\beta)$ and the chosen microlocal window.
  \item A power--saving remainder $O(R^{1-\varepsilon(\theta,\beta)})$ with an
        \emph{explicit} error exponent $\varepsilon(\theta,\beta)>0$.
  \item A \emph{windowed Weyl law} with constants that depend at most polynomially on the
        geometric data of $X$ (volume, injectivity radius, cusp parameters).
\end{itemize}

\subsection{Historical context and related work}\label{subsec:history}
The original ideas go back to Selberg \cite{selberg1956} and the systematic exposition in
Hejhal \cite{hejhal1976,hejhal1983}. On finite--volume, noncompact surfaces the analytic
foundations of the spectral side (including Eisenstein series and scattering theory) were
clarified in work such as M\"uller \cite{mueller1983}. The geometric control afforded by
Buser \cite{buser1992} remains fundamental.

On the microlocal and semiclassical side, the relation between spectra and classical
dynamics was developed by Duistermaat--Guillemin \cite{duistermaatguillemin1975}, and
has been refined in the semiclassical framework, see e.g.
\cite{zworski2012,dyatlovzworski2019}. Classical tools relevant for our analysis include
pseudodifferential and Fourier integral techniques \cite{hormander1994III,sogge1993},
together with ideas going back to Chazarain \cite{chazarain1974}. For automorphic forms
and sup--norm/eigenfunction issues see Iwaniec--Sarnak \cite{iwaniec1995}. Recent works
that touch the themes of short windows, cusp effects, or remainder bounds include
\cite{canzanigalkowski2019,deleporte2024,gansemer2024,le_masson2024,zhuwu2024}; see also
developments connected to hyperbolic dynamics and resonances
\cite{dyatlov2018,dyatlov2019,danetel2018}.

What distinguishes the present paper from Gaussian--projector based localizations is a
\emph{microlocal window} $\TR$ engineered to respect the underlying geometry and the cusp
structure, thereby allowing explicit polynomial control of constants and a clean
separation from the continuous spectrum.

\subsection{Motivation and difficulties}\label{subsec:difficulties}
Localizing a global trace identity in short spectral windows presents several obstacles:

\begin{enumerate}
  \item \textbf{Continuous spectrum.} In the finite--volume case, Eisenstein series
        contribute in a delicate way; truncation must be arranged so that the continuous
        spectrum is suppressed without disturbing the discrete cuspidal part. This
        requires a height cutoff compatible with microlocal propagation near the cusp.
  \item \textbf{Effective constants.} Many arguments produce only implicit or exponential
        bounds for remainder constants. For quantitative applications (e.g., sup--norm
        bounds or short--interval counting) one needs constants with \emph{polynomial}
        dependence on geometric data (volume, injectivity radius, cusp parameters).
  \item \textbf{Microlocalization at scale $R^\theta$.} Constructing a projector that
        resolves $[R-R^\theta,R+R^\theta]$ while preserving orthogonality and avoiding
        smearing across scales is subtle. Na\"ive spectral cutoffs do not suffice; one
        needs a microlocally adapted operator whose kernel exhibits the correct
        oscillations and decay.
\end{enumerate}

\subsection{Overview of the method}\label{subsec:method}
We sketch the main ideas. The window operator $\TR$ is constructed from a compactly
supported even function $f$ with $\widehat{f}$ supported at the geodesic scale, combined
with a microlocal cutoff that isolates frequencies near $R$ within width $R^\theta$. The
resulting operator may be viewed as a carefully engineered function of $\sqrt{\Lap}$,
with additional microlocalization in phase space. The height truncation $y\le Y=R^\beta$
is then incorporated to eliminate contributions associated with the continuous spectrum.

On the \emph{identity term}, the kernel of $\TR$ is analyzed on the diagonal and the
height cutoff leads to an \emph{effective volume} $\vol_{\mathrm{eff}}(X;R,\theta,\beta)$.
This is stable across the parameter regime and matches the expected leading order of a
windowed Weyl law. On the \emph{geometric side}, the contribution of a closed geodesic
$\gamma$ of length $\ell(\gamma)$ has an amplitude
\[
  \frac{\ell(\gamma_0)}{2\sinh(\ell(\gamma)/2)}\,\widehat{f}(\ell(\gamma))\,R^\theta,
\]
with $\gamma_0$ the primitive component. This comes from stationary phase analysis of the
off--diagonal kernel and the usual relation between propagation of singularities and the
geodesic flow.

The \emph{remainder} is controlled by combining microlocal estimates with non--stationary
phase and hyperbolic dynamics. The exponent $\varepsilon(\theta,\beta)>0$ is explicit and
reflects the choices of window and height parameters. While we do not rely on the fractal
uncertainty principle in the present paper, the viewpoint of
\cite{dyatlov2018,dyatlov2019,danetel2018} informs our understanding of how localization
interacts with hyperbolic dispersion.

\subsection{Informal statement}
We summarize the outcome informally; a precise formulation appears in
Section~\ref{sec:results}. For $R\to\infty$ and $0<\theta<1$, one has
\[
  \Tr\,\TR \;=\;
  \vol_{\mathrm{eff}}(X;R,\theta,\beta) \;+\;
  \sum_{\gamma}\mathcal{A}_\gamma(R,\theta) \;+\;
  O\!\left(R^{1-\varepsilon(\theta,\beta)}\right),
\]
where the sum runs over closed geodesics, $\mathcal{A}_\gamma$ is as above, and the
remainder exponent $\varepsilon(\theta,\beta)>0$ is explicit. This yields a
\emph{windowed Weyl law} with polynomially controlled constants.

\subsection{Parameter choices and flexibility}\label{subsec:params}
The parameters $(\theta,\beta)$ quantify two complementary aspects of localization:
$\theta$ controls the spectral window width $R^\theta$, while $\beta$ controls the cusp
truncation scale $Y=R^\beta$. Larger $\theta$ yields stronger concentration in frequency,
which sharpens the weight of short geodesics but increases the sensitivity to propagation
effects; increasing $\beta$ suppresses more of the cusp but can alter the effective
volume. Our analysis tracks these effects quantitatively, resulting in explicit trade--off
curves visible in the final error exponent $\varepsilon(\theta,\beta)$.

\subsection{Comparison with Gaussian projectors}\label{subsec:gaussian}
Gaussian projectors, while convenient, introduce tails that complicate the separation of
discrete and continuous spectra and often obscure the dependence of constants on
geometric data. Our microlocal projector $\TR$ is \emph{compactly supported in frequency}
at the relevant scale and incorporates a phase--space cutoff adapted to the hyperbolic
flow; this makes continuous--spectrum suppression transparent and keeps constants under
polynomial control. In particular, the identity term can be written directly in terms of
$\vol_{\mathrm{eff}}$, and the geometric term is localized both in frequency and in phase
space, which is crucial for obtaining power savings.

\subsection{Applications and outlook}\label{subsec:applications}
The localized trace formula developed here provides a flexible tool for several
directions:
\begin{itemize}
  \item \emph{Windowed eigenvalue counting and statistics.} One obtains refined Weyl laws
        in $[R-R^\theta,R+R^\theta]$, opening the door to local statistics and short--range
        correlations.
  \item \emph{Sup--norm bounds and quantum ergodicity in windows.} The microlocal
        projectors $\TR$ can be used to restrict analysis to frequency bands, yielding
        sharper control on eigenfunctions and Eisenstein series.
  \item \emph{Number--theoretic interfaces.} The explicit geometric term and polynomially
        controlled constants can be leveraged in short--interval versions of the prime
        geodesic theorem.
  \item \emph{Dynamics and chaos.} The construction interacts naturally with hyperbolic
        dynamics, creating a platform to test predictions from quantum chaos and to
        compare with semiclassical frameworks (cf.~\cite{zworski2012,dyatlovzworski2019}).
\end{itemize}

\subsection{Notation}\label{subsec:notation}
We write $\Lap$ for the nonnegative Laplace--Beltrami operator on $X$, $\vol(\cdot)$ for
hyperbolic area, and $A\lesssim B$ if $A\le C\,B$ for a constant $C$ depending at most
polynomially on the geometric data of $X$. The localized operator is denoted by $\TR$.
We use $\injrad(x)$ for the injectivity radius at $x\in X$ and simply $\injrad(X)$ for a
global lower bound when available. Primitive closed geodesics are denoted by $\gamma_0$
and have length $\ell(\gamma_0)$; a general closed geodesic is a power of a primitive
one. We use standard big--Oh and Vinogradov notation with constants controlled as above.

\paragraph{Bibliographic conventions.}
We follow the AMS style for references. When available we include DOIs and arXiv
identifiers in the form ``arXiv:YYMM.NNNNv\#''. Classical background includes
\cite{selberg1956,hejhal1976,hejhal1983,mueller1983,buser1992,duistermaatguillemin1975,
hormander1994III,sogge1993,chazarain1974}, and more recent perspectives and applications
appear in \cite{iwaniec1995,canzanigalkowski2019,deleporte2024,gansemer2024,
le_masson2024,zhuwu2024,dyatlov2018,dyatlov2019,dyatlovzworski2019,zworski2012}.

\subsection{Organization of the paper}\label{subsec:outline}
Section~\ref{sec:preliminaries} recalls preliminaries and sets the notation.
Section~\ref{sec:kernel} constructs and analyzes the kernel of the localized operator.
Section~\ref{sec:projector} builds the spectral projector and records mapping properties.
Section~\ref{sec:microlocal} develops the microlocal construction and the cusp cutoff.
Section~\ref{sec:geometric} evaluates the geometric contributions from closed geodesics.
Section~\ref{sec:results} states the final localized trace formula and the windowed Weyl
law, with explicit constants and error terms. Section~\ref{sec:conclusion} discusses
extensions and open directions. Appendix~A contains the computation of the
effective volume; Appendix~B collects auxiliary estimates needed in the proofs.
