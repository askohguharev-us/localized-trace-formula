\section{Introduction}\label{sec:intro}

The Selberg trace formula is one of the central tools of modern spectral geometry and the analytic theory of automorphic forms. It provides an identity equating two fundamental aspects of the geometry of hyperbolic surfaces: on the one hand, the spectral data of the Laplace--Beltrami operator, and on the other hand, the length spectrum of closed geodesics. This duality has inspired entire branches of mathematics, from number theory and quantum chaos to spectral geometry and microlocal analysis.

In its classical formulation, however, the Selberg trace formula is inherently global. It involves the entire discrete and continuous spectrum of a finite-area hyperbolic surface
\[
  X = \Gamma \backslash \HH,
\]
and averages over long spectral intervals. While extremely powerful, this global nature means that the classical trace formula does not directly resolve fine spectral information such as the distribution of eigenvalues in short intervals, local spectral statistics, or the microscopic behaviour of eigenfunctions. For many applications in analytic number theory and quantum chaos, what is required is not only global asymptotics but also \emph{localized} information on the spectrum.

The present paper develops a localized version of the Selberg trace formula that isolates the discrete cuspidal spectrum in spectral windows of the form
\[
  [R - R^\theta,\; R + R^\theta], \qquad 0 < \theta < 1,
\]
while simultaneously suppressing the contribution of the continuous spectrum through an explicit cusp cutoff. This leads to a localized trace identity whose constants depend at most polynomially on the geometric data of $X$ (volume, injectivity radius, cusp parameters), and from which we deduce a short-window Weyl law with power-saving remainder. Beyond this, the construction provides a new microlocal operator---a family of window projectors $\TR$---that can be used as a versatile tool in spectral geometry, number theory, and quantum chaos.

\paragraph{Main result and novelty.}
The core novelty of this work is the design of a microlocal spectral projector $\TR$ adapted to short windows of width $R^\theta$, combined with a cusp truncation at height $Y=R^\beta$. Unlike previous approaches based on Gaussian projectors, our construction yields:
\begin{itemize}
  \item Polynomial dependence of all constants on the geometric data of $X$;
  \item Compatibility with cusp truncation, ensuring that continuous-spectrum contributions are eliminated in a controlled fashion;
  \item A localized Weyl law with an explicit error exponent $\varepsilon(\theta,\beta) > 0$;
  \item Microlocal concentration in phase space, aligning the spectral window with the underlying geodesic dynamics.
\end{itemize}
This framework provides a bridge between the global Selberg trace formula and the fine-scale spectral analysis required in arithmetic and dynamical applications.

\paragraph{Principal contributions.}
The detailed contributions of this paper are as follows:
\begin{enumerate}
  \item Construction of the first microlocal \emph{localized trace formula} for finite-area hyperbolic surfaces, isolating the cuspidal spectrum within a short window $[R-R^\theta, R+R^\theta]$ under a cusp cutoff $y \le R^\beta$.
  \item Definition and analysis of microlocal spectral projectors $\TR$ that act almost idempotently on the cuspidal spectrum, annihilate the continuous spectrum, and satisfy super-polynomial orthogonality across disjoint windows.
  \item Computation of the identity term in terms of an \emph{effective volume} $\vol_{\mathrm{eff}}(X;R,\theta,\beta)$, stable under parameter variation and compatible with truncation.
  \item Evaluation of the geometric side, summing contributions of short closed geodesics with amplitudes explicitly expressed in terms of the window function and geodesic length.
  \item Establishment of a \emph{windowed Weyl law} with power-saving error:
  \[
     N(R,\theta) = \frac{\vol(X)}{2\pi} R^{1+\theta} + O\!\left(R^{1-\varepsilon(\theta,\beta)}\right),
  \]
  valid for an explicit range of $(\theta,\beta)$.
  \item Applications to eigenvalue statistics, sup-norm bounds, quantum ergodicity in windows, and spectral statistics, all with polynomial control of constants across families of surfaces.
\end{enumerate}

% ---------------------------
\subsection{Historical context and related work}\label{subsec:history}

The origins of the trace formula trace back to Selberg~\cite{Selberg1956}, who introduced it in the mid-20th century as a far-reaching generalization of the Poisson summation formula to the non-Euclidean setting. His pioneering work revealed the profound connection between the spectral and geometric sides of hyperbolic surfaces.

Hejhal’s monographs~\cite{Hejhal1976,Hejhal1983} provided the first systematic treatment of the Selberg trace formula, both in compact and finite-volume cases, laying the analytic foundations that continue to underpin the field. M\"uller~\cite{Mueller1983} addressed the delicate role of the continuous spectrum and Eisenstein series, establishing spectral theory for manifolds with cusps. Buser’s work~\cite{Buser1992} developed the geometric control and inequalities necessary for applications in spectral geometry.

Parallel developments in microlocal and semiclassical analysis reshaped the understanding of trace formulas. Duistermaat and Guillemin~\cite{DuistermaatGuillemin1975} analyzed singularities of the wave trace, revealing the connection between periodic geodesics and spectral asymptotics. Chazarain~\cite{Chazarain1974} studied the Poisson formula for Riemannian manifolds. Hörmander’s theory of pseudodifferential and Fourier integral operators~\cite{Hormander1994III} provided the analytic machinery that now forms the backbone of microlocal spectral analysis. These works collectively established the principle that spectral data encodes—and is constrained by—the underlying classical dynamics.

In the arithmetic setting, Iwaniec and Sarnak~\cite{IwaniecSarnak1995} applied trace formula techniques to sup-norm bounds for eigenfunctions, demonstrating the reach of spectral methods in analytic number theory. More recently, Canzani and Galkowski~\cite{CanzaniGalkowski2019} proved sharp Weyl laws on manifolds with boundary; Deleporte~\cite{Deleporte2024} investigated spectral estimates on hyperbolic surfaces with cusp effects; and Gansemer~\cite{Gansemer2024a,Gansemer2024b} developed improvements in remainder terms via refined spectral windows.

The interplay between spectral geometry and quantum chaos has been advanced by Dyatlov, Zworski, and collaborators~\cite{Dyatlov2018,DyatlovZworski2019,DyatlovZworski2019b}, exploring fractal uncertainty principles and microlocal properties of resonances. Surveys such as Zelditch~\cite{Zelditch2009} provide broader context. Recent works by Le Masson~\cite{LeMasson2024}, Zhu--Wu--Zhang~\cite{ZhuWuZhang2024}, and Anantharaman~\cite{Anantharaman2024} emphasize the importance of localized spectral tools for hyperbolic dynamics and eigenfunction statistics.

What distinguishes the present paper is the explicit localization of the trace formula in both spectral and geometric variables, with constants controlled polynomially in the invariants of $X$. This makes the formula robust across families and suitable for quantitative applications.

% ---------------------------
\subsection{Motivation and central difficulties}\label{subsec:difficulties}

Localizing a global identity such as the Selberg trace formula presents a series of fundamental challenges:

\begin{enumerate}
  \item \textbf{Continuous spectrum.} On noncompact surfaces, Eisenstein series contribute delicately to the spectral side. Their analytic continuation and scattering matrices are subtle, and truncating them without disrupting the cuspidal spectrum requires careful design.
  \item \textbf{Effective constants.} Many spectral asymptotic results provide remainders with implicit constants or with exponential dependence on geometric parameters. For arithmetic applications, uniformity across families requires explicit polynomial control.
  \item \textbf{Microlocalization at scale $R^\theta$.} The spectral window $[R-R^\theta,R+R^\theta]$ shrinks with $R$, demanding projectors adapted to fine scales. Gaussian functions fail here: their fixed width is incompatible with shrinking windows, and they cannot suppress cuspidal contributions cleanly.
\end{enumerate}

Overcoming these difficulties forms the backbone of our approach. Our construction of $\TR$ solves each in turn: it suppresses continuous spectrum via cusp truncation; it ensures polynomial control of constants; and it achieves microlocal localization adapted to hyperbolic dynamics.

% ---------------------------
\subsection{Overview of the method}\label{subsec:method}

The fundamental object is the microlocal projector $\TR$, constructed from a test function
\[
   h_R(t) = \eta\!\left(\frac{t-R}{R^\theta}\right),
\]
with $\eta \in \mathcal{S}(\RR)$ even, real-valued, and $\eta(0)=1$. Its spherical transform defines a radial kernel on $\HH$, which is then automorphized to $X$ and truncated at cusp height $Y=R^\beta$.

This kernel $K_R^Y$ exhibits oscillatory behaviour aligned with the geodesic flow, with phase $\sim e^{iR\rho}$ and spatial cutoff at scale $\rho \lesssim R^\theta$. The cusp truncation suppresses Eisenstein contributions with $L^2$-mass $\ll R^{-\beta/2+\epsilon}$. Acting by convolution with $K_R^Y$, the operator $\TR$ multiplies cusp eigenfunctions with $t_j \approx R$ by $h_R(t_j)$ and annihilates others, while suppressing the continuous spectrum.

The trace $\Tr \TR$ admits both spectral and geometric expansions. Spectrally, it counts eigenvalues in the window up to negligible errors. Geometrically, it decomposes into an effective volume term and contributions from short closed geodesics, with amplitudes derived from stationary phase. Comparing the two sides yields the localized trace formula.

The error terms come from four sources: tails of $h_R$, long geodesics, cusp truncation, and projector approximation. Balancing these yields the explicit error exponent
\[
   \varepsilon(\theta,\beta) = \min\Big\{ \theta,\, 1-\theta+\beta,\, \tfrac{1}{2},\, 1-2\theta+\beta \Big\} - \delta,
\]
for any $\delta > 0$.

% ---------------------------
\subsection{Informal statement of results}\label{subsec:informal}

For a finite-area hyperbolic surface $X = \Gamma\backslash \HH$, parameters $0<\theta<1$, $\beta \ge 0$, and $R \to \infty$, we obtain
\[
   \Tr \TR = \frac{\vol(X)}{2\pi} C_\eta R^{1+\theta} + \sum_{\gamma \;\text{primitive},\;\ell(\gamma)\ll R^{-\theta}} 
      \frac{\ell(\gamma_0)}{2\sinh(\ell(\gamma)/2)} R^\theta \widehat{\eta}(\ell(\gamma) R^\theta) e^{iR\ell(\gamma)}
      + O(R^{1-\varepsilon(\theta,\beta)}),
\]
where $C_\eta = \int_\RR \eta(u)\,du$ and $\ell(\gamma_0)$ is the primitive length. The error exponent $\varepsilon(\theta,\beta)>0$ is explicit, as above.

From this identity follows the \emph{windowed Weyl law}:
\[
   N(R,\theta) = \#\{ j : |t_j-R|\le R^\theta\} 
      = \frac{\vol(X)}{2\pi} R^{1+\theta} + O(R^{1-\varepsilon(\theta,\beta)}).
\]

% ---------------------------
\subsection{Parameter interpretation}\label{subsec:params}

The parameters $(\theta,\beta)$ encode the trade-off between spectral resolution and cusp truncation:
\begin{itemize}
  \item Smaller $\theta$ yields finer resolution but tighter error terms; larger $\theta$ widens the window.
  \item Larger $\beta$ more strongly suppresses the continuous spectrum but reduces effective volume.
\end{itemize}
The admissible region $\varepsilon(\theta,\beta)>0$ is nonempty and includes natural choices such as $\theta=1/2-\epsilon$, $\beta=1/2$.

% ---------------------------
\subsection{Comparison with Gaussian projectors}\label{subsec:gaussian}

Gaussian projectors use $h(t)=e^{-(t-R)^2}$, whose fixed width cannot shrink with $R$, and whose exponential tails are poorly suited for cusp suppression. They yield constants depending exponentially on geometry and fail to handle noncompactness. Our microlocal $\TR$ avoids these pitfalls by being scale-adapted and cusp-compatible.

% ---------------------------
\subsection{Applications and future directions}\label{subsec:applications}

The localized trace formula has broad applications:
\begin{itemize}
  \item Short-window eigenvalue statistics and pair correlation;
  \item Sup-norm bounds via amplification within windows;
  \item Quantum ergodicity and variance bounds on fine scales;
  \item Arithmetic applications such as short-interval prime geodesic theorems;
  \item Extensions to higher-rank symmetric spaces and arithmetic manifolds.
\end{itemize}

% ---------------------------
\subsection{Notation and conventions}\label{subsec:notation}

We use the following conventions:
\begin{itemize}
  \item $\HH = \{x+iy:y>0\}$ with metric $ds^2 = (dx^2+dy^2)/y^2$;
  \item $X=\Gamma\backslash \HH$, finite-area, torsion-free;
  \item $\Delta$ Laplace--Beltrami operator, eigenvalues $\lambda_j=1/4+t_j^2$;
  \item $\TR$ localized projector, $A\lesssim B$ meaning $A \le CB$ with $C$ polynomial in $\vol(X),\inj(X)^{-1}$, cusp data;
  \item Fourier transform $\widehat{f}(\xi)=\int_\RR f(x) e^{-2\pi i x\xi}\,dx$.
\end{itemize}

% ---------------------------
\subsection{Organization of the paper}\label{subsec:outline}

Section~\ref{sec:preliminaries} recalls background on hyperbolic surfaces and spectral theory.  
Section~\ref{sec:kernel} constructs the kernel $K_R^Y$ and establishes its properties.  
Section~\ref{sec:projector} defines $\TR$ and proves its near-idempotence and orthogonality.  
Section~\ref{sec:microlocal} analyzes its microlocal structure and phase space concentration.  
Section~\ref{sec:geometric} evaluates geometric contributions.  
Section~\ref{sec:results} synthesizes into the localized trace formula and windowed Weyl law.  
Section~\ref{sec:conclusion} discusses outlook.  
Appendix~A contains effective volume computations; Appendix~B provides technical estimates.

