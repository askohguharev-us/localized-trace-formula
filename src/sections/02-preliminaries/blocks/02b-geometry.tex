% --- 02b-geometry (Part 1/4): Hyperbolic plane, isometries, geodesics, volumes

\noindent
We recall and fix the geometric framework for the hyperbolic plane $\mathbb{H}$
and its quotients $M=\Gamma\backslash\mathbb{H}$, with normalizations that are used
throughout the monograph.

\medskip

\noindent\textbf{Hyperbolic plane and metric.}
Let
\[
  \mathbb{H}=\{z=x+iy\in\mathbb{C}: y>0\}
\]
with hyperbolic metric
\[
  ds^{2}=\frac{dx^{2}+dy^{2}}{y^{2}}, \qquad
  d\mu(z)=\frac{dx\,dy}{y^{2}}.
\]
For $z,w\in\mathbb{H}$ the distance is determined by
\[
  \cosh d(z,w)=1+\frac{|z-w|^{2}}{2\,\Im z\,\Im w}.
\]
The Riemannian volume of a measurable set $\Omega\subset\mathbb{H}$ is
$\vol(\Omega)=\int_{\Omega} d\mu$.

\medskip

\noindent\textbf{Isometries.}
The group $PSL_{2}(\mathbb{R})$ acts by fractional linear transformations
$\gamma z=\dfrac{az+b}{cz+d}$ with $\gamma=\bigl(\begin{smallmatrix}a&b\\ c&d\end{smallmatrix}\bigr)$
and preserves $ds^{2}$ and $d\mu$.
Elements are classified by $|\tr \gamma|$:
elliptic ($<2$), parabolic ($=2$), hyperbolic ($>2$).
Two hyperbolic elements are conjugate iff they have the same translation length,
defined below.

\medskip

\noindent\textbf{Geodesics.}
Geodesics in $\mathbb{H}$ are vertical lines and semicircles orthogonal to $\mathbb{R}$.
Let $\gamma\in PSL_{2}(\mathbb{R})$ be hyperbolic.
Its axis is the unique geodesic stabilized by $\gamma$.
The \emph{translation length} $\ell(\gamma)>0$ satisfies
\[
  2\cosh\!\left(\frac{\ell(\gamma)}{2}\right)=|\tr(\gamma)|.
\]
If $\gamma=\gamma_{0}^{k}$ with $\gamma_{0}$ primitive, then
$\ell(\gamma)=k\,\ell(\gamma_{0})$.

\medskip

\noindent\textbf{Balls and volume growth.}
The ball of radius $R$ has area
\[
  \vol(B(R))=2\pi(\cosh R-1).
\]
Hence $\vol(B(R))\sim \pi e^{R}$ as $R\to\infty$.
This exponential growth permeates all orbit and lattice point estimates.

\medskip

\noindent\textbf{Fundamental and Dirichlet domains.}
Let $\Gamma\subset PSL_{2}(\mathbb{R})$ be a discrete group of finite covolume.
A fundamental domain $\mathcal{F}$ for $\Gamma$ has finitely many geodesic sides,
paired by elements of $\Gamma$.
For a basepoint $z_{0}\in\mathbb{H}$ the Dirichlet domain is
\[
  \mathcal{D}(z_{0})=\{z\in\mathbb{H}: d(z,z_{0})\le d(z,\gamma z_{0})\ \forall \gamma\in\Gamma\}.
\]
It is convex, locally finite, and $\vol(\Gamma\backslash\mathbb{H})=\int_{\mathcal{D}(z_{0})} d\mu$.

\medskip

\noindent\textbf{Lattice point estimate (hyperbolic).}
Let $N_{\Gamma}(R;z_{0})=\#\{\gamma\in\Gamma: d(z_{0},\gamma z_{0})\le R\}$.
Then
\[
  N_{\Gamma}(R;z_{0})=\frac{\vol(B(R))}{\vol(\Gamma\backslash\mathbb{H})}+O_{\Gamma}\!\left(\frac{e^{R}}{R^{2}}\right),
\]
uniformly for $R\to\infty$. The implied constant depends only on $\Gamma$.

\medskip

\noindent\textbf{Separation lemma (Selberg).}
There exists $\epsilon_{\Gamma}>0$ such that
$d(\gamma\mathcal{F},\gamma'\mathcal{F})\ge \epsilon_{\Gamma}$ for distinct $\gamma,\gamma'\in\Gamma$.
This quantitative separation underlies the absolute convergence of orbital sums
for compactly supported or sufficiently decaying kernels.

\medskip

\noindent\textbf{Closed geodesics in the quotient.}
Let $M=\Gamma\backslash\mathbb{H}$. Conjugacy classes of primitive hyperbolic elements
$\{\gamma_{0}\}$ correspond to primitive closed geodesics $\{\mathcal{C}_{\gamma_{0}}\}$ on $M$,
with $\ell(\mathcal{C}_{\gamma_{0}})=\ell(\gamma_{0})$.
The number $\pi_{M}(R)$ of primitive closed geodesics with $\ell\le R$ satisfies
\[
  \pi_{M}(R)\sim \frac{e^{R}}{R}\qquad (R\to\infty).
\]
This “prime geodesic theorem’’ will be used later to bound hyperbolic orbital sums.

\medskip

\noindent\textbf{Green’s function (model).}
The resolvent kernel $G_{s}(z,w)$ on $\mathbb{H}$ solves
\[
  (\Delta_{z}-s(1-s))G_{s}(z,w)=-\delta_{w}(z),
\]
with explicit expression via the hypergeometric function of $\cosh d(z,w)$. On quotients,
$G_{s}^{\Gamma}(z,w)=\sum_{\gamma\in\Gamma} G_{s}(z,\gamma w)$ (in the sense of meromorphic continuation).
We record this for later spectral transforms; details are standard and omitted here.

% --- 02b-geometry (Part 2/4): Cusps, injectivity, thick-thin, collars

\noindent\textbf{Cusps and scaling matrices.}
Let $\mathfrak{a}$ be a cusp (a $\Gamma$-orbit of a parabolic fixed point on $\partial\mathbb{H}$).
There exists a scaling matrix $\sigma_{\mathfrak{a}}\in PSL_{2}(\mathbb{R})$ such that
$\sigma_{\mathfrak{a}}(\infty)=\mathfrak{a}$ and
\[
  \sigma_{\mathfrak{a}}^{-1}\Gamma_{\mathfrak{a}}\sigma_{\mathfrak{a}}
  =
  \Bigl\langle
  \begin{pmatrix}1&w_{\mathfrak{a}}\\ 0&1\end{pmatrix}
  \Bigr\rangle,
\]
where $w_{\mathfrak{a}}>0$ is the \emph{cusp width}.
The matrix $\sigma_{\mathfrak{a}}$ is unique up to left multiplication by the unipotent subgroup
$N=\{\bigl(\begin{smallmatrix}1&t\\0&1\end{smallmatrix}\bigr): t\in\mathbb{R}\}$ and right multiplication by $\{\pm I\}$.
Set the standard strip $\mathcal{S}(w)=\{x+iy\in\mathbb{H}: 0\le x<w,\ y>0\}$ and
the cusp region $C_{\mathfrak{a}}=\sigma_{\mathfrak{a}}(\mathcal{S}(w_{\mathfrak{a}}))$.

\medskip

\noindent\textbf{Height truncation.}
For $Y>1$ let
\[
  C_{\mathfrak{a}}(Y)=\sigma_{\mathfrak{a}}\{x+iy\in\mathbb{H}: 0\le x<w_{\mathfrak{a}},\ y\ge Y\},
  \qquad
  M(Y)=M\setminus\bigcup_{\mathfrak{a}}\pi(C_{\mathfrak{a}}(Y)).
\]
For $Y$ large enough the sets $\pi(C_{\mathfrak{a}}(Y))$ are disjoint and $M(Y)$ is compact with geodesic boundary.

\medskip

\noindent\textbf{Exact volume and boundary length.}
With $d\mu=dx\,dy/y^{2}$ one computes
\[
  \vol\bigl(\pi(C_{\mathfrak{a}})\setminus \pi(C_{\mathfrak{a}}(Y))\bigr)=\frac{w_{\mathfrak{a}}}{Y},
  \qquad
  \operatorname{length}\bigl(\partial \pi(C_{\mathfrak{a}}(Y))\bigr)=\frac{w_{\mathfrak{a}}}{Y}.
\]
Summing over cusps:
\[
  \vol(M\setminus M(Y))=\sum_{\mathfrak{a}}\frac{w_{\mathfrak{a}}}{Y},
  \qquad
  \operatorname{length}(\partial M(Y))=\sum_{\mathfrak{a}}\frac{w_{\mathfrak{a}}}{Y}.
\]

\medskip

\begin{lemma}\label{lem:inj-cusp}
There exists $c>0$ depending only on $\min_{\mathfrak{a}} w_{\mathfrak{a}}$ such that
for all $z=\sigma_{\mathfrak{a}}(x+iy)$ with $y\ge Y$ one has
\[
  \inj(z)\ \ge\ c\cdot \min\{1, Y^{-1}\}.
\]
\end{lemma}

\begin{proof}
In cusp coordinates the shortest nontrivial displacement is horizontal by $w_{\mathfrak{a}}$,
whose hyperbolic length equals $w_{\mathfrak{a}}/y$.
Hence $\inj(z)\asymp \min\{1,w_{\mathfrak{a}}/y\}$; for $y\ge Y$ this gives the stated bound with
$c=\tfrac12 \min_{\mathfrak{a}} w_{\mathfrak{a}}$.
\end{proof}

\medskip

\noindent\textbf{Thick–thin decomposition.}
For $\epsilon\in(0, \epsilon_{0}(\Gamma))$ define
\[
  M_{\ge \epsilon}=\{z\in M:\inj(z)\ge \epsilon\},\qquad
  M_{<\epsilon}=M\setminus M_{\ge \epsilon}.
\]
Then $M_{\ge \epsilon}$ is compact, and $M_{<\epsilon}$ is a disjoint union of cusp neighborhoods
and collars around short closed geodesics. This is the standard Margulis decomposition.

\medskip

\noindent\textbf{Collar lemma.}
Let $\gamma$ be a simple closed geodesic of length $\ell=\ell(\gamma)$.
Its embedded collar
\[
  \mathcal{C}(\gamma)=\{z\in M : d(z,\gamma)< w(\ell)\},\qquad
  w(\ell)=\operatorname{arcsinh}\!\left(\frac{1}{\sinh(\ell/2)}\right),
\]
is disjoint from all collars of geodesics freely homotopic to $\gamma$.
As $\ell\to 0$, $w(\ell)\sim \log(2/\ell)$.

\medskip

\noindent\textbf{Compact core.}
Choose $Y$ large so that all cusp neighborhoods $\pi(C_{\mathfrak{a}}(Y))$ are disjoint.
The compact core
\[
  M_{\mathrm{core}}=M\setminus\bigcup_{\mathfrak{a}}\pi(C_{\mathfrak{a}}(Y))
\]
is a compact surface with geodesic boundary. Many local analytic estimates
(Sobolev, elliptic regularity) can be verified first on $M_{\mathrm{core}}$ and then extended
to $M$ by patching with the explicit cusp model.

% --- 02b-geometry (Part 3/4): Laplacian, spectral parameter, heat/wave/resolvent kernels

\noindent\textbf{Laplacian and spectral parameter.}
On functions the (nonnegative) Laplace–Beltrami operator is
\[
  \Delta=-y^{2}\bigl(\partial_{x}^{2}+\partial_{y}^{2}\bigr).
\]
On $\mathbb{H}$ the spherical spectral parameter is $t\in\mathbb{R}$ with eigenvalue
$\lambda=\tfrac14+t^{2}$. This normalization matches the Plancherel measure
$\tfrac{1}{4\pi}t\tanh(\pi t)\,dt$ for radial analysis.

\medskip

\noindent\textbf{Spherical functions.}
Let $\varphi_{t}(r)$ be the $K$–biinvariant spherical function with $\varphi_{t}(0)=1$.
It admits
\[
  \varphi_{t}(r)=P_{-1/2+it}(\cosh r)
\]
and the asymptotic expansion as $r\to\infty$
\[
  \varphi_{t}(r)=c(t)\,e^{(it-1/2)r}+c(-t)\,e^{(-it-1/2)r}+O(e^{-3r/2}),
  \qquad
  c(t)=\frac{\Gamma(it)}{\Gamma(1/2+it)}.
\]

\medskip

\noindent\textbf{Heat kernel.}
The heat kernel $K_{t}(z,w)$ on $\mathbb{H}$ solves
$\partial_{t}K_{t}=\Delta_{z}K_{t}$ with $K_{0}(z,w)=\delta_{w}(z)$ and admits the spectral
representation
\[
  K_{t}(z,w)=\frac{1}{4\pi}\int_{-\infty}^{\infty}
  e^{-(1/4+u^{2})t}\,\varphi_{u}\bigl(d(z,w)\bigr)\,u\tanh(\pi u)\,du.
\]
As $t\to 0^{+}$, $K_{t}(z,w)\sim (4\pi t)^{-1} e^{-d(z,w)^{2}/4t}$; as $t\to\infty$ it decays exponentially.

\medskip

\noindent\textbf{Wave kernel.}
Let $\cos(t\sqrt{\Delta-1/4})$ denote the even wave group on $\mathbb{H}$.
Its kernel is
\[
  W_{t}(z,w)=\frac{1}{2\pi}\int_{-\infty}^{\infty}
  \cos(tu)\,\varphi_{u}\bigl(d(z,w)\bigr)\,u\tanh(\pi u)\,du.
\]
For fixed $t\ne 0$ and $d(z,w)\to\infty$, stationary phase yields
\[
  W_{t}(z,w)\ll (1+d(z,w))^{-1/2} e^{-|t|/2}\quad\text{uniformly in }t.
\]
This propagation estimate is used when localizing kernels to scales $t\ll \log\lambda$.

\medskip

\noindent\textbf{Resolvent and Green’s function.}
For $s\in\mathbb{C}$ with $\Re s>1/2$, define
\[
  R(s)=(\Delta-s(1-s))^{-1}.
\]
On $\mathbb{H}$
\[
  R(s;z,w)=\frac{1}{2\pi}\int_{0}^{\infty}
  \frac{\varphi_{u}\bigl(d(z,w)\bigr)}{u^{2}+s(1-s)-1/4}\,u\tanh(\pi u)\,du,
\]
and $R(s;z,w)$ extends meromorphically in $s$.

\medskip

\noindent\textbf{Quotients and kernel summation.}
On $M=\Gamma\backslash\mathbb{H}$, invariant kernels are obtained by
Poincaré series
\[
  K^{\Gamma}(z,w)=\sum_{\gamma\in\Gamma} K(z,\gamma w),
\]
whenever the sum converges absolutely (or via analytic continuation).
The separation lemma and exponential decay of $\varphi_{u}(r)$ ensure convergence for the
heat kernel for $t>0$, and for compactly supported wave packets.

\medskip

\noindent\textbf{Sobolev inequalities (model statement).}
For $f\in H^{2}(M)$ one has
\[
  \|f\|_{L^{\infty}(M)}\ \ll\ C_{\mathrm{Sob}}(M)\,\|f\|_{H^{2}(M)},
\]
with $C_{\mathrm{Sob}}(M)$ depending only on a lower bound for the injectivity radius on a fixed compact core and on cusp data (via Lemma~\ref{lem:inj-cusp}). Exact constants are tracked in Appendix~B.

\medskip

\noindent\textbf{Geodesic flow.}
The geodesic flow on $T^{1}M$ is Anosov; it is mixing w.r.t.\ the Liouville measure.
We record two consequences used later:
(i) uniform distribution of long closed geodesics along their axes,
(ii) decay of correlations for smooth observables,
both entering the control of oscillatory orbital sums.

% --- 02b-geometry (Part 4/4): Topology, Gauss–Bonnet, elliptic points, links, audit

\noindent\textbf{Gauss–Bonnet.}
Let $M$ be a finite-area hyperbolic surface (possibly with elliptic points).
Denote by $g$ its genus, by $n$ the number of cusps, and by
$\{e_{j}\}$ the orders of elliptic points (cone points).
Then
\[
  \vol(M)=2\pi\Bigl(2g-2+n+\sum_{j}(1-1/e_{j})\Bigr).
\]
In particular, for torsion-free $\Gamma$ one has $\vol(M)=2\pi(2g-2+n)$.

\medskip

\noindent\textbf{Elliptic points.}
If $\Gamma$ contains elliptic elements, then $M$ is an orbifold.
Local neighborhoods are cones of angle $2\pi/e$.
Elliptic conjugacy classes contribute additional (explicit) terms to geometric trace identities.
In the present monograph we may include or exclude torsion by assumption;
when included, the required orbital integrals are standard and will be accounted for in Chapter~6.

\medskip

\noindent\textbf{From geometry to analysis (reader’s map).}
The geometric material above feeds the analytic chapters as follows:
\begin{itemize}
  \item \emph{Chapter 3 (Truncated kernels).} Height truncation $M(Y)$, explicit boundary lengths, and separation lemma
        justify absolute convergence of Poincaré series for truncated kernels and control remainders near the boundary.
  \item \emph{Chapter 4 (Spectral projector).} Lower bounds for $\inj$ and Sobolev constants yield
        operator norm bounds and commutator estimates with smoothed truncations.
  \item \emph{Chapter 5 (Microlocal analysis).} Wave kernel asymptotics
        and spherical function expansions match the semiclassical parametrix;
        propagation scale $T\asymp \log\lambda$ is geometrically natural because of exponential volume growth.
  \item \emph{Chapter 6 (Geometric expansion).} Hyperbolic/elliptic/parabolic classifications, collar lemma,
        and explicit length/volume formulas enter orbital integrals and the organization of conjugacy classes.
\end{itemize}

\medskip

\noindent\textbf{Internal consistency checks (Geometry block audit).}
\begin{itemize}
  \item Distances, isometries, and classification agree with $PSL_{2}(\mathbb{R})$ conventions used in the Selberg transform.
  \item Cusp widths $w_{\mathfrak{a}}$ and scaling matrices $\sigma_{\mathfrak{a}}$ are fixed once and for all;
        all cusp-dependent constants in later chapters are functions only of $\{w_{\mathfrak{a}}\}$.
  \item Exact identities:
        $\vol(\pi(C_{\mathfrak{a}})\setminus\pi(C_{\mathfrak{a}}(Y)))=w_{\mathfrak{a}}/Y$
        and $\operatorname{length}(\partial \pi(C_{\mathfrak{a}}(Y)))=w_{\mathfrak{a}}/Y$
        are recorded for subsequent use.
  \item Lemma~\ref{lem:inj-cusp} pins down the injectivity radius in cusps with explicit dependence on $Y$.
  \item Kernel formulas (heat, wave, resolvent) are stated with the same spectral measure
        that appears in the Selberg/Harish–Chandra transform in the next block.
  \item No dependence on later localization parameters $(\lambda,\eta)$ is introduced in this block.
\end{itemize}

\medskip

\noindent\textbf{Conclusion.}
This completes the geometry block for the preliminaries.
It fixes the global normalizations and the local cusp and collar models that
will be used uniformly in Chapters~3–7.
All statements are classical and appear here with explicit constants and
dependencies to ensure reproducibility of later error estimates.
