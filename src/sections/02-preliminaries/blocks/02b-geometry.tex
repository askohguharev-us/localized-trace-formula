% --- Hyperbolic geometry: metric, volume, and injectivity radius (Chapter 2 block) ---

We recall the geometric background needed for the construction of kernels and
for the estimates in later chapters.
All constants are explicit and depend only on $\Gamma$.

\medskip
\noindent\textbf{Hyperbolic metric and measure.}
On the upper half–plane $\mathbb{H}=\{x+iy : y>0\}$ the hyperbolic metric is
\[
  ds^{2} = \frac{dx^{2}+dy^{2}}{y^{2}},
  \qquad
  d\mu(z) = \frac{dx\,dy}{y^{2}}.
\]
The geodesic distance $d(z,w)$ satisfies
\[
  \cosh d(z,w) = 1 + \frac{|z-w|^{2}}{2\,\Im z\,\Im w}.
\]
The Laplace–Beltrami operator (with negative spectrum convention) is
\[
  \Delta = -y^{2}\Big(\partial_{x}^{2} + \partial_{y}^{2}\Big).
\]
Its spectrum on $M=\Gamma\backslash\mathbb{H}$ consists of discrete eigenvalues
$0=\lambda_{0}<\lambda_{1}\le\lambda_{2}\le\cdots$ tending to infinity,
together with the continuous spectrum $[1/4,\infty)$.

\medskip
\noindent\textbf{Balls and volumes.}
The volume of a hyperbolic ball of radius $R$ is
\[
  \vol B(R) = 2\pi\!\big(\cosh R - 1\big) \asymp e^{R}, \quad (R\to\infty).
\]
For small $R$ one has $\vol B(R)\sim \pi R^{2}$.
These identities allow comparison between hyperbolic and Euclidean scales,
especially in local Sobolev inequalities.

\medskip
\noindent\textbf{Fundamental domains.}
Let $F\subset\mathbb{H}$ be a fundamental domain for $\Gamma$.
It can be chosen as a union of finitely many hyperbolic polygons with geodesic sides,
together with cusp neighborhoods of the form
\[
  \{ x+iy : 0\le x<w,\, y>Y_{0}\}
\]
after application of a scaling matrix.
The boundary of $F$ has finite length modulo cusps.
For practical purposes we work with truncated domains $F(Y)$ where cuspidal regions $y>Y$
are cut off and replaced by boundary horocycles.

\medskip
\noindent\textbf{Injectivity radius.}
For $z\in M$ define the injectivity radius
\[
  \inj(z) = \tfrac12 \inf_{\gamma\in\Gamma\setminus\{\pm I\}} d(z,\gamma z).
\]
It is uniformly positive on compact subsets of $M$.
At cusp neighborhoods, $\inj(z)$ decays like $c/y$ in terms of the imaginary coordinate,
with $c$ depending on the cusp width.
In particular,
\[
  \inj(M(Y)) := \inf_{z\in M(Y)}\inj(z) > 0
\]
for $Y$ large enough, where $M(Y)$ is the surface truncated at height $Y$.
This fact is critical when applying Sobolev inequalities and stationary phase arguments.

\medskip
\noindent\textbf{Geodesic flow and unit tangent bundle.}
Let $SM$ be the unit tangent bundle of $M$.
The geodesic flow $\varphi^{t}:SM\to SM$ preserves the Liouville measure.
On the universal cover $S\mathbb{H}\cong\mathrm{PSL}_{2}(\mathbb{R})$,
$\varphi^{t}$ corresponds to right multiplication by
$\begin{psmallmatrix} e^{t/2} & 0 \\ 0 & e^{-t/2}\end{psmallmatrix}$.
The mixing properties of this flow underlie the semiclassical estimates of Chapter~5.

\medskip
\noindent\textbf{Length spectrum.}
Every primitive closed geodesic $\gamma$ on $M$ corresponds to a hyperbolic conjugacy class in $\Gamma$,
with length $\ell(\gamma)>0$ given by
\[
  2\cosh\!\Big(\tfrac{\ell(\gamma)}{2}\Big) = |\operatorname{tr} \gamma|.
\]
The set $\{\ell(\gamma)\}$ (counted with multiplicity) is the \emph{length spectrum}.
It satisfies the asymptotic
\[
  \#\{\gamma : \ell(\gamma)\le L\} \sim \frac{e^{L}}{L},
  \qquad L\to\infty,
\]
which mirrors the Weyl law for the eigenvalue spectrum.
The length spectrum enters explicitly into the geometric side of the trace formula.

\medskip
\noindent\textbf{Wave kernel on $\mathbb{H}$.}
Let $K_{t}(z,w)$ denote the kernel of $\cos(t\sqrt{\Delta})$ on $\mathbb{H}$.
It depends only on $d(z,w)$ and admits an explicit expression
\[
  K_{t}(d) = -\frac{1}{\pi}\,\frac{\partial}{\partial t}\Big(\frac{\sin(t\sqrt{d^{2}-1})}{\sqrt{d^{2}-1}}\Big), \qquad d>1,
\]
continued analytically elsewhere.
This kernel satisfies finite propagation speed:
$K_{t}(z,w)=0$ if $d(z,w)>|t|$.
On the quotient $M$, the periodization
\[
  K^{M}_{t}(z,w) = \sum_{\gamma\in\Gamma} K_{t}(d(z,\gamma w))
\]
is convergent for each fixed $t$ and smooth in $(z,w)$.
It plays a central role in constructing microlocal projectors.

\medskip
\noindent\textbf{Sobolev inequalities.}
For $f\in C_{c}^{\infty}(M)$ one has the hyperbolic Sobolev inequality
\[
  \|f\|_{\infty} \le C \|f\|_{H^{s}(M)} \qquad (s>1),
\]
with $C$ depending only on $\inj(M)$.
In cusp neighborhoods the inequality holds with $C$ depending on $Y$,
and with the $H^{s}$–norm taken over the truncated region $M(Y)$.
These bounds control the growth of eigenfunctions and Eisenstein series.

\medskip
\noindent\textbf{Consistency check and forward link.}
We have fixed conventions for the hyperbolic metric, measure, and Laplacian sign.
Explicit formulas for ball volumes and injectivity radii have been recorded.
The geodesic flow and length spectrum are normalized compatibly with Selberg’s trace formula.
These foundations will be used in the next block to introduce the Selberg transform,
which links radial kernels to their spectral multipliers.
