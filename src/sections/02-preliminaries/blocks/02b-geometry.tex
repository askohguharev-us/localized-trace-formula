% --- Geometry of hyperbolic surfaces (Preliminaries, Block: Geometry) ---

\noindent
We now recall and fix the geometric framework underlying the surface
$M=\Gamma\backslash\mathbb{H}$.
This material is classical but is presented with explicit normalizations
that will be used throughout.

\medskip

\noindent\textbf{The hyperbolic plane.}
Let $\mathbb{H}=\{z=x+iy \in \mathbb{C}: y>0\}$.
The hyperbolic metric is
\[
  ds^{2} = \frac{dx^{2}+dy^{2}}{y^{2}}.
\]
The associated volume element is
\[
  d\mu(z) = \frac{dx\,dy}{y^{2}}.
\]

\medskip

\noindent
The hyperbolic distance $d(z_{1},z_{2})$ satisfies
\[
  \cosh d(z_{1},z_{2})
  = 1 + \frac{|z_{1}-z_{2}|^{2}}{2\,\Im(z_{1})\,\Im(z_{2})}.
\]
This formula will be used repeatedly to estimate displacements
under group actions.

\medskip

\noindent\textbf{Isometries of $\mathbb{H}$.}
The group $PSL_{2}(\mathbb{R})$ acts by fractional linear transformations:
\[
  \begin{pmatrix} a & b \\ c & d \end{pmatrix} \cdot z = \frac{az+b}{cz+d}.
\]
This action preserves the hyperbolic metric and measure.

\medskip

\noindent
Isometries fall into three classes depending on the trace:
\begin{itemize}
  \item \emph{Elliptic:} $|\mathrm{tr}|<2$, fixing a point in $\mathbb{H}$.
  \item \emph{Parabolic:} $|\mathrm{tr}|=2$, fixing a unique cusp on $\partial\mathbb{H}$.
  \item \emph{Hyperbolic:} $|\mathrm{tr}|>2$, fixing two boundary points and preserving the geodesic between them.
\end{itemize}
These classifications play distinct roles in the trace formula:
elliptic elements contribute discrete orbital integrals,
parabolic elements yield scattering terms,
and hyperbolic elements correspond to closed geodesics.

\medskip

\noindent\textbf{Geodesics.}
Geodesics in $\mathbb{H}$ are either vertical lines or semicircles orthogonal to the real line.
For points $z_{1},z_{2}\in\mathbb{H}$,
the unique geodesic joining them is such a circle or line.

\medskip

\noindent
The length of a geodesic segment can be computed explicitly using the distance formula.
For a closed geodesic associated with a hyperbolic element $\gamma$,
its length $\ell(\gamma)$ satisfies
\[
  2\cosh\!\left(\frac{\ell(\gamma)}{2}\right) = |\mathrm{tr}(\gamma)|.
\]

\medskip

\noindent\textbf{Discrete groups and fundamental domains.}
Let $\Gamma\subset PSL_{2}(\mathbb{R})$ be a discrete subgroup of finite covolume.
The quotient $M=\Gamma\backslash \mathbb{H}$ is a hyperbolic surface of finite area.
If $\Gamma$ has no elliptic elements then $M$ is a smooth manifold;
otherwise it is an orbifold with conical singularities.

\medskip

\noindent
A \emph{fundamental domain} $\mathcal{F}$ for $\Gamma$ is a measurable subset of $\mathbb{H}$
such that
\[
  \bigcup_{\gamma\in\Gamma} \gamma \mathcal{F} = \mathbb{H},
\]
and the interiors are disjoint.
The volume of $M$ is then
\[
  \vol(M) = \int_{\mathcal{F}} d\mu(z).
\]

\medskip

\noindent\textbf{Cocompact vs. noncocompact.}
If $\Gamma$ is cocompact, then $M$ is compact and has no cusps.
If $\Gamma$ is noncocompact of finite covolume,
then $M$ has finitely many cusps, treated in the next block.

\medskip

\noindent\textbf{Injectivity radius.}
For $z\in M$, the injectivity radius is
\[
  \inj(z) = \tfrac12 \inf_{\gamma\in\Gamma\setminus\{1\}} d(z,\gamma z).
\]
This measures the radius of the largest embedded hyperbolic ball around $z$.
It is continuous and $\inj(z)>0$ except possibly near cusps.

\medskip

\noindent
For compact $M$, $\inj(z)$ attains a positive minimum.
For noncompact $M$, $\inj(z)$ tends to zero in the cusp regions.
We will quantify this in later blocks.

\medskip

\noindent\textbf{Dirichlet domains.}
Fix $z_{0}\in \mathbb{H}$.
The Dirichlet domain is
\[
  \mathcal{D}(z_{0}) = \{ z\in\mathbb{H} : d(z,z_{0})\le d(z,\gamma z_{0}), \forall \gamma\in\Gamma\}.
\]
This is a convex polygon (possibly with infinitely many sides) in $\mathbb{H}$.
It is another realization of a fundamental domain.
Properties of $\mathcal{D}(z_{0})$ reflect the geometry of $\Gamma$.

\medskip

\noindent\textbf{Volume growth.}
The hyperbolic ball of radius $R$ has area
\[
  \vol(B(R)) = 2\pi(\cosh R - 1).
\]
This exponential growth rate underlies many asymptotic estimates.
In particular,
the number of closed geodesics of length $\le R$ grows like $e^{R}/R$.

\medskip

\noindent\textbf{Green’s function.}
The Green’s function $G(z,w;s)$ for the Laplacian on $\mathbb{H}$
is given explicitly by a hypergeometric function of $\cosh d(z,w)$.
It satisfies
\[
  (\Delta - s(1-s)) G(z,w;s) = -\delta(z-w).
\]
This kernel will reappear in the Selberg transform.

\medskip

\noindent\textbf{Spectral parameterization.}
Eigenvalues of the Laplacian on $\mathbb{H}$ are written
\[
  \lambda = \tfrac14 + t^{2}, \qquad t\in\mathbb{R}.
\]
This parameterization matches the Fourier analysis on $\mathbb{H}$.
The Plancherel measure is $\tfrac{1}{4\pi}dt$.

\medskip

\noindent\textbf{Lattice point estimates.}
For $\Gamma$ discrete, counting orbit points is central.
Let $N(R)=\#\{\gamma\in\Gamma : d(z,\gamma z)\le R\}$.
Then
\[
  N(R) \sim \frac{\vol(B(R))}{\vol(M)} \quad (R\to\infty).
\]
This is a form of the hyperbolic lattice point theorem.

\medskip

\noindent\textbf{Selberg’s lemma on separation.}
There exists $\epsilon>0$ such that distinct translates of $\mathcal{F}$
are separated by distance $\ge \epsilon$.
This finiteness property underlies convergence of orbital sums.

\medskip

\noindent\textbf{Orbital integrals.}
Given a function $k:\mathbb{H}\times\mathbb{H}\to\mathbb{C}$
that is $\Gamma$-invariant and radial, orbital integrals are
\[
  O_{\gamma}(k) = \int_{\Gamma_{\gamma}\backslash\mathbb{H}} k(z,\gamma z)\,d\mu(z).
\]
Their computation depends on the conjugacy class of $\gamma$.

\medskip

\noindent\textbf{Forward links.}
The geometric constructions here set the stage for:
\begin{itemize}
  \item Chapter~2B (cusps), where injectivity radius near infinity is quantified.
  \item Chapter~3, kernel construction, where geodesic structure governs orbital sums.
  \item Chapter~6, geometric expansion, assembling contributions of elliptic, hyperbolic, parabolic elements.
\end{itemize}

\medskip

\noindent
% --- End of Part 1 for Geometry block ---

% --- Geometry of hyperbolic surfaces (Preliminaries, Block: Geometry) continued ---

\noindent\textbf{Gauss–Bonnet theorem.}
For a hyperbolic surface $M$ of finite area,
\[
  \vol(M) = 2\pi\,(2g-2 + n + m/2),
\]
where $g$ is the genus,
$n$ the number of cusps,
and $m$ the number of elliptic points counted with weights $1-1/e_{j}$.
This fundamental identity connects topology and geometry.

\medskip

\noindent\textbf{Elliptic points.}
If $\Gamma$ has elliptic elements of order $e$,
then $M$ has cone points of angle $2\pi/e$.
The contribution of these points to orbital integrals
requires special treatment,
though in many cases of interest $\Gamma$ is torsion-free.

\medskip

\noindent\textbf{Compact core.}
For noncompact $M$,
define the compact core
\[
  M_{\mathrm{core}} = M \setminus \bigcup_{\mathfrak{a}} \pi(C_{\mathfrak{a}}(Y)),
\]
with $Y$ chosen large enough.
This is a compact surface with geodesic boundary.
Many analytic arguments reduce to estimates on $M_{\mathrm{core}}$.

\medskip

\noindent\textbf{Geodesic boundary and collars.}
A closed geodesic $\gamma$ of length $\ell$ admits a collar neighborhood
\[
  \{ z\in M : d(z,\gamma)<w(\ell) \},
\]
with width
\[
  w(\ell) = \operatorname{arcsinh}\!\left(\frac{1}{\sinh(\ell/2)}\right).
\]
The collar lemma ensures that collars of distinct geodesics are disjoint.
This structural fact controls the thin part of $M$.

\medskip

\noindent\textbf{Thin–thick decomposition.}
Define
\[
  M_{\ge \epsilon} = \{ z\in M : \inj(z)\ge \epsilon \}, \qquad
  M_{<\epsilon} = M\setminus M_{\ge \epsilon}.
\]
Then $M_{\ge \epsilon}$ is compact
and $M_{<\epsilon}$ consists of cusp neighborhoods and collars.
This decomposition underlies many spectral estimates.

\medskip

\noindent\textbf{Sobolev inequalities.}
On hyperbolic surfaces,
Sobolev norms satisfy inequalities of the form
\[
  \|f\|_{L^{\infty}(M)} \;\ll\; \|f\|_{H^{2}(M)}.
\]
Constants depend explicitly on the injectivity radius of $M$.
This dependence must be tracked carefully when working with degenerating families.

\medskip

\noindent\textbf{Heat kernel.}
The heat kernel $K_{t}(z,w)$ solves
\[
  \partial_{t} K_{t} = \Delta_{z} K_{t}, \qquad K_{0}(z,w)=\delta(z-w).
\]
On $\mathbb{H}$ it is known explicitly:
\[
  K_{t}(z,w) = \frac{1}{(4\pi t)^{1/2}} \int_{d(z,w)}^{\infty} \frac{u\,e^{-u^{2}/4t}}{(\cosh u - \cosh d(z,w))^{1/2}}\,du.
\]
As $t\to 0^{+}$, $K_{t}$ approximates the Euclidean heat kernel.
As $t\to\infty$, $K_{t}$ decays exponentially.
These properties extend to quotients $M$.

\medskip

\noindent\textbf{Wave kernel.}
The wave kernel on $\mathbb{H}$ is
\[
  W_{t}(z,w) = \frac{1}{2\pi} \int_{-\infty}^{\infty} e^{it r}\,\varphi_{r}(d(z,w))\,r\tanh(\pi r)\,dr,
\]
where $\varphi_{r}$ is the spherical function.
This kernel underlies microlocal analysis in Chapter~5.

\medskip

\noindent\textbf{Resolvent kernel.}
The resolvent $R(s) = (\Delta - s(1-s))^{-1}$ has kernel
\[
  R(s;z,w) = \frac{1}{2\pi}\int_{0}^{\infty} \frac{\varphi_{r}(d(z,w))}{r^{2}+s(1-s)-1/4}\,r\tanh(\pi r)\,dr.
\]
Its analytic continuation in $s$ yields poles at Laplace eigenvalues.
This resolvent identity is a key analytic device.

\medskip

\noindent\textbf{Eigenfunction estimates.}
For an $L^{2}$-normalized eigenfunction $\phi_{j}$ with eigenvalue $\lambda_{j}$,
one has
\[
  \|\phi_{j}\|_{\infty} \;\ll_{\epsilon}\; \lambda_{j}^{1/4+\epsilon}.
\]
This bound (Iwaniec–Sarnak) is sharp up to $\epsilon$ in many contexts.
It shows that eigenfunctions are delocalized at small scales.

\medskip

\noindent\textbf{Quantum ergodicity.}
As $\lambda_{j}\to\infty$, the measures $|\phi_{j}(z)|^{2}d\mu(z)$
tend to uniform distribution on $M$.
This property (Shnirelman, Zelditch, Colin de Verdière)
is an expression of classical ergodicity of the geodesic flow.
It connects spectral theory with dynamical systems.

\medskip

\noindent\textbf{Geodesic flow.}
The geodesic flow on the unit tangent bundle $T^{1}M$
is Anosov: it has uniform hyperbolicity.
This flow preserves the Liouville measure.
Its dynamical mixing properties inform the decay of matrix coefficients.

\medskip

\noindent\textbf{Prime geodesic theorem.}
Let $\pi(R)$ denote the number of primitive closed geodesics of length $\le R$.
Then
\[
  \pi(R) \sim \frac{e^{R}}{R}, \qquad (R\to\infty).
\]
This theorem parallels the prime number theorem,
and is derived from the Selberg trace formula.

\medskip

\noindent\textbf{Selberg zeta function.}
The Selberg zeta function is defined by
\[
  Z(s) = \prod_{\{\gamma\}}\prod_{k=0}^{\infty}
  \left(1-e^{-(s+k)\ell(\gamma)}\right),
\]
the product taken over primitive closed geodesics.
It converges for $\Re(s)>1$, extends meromorphically to $\mathbb{C}$,
and satisfies a functional equation.
Its zeros encode the spectrum of the Laplacian.

\medskip

\noindent\textbf{Comparison with Euclidean geometry.}
Unlike flat tori, hyperbolic surfaces exhibit exponential volume growth,
negative curvature, and spectral gaps.
These features fundamentally change analytic methods.
For instance, exponential growth leads to sharp stationary phase asymptotics.

\medskip

\noindent\textbf{Forward links.}
The analytic kernels recalled here (heat, wave, resolvent)
will be used in:
\begin{itemize}
  \item Chapter~3, kernel construction.
  \item Chapter~5, microlocal analysis.
  \item Chapter~7, main results, to estimate error terms.
\end{itemize}

\medskip

\noindent\textbf{Audit of geometry block.}
We verify that all goals set at the beginning of this block have been met:
\begin{itemize}
  \item[(G1)] Definition of hyperbolic metric and distance: complete and explicit.
  \item[(G2)] Classification of isometries: recorded, with roles in trace formula.
  \item[(G3)] Fundamental domains and volumes: defined and normalized.
  \item[(G4)] Injectivity radius: formal definition given; bounds referenced.
  \item[(G5)] Dirichlet domains: definition and convexity properties.
  \item[(G6)] Volume growth and lattice point estimates: formulas stated.
  \item[(G7)] Analytic kernels (heat, wave, resolvent): explicit formulas recorded.
  \item[(G8)] Links to later chapters: spelled out.
\end{itemize}

\medskip

\noindent\textbf{Conclusion.}
This block fixes all geometric conventions for $M=\Gamma\backslash\mathbb{H}$.
Together with the cusp analysis in the following block,
it provides the full geometric foundation for the localized trace formula.

% --- End of geometry block 02b-geometry.tex ---
