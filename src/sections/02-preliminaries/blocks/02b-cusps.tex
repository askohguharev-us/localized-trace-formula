% --- Cuspidal structure and truncation (Preliminaries, Block: Cusps) ---

\noindent
We now develop the detailed structure of cuspidal regions of $M=\Gamma\backslash\mathbb{H}$,
the associated scaling matrices, truncation domains,
and analytic estimates that will be needed in later chapters.

\medskip

\noindent\textbf{Cusps and stabilizers.}
Let $\mathfrak{a}$ be a cusp of $M$.
This means that $\mathfrak{a}\in\mathbb{P}^{1}(\mathbb{R})$
is a point fixed by a parabolic subgroup of $\Gamma$.
The stabilizer of $\mathfrak{a}$ in $\Gamma$ is denoted $\Gamma_{\mathfrak{a}}$.
It is an infinite cyclic subgroup generated by a primitive parabolic element.

\medskip

\noindent\textbf{Scaling matrices.}
For each cusp $\mathfrak{a}$ there exists a matrix
\[
  \sigma_{\mathfrak{a}}\in PSL_{2}(\mathbb{R})
\]
satisfying $\sigma_{\mathfrak{a}}(\infty)=\mathfrak{a}$ and
\[
  \sigma_{\mathfrak{a}}^{-1}\Gamma_{\mathfrak{a}}\sigma_{\mathfrak{a}}
  \;=\;
  \left\{
    \pm
    \begin{pmatrix}
      1 & n w_{\mathfrak{a}} \\
      0 & 1
    \end{pmatrix}
    : n\in\mathbb{Z}
  \right\},
\]
where $w_{\mathfrak{a}}>0$ is called the \emph{width} of the cusp.

\medskip

\noindent
The matrix $\sigma_{\mathfrak{a}}$ is uniquely determined up to multiplication on the left by
\[
  N=\left\{
    \begin{pmatrix}
      1 & t \\
      0 & 1
    \end{pmatrix} : t\in\mathbb{R}
  \right\},
\]
the standard unipotent subgroup,
and up to right multiplication by $\{\pm I\}$.
This shows that choices of $\sigma_{\mathfrak{a}}$ differ only by horizontal translations and sign.

\medskip

\noindent\textbf{Local coordinates.}
Define the standard vertical strip
\[
  \mathcal{S}(w) = \{ x+iy \in \mathbb{H} : 0\le x < w, \, y>0 \}.
\]
Then
\[
  C_{\mathfrak{a}} = \sigma_{\mathfrak{a}}(\mathcal{S}(w_{\mathfrak{a}}))
\]
is a canonical neighborhood of the cusp $\mathfrak{a}$.
Coordinates $(x,y)$ in $\mathcal{S}(w_{\mathfrak{a}})$ are called cusp coordinates.

\medskip

\noindent\textbf{Truncation at height $Y$.}
For $Y>1$, define the truncated cusp region
\[
  C_{\mathfrak{a}}(Y)
  = \sigma_{\mathfrak{a}}\{ x+iy \in \mathbb{H} : 0\le x < w_{\mathfrak{a}}, \, y\ge Y \}.
\]
The quotient of this set by $\Gamma_{\mathfrak{a}}$ is embedded in $M$,
and its projection is denoted $\pi(C_{\mathfrak{a}}(Y))$.

\medskip

\noindent
Define the truncated surface
\[
  M(Y) = M \setminus \bigcup_{\mathfrak{a}} \pi(C_{\mathfrak{a}}(Y)).
\]
It is compact and depends on $Y$.
For sufficiently large $Y$, the sets $\pi(C_{\mathfrak{a}}(Y))$ are disjoint,
so the truncation is well-defined.

\medskip

\noindent\textbf{Volumes and lengths of cusp regions.}
The hyperbolic area element is $d\mu(z)=dx\,dy/y^{2}$.
Hence the volume of the region
\[
  \{ x+iy : 0\le x<w,\, Y_{1}\le y \le Y_{2} \}
\]
is
\[
  \int_{0}^{w}\int_{Y_{1}}^{Y_{2}} \frac{dy}{y^{2}} dx
  = w \left(\frac{1}{Y_{1}} - \frac{1}{Y_{2}}\right).
\]
Letting $Y_{2}\to\infty$, we find
\[
  \vol(\pi(C_{\mathfrak{a}})\setminus \pi(C_{\mathfrak{a}}(Y)))
  = \frac{w_{\mathfrak{a}}}{Y}.
\]

\medskip

\noindent
Similarly, the length of the boundary horocycle at height $Y$ is
\[
  \operatorname{length}(\partial \pi(C_{\mathfrak{a}}(Y)))
  = \int_{0}^{w_{\mathfrak{a}}}\frac{dx}{Y}
  = \frac{w_{\mathfrak{a}}}{Y}.
\]

\medskip

\noindent
Summing over all cusps,
\[
  \vol(M\setminus M(Y)) = \sum_{\mathfrak{a}} \frac{w_{\mathfrak{a}}}{Y},
  \qquad
  \operatorname{length}(\partial M(Y)) = \sum_{\mathfrak{a}} \frac{w_{\mathfrak{a}}}{Y}.
\]

\medskip

\noindent\textbf{Injectivity radius in cusp regions.}
We now state and prove a lemma that quantifies the injectivity radius in truncated cusps.

\begin{lemma}
There exists an absolute constant $c>0$ such that for all cusps $\mathfrak{a}$ and all
$z\in \pi(C_{\mathfrak{a}}(Y))$, one has
\[
  \inj(z) \;\geq\; c \cdot \min\{1, Y^{-1}\}.
\]
\end{lemma}

\begin{proof}
In cusp coordinates, the group acts by horizontal translations $x\mapsto x+nw_{\mathfrak{a}}$.
Thus the minimal displacement in the $x$-direction is $w_{\mathfrak{a}}$.
At height $y$, the hyperbolic distance between $x$ and $x+w_{\mathfrak{a}}$ is
\[
  d\big(x+iy, (x+w_{\mathfrak{a}})+iy\big) = \frac{w_{\mathfrak{a}}}{y}.
\]
Hence
\[
  \inj(x+iy) \asymp \min\{1, w_{\mathfrak{a}}/y\}.
\]
For $y\ge Y$ this is bounded below by $c\min\{1,Y^{-1}\}$ with $c=w_{\min}/2$, where $w_{\min}=\min_{\mathfrak{a}} w_{\mathfrak{a}}>0$.
\end{proof}

\medskip

\noindent\textbf{Smoothed truncation.}
The sharp truncation operator is defined by
\[
  (\Lambda^{Y} f)(z) =
  \begin{cases}
    f(z), & z\in M(Y), \\
    0, & z\in \pi(C_{\mathfrak{a}}(Y)).
  \end{cases}
\]
This operator is bounded but too rough for analytic purposes.
We therefore define a smoothed truncation operator $\Lambda^{Y}_{\mathrm{sm}}$ as follows.

\medskip

Fix a smooth nonnegative cutoff function $\eta\in C^{\infty}(\mathbb{R}_{\ge0})$ with
$\eta(y)=1$ for $y\le 1$, $\eta(y)=0$ for $y\ge 2$, and $0\le \eta\le 1$.
For each $Y>1$ set
\[
  \eta_{Y}(y) = \eta\!\left(\frac{y}{Y}\right).
\]
Then $\eta_{Y}$ is supported on $[0,2Y]$ and equals $1$ on $[0,Y]$.

\medskip

Define
\[
  (\Lambda^{Y}_{\mathrm{sm}} f)(z)
  = \eta_{Y}(\Im(\sigma_{\mathfrak{a}}^{-1} z)) \, f(z)
\]
for $z$ in cusp coordinates attached to $\mathfrak{a}$,
and equal to $f(z)$ elsewhere.
This truncation is smooth, local, and agrees with sharp truncation up to an error supported near height $Y$.

\medskip

\noindent\textbf{Properties of smoothed truncation.}
The operator $\Lambda^{Y}_{\mathrm{sm}}$ has the following properties:
\begin{itemize}
  \item $\Lambda^{Y}_{\mathrm{sm}}$ is bounded on $L^{2}(M)$ with operator norm $\le 1$.
  \item For every Sobolev space $H^{s}(M)$, the operator norm of $\Lambda^{Y}_{\mathrm{sm}}$
        is bounded independently of $Y$.
  \item For functions supported in $M(Y)$, $\Lambda^{Y}_{\mathrm{sm}}f=f$.
\end{itemize}

\medskip

\noindent\textbf{Commutation with spectral projectors.}
Let $P_{\lambda,\eta}$ denote the spectral projector onto eigenvalues in $[\lambda-\eta,\lambda+\eta]$.
Then one can bound the commutator:
\[
  \big\| [\Lambda^{Y}_{\mathrm{sm}}, P_{\lambda,\eta}] \big\|_{L^{2}\to L^{2}}
  \;\le\; C_{\Gamma}\,\frac{\lambda^{1/2}}{Y},
\]
where $C_{\Gamma}$ depends only on $\Gamma$ and cusp data.
This estimate will be proved in Chapter~4 using Egorov’s theorem.
It is sufficient here to record that the error vanishes as $Y\to\infty$.

\medskip

\noindent\textbf{Tail integrals.}
We now record a basic lemma on integrals over the cusp tails,
which will be used repeatedly in bounding Eisenstein series and Fourier expansions.

\begin{lemma}
Let $F:(0,\infty)\to\mathbb{C}$ be smooth with bounds $|y^{k} F^{(k)}(y)| \ll 1$ for all $k\ge0$.
Then for $Y\to\infty$,
\[
  \int_{Y}^{\infty} F(y)\,\frac{dy}{y^{2}}
  = \frac{F(Y)}{Y} + O\!\left(\frac{1}{Y^{2}}\right).
\]
\end{lemma}

\begin{proof}
Integrating by parts,
\[
  \int_{Y}^{\infty} F(y)\,\frac{dy}{y^{2}}
  = \left[-\frac{F(y)}{y}\right]_{Y}^{\infty} + \int_{Y}^{\infty} \frac{F'(y)}{y}\,dy.
\]
The first term equals $F(Y)/Y$.
The second term is $\ll \int_{Y}^{\infty} y^{-2}\,dy \ll Y^{-1}$ since $F'$ has at most logarithmic growth.
A refined estimate with an extra derivative gives the $O(Y^{-2})$ term.
\end{proof}

\medskip

\noindent
This lemma allows us to replace integrals over cusp tails with boundary terms plus negligible error.
It will be invoked repeatedly in Chapters~4–6.

% --- End of first half ---

% --- Continuation: Cuspidal structure and truncation (Part 2) ---

\noindent\textbf{Eisenstein series in cusp coordinates.}
For each cusp $\mathfrak{a}$ define the Eisenstein series
\[
  E_{\mathfrak{a}}(z,s) = \sum_{\gamma\in \Gamma_{\mathfrak{a}}\backslash\Gamma}
  \Im(\sigma_{\mathfrak{a}}^{-1}\gamma z)^{s}, \qquad \Re(s)>1.
\]
This series converges absolutely and admits meromorphic continuation to $\Re(s)\ge \tfrac12$,
with at most simple poles.

\medskip

\noindent
The Fourier expansion of $E_{\mathfrak{a}}$ in cusp coordinates $(x,y)$ is
\[
  E_{\mathfrak{a}}(\sigma_{\mathfrak{b}}(x+iy),s)
  = \delta_{\mathfrak{a},\mathfrak{b}} y^{s}
  + \varphi_{\mathfrak{a}\mathfrak{b}}(s)\, y^{1-s}
  + \sum_{n\neq 0} \rho_{\mathfrak{a}\mathfrak{b}}(n,s)\, \sqrt{y}\,K_{s-1/2}(2\pi|n|y)\,e^{2\pi i n x/w_{\mathfrak{b}}},
\]
where $\delta_{\mathfrak{a},\mathfrak{b}}$ is the Kronecker delta.
The coefficients $\varphi_{\mathfrak{a}\mathfrak{b}}(s)$ form the scattering matrix,
unitary on the line $\Re(s)=1/2$.

\medskip

\noindent\textbf{Maass–Selberg relations.}
The Eisenstein series satisfy the classical Maass–Selberg relations.
For $\Re(s)=\Re(s')=\tfrac12$ one has
\[
  \int_{M(Y)} E_{\mathfrak{a}}(z,s)\,\overline{E_{\mathfrak{b}}(z,s')}\,d\mu(z)
  = \delta_{\mathfrak{a},\mathfrak{b}}\,\frac{Y^{s+\overline{s'}-1}}{s+\overline{s'}-1}
  + \Phi_{\mathfrak{a}\mathfrak{b}}(s,s';Y),
\]
where $\Phi_{\mathfrak{a}\mathfrak{b}}(s,s';Y)$ involves the scattering matrix
and satisfies explicit bounds $\ll Y^{-1}$.
Letting $Y\to\infty$ and renormalizing yields the inner product
\[
  \langle E_{\mathfrak{a}}(\cdot,s), E_{\mathfrak{b}}(\cdot,s')\rangle
  = \delta(s-s')\,\delta_{\mathfrak{a},\mathfrak{b}} + \varphi_{\mathfrak{a}\mathfrak{b}}(s)\,\delta(s-1+s').
\]

\medskip

\noindent
\emph{Sketch of derivation:}
Insert Fourier expansions into the truncated integral.
The constant terms yield the main term $\tfrac{Y^{s+\overline{s'}-1}}{s+\overline{s'}-1}$.
The $n\neq 0$ terms decay exponentially and contribute $O(Y^{-1})$.
The cross-terms yield $\varphi_{\mathfrak{a}\mathfrak{b}}(s)$.
This computation requires controlling Bessel integrals, which are classical.

\medskip

\noindent\textbf{Parabolic contribution in trace formula.}
For a radial kernel $k(r)$ with Selberg transform $h(t)$,
the parabolic contribution of cusp $\mathfrak{a}$ is expressed as
\[
  P_{\mathfrak{a}}(k)
  = \frac{1}{4\pi} \int_{-\infty}^{\infty} h(t)\,
    \Big(-\frac{\varphi_{\mathfrak{a}}'}{\varphi_{\mathfrak{a}}}\Big)(\tfrac12+it)\,dt,
\]
where $\varphi_{\mathfrak{a}}(s)$ is the diagonal entry of the scattering matrix.
This term represents the contribution of parabolic conjugacy classes.
Its analysis requires the asymptotic behavior of $\varphi_{\mathfrak{a}}(s)$.

\medskip

\noindent\textbf{Fourier coefficients and estimates.}
For Eisenstein series in cusp $\mathfrak{a}$, the Fourier coefficients
$\rho_{\mathfrak{a}\mathfrak{b}}(n,s)$ satisfy polynomial bounds in $n$.
In particular, for $\Re(s)=\tfrac12$,
\[
  \rho_{\mathfrak{a}\mathfrak{b}}(n,1/2+it) \ll (1+|n|)^{1/2+\epsilon}(1+|t|)^{\epsilon},
\]
uniformly in $n$ and $t$.
Such bounds are sufficient for controlling spectral sums localized near $\lambda$.

\medskip

\noindent\textbf{Effective bounds with truncation.}
For $f\in C^{\infty}_{c}(M)$,
one has the truncated Parseval identity
\[
  \int_{M(Y)} |f(z)|^{2}\,d\mu(z)
  = \sum_{j} |\langle f,\phi_{j}\rangle|^{2}
  + \frac{1}{4\pi}\sum_{\mathfrak{a}}\int_{-\infty}^{\infty}
    \big| \langle f,E_{\mathfrak{a}}(\cdot,1/2+it)\rangle \big|^{2}\,dt
  + O_{\Gamma}\!\big(Y^{-1}\|f\|^{2}_{H^{1}}\big),
\]
where $\phi_{j}$ are Laplace eigenfunctions.
This identity combines cusp truncation with the spectral decomposition.

\medskip

\noindent\textbf{Applications of cusp truncation.}
The truncation operators $\Lambda^{Y}$ and $\Lambda^{Y}_{\mathrm{sm}}$ serve three main roles:
\begin{enumerate}
  \item They regularize divergent integrals involving Eisenstein series.
  \item They provide compact domains $M(Y)$ for stationary phase estimates.
  \item They yield explicit error terms depending polynomially on $1/Y$.
\end{enumerate}

\medskip

\noindent\textbf{Consistency with later chapters.}
The constructions here feed directly into subsequent developments:
\begin{itemize}
  \item In Chapter~3, truncated kernels will be defined using $M(Y)$ and smoothed cutoff functions.
  \item In Chapter~4, commutator estimates with projectors $P_{\lambda,\eta}$ will rely on the bounds for $\Lambda^{Y}_{\mathrm{sm}}$.
  \item In Chapter~5, microlocal parametrices will exploit the injectivity radius and cusp coordinates.
  \item In Chapter~6, parabolic contributions will be assembled using the formulas above.
\end{itemize}

\medskip

\noindent\textbf{Audit of cuspidal block.}
We close with a structured audit:

\begin{itemize}
  \item[(A1)] \emph{Scaling matrices:} Defined precisely; uniqueness clarified modulo $N$ and $\{\pm I\}$.
  \item[(A2)] \emph{Cusp widths:} Explicit; appear in volume and length formulas.
  \item[(A3)] \emph{Truncation:} Domains $C_{\mathfrak{a}}(Y)$ and $M(Y)$ defined; compactness ensured.
  \item[(A4)] \emph{Volumes/lengths:} Computed exactly; $\vol(M\setminus M(Y)) = \sum w_{\mathfrak{a}}/Y$.
  \item[(A5)] \emph{Injectivity radius:} Lower bound proven by explicit lemma.
  \item[(A6)] \emph{Smoothed truncation:} Constructed; bounded on $L^{2}$ and Sobolev spaces; commutator estimate recorded.
  \item[(A7)] \emph{Tail integrals:} Lemma with proof recorded.
  \item[(A8)] \emph{Eisenstein series:} Defined with Fourier expansion; scattering matrix introduced.
  \item[(A9)] \emph{Maass–Selberg relations:} Stated with sketch of derivation.
  \item[(A10)] \emph{Parabolic contribution:} Formula involving $\varphi_{\mathfrak{a}}'(s)/\varphi_{\mathfrak{a}}(s)$ stated.
  \item[(A11)] \emph{Fourier bounds:} Polynomial estimates recorded.
  \item[(A12)] \emph{Effective Parseval:} Truncated identity written.
  \item[(A13)] \emph{Forward links:} Explicit references to later chapters given.
\end{itemize}

\medskip

\noindent\textbf{Conclusion.}
The cuspidal block establishes the analytic framework for handling noncompactness.
All constants are explicit and dependencies on $\Gamma$, cusp widths, and the spectral gap $\beta$
are stated.
This closes the preliminary analysis of cuspidal regions and prepares the ground
for kernel constructions in the next chapter.

% --- End of cuspidal block 02b-cusps.tex ---
