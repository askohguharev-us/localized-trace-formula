% =====================================================================
% Block 1: Cuspidal structure and scaling matrices
% =====================================================================

\subsection{Cuspidal structure and scaling matrices}

In this block we recall the precise structure of cuspidal regions
of the finite-area hyperbolic surface
\[
  M = \Gamma \backslash \mathbb{H}, \qquad 
  \Gamma \subset PSL_{2}(\mathbb{R})
\]
and fix the notational conventions that will be used throughout
for truncations, coordinates, and scaling matrices. 
The material is classical but must be spelled out carefully,
since all constants and asymptotic estimates in the trace formula
ultimately depend on the cusp data.

\medskip
\noindent
\textbf{Cusps as parabolic fixed points.}
A cusp $\mathfrak{a}$ of $M$ is, by definition, a $\Gamma$-equivalence
class of points in $\mathbb{P}^{1}(\mathbb{R})$ that are fixed by
a parabolic subgroup of $\Gamma$. 
If $\Gamma$ is a cofinite Fuchsian group,
then it has finitely many cusps
\[
  \{\mathfrak{a}_{1}, \dots, \mathfrak{a}_{h}\}.
\]
Each cusp admits an associated stabilizer subgroup
\[
  \Gamma_{\mathfrak{a}} = \{\gamma \in \Gamma : \gamma \mathfrak{a} = \mathfrak{a}\}.
\]
This stabilizer is infinite cyclic, generated by a primitive parabolic
element of $\Gamma$.

\medskip
\noindent
\textbf{Scaling matrices.}
For each cusp $\mathfrak{a}$ we choose a scaling matrix
$\sigma_{\mathfrak{a}} \in PSL_{2}(\mathbb{R})$ satisfying
\[
  \sigma_{\mathfrak{a}}(\infty) = \mathfrak{a},
  \qquad
  \sigma_{\mathfrak{a}}^{-1} \Gamma_{\mathfrak{a}} \sigma_{\mathfrak{a}}
  = \left\{ 
      \pm \begin{pmatrix} 1 & n w_{\mathfrak{a}} \\ 0 & 1 \end{pmatrix}
      : n \in \mathbb{Z} \right\}.
\]
Here $w_{\mathfrak{a}} > 0$ is the \emph{width} of the cusp $\mathfrak{a}$.
The set of widths $\{w_{\mathfrak{a}}\}$ will appear repeatedly
in the volume formulas and error terms of later chapters.

\begin{remark}
The scaling matrix $\sigma_{\mathfrak{a}}$ is uniquely determined up to
left multiplication by the standard unipotent group
\[
  N = \left\{ \begin{pmatrix} 1 & t \\ 0 & 1 \end{pmatrix} : t \in \mathbb{R} \right\},
\]
and right multiplication by $\{\pm I\}$. 
This freedom reflects the fact that cusp coordinates are defined only
up to horizontal translation and sign.
All subsequent constructions are invariant under these changes.
\end{remark}

\medskip
\noindent
\textbf{Canonical cusp neighborhoods.}
Define the standard strip
\[
  \mathcal{S}(w) = \{ x+iy \in \mathbb{H} : 0 \le x < w, \ y > 0 \}.
\]
Then the set
\[
  C_{\mathfrak{a}} = \sigma_{\mathfrak{a}}(\mathcal{S}(w_{\mathfrak{a}}))
\]
is a canonical neighborhood of the cusp $\mathfrak{a}$. 
Coordinates $(x,y)$ on $\mathcal{S}(w_{\mathfrak{a}})$
are called \emph{cusp coordinates} attached to $\mathfrak{a}$.
They identify the stabilizer $\Gamma_{\mathfrak{a}}$
with horizontal translations $x \mapsto x+n w_{\mathfrak{a}}$.

\begin{lemma}[Uniqueness of cusp coordinates]
Let $\sigma_{\mathfrak{a}}$ and $\sigma'_{\mathfrak{a}}$
be two scaling matrices attached to the same cusp $\mathfrak{a}$.
Then there exists $t \in \mathbb{R}$ such that
\[
  \sigma'_{\mathfrak{a}} = \sigma_{\mathfrak{a}}
  \begin{pmatrix} 1 & t \\ 0 & 1 \end{pmatrix}.
\]
Thus cusp coordinates differ only by horizontal translation.
\end{lemma}

\begin{proof}
By construction both $\sigma_{\mathfrak{a}}$ and $\sigma'_{\mathfrak{a}}$
map $\infty$ to $\mathfrak{a}$,
and both conjugate $\Gamma_{\mathfrak{a}}$ to translations by multiples
of $w_{\mathfrak{a}}$. 
It follows that $\sigma'_{\mathfrak{a}} \sigma_{\mathfrak{a}}^{-1}$
fixes $\infty$ and normalizes the translation subgroup,
so it belongs to $N$, as claimed.
\end{proof}

\medskip
\noindent
\textbf{Cusp data as invariants.}
The full cusp structure of $M$ is encoded in:
\begin{itemize}
  \item the number of cusps $h$,
  \item the widths $w_{\mathfrak{a}}$,
  \item the choice of scaling matrices $\sigma_{\mathfrak{a}}$.
\end{itemize}
All cusp-dependent constants in later chapters will be functions
of this tuple of data. 
In particular, volume and length formulas
for truncated cusps depend linearly on $w_{\mathfrak{a}}$.

\medskip
\noindent
\textbf{Forward links.}
\begin{itemize}
  \item In Chapter~3, cusp coordinates enter the definition
        of truncated kernels $K_{Y}$.
  \item In Chapter~4, commutator estimates with projectors
        require bounds uniform in $w_{\mathfrak{a}}$.
  \item In Chapter~6, parabolic contributions in the geometric expansion
        involve the scattering coefficients attached to cusps.
\end{itemize}

\medskip
\noindent
\textbf{Backward links.}
\begin{itemize}
  \item From Chapter~1: the motivation for localization depends
        on explicit cusp geometry to regularize divergent integrals.
  \item From the Executive Summary: effective error terms depend on
        making all cusp constants explicit at the start.
\end{itemize}

\medskip
\noindent
\textbf{Audit of Block 1.}
\begin{itemize}
  \item[(A1)] Cusps defined as $\Gamma$-equivalence classes of parabolic fixed points.
  \item[(A2)] Stabilizers identified with infinite cyclic groups.
  \item[(A3)] Scaling matrices fixed, uniqueness clarified.
  \item[(A4)] Canonical cusp neighborhoods constructed.
  \item[(A5)] Lemma on uniqueness of cusp coordinates proved.
  \item[(A6)] Invariants $(h, \{w_{\mathfrak{a}}\}, \{\sigma_{\mathfrak{a}}\})$ declared.
  \item[(A7)] Forward and backward links documented.
\end{itemize}

\medskip
\noindent
\textbf{Conclusion.}
Block~1 has established the structural data of cusps,
fixed once and for all for the entire monograph.
This prepares the ground for truncated cusp regions,
volume and length formulas, and Sobolev estimates in Block~2.

% =====================================================================
% Block 2: Truncation and cusp volumes
% =====================================================================

\subsection{Truncation at height and cusp volumes}

In order to control divergent integrals arising from Eisenstein series
and to define kernels with compact support, it is necessary to introduce
height truncations in cusp neighborhoods. 
We now define these truncations, compute their hyperbolic volumes and
boundary lengths explicitly, and record lemmas that will be used
systematically in later chapters.

\medskip
\noindent
\textbf{Truncated cusp neighborhoods.}
Fix a cusp $\mathfrak{a}$ with width $w_{\mathfrak{a}}$.
For $Y>1$ define the truncated region
\[
  C_{\mathfrak{a}}(Y)
  = \sigma_{\mathfrak{a}}
    \Big\{ x+iy \in \mathbb{H} : 0 \le x < w_{\mathfrak{a}},\ y \ge Y \Big\}.
\]
The projection of $C_{\mathfrak{a}}(Y)$ to $M$
is denoted $\pi(C_{\mathfrak{a}}(Y))$.
It represents the tail of the cusp above height $Y$.

\medskip
\noindent
\textbf{Truncated surface.}
The truncated surface at height $Y$ is defined as
\[
  M(Y) \;=\; M \setminus \bigcup_{\mathfrak{a}} \pi(C_{\mathfrak{a}}(Y)).
\]
For $Y$ sufficiently large, the sets $\pi(C_{\mathfrak{a}}(Y))$
are disjoint, hence $M(Y)$ is compact with geodesic boundary.
The family $\{M(Y)\}_{Y\ge1}$ provides an exhaustion of $M$ by compact sets.

\begin{remark}
Truncations will be introduced twice:
\begin{itemize}
  \item At the geometric level, to restrict kernels to compact domains
        where stationary phase methods apply.
  \item At the spectral level, to regularize integrals of Eisenstein series.
\end{itemize}
Both uses rely on the explicit volume formulas given below.
\end{remark}

\medskip
\noindent
\textbf{Volume formulas.}
Let $\mu(z)=dx\,dy/y^{2}$ denote the hyperbolic area element.
For $0<Y_{1}<Y_{2}$ we compute
\[
  \vol\!\left(
    \sigma_{\mathfrak{a}}
      \{ x+iy : 0\le x < w_{\mathfrak{a}},\ Y_{1} \le y \le Y_{2} \}
  \right)
  = w_{\mathfrak{a}} \int_{Y_{1}}^{Y_{2}} \frac{dy}{y^{2}}
  = w_{\mathfrak{a}}
    \left( \frac{1}{Y_{1}} - \frac{1}{Y_{2}} \right).
\]
Letting $Y_{2}\to\infty$, we obtain
\[
  \vol\big(\pi(C_{\mathfrak{a}}) \setminus \pi(C_{\mathfrak{a}}(Y))\big)
  = \frac{w_{\mathfrak{a}}}{Y}.
\]
Thus the contribution of the cusp tail decreases linearly with $1/Y$.

\medskip
\noindent
\textbf{Boundary length formulas.}
The boundary of $\pi(C_{\mathfrak{a}}(Y))$ is a horocycle at height $Y$.
Its length is
\[
  \operatorname{length}\big(\partial \pi(C_{\mathfrak{a}}(Y))\big)
  = \int_{0}^{w_{\mathfrak{a}}} \frac{dx}{Y}
  = \frac{w_{\mathfrak{a}}}{Y}.
\]

\medskip
\noindent
\textbf{Global sums over cusps.}
Summing over all cusps, we obtain
\[
  \vol(M \setminus M(Y)) \;=\; \sum_{\mathfrak{a}} \frac{w_{\mathfrak{a}}}{Y},
  \qquad
  \operatorname{length}(\partial M(Y)) \;=\; \sum_{\mathfrak{a}} \frac{w_{\mathfrak{a}}}{Y}.
\]
These formulas will serve as input in Chapter~6 for bounding
parabolic contributions to the geometric expansion.

\begin{lemma}[Linear decay of cusp volumes]
There exists a constant $C_{\Gamma}>0$ depending only on cusp widths
such that
\[
  \vol(M \setminus M(Y)) \le \frac{C_{\Gamma}}{Y}, \qquad Y \ge 1.
\]
\end{lemma}

\begin{proof}
The formula above shows that
\[
  \vol(M \setminus M(Y)) = \sum_{\mathfrak{a}} w_{\mathfrak{a}} / Y.
\]
Setting $C_{\Gamma} = \sum_{\mathfrak{a}} w_{\mathfrak{a}}$ proves the claim.
\end{proof}

\medskip
\noindent
\textbf{Injectivity radius in cusp regions.}
For $z=x+iy \in C_{\mathfrak{a}}(Y)$, the group $\Gamma_{\mathfrak{a}}$
acts by translations $x \mapsto x+ n w_{\mathfrak{a}}$.
At height $y$, the hyperbolic displacement is
\[
  d(x+iy,\ (x+w_{\mathfrak{a}})+iy) = \frac{w_{\mathfrak{a}}}{y}.
\]
Hence
\[
  \inj(z) \asymp \min\{1,\ w_{\mathfrak{a}}/y\}.
\]

\begin{lemma}[Lower bound for injectivity radius in cusps]
There exists $c>0$ such that for all $z \in \pi(C_{\mathfrak{a}}(Y))$,
\[
  \inj(z) \;\ge\; c \cdot \min\{1, Y^{-1}\}.
\]
\end{lemma}

\begin{proof}
Since $y \ge Y$, the minimal displacement satisfies
$w_{\mathfrak{a}}/y \ge w_{\min}/Y$, 
where $w_{\min} = \min_{\mathfrak{a}} w_{\mathfrak{a}}>0$.
Taking $c=w_{\min}/2$ gives the claim.
\end{proof}

\medskip
\noindent
\textbf{Forward links.}
\begin{itemize}
  \item To Chapter~3: truncated kernels $K_{Y}$ inherit these bounds.
  \item To Chapter~5: stationary phase analysis requires compact support,
        achieved by working on $M(Y)$.
  \item To Chapter~6: cusp volume and boundary length formulas
        are used in estimating parabolic orbital integrals.
\end{itemize}

\medskip
\noindent
\textbf{Audit of Block 2.}
\begin{itemize}
  \item[(B1)] Definition of truncated cusp regions $C_{\mathfrak{a}}(Y)$ fixed.
  \item[(B2)] Truncated surface $M(Y)$ defined, compactness ensured.
  \item[(B3)] Exact volume and boundary length formulas computed.
  \item[(B4)] Lemma on linear decay of cusp volume proved.
  \item[(B5)] Injectivity radius in cusp regions quantified.
  \item[(B6)] Forward links to later chapters recorded.
\end{itemize}

\medskip
\noindent
\textbf{Conclusion.}
Block~2 has formalized the truncation process at height $Y$,
computed the explicit volume and boundary contributions of cusp regions,
and established injectivity radius bounds.
These results will play a central role in controlling
the parabolic contributions in the trace formula
and in ensuring that constants remain explicit and uniform.

% =====================================================================
% Block 3: Smoothed truncation and tail integrals
% =====================================================================

\subsection{Smoothed truncation and analytic control}

The sharp truncation operator $\Lambda^{Y}$ introduced above is bounded
but not smooth.  
When applied to eigenfunctions or Eisenstein series, it introduces
artificial discontinuities at the boundary horocycles $y=Y$,
which propagate into error terms of order $O(1)$.
For analytic purposes it is essential to replace sharp truncation by
a smoothed operator that preserves compactness while eliminating
boundary discontinuities.  
We also record a lemma on tail integrals that will be used repeatedly
to replace cusp integrals by boundary contributions.

\medskip
\noindent
\textbf{Definition of sharp truncation.}
For $f \in L^{2}(M)$, the sharp truncation operator is
\[
  (\Lambda^{Y} f)(z) =
  \begin{cases}
    f(z), & z \in M(Y), \\
    0, & z \in \pi(C_{\mathfrak{a}}(Y)).
  \end{cases}
\]
This operator is projection onto the compact core $M(Y)$.
While bounded on $L^{2}(M)$, it is not well adapted to Sobolev spaces
and interacts poorly with spectral projectors.

\medskip
\noindent
\textbf{Definition of smoothed truncation.}
Choose a smooth nonnegative cutoff function
$\eta \in C^{\infty}(\mathbb{R}_{\ge0})$ with
\[
  \eta(y) = 1 \;\; (y \le 1), \qquad
  \eta(y) = 0 \;\; (y \ge 2), \qquad
  0 \le \eta(y) \le 1.
\]
For each $Y>1$ define the rescaled cutoff
\[
  \eta_{Y}(y) = \eta\!\left(\frac{y}{Y}\right).
\]
Then $\eta_{Y}$ is supported in $[0,2Y]$ and equals $1$ on $[0,Y]$.

\medskip
\noindent
Define the smoothed truncation operator
\[
  (\Lambda^{Y}_{\mathrm{sm}} f)(z)
  = \eta_{Y}(\Im(\sigma_{\mathfrak{a}}^{-1} z)) \, f(z),
\]
for $z$ in cusp coordinates attached to $\mathfrak{a}$,
and set $\Lambda^{Y}_{\mathrm{sm}} f(z)=f(z)$ elsewhere.
Thus $\Lambda^{Y}_{\mathrm{sm}}$ acts as a multiplicative cutoff,
transitioning smoothly from $1$ to $0$ around height $Y$.

\medskip
\noindent
\textbf{Properties.}
\begin{itemize}
  \item[(P1)] $\Lambda^{Y}_{\mathrm{sm}}$ is bounded on $L^{2}(M)$ with norm $\le 1$.
  \item[(P2)] For every Sobolev space $H^{s}(M)$, the operator norm
              of $\Lambda^{Y}_{\mathrm{sm}}$ is bounded independently of $Y$.
  \item[(P3)] For $f$ supported in $M(Y)$, one has $\Lambda^{Y}_{\mathrm{sm}} f=f$.
  \item[(P4)] The difference $\Lambda^{Y}-\Lambda^{Y}_{\mathrm{sm}}$ is supported
              in a thin horocyclic strip of height $\asymp Y$.
\end{itemize}

\medskip
\noindent
\textbf{Commutators with spectral projectors.}
Let $P_{\lambda,\eta}$ denote the spectral projector
onto eigenvalues in $[\lambda-\eta,\lambda+\eta]$.
Then
\[
  \big\| [\Lambda^{Y}_{\mathrm{sm}}, P_{\lambda,\eta}] \big\|_{L^{2}\to L^{2}}
  \;\le\; C_{\Gamma} \,\frac{\lambda^{1/2}}{Y},
\]
for some constant $C_{\Gamma}$ depending only on cusp data.
This bound will be proved in Chapter~4 using Egorov’s theorem and semiclassical
propagation estimates.  
It suffices here to record that the commutator vanishes as $Y\to\infty$.

\begin{remark}
The error term $\lambda^{1/2}/Y$ shows that $Y$ must grow at least
like $\lambda^{1/2+\epsilon}$ to make the commutator negligible.
This tradeoff between truncation height and spectral localization
is central in later stationary phase arguments.
\end{remark}

\medskip
\noindent
\textbf{Tail integral lemma.}
We now record a simple but frequently used estimate on integrals
over cusp tails.

\begin{lemma}[Tail integrals]\label{lem:tail-integral}
Let $F:(0,\infty)\to\mathbb{C}$ be smooth with bounds
$|y^{k} F^{(k)}(y)| \ll 1$ for all $k \ge 0$.
Then as $Y\to\infty$,
\[
  \int_{Y}^{\infty} F(y)\,\frac{dy}{y^{2}}
  = \frac{F(Y)}{Y} + O\!\left(\frac{1}{Y^{2}}\right).
\]
\end{lemma}

\begin{proof}
Integrating by parts,
\[
  \int_{Y}^{\infty} F(y)\,\frac{dy}{y^{2}}
  = \left[-\frac{F(y)}{y}\right]_{Y}^{\infty}
    + \int_{Y}^{\infty} \frac{F'(y)}{y}\,dy.
\]
The boundary term equals $F(Y)/Y$.
Since $|y F'(y)|\ll 1$, the second integral is $\ll 1/Y^{2}$.
\end{proof}

\begin{corollary}
Integrals over cusp tails can be replaced by boundary evaluations
at $y=Y$ with an error of order $O(Y^{-2})$.
\end{corollary}

\medskip
\noindent
\textbf{Forward links.}
\begin{itemize}
  \item Chapter~4: commutator bounds with $P_{\lambda,\eta}$
        are used in defining localized projectors.
  \item Chapter~5: stationary phase analysis uses tail integral lemma
        to approximate cusp integrals.
  \item Chapter~6: parabolic orbital integrals are expressed
        via logarithmic derivatives of scattering matrices,
        with error controlled by Lemma~\ref{lem:tail-integral}.
\end{itemize}

\medskip
\noindent
\textbf{Audit of Block 3.}
\begin{itemize}
  \item[(B7)] Sharp truncation $\Lambda^{Y}$ defined explicitly.
  \item[(B8)] Smoothed truncation $\Lambda^{Y}_{\mathrm{sm}}$ constructed.
  \item[(B9)] Properties (boundedness, Sobolev stability, support) verified.
  \item[(B10)] Commutator bounds with spectral projectors recorded.
  \item[(B11)] Tail integral lemma proved and corollary stated.
  \item[(B12)] Forward links to later chapters documented.
\end{itemize}

\medskip
\noindent
\textbf{Conclusion.}
Block~3 has introduced the smoothed truncation operator,
proved its stability properties, quantified its commutators with projectors,
and established a tail integral lemma.
These tools guarantee that cusp contributions can be controlled
with explicit and uniform error bounds in the localized trace formula.

% =====================================================================
% Block 4: Eisenstein series and Maass–Selberg relations
% =====================================================================

\subsection{Eisenstein series and parabolic contribution}

The non-compactness of $M=\Gamma\backslash\mathbb{H}$ manifests analytically
through the Eisenstein series attached to each cusp.
These series encode the continuous spectrum of the Laplacian
and play a central role in the trace formula.
We recall their definition, Fourier expansion,
and the Maass–Selberg relations governing their inner products.

\medskip
\noindent
\textbf{Definition of Eisenstein series.}
For each cusp $\mathfrak{a}$ with scaling matrix $\sigma_{\mathfrak{a}}$,
the Eisenstein series is
\[
  E_{\mathfrak{a}}(z,s)
  = \sum_{\gamma \in \Gamma_{\mathfrak{a}}\backslash\Gamma}
    \Im(\sigma_{\mathfrak{a}}^{-1}\gamma z)^{s}, \qquad \Re(s)>1.
\]
This series converges absolutely and uniformly on compact subsets
for $\Re(s)>1$ and extends meromorphically to $\mathbb{C}$,
with functional equation $s \mapsto 1-s$.

\medskip
\noindent
\textbf{Fourier expansion at a cusp.}
At the cusp $\mathfrak{b}$, the Eisenstein series attached to $\mathfrak{a}$
has expansion
\[
  E_{\mathfrak{a}}(\sigma_{\mathfrak{b}} z, s)
  = \delta_{\mathfrak{a}\mathfrak{b}}\, y^{s}
    + \varphi_{\mathfrak{a}\mathfrak{b}}(s)\, y^{1-s}
    + \sum_{n\neq 0}
      \rho_{\mathfrak{a}\mathfrak{b}}(n,s)\,
      \sqrt{y}\, K_{s-1/2}(2\pi |n|y)\, e^{2\pi i n x}.
\]
Here:
\begin{itemize}
  \item $\delta_{\mathfrak{a}\mathfrak{b}}$ is the Kronecker delta.
  \item $\varphi_{\mathfrak{a}\mathfrak{b}}(s)$ is the scattering matrix entry.
  \item $\rho_{\mathfrak{a}\mathfrak{b}}(n,s)$ are Fourier coefficients.
  \item $K_{\nu}$ is the modified Bessel function of the second kind.
\end{itemize}

\medskip
\noindent
\textbf{Scattering matrix.}
The collection of coefficients $\varphi_{\mathfrak{a}\mathfrak{b}}(s)$
forms the scattering matrix $\Phi(s)$.
It is unitary on the critical line $\Re(s)=1/2$ and satisfies
$\Phi(s)\Phi(1-s)=I$.
Its determinant admits the functional equation
\[
  \det \Phi(s)\,\det \Phi(1-s) = 1.
\]

\medskip
\noindent
\textbf{Maass–Selberg relations.}
Let $E_{\mathfrak{a}}(z,s)$ and $E_{\mathfrak{b}}(z,\bar{s})$ be Eisenstein
series attached to cusps $\mathfrak{a},\mathfrak{b}$.
For $\Re(s)=1/2$,
\[
  \langle \Lambda^{Y} E_{\mathfrak{a}}(\cdot,s),\,
          \Lambda^{Y} E_{\mathfrak{b}}(\cdot,\bar{s}) \rangle
  = \delta_{\mathfrak{a}\mathfrak{b}} \,\log Y
    + \Re\!\left(\frac{\varphi'_{\mathfrak{a}\mathfrak{b}}}{\varphi_{\mathfrak{a}\mathfrak{b}}}(s)\right)
    + O(Y^{-1}),
\]
where the error arises from tail integrals as in Lemma~\ref{lem:tail-integral}.
This relation identifies the logarithmic divergence of Eisenstein inner products
with the derivative of the scattering matrix.

\begin{remark}
The formula is exact in the limit $Y\to\infty$ and forms the analytic bridge
between cusp geometry and spectral contributions of the continuous spectrum.
\end{remark}

\medskip
\noindent
\textbf{Parabolic contribution in the trace formula.}
Inserting Eisenstein series into the pre-trace identity and applying the
Maass–Selberg relations yields a parabolic term of the form
\[
  G_{\mathrm{para}}(\lambda,\eta)
  = \sum_{\mathfrak{a},\mathfrak{b}}
    \int_{-\infty}^{\infty}
    h_{\lambda,\eta}(r)\,
    \frac{\varphi'_{\mathfrak{a}\mathfrak{b}}}{\varphi_{\mathfrak{a}\mathfrak{b}}}
      \!\left(\tfrac{1}{2}+ir\right) dr,
\]
where $h_{\lambda,\eta}$ is the spectral test function.
This term reflects the influence of the continuous spectrum
and contributes to error terms of size $O(\lambda^{1-\delta_{1}})$,
with $\delta_{1}$ depending on the spectral gap.

\medskip
\noindent
\textbf{Effect of smoothed truncation.}
Replacing $\Lambda^{Y}$ by $\Lambda^{Y}_{\mathrm{sm}}$ modifies the inner product
by an error $O(Y^{-1})$,
which can be absorbed into the cusp error analysis.
Thus the Maass–Selberg relations hold equally with smoothed truncation,
with constants depending only on cusp geometry.

\medskip
\noindent
\textbf{Forward links.}
\begin{itemize}
  \item Chapter~3: pre-trace identity with cusp terms requires Maass–Selberg.
  \item Chapter~6: parabolic orbital integrals are analyzed using scattering data.
  \item Chapter~7: synthesis of spectral and geometric sides includes
        $G_{\mathrm{para}}(\lambda,\eta)$ explicitly.
\end{itemize}

\medskip
\noindent
\textbf{Audit of Block 4.}
\begin{itemize}
  \item[(B13)] Definition of Eisenstein series recalled.
  \item[(B14)] Fourier expansion at cusps written in detail.
  \item[(B15)] Scattering matrix properties recorded.
  \item[(B16)] Maass–Selberg relations stated with proof sketch.
  \item[(B17)] Parabolic contribution formula derived.
  \item[(B18)] Effect of smoothed truncation noted.
  \item[(B19)] Forward links to later chapters documented.
\end{itemize}

\medskip
\noindent
\textbf{Conclusion.}
Block~4 has introduced Eisenstein series, their Fourier expansions,
and the scattering matrix, culminating in the Maass–Selberg relations.
These yield the parabolic contribution in the trace formula.
Together with Blocks~1–3,
this completes the analytic foundation for handling cusp geometry
and continuous spectrum in the localized trace formula.
