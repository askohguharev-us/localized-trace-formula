% --- Cuspidal structure, truncation, and Eisenstein setup (Chapter 2 block) ---

We record the standard description of cuspidal regions, height truncations,
and the analytic objects attached to the cusps.
All normalizations are consistent with the global glossary and with the
Plancherel conventions used in later chapters.

Let $M=\Gamma\backslash\mathbb{H}$ be a finite-area hyperbolic surface with cusps.
Choose a complete set of cusp representatives
$\{\mathfrak{a}_1,\dots,\mathfrak{a}_h\}$, one for each $\Gamma$–orbit of parabolic fixed points.
For each cusp $\mathfrak{a}$ fix a \emph{scaling matrix} $\sigma_{\mathfrak{a}}\in\mathrm{PSL}_2(\mathbb{R})$
such that $\sigma_{\mathfrak{a}}\infty=\mathfrak{a}$ and
\[
  \sigma_{\mathfrak{a}}^{-1}\Gamma_{\mathfrak{a}}\sigma_{\mathfrak{a}}
  =
  \Big\{ \begin{psmallmatrix}1&w_{\mathfrak{a}} n\\0&1\end{psmallmatrix} : n\in\mathbb{Z} \Big\},
\]
where $\Gamma_{\mathfrak{a}}$ is the stabilizer of $\mathfrak{a}$ in $\Gamma$ and
$w_{\mathfrak{a}}>0$ is the \emph{cusp width}.
The matrix $\sigma_{\mathfrak{a}}$ is unique up to left multiplication by
$\begin{psmallmatrix}1&t\\0&1\end{psmallmatrix}$ with $t\in\mathbb{R}$ and up to right multiplication by
$\begin{psmallmatrix}\pm1&0\\0&\pm1\end{psmallmatrix}$.
All constants below depend only on $\Gamma$ through the data
$\{h, w_{\mathfrak{a}}, \sigma_{\mathfrak{a}}\}$.

\medskip
\noindent\textbf{Standard cusp neighborhood.}
For $Y>0$ define
\[
  \mathcal{C}_{\mathfrak{a}}(Y)
  :=
  \sigma_{\mathfrak{a}}\big\{ x+iy \in \mathbb{H} : 0\le x < w_{\mathfrak{a}},\ y>Y \big\}.
\]
If $Y$ is larger than a $\Gamma$–dependent threshold $Y_0(\Gamma)$,
the sets $\mathcal{C}_{\mathfrak{a}}(Y)$ are embedded and pairwise disjoint.
We write
\[
  M(Y) := M \setminus \bigcup_{\mathfrak{a}} \pi\big(\mathcal{C}_{\mathfrak{a}}(Y)\big),
\]
where $\pi:\mathbb{H}\to M$ is the quotient map.
The boundary of the truncated surface consists of horocycles
$H_{\mathfrak{a}}(Y):=\pi\big(\sigma_{\mathfrak{a}}\{ x+iY : 0\le x < w_{\mathfrak{a}}\}\big)$.

\medskip
\noindent\textbf{Volume and boundary length.}
By direct integration with respect to $d\mu(z)=y^{-2}dx\,dy$ one has
\[
  \vol\big(\pi(\mathcal{C}_{\mathfrak{a}}(Y))\big)
  =
  \int_{Y}^{\infty}\!\int_{0}^{w_{\mathfrak{a}}} \frac{dx\,dy}{y^{2}}
  =
  \frac{w_{\mathfrak{a}}}{Y}.
\]
In particular,
\[
  \vol\Big(M \setminus M(Y)\Big)
  =
  \sum_{\mathfrak{a}} \frac{w_{\mathfrak{a}}}{Y}
  \qquad\text{and}\qquad
  \mathrm{length}\big(H_{\mathfrak{a}}(Y)\big)=\frac{w_{\mathfrak{a}}}{Y}.
\]
Hence the cusp volume and boundary length decay like $Y^{-1}$ with explicit constants.
These identities will be used repeatedly to estimate truncation errors.

\medskip
\noindent\textbf{Horocyclic coordinates.}
Write $z=x+iy$ with respect to the chart $\sigma_{\mathfrak{a}}$.
The hyperbolic distance in the strip $0\le x < w_{\mathfrak{a}}$ satisfies
$d(x_1+i y_1, x_2+i y_2) \ge \big|\log(y_1/y_2)\big|$,
and the metric restricted to a horizontal horocycle $\{y=\mathrm{const}\}$ has length element $dx/y$.
The injectivity radius on $\pi(\mathcal{C}_{\mathfrak{a}}(Y))$ equals $\frac{1}{2}\min(1,Y^{-1})$ up to absolute factors,
uniformly in $x$.

\medskip
\noindent\textbf{Truncation operator.}
Let $\chi_{\mathfrak{a},Y}$ be the characteristic function of $\pi(\mathcal{C}_{\mathfrak{a}}(Y))$.
For a function $f$ on $M$ define the \emph{cusp truncation}
\[
  \Lambda^{Y} f
  :=
  f
  - \sum_{\mathfrak{a}}
    \big(\chi_{\mathfrak{a},Y}\circ\pi\big)
    \cdot \big( f \big)_{\mathfrak{a}}^{\mathrm{ct}},
\]
where $\big( f \big)_{\mathfrak{a}}^{\mathrm{ct}}$ denotes the constant term of $f$ at $\mathfrak{a}$
pulled back to $M$ via $\sigma_{\mathfrak{a}}$.
This is the classical Arthur–Selberg truncation adapted to surfaces.
It removes the divergent constant terms on the cusp regions while leaving the compact part unchanged.
When $f$ is $\Gamma$–invariant and of moderate growth, the truncated function $\Lambda^{Y}f$ is $L^{2}(M)$.

\medskip
\noindent\textbf{Eisenstein series and constant terms.}
For each cusp $\mathfrak{a}$ and $s\in\mathbb{C}$ with $\Re s>1$ define
\[
  E_{\mathfrak{a}}(z,s)
  =
  \sum_{\gamma\in \Gamma_{\mathfrak{a}}\backslash\Gamma}
  \Im\!\big( \sigma_{\mathfrak{a}}^{-1}\gamma z \big)^{s}.
\]
This admits meromorphic continuation to $s\in\mathbb{C}$ and satisfies the functional equation
$E_{\mathfrak{a}}(z,s)=\sum_{\mathfrak{b}}\varphi_{\mathfrak{a}\mathfrak{b}}(s)E_{\mathfrak{b}}(z,1-s)$,
where $\Phi(s)=(\varphi_{\mathfrak{a}\mathfrak{b}}(s))$ is the scattering matrix.
At the cusp $\mathfrak{b}$ one has the Fourier expansion
\[
  E_{\mathfrak{a}}(\sigma_{\mathfrak{b}}(x+iy),s)
  =
  \delta_{\mathfrak{a}\mathfrak{b}}\,y^{s}
  +
  \varphi_{\mathfrak{a}\mathfrak{b}}(s)\,y^{1-s}
  +
  \sum_{n\neq 0} \rho_{\mathfrak{a}\mathfrak{b}}(n,s)\, \sqrt{y}\,K_{s-\frac12}(2\pi|n|y)\,e^{2\pi i n x/w_{\mathfrak{b}}},
\]
valid for $y>0$ and $0\le x < w_{\mathfrak{b}}$.
Here $K_{\nu}$ is the $K$–Bessel function and the coefficients $\rho_{\mathfrak{a}\mathfrak{b}}(n,s)$ are explicit.
At the spectral line $s=\tfrac12+ir$ the constant term is
$y^{1/2+ir}+\varphi_{\mathfrak{a}\mathfrak{b}}(\tfrac12+ir)y^{1/2-ir}$.

\medskip
\noindent\textbf{Maass–Selberg relations (schematic form).}
For $Y\ge Y_0(\Gamma)$ one has
\[
  \int_{M(Y)} E_{\mathfrak{a}}(z,\tfrac12+ir)\,\overline{E_{\mathfrak{b}}(z,\tfrac12+ir)}\,d\mu(z)
  =
  \delta_{\mathfrak{a}\mathfrak{b}}\,\log Y
  + \frac{1}{2}\frac{\varphi_{\mathfrak{a}\mathfrak{b}}'}{\varphi_{\mathfrak{a}\mathfrak{b}}}\Big(\tfrac12+ir\Big)
  + O_{\Gamma}\!\big( Y^{-1} \big),
\]
where the derivative term is shorthand for an explicit expression in scattering data.
All constants are explicit and depend only on $\Gamma$.
We use only the consequences that the $L^{2}$–mass of Eisenstein series on $M(Y)$ is controlled by $\log Y$ and scattering data,
and that the error $O_{\Gamma}(Y^{-1})$ is uniform in $r$ on compact sets.

\medskip
\noindent\textbf{Uniform bounds in cusp regions.}
Fix $Y\ge Y_0(\Gamma)$.
There exists $C_\Gamma>0$ such that for any smooth $f$ on $M$ with moderate growth,
\[
  \int_{\pi(\mathcal{C}_{\mathfrak{a}}(Y))} |f(z)|^{2}\,d\mu(z)
  \le
  C_\Gamma
  \int_{\pi(\mathcal{C}_{\mathfrak{a}}(Y))} \big( |y\,\partial_y f|^{2} + |y\,\partial_x f|^{2} + |f|^{2} \big)\, \frac{dx\,dy}{y^{2}},
\]
uniformly in $Y$.
This is a hyperbolic Hardy–Poincaré inequality and follows from integration by parts on the horocyclic strips.
It implies that Sobolev norms on $M(Y)$ control $L^{2}$–mass leaking into the cusps.

\medskip
\noindent\textbf{Cutoff functions and stability under truncation.}
Let $\eta_Y:\mathbb{R}_{>0}\to[0,1]$ be a smooth profile with
$\eta_Y(y)=0$ for $y\le Y$, $\eta_Y(y)=1$ for $y\ge 2Y$, and $|y^{k}\eta_Y^{(k)}(y)|\le C_k$.
Define $\chi_{\mathfrak{a},Y}^{\mathrm{sm}}(x+iy) := \eta_Y(y)$ on $0\le x<w_{\mathfrak{a}}$ and extend by $\Gamma$–invariance.
Then the smoothed truncation
\[
  \Lambda_{\mathrm{sm}}^{Y} f
  :=
  f - \sum_{\mathfrak{a}}
  \big(\chi_{\mathfrak{a},Y}^{\mathrm{sm}}\circ\pi\big)\cdot \big(f\big)_{\mathfrak{a}}^{\mathrm{ct}}
\]
satisfies $\|\Lambda_{\mathrm{sm}}^{Y} f\|_{H^{s}(M)} \le C_{s,\Gamma}\,\|f\|_{H^{s}(M)}$ for each $s\ge 0$,
with constants independent of $Y$.
In particular, the truncation is stable in Sobolev scales relevant to microlocal arguments.

\medskip
\noindent\textbf{Choice of truncation height.}
Later we will choose a height $Y=Y(\lambda)$ that grows slowly with the spectral parameter,
for instance $Y(\lambda)=\lambda^{\kappa}$ with a small fixed $\kappa>0$.
This choice ensures that
\[
  \vol\big(M\setminus M(Y(\lambda))\big)
  =
  \sum_{\mathfrak{a}}\frac{w_{\mathfrak{a}}}{Y(\lambda)}
  \ll_{\Gamma} \lambda^{-\kappa},
\]
and that boundary integrals over $H_{\mathfrak{a}}(Y(\lambda))$ contribute only to lower-order terms.
Any alternative choice with $Y(\lambda)\to\infty$ and $Y(\lambda)=O(\lambda^{\varepsilon})$ for fixed $\varepsilon>0$
is equally admissible for the purposes of the localized trace identity.

\medskip
\noindent\textbf{Parabolic contribution bookkeeping.}
Write the parabolic term in the geometric side of the trace formula as
\[
  \mathcal{P}_{Y}(k)
  =
  \sum_{\mathfrak{a}} \int_{\pi(\mathcal{C}_{\mathfrak{a}}(Y))} K_{k}(z,z)\,d\mu(z)
  - \sum_{\mathfrak{a}} \int_{H_{\mathfrak{a}}(Y)} B_{k}(z)\, ds(z),
\]
where $K_{k}$ is the kernel associated to a radial test function $k$ and $B_{k}$ is a boundary correction term determined by the constant terms of $K_{k}$.
With our normalizations one has the identity
\[
  \mathcal{P}_{Y}(k)
  =
  \sum_{\mathfrak{a}}
  \Big( \frac{w_{\mathfrak{a}}}{Y}\, \mathcal{M}_{0}(k) + \mathcal{E}_{\mathfrak{a}}(k;Y) \Big),
\]
where $\mathcal{M}_{0}(k)$ is an explicit moment of $k$ and
$\mathcal{E}_{\mathfrak{a}}(k;Y)=O_{\Gamma}\big( Y^{-1-\delta_{0}}\|k\|_{\mathscr{S}} \big)$
for some $\delta_{0}>0$ depending only on $\Gamma$ and on symbol seminorms of $k$.
This isolates the principal $Y^{-1}$–contribution and shows that parabolic tails are effectively controlled.

\medskip
\noindent\textbf{Fourier expansions for $\Gamma$–invariant kernels.}
If $K(z,w)=\sum_{\gamma\in\Gamma}k\!\big(d(z,\gamma w)\big)$ is the $\Gamma$–periodization of a radial kernel,
its restriction to a cusp strip admits a Fourier expansion in $x$ with respect to the lattice of width $w_{\mathfrak{a}}$.
The constant term coincides with the radial average of $k$ against the hyperbolic measure in horocyclic coordinates,
and higher modes are exponentially small in $y$ when $k$ is supported in a geodesic ball of bounded radius.
Consequently, for $y\ge Y$ one has
\[
  K(z,z)
  =
  \mathcal{K}_{0}(y) + O\!\big( e^{-cY} \|k\|_{C^{m}} \big),
\]
with $c,m>0$ explicit and $\mathcal{K}_{0}$ determined by the Selberg transform of $k$.
This observation underlies the boundary correction in $\mathcal{P}_{Y}(k)$.

\medskip
\noindent\textbf{Quantitative tail integrals.}
For any smooth function $F(y)$ with $|y^{j}F^{(j)}(y)|\le C_{j}$ for $0\le j\le 2$ one has
\[
  \int_{Y}^{\infty} F(y)\,\frac{dy}{y^{2}}
  =
  \frac{F(Y)}{Y}
  + O\!\Big( \frac{1}{Y^{2}} \Big),
\]
with an absolute implied constant depending only on the bounds $C_{j}$.
Applied to $F(y)=\int_{0}^{w_{\mathfrak{a}}} G(x+iy)\,dx$ this yields explicit control of tails in cusp integrals.

\medskip
\noindent\textbf{Compatibility with spectral localization.}
All cusp manipulations above commute with the spectral projector $P_{\lambda,\eta}$
up to errors that are uniform in $\lambda$ and $\eta$.
In particular, for the smoothed truncation $\Lambda_{\mathrm{sm}}^{Y}$ one has
\[
  \| \,[P_{\lambda,\eta}, \Lambda_{\mathrm{sm}}^{Y}]\, \|_{L^{2}\to L^{2}}
  \le
  C_{\Gamma}\, Y^{-1}\, \langle \lambda \rangle^{0},
\]
reflecting that $P_{\lambda,\eta}$ is pseudolocal while $\Lambda_{\mathrm{sm}}^{Y}$ depends only on $y$ in cusp charts.
This bound will be sharpened in Chapter~4 using Egorov’s theorem.

\medskip
\noindent\textbf{Summary and forward link.}
We have fixed the cusp coordinates, truncation sets, and scattering-theoretic normalizations.
Volumes and boundary lengths of cusp regions are explicit and scale like $Y^{-1}$ with the constants $w_{\mathfrak{a}}$.
Eisenstein series and their constant terms are normalized compatibly with the Plancherel measure $dr/(4\pi)$.
Truncation operators preserve Sobolev scales uniformly in $Y$ and allow precise bookkeeping of parabolic terms.
These facts will be used in Chapter~3 to define the truncated kernel and in Chapter~6 to separate the parabolic contribution on the geometric side.
