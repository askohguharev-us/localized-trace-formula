% --- Selberg transform and radial kernels (Chapter 2 block) ---

We now review the Selberg transform, which connects radial kernels on
the hyperbolic plane with multipliers acting on the spectral side.
This device is indispensable for formulating and analyzing the trace formula.

\medskip
\noindent\textbf{Radial kernels.}
Let $k:\mathbb{R}_{\ge0}\to\mathbb{C}$ be a smooth compactly supported function,
and define a $\mathrm{PSL}_{2}(\mathbb{R})$–invariant kernel on $\mathbb{H}$ by
\[
  K(z,w) = k\!\big(d(z,w)\big).
\]
Such kernels are central objects in harmonic analysis on $\mathbb{H}$.
They descend to automorphic kernels on $M=\Gamma\backslash\mathbb{H}$ via
\[
  K^{M}(z,w) = \sum_{\gamma\in\Gamma} k\!\big(d(z,\gamma w)\big).
\]
This series converges absolutely for compactly supported $k$ and defines a bounded
operator on $L^{2}(M)$.

\medskip
\noindent\textbf{Spectral action.}
The kernel $K$ acts diagonally on eigenfunctions.
For $\phi$ satisfying $\Delta\phi=\big(\tfrac14+t^{2}\big)\phi$ one has
\[
  \int_{M} K^{M}(z,w)\,\phi(w)\,d\mu(w)
  \;=\; h(t)\,\phi(z),
\]
where $h(t)$ is the \emph{Selberg transform} of $k$.
Thus $K^{M}$ corresponds to the spectral multiplier $h(t)$.

\medskip
\noindent\textbf{Definition of the transform.}
Given $k$, define
\[
  h(t) = \int_{0}^{\infty} k(r)\,\varphi_{t}(r)\,\sinh r\,dr,
\]
where
\[
  \varphi_{t}(r) = \frac{\sin(t r)}{t\sinh r}.
\]
The function $h(t)$ is even, real for real $t$, and rapidly decaying when $k$
is smooth and compactly supported.
The inversion formula states that
\[
  k(r) = \frac{1}{2\pi}\int_{-\infty}^{\infty} h(t)\,\varphi_{t}(r)\,t\tanh(\pi t)\,dt.
\]

\medskip
\noindent\textbf{Analytic properties.}
The transform $h(t)$ extends holomorphically to $t\in\mathbb{C}$
with exponential type determined by the support of $k$.
If $k$ is supported in $[0,R]$, then $h(t)$ grows at most like $e^{R|\,\Im t\,|}$.
This Paley–Wiener type property is essential for localization in the spectral parameter.

\medskip
\noindent\textbf{Relation to spherical functions.}
The kernel $\varphi_{t}(r)$ is the elementary spherical function on $\mathbb{H}$,
satisfying
\[
  \Delta_{z}\,\varphi_{t}(d(z,w)) = \Big(\tfrac14+t^{2}\Big)\varphi_{t}(d(z,w)).
\]
Thus the Selberg transform is nothing but the spherical Fourier transform
for radial functions on $\mathbb{H}$.
This places the trace formula within the general framework of non–Euclidean
harmonic analysis.

\medskip
\noindent\textbf{Automorphic kernels.}
For $f\in C_{c}^{\infty}(\mathbb{R})$ one defines the automorphic kernel
\[
  K_{f}(z,w) = \sum_{\gamma\in\Gamma} k_{f}\!\big(d(z,\gamma w)\big),
\]
where $k_{f}$ is the radial kernel corresponding to spectral multiplier $f(t)$.
Then
\[
  \langle K_{f}\phi,\phi\rangle = f(t)\,\|\phi\|^{2}
\]
for each eigenfunction $\phi$ of spectral parameter $t$.
This correspondence $f \mapsto k_{f}$ is a bijection between admissible multipliers
and radial kernels.

\medskip
\noindent\textbf{Normalization conventions.}
We adopt the convention that the Plancherel measure on the spectral side is
\[
  d\mu_{\mathrm{pl}}(t) = \frac{1}{2\pi}\,t\tanh(\pi t)\,dt,
\]
so that the inversion formula above holds.
This normalization is standard (cf. Selberg~\cite{Selberg1956}, Hejhal~\cite{Hejhal1983}).

\medskip
\noindent\textbf{Consistency check and forward link.}
We have defined the Selberg transform, its inversion formula, and analytic properties.
We fixed the Plancherel normalization and spherical function conventions.
These elements establish the spectral dictionary needed for Chapter~3,
where the truncated kernel will be introduced and analyzed.
