% --- Selberg transform and kernels (Preliminaries, Block: Selberg Transform) ---

\noindent
We now develop the Selberg transform,
the analytic device that relates radial kernels on $\mathbb{H}$ to spectral multipliers.
This transform is the analytic heart of the trace formula,
bridging geometry and spectrum.

\medskip

\noindent\textbf{Radial kernels.}
Let $k:\mathbb{H}\times \mathbb{H}\to\mathbb{C}$ be a $\Gamma$-invariant kernel
depending only on the hyperbolic distance $d(z,w)$.
That is,
\[
  k(z,w) = q(d(z,w)), \qquad q:[0,\infty)\to \mathbb{C}.
\]
The function $q$ is called the radial profile of $k$.

\medskip

\noindent
Examples of radial kernels include:
\begin{itemize}
  \item The heat kernel $K_{t}(z,w)$.
  \item The wave kernel $W_{t}(z,w)$.
  \item Approximate spectral projectors, constructed from smooth cutoffs.
\end{itemize}

\medskip

\noindent\textbf{Selberg transform.}
For a radial function $q$, the Selberg transform is
\[
  h(t) = \int_{0}^{\infty} q(r)\,\varphi_{t}(r)\,\sinh r\,dr,
\]
where $\varphi_{t}(r)$ is the spherical function on $\mathbb{H}$,
normalized by $\varphi_{t}(0)=1$.
Equivalently,
\[
  \varphi_{t}(r) = P_{-1/2+it}(\cosh r),
\]
the Legendre function of the first kind.

\medskip

\noindent
The transform $q\mapsto h$ is invertible:
\[
  q(r) = \frac{1}{4\pi}\int_{-\infty}^{\infty} h(t)\,\varphi_{t}(r)\,t\tanh(\pi t)\,dt.
\]
This inversion formula expresses $q$ in terms of its spectral data $h$.

\medskip

\noindent\textbf{Spectral interpretation.}
Let $\Delta$ denote the Laplacian on $\mathbb{H}$.
Then $h(t)$ is the spectral multiplier corresponding to the operator $K$ with kernel $k$:
\[
  (Kf)(z) = \int_{\mathbb{H}} k(z,w)f(w)\,d\mu(w).
\]
On the spectral side,
\[
  K \phi_{t} = h(t)\phi_{t},
\]
for each eigenfunction $\phi_{t}$ with eigenvalue $1/4+t^{2}$.

\medskip

\noindent\textbf{Normalization of spherical functions.}
The spherical function $\varphi_{t}(r)$ has explicit expression
\[
  \varphi_{t}(r) = \frac{1}{2\pi}\int_{0}^{2\pi} \big(\cosh r + \sinh r\cos\theta\big)^{-1/2-it}\,d\theta.
\]
As $r\to\infty$,
\[
  \varphi_{t}(r) \sim c(t) e^{(it-1/2)r} + c(-t)e^{(-it-1/2)r}.
\]
These asymptotics are crucial for stationary phase estimates.

\medskip

\noindent\textbf{Decay properties.}
If $q(r)$ is compactly supported or sufficiently smooth,
then $h(t)$ decays faster than any polynomial in $t$.
Conversely,
if $h(t)$ is compactly supported,
then $q(r)$ decays exponentially in $r$.
This duality mirrors the Fourier transform.

\medskip

\noindent\textbf{Approximate identities.}
Choose $q(r)$ supported in $[0,R]$ with $\int q(r)\,d\mu(r)=1$.
Then $h(t)\approx 1$ for $|t|\le 1/R$ and decays for larger $t$.
Such kernels act as approximate identities in the spectral decomposition.

\medskip

\noindent\textbf{Example: heat kernel.}
The heat kernel satisfies
\[
  h_{t}(u) = e^{-(1/4+u^{2})t}.
\]
This exhibits Gaussian decay in $u$, reflecting parabolic smoothing.
On the geometric side,
$q(r)$ is given by an explicit integral representation.

\medskip

\noindent\textbf{Example: wave kernel.}
The wave kernel corresponds to
\[
  h_{t}(u) = \cos(tu).
\]
This shows that the wave operator is unitary on $L^{2}$.

\medskip

\noindent\textbf{Example: cutoff projector.}
Let $\psi\in C^{\infty}_{c}(\mathbb{R})$ be even, $\psi(0)=1$.
Define
\[
  h(t) = \psi\!\left(\frac{t-\lambda}{\eta}\right).
\]
Then $q(r)$ is an oscillatory function concentrated near $r\sim \log(\lambda)$.
Such kernels are used in constructing localized spectral projectors.

\medskip

\noindent\textbf{Selberg/Harish-Chandra transform.}
The Selberg transform is a special case of the general Harish-Chandra spherical transform
for $G=PSL_{2}(\mathbb{R})$.
The inversion and Plancherel theorems follow from representation theory of $G$.
In our setting we restrict to radial kernels on $\mathbb{H}$.

\medskip

\noindent\textbf{Plancherel identity.}
For $q$ radial with transform $h$,
\[
  \int_{0}^{\infty} |q(r)|^{2}\sinh r\,dr
  = \frac{1}{4\pi}\int_{-\infty}^{\infty} |h(t)|^{2}\,t\tanh(\pi t)\,dt.
\]
This is the exact analogue of Parseval’s identity for the Fourier transform.

\medskip

\noindent\textbf{Comparison with Euclidean Fourier analysis.}
The table below summarizes the dictionary:

\begin{center}
\begin{tabular}{|c|c|}
\hline
Euclidean $\mathbb{R}^{n}$ & Hyperbolic $\mathbb{H}$ \\
\hline
Exponential $e^{i\xi\cdot x}$ & Spherical function $\varphi_{t}(r)$ \\
Fourier transform $\hat f(\xi)$ & Selberg transform $h(t)$ \\
Plancherel measure $d\xi$ & $(4\pi)^{-1} t\tanh(\pi t)\,dt$ \\
\hline
\end{tabular}
\end{center}

\medskip

\noindent\textbf{Forward links.}
The Selberg transform plays several critical roles in later chapters:
\begin{itemize}
  \item In Chapter~3, kernels are defined via their transforms $h(t)$.
  \item In Chapter~4, projectors $P_{\lambda,\eta}$ are constructed using cutoff transforms.
  \item In Chapter~6, the geometric expansion is organized according to conjugacy classes.
\end{itemize}

\medskip

\noindent\textbf{Estimates for transforms.}
If $q$ is smooth with compact support of radius $R$,
then integration by parts in the inversion formula yields
\[
  h(t) \ll_{N} (1+|t|)^{-N}, \quad \forall N.
\]
Conversely, if $h(t)$ is compactly supported in $[-T,T]$,
then stationary phase gives
\[
  q(r) \ll (1+r)^{-1/2} e^{Tr}.
\]
These dual bounds are indispensable for microlocal arguments.

\medskip

\noindent\textbf{Harish-Chandra’s c-function.}
The asymptotics of $\varphi_{t}(r)$ involve the c-function:
\[
  \varphi_{t}(r) \sim c(t) e^{(it-1/2)r} + c(-t)e^{(-it-1/2)r},
\]
with
\[
  c(t) = \frac{\Gamma(it)}{\Gamma(1/2+it)}.
\]
This function encodes the scattering behavior at infinity.

\medskip

\noindent\textbf{Stationary phase analysis.}
For large $t$, integrals of the form
\[
  h(t) = \int q(r)\,\varphi_{t}(r)\,\sinh r\,dr
\]
are analyzed by stationary phase.
Critical points occur at $r=0$ and at the support edges of $q$.
This yields asymptotics for $h(t)$ depending on the smoothness of $q$.

\medskip

\noindent\textbf{Application: spectral projectors.}
Let $\chi\in C^{\infty}_{c}(\mathbb{R})$.
Define
\[
  h(t) = \chi\!\left(\frac{t-\lambda}{\eta}\right).
\]
The corresponding kernel localizes eigenvalues in the window $[\lambda-\eta,\lambda+\eta]$.
This is the device $P_{\lambda,\eta}$ of Chapter~4.
Error terms depend on the decay of $\widehat{\chi}$.

\medskip

\noindent\textbf{Consistency with microlocal analysis.}
The spectral multipliers $h(t)$ constructed here
will be matched with microlocal parametrices in Chapter~5.
The asymptotic expansions of $\varphi_{t}(r)$ ensure consistency
between spectral and geometric sides.

\medskip

\noindent
% --- End of Part 1 for Selberg transform block ---

% --- Selberg transform and kernels (Preliminaries, Block: Selberg Transform) continued ---

\noindent\textbf{Spectral convolution.}
If $k_{1},k_{2}$ are radial kernels with transforms $h_{1},h_{2}$,
then the convolution kernel $k_{1}\ast k_{2}$ has transform
\[
  h_{1\ast 2}(t) = h_{1}(t)\,h_{2}(t).
\]
Thus the Selberg transform diagonalizes convolution operators,
just as the Fourier transform diagonalizes convolution on $\mathbb{R}^{n}$.

\medskip

\noindent\textbf{Spectral projector operators.}
For a cutoff $h(t)$,
define the operator
\[
  (Kf)(z) = \int_{M} k(z,w)f(w)\,d\mu(w).
\]
If $h(t)$ is approximately the indicator of an interval $[\lambda-\eta,\lambda+\eta]$,
then $K$ approximates the spectral projector $P_{\lambda,\eta}$.
This connection is central to Chapters~4 and 7.

\medskip

\noindent\textbf{Trace identities.}
Let $k$ have Selberg transform $h$.
Then
\[
  \sum_{j} h(t_{j}) + \frac{1}{4\pi}\int_{-\infty}^{\infty} h(t)\,d\mu_{\mathrm{cont}}(t)
  = \sum_{\{\gamma\}} O_{\gamma}(k),
\]
where the left-hand side is spectral,
the right-hand side is geometric.
This is the essence of the Selberg trace formula.

\medskip

\noindent\textbf{Geometric orbital integrals.}
For $\gamma$ hyperbolic with length $\ell(\gamma)$,
\[
  O_{\gamma}(k) = \frac{\ell(\gamma_{0})}{2\sinh(\ell(\gamma)/2)}
  \,q(\ell(\gamma)),
\]
where $\gamma_{0}$ is the primitive element underlying $\gamma$.
For parabolic and elliptic $\gamma$, formulas involve scattering terms and weighted integrals.
These orbital integrals converge because of decay of $q(r)$.

\medskip

\noindent\textbf{Parabolic contribution via scattering.}
For parabolic elements,
the orbital integral is expressed through the logarithmic derivative of the scattering determinant.
Explicitly,
\[
  O_{\mathrm{par}}(k)
  = \frac{1}{4\pi}\int_{-\infty}^{\infty} h(t)\,
    \left(-\frac{\varphi'(1/2+it)}{\varphi(1/2+it)}\right)\,dt,
\]
where $\varphi(s)$ is the determinant of the scattering matrix.

\medskip

\noindent\textbf{Spectral test functions.}
A function $h(t)$ is called an \emph{admissible test function} if:
\begin{itemize}
  \item $h$ is even and holomorphic in a strip $|\Im t|<1/2+\epsilon$,
  \item $h(t)\ll (1+|t|)^{-2-\delta}$ as $|t|\to\infty$.
\end{itemize}
Such conditions guarantee that the kernel $k(z,w)$ is absolutely summable over $\Gamma$,
ensuring convergence of the trace formula.

\medskip

\noindent\textbf{Analytic continuation.}
The Selberg transform $h(t)$ extends holomorphically
in vertical strips determined by the analytic continuation of $\varphi_{t}(r)$.
This allows the use of contour-shifting arguments in spectral analysis.

\medskip

\noindent\textbf{Explicit kernel formulas.}
If $h(t)$ is Gaussian,
\[
  h(t) = e^{-\sigma^{2}t^{2}},
\]
then $q(r)$ has explicit expression involving
\[
  q(r) = \frac{1}{\sqrt{4\pi\sigma^{2}}} \, \frac{1}{\sinh r}
  \exp\!\left(-\frac{r^{2}}{4\sigma^{2}}\right).
\]
This illustrates the localization of $q(r)$ around $r=0$.

\medskip

\noindent\textbf{Paley–Wiener theorem.}
The Selberg transform satisfies a Paley–Wiener theorem:
$h(t)$ is entire of exponential type $R$
if and only if $q(r)$ is supported in $[0,R]$.
This duality is fundamental for constructing compactly supported kernels.

\medskip

\noindent\textbf{Stationary phase for $q(r)$.}
Suppose $h(t)$ is compactly supported and smooth.
Then
\[
  q(r) \asymp r^{-1/2}\cos(Tr-r_{0}) + O(r^{-3/2}),
\]
where $T$ is the radius of support of $h$.
This oscillatory decay governs error terms in the trace formula.

\medskip

\noindent\textbf{Microlocal matching.}
In Chapter~5 we will construct parametrices for the wave kernel.
The asymptotics of $q(r)$ derived from compactly supported $h(t)$
match the oscillatory integrals in microlocal analysis.
This ensures that the spectral and geometric constructions are consistent.

\medskip

\noindent\textbf{Applications to Weyl’s law.}
The local Weyl law can be derived by choosing $h(t)$
to approximate the characteristic function of $[0,\lambda]$.
The corresponding $q(r)$ has decay of order $\lambda^{-1}$.
The trace formula then yields $N(\lambda)\sim C\lambda^{2}$ with effective error bounds.

\medskip

\noindent\textbf{Applications to quantum chaos.}
By choosing localized test functions $h(t)$,
one isolates spectral windows of size $\eta$.
This allows analysis of variance of eigenvalue distributions
and of Fourier coefficients of cusp forms.
Such applications will be detailed in Chapter~8.

\medskip

\noindent\textbf{Connections to automorphic $L$-functions.}
Through the Kuznetsov formula,
which can be viewed as a relative of the Selberg transform,
spectral sums are connected to sums of Kloosterman sums.
The admissibility of test functions $h(t)$ is crucial
for deriving analytic continuation of $L$-functions.

\medskip

\noindent\textbf{Forward links.}
The transform developed here enters:
\begin{itemize}
  \item Chapter~3: kernels $K_{\lambda,\eta}$ are defined via $h(t)$.
  \item Chapter~4: projectors $P_{\lambda,\eta}$ use cutoff $h(t)$.
  \item Chapter~6: geometric side expansion computed via $q(r)$.
  \item Appendix~B: auxiliary estimates on $\varphi_{t}(r)$.
\end{itemize}

\medskip

\noindent\textbf{Audit of Selberg transform block.}
\begin{itemize}
  \item[(S1)] Definition of Selberg transform given, with inversion formula.
  \item[(S2)] Spherical functions normalized; asymptotics recorded.
  \item[(S3)] Plancherel identity stated explicitly.
  \item[(S4)] Examples: heat kernel, wave kernel, cutoff projectors.
  \item[(S5)] Decay estimates derived, stationary phase explained.
  \item[(S6)] Paley–Wiener theorem stated.
  \item[(S7)] Convolution property recorded.
  \item[(S8)] Applications to Weyl’s law and quantum chaos indicated.
  \item[(S9)] Forward links to later chapters provided.
\end{itemize}

\medskip

\noindent\textbf{Conclusion.}
The Selberg transform completes the analytic toolkit of the preliminaries.
Together with geometry and cuspidal analysis,
it furnishes the bridge between kernels and spectral multipliers,
essential for the localized trace formula.

% --- End of Selberg transform block 02b-selberg-transform.tex ---
