% =====================================================================
% File: 02-preliminaries.tex
% Block 1 of 4 — Introduction, Goals, Invariants, Organization
% =====================================================================

\chapter{Preliminaries}

% ---------------------------------------------------------------------
% Chapter orientation
% ---------------------------------------------------------------------
\section*{Orientation and Intent}

This chapter establishes the analytic and geometric setting for the entire
monograph. Its role is foundational: it fixes once and for all the
normalizations, conventions, and persistent invariants that will be used
throughout. Every later construction (kernels, projectors, trace formulae)
will cite this chapter to avoid repetition.

The goals are fourfold:
\begin{enumerate}
  \item To specify the geometric background of hyperbolic surfaces of finite area,
        including cusps, elliptic points, and geodesic structure.
  \item To fix the sign and parameterization of the Laplace--Beltrami operator,
        together with the spectral decomposition of $L^{2}(M)$.
  \item To describe truncation conventions and cusp neighborhoods with explicit
        dependence on cusp widths and scaling matrices.
  \item To declare the Selberg/Harish–Chandra transform conventions,
        Sobolev norms, and dependency rules for error terms and constants.
\end{enumerate}

The intent is not only to provide clarity but also reproducibility:
all constants will be annotated explicitly, and dependencies will be
traceable to invariants of $(M,\Gamma)$ rather than hidden parameters.

% ---------------------------------------------------------------------
\section*{Chapter Goals}

\begin{itemize}[leftmargin=2em]
  \item[\textbf{(G1)}] Specify the hyperbolic geometry:
  the upper half-plane $\mathbb{H}$ with metric
  \[
    ds^2=\frac{dx^2+dy^2}{y^2}, \qquad
    d\mu(z)=\frac{dx\,dy}{y^2},
  \]
  and the quotient $M=\Gamma\backslash\mathbb{H}$ by a cofinite
  Fuchsian group $\Gamma\subset\mathrm{PSL}_2(\mathbb{R})$.

  \item[\textbf{(G2)}] Fix the Laplacian:
  $\Delta \ge 0$ with spectral parameter $\lambda=\tfrac14+r^2$,
  $r\in[0,\infty)$.
  This convention aligns with Selberg~\cite{Selberg1956},
  Hejhal~\cite{Hejhal1983}, and Iwaniec~\cite{Iwaniec2002}.

  \item[\textbf{(G3)}] Define cusp neighborhoods and truncations $M(Y)$,
  with canonical coordinates from scaling matrices
  and explicit relation of cusp widths $w_\mathfrak{a}$ to volume and boundary length.

  \item[\textbf{(G4)}] Record the spectral decomposition on $L^{2}(M)$:
  discrete eigenfunctions $\{\varphi_j\}$ and Eisenstein series
  $E_{\mathfrak{a}}(z,1/2+ir)$, with Plancherel measure $dr/(4\pi)$.

  \item[\textbf{(G5)}] Normalize the Selberg/Harish–Chandra transform
  for radial kernels, consistent with later kernel constructions.

  \item[\textbf{(G6)}] State uniform Sobolev bounds and mapping
  properties required for error analysis and stationary phase.

  \item[\textbf{(G7)}] Establish explicit dependency rules for constants:
  all implied constants are annotated as $O_{\Gamma,\beta}(\cdot)$,
  where $\beta$ is a lower bound for the spectral gap.
\end{itemize}

% ---------------------------------------------------------------------
\section*{Invariants and Dependency Ledger}

\begin{itemize}[leftmargin=2em]
  \item[\textbf{(I1)}] \textit{Surface and volume.}
  $M=\Gamma\backslash\mathbb{H}$ with $\vol(M)\in(0,\infty)$.

  \item[\textbf{(I2)}] \textit{Cuspidal data.}
  Number of cusps $h$, cusp widths $(w_1,\dots,w_h)$,
  and scaling matrices $\sigma_\mathfrak{a}$.
  All cusp-dependent constants are functions of this tuple.

  \item[\textbf{(I3)}] \textit{Spectral gap.}
  $\beta\in(0,\tfrac14]$ denotes a fixed lower bound for
  $r_j^2$ of nontrivial spectrum.
  Constants depending on $\beta$ are written $O_{\Gamma,\beta}(\cdot)$.

  \item[\textbf{(I4)}] \textit{Normalization of Eisenstein series.}
  Continuous spectrum integrated with measure $dr/(4\pi)$.
  Constant terms follow standard scattering conventions.

  \item[\textbf{(I5)}] \textit{Asymptotic notation.}
  $A\lesssim B$ means $A\le C\,B$ for some absolute $C$;
  $A\asymp B$ means both $A\lesssim B$ and $B\lesssim A$.
  Dependence on parameters is annotated:
  $O_X(\cdot)$ indicates dependence on $X$ only.

  \item[\textbf{(I6)}] \textit{Propagation scale.}
  The natural comparison time is $T\asymp\log\lambda$.
  Constants are fixed globally (Chapter~5) and will not affect error orders.
\end{itemize}

% ---------------------------------------------------------------------
\section*{Organization of the Chapter}

The chapter consists of modular blocks, each self-contained and audited:

\begin{itemize}[leftmargin=2em]
  \item \textbf{Geometry:} hyperbolic metric, distance, injectivity, and
        global volume conventions.
        (Input: \texttt{02b-geometry.tex})

  \item \textbf{Cusps and truncation:} cusp neighborhoods, truncation
        domains $M(Y)$, smoothed truncations, Eisenstein series, and
        Maass–Selberg relations.
        (Input: \texttt{02b-cusps.tex})

  \item \textbf{Selberg/Harish–Chandra transform:} definition, inversion,
        admissible test functions, Paley–Wiener duality, stationary phase.
        (Input: \texttt{02b-selberg-transform.tex})
\end{itemize}

Each block closes with an internal consistency check and a forward link
to later chapters. The chapter itself concludes with a global audit
summarizing achieved goals, preserved invariants, and dependency rules.

% =====================================================================
% End of File: 02-preliminaries.tex (Block 1 of 4)
% =====================================================================

% ---------------------------------------------------------------------
% Invariants and dependency ledger
% ---------------------------------------------------------------------
\subsection*{Invariants and Dependency Ledger}

In order to ensure clarity and reproducibility throughout the monograph, 
we fix here the list of persistent invariants and the precise rules by 
which constants and error terms may depend on geometric and spectral data.  
This ledger serves as a reference point for all subsequent chapters.  
Every constant introduced later must be justified in terms of the entries below.

\begin{itemize}[leftmargin=2em]

  \item[\textbf{(I1)}] \textit{Surface and volume.}  
  The basic object is the finite–area hyperbolic surface
  \[
    M = \Gamma \backslash \mathbb{H},
  \]
  where $\Gamma \subset PSL_{2}(\mathbb{R})$ is a cofinite Fuchsian group.  
  Its hyperbolic volume
  \[
    \vol(M) = \int_{\Gamma \backslash \mathbb{H}} \frac{dx\,dy}{y^{2}}
  \]
  is finite and positive, and will appear repeatedly in main terms of Weyl–type laws
  (cf. Selberg~\cite{Selberg1956}, Buser~\cite{Buser1992}).

  \item[\textbf{(I2)}] \textit{Cuspidal data.}  
  The number of cusps is $h \geq 0$.  
  To each cusp $\mathfrak{a}$ we attach a width $w_{\mathfrak{a}}$ and a scaling matrix 
  $\sigma_{\mathfrak{a}} \in PSL_{2}(\mathbb{R})$.  
  All cusp–dependent constants in this monograph are functions of the finite tuple
  \[
    (w_{1}, \dots, w_{h}; \, \sigma_{1}, \dots, \sigma_{h}).
  \]
  This convention guarantees that no hidden dependence on $\lambda$ or on truncation
  parameters is introduced.

  \item[\textbf{(I3)}] \textit{Spectral gap.}  
  Let $\{ \lambda_{j} = \tfrac14 + r_{j}^{2} \}$ denote the discrete spectrum of $\Delta$,
  excluding the trivial eigenvalue $\lambda_{0}=0$.  
  We fix once and for all a lower bound
  \[
    \beta \in (0, \tfrac14]
  \]
  such that $r_{j}^{2} \geq \beta$ for all nontrivial $j$.  
  Constants depending on $\beta$ are explicitly annotated as $O_{\Gamma,\beta}(\cdot)$.  
  This convention follows Iwaniec~\cite{Iwaniec2002} and Sarnak~\cite{Sarnak2004}.

  \item[\textbf{(I4)}] \textit{Normalization of Eisenstein series.}  
  For each cusp $\mathfrak{a}$ we adopt the convention
  \[
    E_{\mathfrak{a}}(z, s) = \sum_{\gamma \in \Gamma_{\mathfrak{a}} \backslash \Gamma}
    \Im(\sigma_{\mathfrak{a}}^{-1}\gamma z)^{s}, \qquad \Re(s) > 1,
  \]
  analytically continued to $\Re(s) \geq \tfrac12$.  
  The Plancherel measure for the continuous spectrum is fixed as $dr/(4\pi)$, and
  constant terms use the scattering matrix normalization of Hejhal~\cite{Hejhal1983}.  

  \item[\textbf{(I5)}] \textit{Asymptotic notation.}  
  We adhere to the following conventions:  
  $A \lesssim B$ means there exists an absolute $C>0$ such that $A \leq C\,B$;  
  $A \asymp B$ means both $A \lesssim B$ and $B \lesssim A$;  
  $O_{X}(\cdot)$ indicates dependence only on the parameter set $X$.  
  Whenever constants depend on $(\Gamma,\beta)$ this is made explicit via
  $O_{\Gamma,\beta}(\cdot)$.  
  No error term is allowed to hide a dependence outside this ledger.

  \item[\textbf{(I6)}] \textit{Propagation scale.}  
  In all microlocal arguments, the natural propagation time is
  \[
    T \asymp \log \lambda,
  \]
  reflecting exponential volume growth of hyperbolic balls.  
  The proportionality constants involved in this definition are fixed in Chapter~5
  and remain universal.  
  They affect no asymptotic bounds but must be recorded to ensure reproducibility.

\end{itemize}

\medskip

\noindent
\textbf{Audit of invariants.}  
The ledger above has been checked for consistency with the global glossary
(Chapter~0).  
It isolates all sources of hidden constants and ensures transparency in later
error bounds.  
Forward links: invariants (I1–I6) will be invoked in Chapters~3–6 for kernel
constructions, in Chapter~7 for error terms in theorems, and in Chapter~8
for explicit arithmetic applications.

% ---------------------------------------------------------------------
% Organization of the chapter (reader-facing map)
% ---------------------------------------------------------------------
\subsection*{Organization of the Chapter}

This chapter consists of three self-contained building blocks.  
Each block develops a distinct analytic or geometric component that will be used 
as input for the localized trace formula.  
The organization is deliberately modular: each block closes with an internal consistency check, 
and the chapter itself concludes with a full audit.  
This ensures that every convention fixed here can be cited later without repetition.

\begin{itemize}[leftmargin=2em]
  \item \textbf{Geometry of the hyperbolic plane and quotient surfaces.}  
  We recall the metric, distance function, geodesic structure, 
  volume growth, injectivity radius, fundamental domains,  
  and analytic kernels such as the heat, wave, and resolvent kernels.  
  This material provides the global background for microlocal analysis 
  and orbital decompositions.  
  (Technical input: \texttt{02b-geometry.tex})

  \item \textbf{Cuspidal structure and height truncation.}  
  We develop canonical cusp neighborhoods via scaling matrices, 
  define truncation domains $M(Y)$, compute explicit volume and boundary formulas, 
  and introduce both sharp and smoothed truncation operators $\Lambda^{Y}$ and $\Lambda^{Y}_{\mathrm{sm}}$.  
  Fourier expansions of Eisenstein series, Maass–Selberg relations, 
  scattering matrices, and effective Parseval identities are also recalled.  
  This block handles the analytic difficulties introduced by noncompactness.  
  (Technical input: \texttt{02b-cusps.tex})

  \item \textbf{Selberg/Harish–Chandra transform for radial kernels.}  
  We define the transform $q(r) \mapsto h(t)$ and its inversion, 
  record normalization of spherical functions,  
  state Plancherel and Paley–Wiener theorems, and develop examples 
  (heat kernel, wave kernel, cutoff projectors).  
  We also explain the role of admissible test functions, 
  convolution properties, orbital integrals, and stationary phase analysis.  
  This block is the analytic bridge between geometry and spectrum.  
  (Technical input: \texttt{02b-selberg-transform.tex})
\end{itemize}

\medskip

\noindent
\textbf{Pedagogical note.}  
The reader can treat this chapter as a reference manual.  
Every time a constant, a volume estimate, or a transform normalization appears in later proofs, 
its justification is located here.  
Forward links to later chapters are indicated in each block, 
and backward consistency with the global glossary (Chapter~0) is verified in the final audit.

\medskip

% ---------------------------------------------------------------------
% Inputs: technical blocks for Chapter 2
% ---------------------------------------------------------------------

% Geometry block: metric, distance, volume, injectivity considerations.
% --- Hyperbolic geometry: metric, volume, and injectivity radius (Chapter 2 block) ---

We recall the geometric background needed for the construction of kernels and
for the estimates in later chapters.
All constants are explicit and depend only on $\Gamma$.

\medskip
\noindent\textbf{Hyperbolic metric and measure.}
On the upper half–plane $\mathbb{H}=\{x+iy : y>0\}$ the hyperbolic metric is
\[
  ds^{2} = \frac{dx^{2}+dy^{2}}{y^{2}},
  \qquad
  d\mu(z) = \frac{dx\,dy}{y^{2}}.
\]
The geodesic distance $d(z,w)$ satisfies
\[
  \cosh d(z,w) = 1 + \frac{|z-w|^{2}}{2\,\Im z\,\Im w}.
\]
The Laplace–Beltrami operator (with negative spectrum convention) is
\[
  \Delta = -y^{2}\Big(\partial_{x}^{2} + \partial_{y}^{2}\Big).
\]
Its spectrum on $M=\Gamma\backslash\mathbb{H}$ consists of discrete eigenvalues
$0=\lambda_{0}<\lambda_{1}\le\lambda_{2}\le\cdots$ tending to infinity,
together with the continuous spectrum $[1/4,\infty)$.

\medskip
\noindent\textbf{Balls and volumes.}
The volume of a hyperbolic ball of radius $R$ is
\[
  \vol B(R) = 2\pi\!\big(\cosh R - 1\big) \asymp e^{R}, \quad (R\to\infty).
\]
For small $R$ one has $\vol B(R)\sim \pi R^{2}$.
These identities allow comparison between hyperbolic and Euclidean scales,
especially in local Sobolev inequalities.

\medskip
\noindent\textbf{Fundamental domains.}
Let $F\subset\mathbb{H}$ be a fundamental domain for $\Gamma$.
It can be chosen as a union of finitely many hyperbolic polygons with geodesic sides,
together with cusp neighborhoods of the form
\[
  \{ x+iy : 0\le x<w,\, y>Y_{0}\}
\]
after application of a scaling matrix.
The boundary of $F$ has finite length modulo cusps.
For practical purposes we work with truncated domains $F(Y)$ where cuspidal regions $y>Y$
are cut off and replaced by boundary horocycles.

\medskip
\noindent\textbf{Injectivity radius.}
For $z\in M$ define the injectivity radius
\[
  \inj(z) = \tfrac12 \inf_{\gamma\in\Gamma\setminus\{\pm I\}} d(z,\gamma z).
\]
It is uniformly positive on compact subsets of $M$.
At cusp neighborhoods, $\inj(z)$ decays like $c/y$ in terms of the imaginary coordinate,
with $c$ depending on the cusp width.
In particular,
\[
  \inj(M(Y)) := \inf_{z\in M(Y)}\inj(z) > 0
\]
for $Y$ large enough, where $M(Y)$ is the surface truncated at height $Y$.
This fact is critical when applying Sobolev inequalities and stationary phase arguments.

\medskip
\noindent\textbf{Geodesic flow and unit tangent bundle.}
Let $SM$ be the unit tangent bundle of $M$.
The geodesic flow $\varphi^{t}:SM\to SM$ preserves the Liouville measure.
On the universal cover $S\mathbb{H}\cong\mathrm{PSL}_{2}(\mathbb{R})$,
$\varphi^{t}$ corresponds to right multiplication by
$\begin{psmallmatrix} e^{t/2} & 0 \\ 0 & e^{-t/2}\end{psmallmatrix}$.
The mixing properties of this flow underlie the semiclassical estimates of Chapter~5.

\medskip
\noindent\textbf{Length spectrum.}
Every primitive closed geodesic $\gamma$ on $M$ corresponds to a hyperbolic conjugacy class in $\Gamma$,
with length $\ell(\gamma)>0$ given by
\[
  2\cosh\!\Big(\tfrac{\ell(\gamma)}{2}\Big) = |\operatorname{tr} \gamma|.
\]
The set $\{\ell(\gamma)\}$ (counted with multiplicity) is the \emph{length spectrum}.
It satisfies the asymptotic
\[
  \#\{\gamma : \ell(\gamma)\le L\} \sim \frac{e^{L}}{L},
  \qquad L\to\infty,
\]
which mirrors the Weyl law for the eigenvalue spectrum.
The length spectrum enters explicitly into the geometric side of the trace formula.

\medskip
\noindent\textbf{Wave kernel on $\mathbb{H}$.}
Let $K_{t}(z,w)$ denote the kernel of $\cos(t\sqrt{\Delta})$ on $\mathbb{H}$.
It depends only on $d(z,w)$ and admits an explicit expression
\[
  K_{t}(d) = -\frac{1}{\pi}\,\frac{\partial}{\partial t}\Big(\frac{\sin(t\sqrt{d^{2}-1})}{\sqrt{d^{2}-1}}\Big), \qquad d>1,
\]
continued analytically elsewhere.
This kernel satisfies finite propagation speed:
$K_{t}(z,w)=0$ if $d(z,w)>|t|$.
On the quotient $M$, the periodization
\[
  K^{M}_{t}(z,w) = \sum_{\gamma\in\Gamma} K_{t}(d(z,\gamma w))
\]
is convergent for each fixed $t$ and smooth in $(z,w)$.
It plays a central role in constructing microlocal projectors.

\medskip
\noindent\textbf{Sobolev inequalities.}
For $f\in C_{c}^{\infty}(M)$ one has the hyperbolic Sobolev inequality
\[
  \|f\|_{\infty} \le C \|f\|_{H^{s}(M)} \qquad (s>1),
\]
with $C$ depending only on $\inj(M)$.
In cusp neighborhoods the inequality holds with $C$ depending on $Y$,
and with the $H^{s}$–norm taken over the truncated region $M(Y)$.
These bounds control the growth of eigenfunctions and Eisenstein series.

\medskip
\noindent\textbf{Consistency check and forward link.}
We have fixed conventions for the hyperbolic metric, measure, and Laplacian sign.
Explicit formulas for ball volumes and injectivity radii have been recorded.
The geodesic flow and length spectrum are normalized compatibly with Selberg’s trace formula.
These foundations will be used in the next block to introduce the Selberg transform,
which links radial kernels to their spectral multipliers.


% Cuspidal analysis: coordinates, truncation, scattering data conventions.
% =====================================================================
% Block 1: Cuspidal structure and scaling matrices
% =====================================================================

\subsection{Cuspidal structure and scaling matrices}

In this block we recall the precise structure of cuspidal regions
of the finite-area hyperbolic surface
\[
  M = \Gamma \backslash \mathbb{H}, \qquad 
  \Gamma \subset PSL_{2}(\mathbb{R})
\]
and fix the notational conventions that will be used throughout
for truncations, coordinates, and scaling matrices. 
The material is classical but must be spelled out carefully,
since all constants and asymptotic estimates in the trace formula
ultimately depend on the cusp data.

\medskip
\noindent
\textbf{Cusps as parabolic fixed points.}
A cusp $\mathfrak{a}$ of $M$ is, by definition, a $\Gamma$-equivalence
class of points in $\mathbb{P}^{1}(\mathbb{R})$ that are fixed by
a parabolic subgroup of $\Gamma$. 
If $\Gamma$ is a cofinite Fuchsian group,
then it has finitely many cusps
\[
  \{\mathfrak{a}_{1}, \dots, \mathfrak{a}_{h}\}.
\]
Each cusp admits an associated stabilizer subgroup
\[
  \Gamma_{\mathfrak{a}} = \{\gamma \in \Gamma : \gamma \mathfrak{a} = \mathfrak{a}\}.
\]
This stabilizer is infinite cyclic, generated by a primitive parabolic
element of $\Gamma$.

\medskip
\noindent
\textbf{Scaling matrices.}
For each cusp $\mathfrak{a}$ we choose a scaling matrix
$\sigma_{\mathfrak{a}} \in PSL_{2}(\mathbb{R})$ satisfying
\[
  \sigma_{\mathfrak{a}}(\infty) = \mathfrak{a},
  \qquad
  \sigma_{\mathfrak{a}}^{-1} \Gamma_{\mathfrak{a}} \sigma_{\mathfrak{a}}
  = \left\{ 
      \pm \begin{pmatrix} 1 & n w_{\mathfrak{a}} \\ 0 & 1 \end{pmatrix}
      : n \in \mathbb{Z} \right\}.
\]
Here $w_{\mathfrak{a}} > 0$ is the \emph{width} of the cusp $\mathfrak{a}$.
The set of widths $\{w_{\mathfrak{a}}\}$ will appear repeatedly
in the volume formulas and error terms of later chapters.

\begin{remark}
The scaling matrix $\sigma_{\mathfrak{a}}$ is uniquely determined up to
left multiplication by the standard unipotent group
\[
  N = \left\{ \begin{pmatrix} 1 & t \\ 0 & 1 \end{pmatrix} : t \in \mathbb{R} \right\},
\]
and right multiplication by $\{\pm I\}$. 
This freedom reflects the fact that cusp coordinates are defined only
up to horizontal translation and sign.
All subsequent constructions are invariant under these changes.
\end{remark}

\medskip
\noindent
\textbf{Canonical cusp neighborhoods.}
Define the standard strip
\[
  \mathcal{S}(w) = \{ x+iy \in \mathbb{H} : 0 \le x < w, \ y > 0 \}.
\]
Then the set
\[
  C_{\mathfrak{a}} = \sigma_{\mathfrak{a}}(\mathcal{S}(w_{\mathfrak{a}}))
\]
is a canonical neighborhood of the cusp $\mathfrak{a}$. 
Coordinates $(x,y)$ on $\mathcal{S}(w_{\mathfrak{a}})$
are called \emph{cusp coordinates} attached to $\mathfrak{a}$.
They identify the stabilizer $\Gamma_{\mathfrak{a}}$
with horizontal translations $x \mapsto x+n w_{\mathfrak{a}}$.

\begin{lemma}[Uniqueness of cusp coordinates]
Let $\sigma_{\mathfrak{a}}$ and $\sigma'_{\mathfrak{a}}$
be two scaling matrices attached to the same cusp $\mathfrak{a}$.
Then there exists $t \in \mathbb{R}$ such that
\[
  \sigma'_{\mathfrak{a}} = \sigma_{\mathfrak{a}}
  \begin{pmatrix} 1 & t \\ 0 & 1 \end{pmatrix}.
\]
Thus cusp coordinates differ only by horizontal translation.
\end{lemma}

\begin{proof}
By construction both $\sigma_{\mathfrak{a}}$ and $\sigma'_{\mathfrak{a}}$
map $\infty$ to $\mathfrak{a}$,
and both conjugate $\Gamma_{\mathfrak{a}}$ to translations by multiples
of $w_{\mathfrak{a}}$. 
It follows that $\sigma'_{\mathfrak{a}} \sigma_{\mathfrak{a}}^{-1}$
fixes $\infty$ and normalizes the translation subgroup,
so it belongs to $N$, as claimed.
\end{proof}

\medskip
\noindent
\textbf{Cusp data as invariants.}
The full cusp structure of $M$ is encoded in:
\begin{itemize}
  \item the number of cusps $h$,
  \item the widths $w_{\mathfrak{a}}$,
  \item the choice of scaling matrices $\sigma_{\mathfrak{a}}$.
\end{itemize}
All cusp-dependent constants in later chapters will be functions
of this tuple of data. 
In particular, volume and length formulas
for truncated cusps depend linearly on $w_{\mathfrak{a}}$.

\medskip
\noindent
\textbf{Forward links.}
\begin{itemize}
  \item In Chapter~3, cusp coordinates enter the definition
        of truncated kernels $K_{Y}$.
  \item In Chapter~4, commutator estimates with projectors
        require bounds uniform in $w_{\mathfrak{a}}$.
  \item In Chapter~6, parabolic contributions in the geometric expansion
        involve the scattering coefficients attached to cusps.
\end{itemize}

\medskip
\noindent
\textbf{Backward links.}
\begin{itemize}
  \item From Chapter~1: the motivation for localization depends
        on explicit cusp geometry to regularize divergent integrals.
  \item From the Executive Summary: effective error terms depend on
        making all cusp constants explicit at the start.
\end{itemize}

\medskip
\noindent
\textbf{Audit of Block 1.}
\begin{itemize}
  \item[(A1)] Cusps defined as $\Gamma$-equivalence classes of parabolic fixed points.
  \item[(A2)] Stabilizers identified with infinite cyclic groups.
  \item[(A3)] Scaling matrices fixed, uniqueness clarified.
  \item[(A4)] Canonical cusp neighborhoods constructed.
  \item[(A5)] Lemma on uniqueness of cusp coordinates proved.
  \item[(A6)] Invariants $(h, \{w_{\mathfrak{a}}\}, \{\sigma_{\mathfrak{a}}\})$ declared.
  \item[(A7)] Forward and backward links documented.
\end{itemize}

\medskip
\noindent
\textbf{Conclusion.}
Block~1 has established the structural data of cusps,
fixed once and for all for the entire monograph.
This prepares the ground for truncated cusp regions,
volume and length formulas, and Sobolev estimates in Block~2.

% =====================================================================
% Block 2: Truncation and cusp volumes
% =====================================================================

\subsection{Truncation at height and cusp volumes}

In order to control divergent integrals arising from Eisenstein series
and to define kernels with compact support, it is necessary to introduce
height truncations in cusp neighborhoods. 
We now define these truncations, compute their hyperbolic volumes and
boundary lengths explicitly, and record lemmas that will be used
systematically in later chapters.

\medskip
\noindent
\textbf{Truncated cusp neighborhoods.}
Fix a cusp $\mathfrak{a}$ with width $w_{\mathfrak{a}}$.
For $Y>1$ define the truncated region
\[
  C_{\mathfrak{a}}(Y)
  = \sigma_{\mathfrak{a}}
    \Big\{ x+iy \in \mathbb{H} : 0 \le x < w_{\mathfrak{a}},\ y \ge Y \Big\}.
\]
The projection of $C_{\mathfrak{a}}(Y)$ to $M$
is denoted $\pi(C_{\mathfrak{a}}(Y))$.
It represents the tail of the cusp above height $Y$.

\medskip
\noindent
\textbf{Truncated surface.}
The truncated surface at height $Y$ is defined as
\[
  M(Y) \;=\; M \setminus \bigcup_{\mathfrak{a}} \pi(C_{\mathfrak{a}}(Y)).
\]
For $Y$ sufficiently large, the sets $\pi(C_{\mathfrak{a}}(Y))$
are disjoint, hence $M(Y)$ is compact with geodesic boundary.
The family $\{M(Y)\}_{Y\ge1}$ provides an exhaustion of $M$ by compact sets.

\begin{remark}
Truncations will be introduced twice:
\begin{itemize}
  \item At the geometric level, to restrict kernels to compact domains
        where stationary phase methods apply.
  \item At the spectral level, to regularize integrals of Eisenstein series.
\end{itemize}
Both uses rely on the explicit volume formulas given below.
\end{remark}

\medskip
\noindent
\textbf{Volume formulas.}
Let $\mu(z)=dx\,dy/y^{2}$ denote the hyperbolic area element.
For $0<Y_{1}<Y_{2}$ we compute
\[
  \vol\!\left(
    \sigma_{\mathfrak{a}}
      \{ x+iy : 0\le x < w_{\mathfrak{a}},\ Y_{1} \le y \le Y_{2} \}
  \right)
  = w_{\mathfrak{a}} \int_{Y_{1}}^{Y_{2}} \frac{dy}{y^{2}}
  = w_{\mathfrak{a}}
    \left( \frac{1}{Y_{1}} - \frac{1}{Y_{2}} \right).
\]
Letting $Y_{2}\to\infty$, we obtain
\[
  \vol\big(\pi(C_{\mathfrak{a}}) \setminus \pi(C_{\mathfrak{a}}(Y))\big)
  = \frac{w_{\mathfrak{a}}}{Y}.
\]
Thus the contribution of the cusp tail decreases linearly with $1/Y$.

\medskip
\noindent
\textbf{Boundary length formulas.}
The boundary of $\pi(C_{\mathfrak{a}}(Y))$ is a horocycle at height $Y$.
Its length is
\[
  \operatorname{length}\big(\partial \pi(C_{\mathfrak{a}}(Y))\big)
  = \int_{0}^{w_{\mathfrak{a}}} \frac{dx}{Y}
  = \frac{w_{\mathfrak{a}}}{Y}.
\]

\medskip
\noindent
\textbf{Global sums over cusps.}
Summing over all cusps, we obtain
\[
  \vol(M \setminus M(Y)) \;=\; \sum_{\mathfrak{a}} \frac{w_{\mathfrak{a}}}{Y},
  \qquad
  \operatorname{length}(\partial M(Y)) \;=\; \sum_{\mathfrak{a}} \frac{w_{\mathfrak{a}}}{Y}.
\]
These formulas will serve as input in Chapter~6 for bounding
parabolic contributions to the geometric expansion.

\begin{lemma}[Linear decay of cusp volumes]
There exists a constant $C_{\Gamma}>0$ depending only on cusp widths
such that
\[
  \vol(M \setminus M(Y)) \le \frac{C_{\Gamma}}{Y}, \qquad Y \ge 1.
\]
\end{lemma}

\begin{proof}
The formula above shows that
\[
  \vol(M \setminus M(Y)) = \sum_{\mathfrak{a}} w_{\mathfrak{a}} / Y.
\]
Setting $C_{\Gamma} = \sum_{\mathfrak{a}} w_{\mathfrak{a}}$ proves the claim.
\end{proof}

\medskip
\noindent
\textbf{Injectivity radius in cusp regions.}
For $z=x+iy \in C_{\mathfrak{a}}(Y)$, the group $\Gamma_{\mathfrak{a}}$
acts by translations $x \mapsto x+ n w_{\mathfrak{a}}$.
At height $y$, the hyperbolic displacement is
\[
  d(x+iy,\ (x+w_{\mathfrak{a}})+iy) = \frac{w_{\mathfrak{a}}}{y}.
\]
Hence
\[
  \inj(z) \asymp \min\{1,\ w_{\mathfrak{a}}/y\}.
\]

\begin{lemma}[Lower bound for injectivity radius in cusps]
There exists $c>0$ such that for all $z \in \pi(C_{\mathfrak{a}}(Y))$,
\[
  \inj(z) \;\ge\; c \cdot \min\{1, Y^{-1}\}.
\]
\end{lemma}

\begin{proof}
Since $y \ge Y$, the minimal displacement satisfies
$w_{\mathfrak{a}}/y \ge w_{\min}/Y$, 
where $w_{\min} = \min_{\mathfrak{a}} w_{\mathfrak{a}}>0$.
Taking $c=w_{\min}/2$ gives the claim.
\end{proof}

\medskip
\noindent
\textbf{Forward links.}
\begin{itemize}
  \item To Chapter~3: truncated kernels $K_{Y}$ inherit these bounds.
  \item To Chapter~5: stationary phase analysis requires compact support,
        achieved by working on $M(Y)$.
  \item To Chapter~6: cusp volume and boundary length formulas
        are used in estimating parabolic orbital integrals.
\end{itemize}

\medskip
\noindent
\textbf{Audit of Block 2.}
\begin{itemize}
  \item[(B1)] Definition of truncated cusp regions $C_{\mathfrak{a}}(Y)$ fixed.
  \item[(B2)] Truncated surface $M(Y)$ defined, compactness ensured.
  \item[(B3)] Exact volume and boundary length formulas computed.
  \item[(B4)] Lemma on linear decay of cusp volume proved.
  \item[(B5)] Injectivity radius in cusp regions quantified.
  \item[(B6)] Forward links to later chapters recorded.
\end{itemize}

\medskip
\noindent
\textbf{Conclusion.}
Block~2 has formalized the truncation process at height $Y$,
computed the explicit volume and boundary contributions of cusp regions,
and established injectivity radius bounds.
These results will play a central role in controlling
the parabolic contributions in the trace formula
and in ensuring that constants remain explicit and uniform.

% =====================================================================
% Block 3: Smoothed truncation and tail integrals
% =====================================================================

\subsection{Smoothed truncation and analytic control}

The sharp truncation operator $\Lambda^{Y}$ introduced above is bounded
but not smooth.  
When applied to eigenfunctions or Eisenstein series, it introduces
artificial discontinuities at the boundary horocycles $y=Y$,
which propagate into error terms of order $O(1)$.
For analytic purposes it is essential to replace sharp truncation by
a smoothed operator that preserves compactness while eliminating
boundary discontinuities.  
We also record a lemma on tail integrals that will be used repeatedly
to replace cusp integrals by boundary contributions.

\medskip
\noindent
\textbf{Definition of sharp truncation.}
For $f \in L^{2}(M)$, the sharp truncation operator is
\[
  (\Lambda^{Y} f)(z) =
  \begin{cases}
    f(z), & z \in M(Y), \\
    0, & z \in \pi(C_{\mathfrak{a}}(Y)).
  \end{cases}
\]
This operator is projection onto the compact core $M(Y)$.
While bounded on $L^{2}(M)$, it is not well adapted to Sobolev spaces
and interacts poorly with spectral projectors.

\medskip
\noindent
\textbf{Definition of smoothed truncation.}
Choose a smooth nonnegative cutoff function
$\eta \in C^{\infty}(\mathbb{R}_{\ge0})$ with
\[
  \eta(y) = 1 \;\; (y \le 1), \qquad
  \eta(y) = 0 \;\; (y \ge 2), \qquad
  0 \le \eta(y) \le 1.
\]
For each $Y>1$ define the rescaled cutoff
\[
  \eta_{Y}(y) = \eta\!\left(\frac{y}{Y}\right).
\]
Then $\eta_{Y}$ is supported in $[0,2Y]$ and equals $1$ on $[0,Y]$.

\medskip
\noindent
Define the smoothed truncation operator
\[
  (\Lambda^{Y}_{\mathrm{sm}} f)(z)
  = \eta_{Y}(\Im(\sigma_{\mathfrak{a}}^{-1} z)) \, f(z),
\]
for $z$ in cusp coordinates attached to $\mathfrak{a}$,
and set $\Lambda^{Y}_{\mathrm{sm}} f(z)=f(z)$ elsewhere.
Thus $\Lambda^{Y}_{\mathrm{sm}}$ acts as a multiplicative cutoff,
transitioning smoothly from $1$ to $0$ around height $Y$.

\medskip
\noindent
\textbf{Properties.}
\begin{itemize}
  \item[(P1)] $\Lambda^{Y}_{\mathrm{sm}}$ is bounded on $L^{2}(M)$ with norm $\le 1$.
  \item[(P2)] For every Sobolev space $H^{s}(M)$, the operator norm
              of $\Lambda^{Y}_{\mathrm{sm}}$ is bounded independently of $Y$.
  \item[(P3)] For $f$ supported in $M(Y)$, one has $\Lambda^{Y}_{\mathrm{sm}} f=f$.
  \item[(P4)] The difference $\Lambda^{Y}-\Lambda^{Y}_{\mathrm{sm}}$ is supported
              in a thin horocyclic strip of height $\asymp Y$.
\end{itemize}

\medskip
\noindent
\textbf{Commutators with spectral projectors.}
Let $P_{\lambda,\eta}$ denote the spectral projector
onto eigenvalues in $[\lambda-\eta,\lambda+\eta]$.
Then
\[
  \big\| [\Lambda^{Y}_{\mathrm{sm}}, P_{\lambda,\eta}] \big\|_{L^{2}\to L^{2}}
  \;\le\; C_{\Gamma} \,\frac{\lambda^{1/2}}{Y},
\]
for some constant $C_{\Gamma}$ depending only on cusp data.
This bound will be proved in Chapter~4 using Egorov’s theorem and semiclassical
propagation estimates.  
It suffices here to record that the commutator vanishes as $Y\to\infty$.

\begin{remark}
The error term $\lambda^{1/2}/Y$ shows that $Y$ must grow at least
like $\lambda^{1/2+\epsilon}$ to make the commutator negligible.
This tradeoff between truncation height and spectral localization
is central in later stationary phase arguments.
\end{remark}

\medskip
\noindent
\textbf{Tail integral lemma.}
We now record a simple but frequently used estimate on integrals
over cusp tails.

\begin{lemma}[Tail integrals]\label{lem:tail-integral}
Let $F:(0,\infty)\to\mathbb{C}$ be smooth with bounds
$|y^{k} F^{(k)}(y)| \ll 1$ for all $k \ge 0$.
Then as $Y\to\infty$,
\[
  \int_{Y}^{\infty} F(y)\,\frac{dy}{y^{2}}
  = \frac{F(Y)}{Y} + O\!\left(\frac{1}{Y^{2}}\right).
\]
\end{lemma}

\begin{proof}
Integrating by parts,
\[
  \int_{Y}^{\infty} F(y)\,\frac{dy}{y^{2}}
  = \left[-\frac{F(y)}{y}\right]_{Y}^{\infty}
    + \int_{Y}^{\infty} \frac{F'(y)}{y}\,dy.
\]
The boundary term equals $F(Y)/Y$.
Since $|y F'(y)|\ll 1$, the second integral is $\ll 1/Y^{2}$.
\end{proof}

\begin{corollary}
Integrals over cusp tails can be replaced by boundary evaluations
at $y=Y$ with an error of order $O(Y^{-2})$.
\end{corollary}

\medskip
\noindent
\textbf{Forward links.}
\begin{itemize}
  \item Chapter~4: commutator bounds with $P_{\lambda,\eta}$
        are used in defining localized projectors.
  \item Chapter~5: stationary phase analysis uses tail integral lemma
        to approximate cusp integrals.
  \item Chapter~6: parabolic orbital integrals are expressed
        via logarithmic derivatives of scattering matrices,
        with error controlled by Lemma~\ref{lem:tail-integral}.
\end{itemize}

\medskip
\noindent
\textbf{Audit of Block 3.}
\begin{itemize}
  \item[(B7)] Sharp truncation $\Lambda^{Y}$ defined explicitly.
  \item[(B8)] Smoothed truncation $\Lambda^{Y}_{\mathrm{sm}}$ constructed.
  \item[(B9)] Properties (boundedness, Sobolev stability, support) verified.
  \item[(B10)] Commutator bounds with spectral projectors recorded.
  \item[(B11)] Tail integral lemma proved and corollary stated.
  \item[(B12)] Forward links to later chapters documented.
\end{itemize}

\medskip
\noindent
\textbf{Conclusion.}
Block~3 has introduced the smoothed truncation operator,
proved its stability properties, quantified its commutators with projectors,
and established a tail integral lemma.
These tools guarantee that cusp contributions can be controlled
with explicit and uniform error bounds in the localized trace formula.

% =====================================================================
% Block 4: Eisenstein series and Maass–Selberg relations
% =====================================================================

\subsection{Eisenstein series and parabolic contribution}

The non-compactness of $M=\Gamma\backslash\mathbb{H}$ manifests analytically
through the Eisenstein series attached to each cusp.
These series encode the continuous spectrum of the Laplacian
and play a central role in the trace formula.
We recall their definition, Fourier expansion,
and the Maass–Selberg relations governing their inner products.

\medskip
\noindent
\textbf{Definition of Eisenstein series.}
For each cusp $\mathfrak{a}$ with scaling matrix $\sigma_{\mathfrak{a}}$,
the Eisenstein series is
\[
  E_{\mathfrak{a}}(z,s)
  = \sum_{\gamma \in \Gamma_{\mathfrak{a}}\backslash\Gamma}
    \Im(\sigma_{\mathfrak{a}}^{-1}\gamma z)^{s}, \qquad \Re(s)>1.
\]
This series converges absolutely and uniformly on compact subsets
for $\Re(s)>1$ and extends meromorphically to $\mathbb{C}$,
with functional equation $s \mapsto 1-s$.

\medskip
\noindent
\textbf{Fourier expansion at a cusp.}
At the cusp $\mathfrak{b}$, the Eisenstein series attached to $\mathfrak{a}$
has expansion
\[
  E_{\mathfrak{a}}(\sigma_{\mathfrak{b}} z, s)
  = \delta_{\mathfrak{a}\mathfrak{b}}\, y^{s}
    + \varphi_{\mathfrak{a}\mathfrak{b}}(s)\, y^{1-s}
    + \sum_{n\neq 0}
      \rho_{\mathfrak{a}\mathfrak{b}}(n,s)\,
      \sqrt{y}\, K_{s-1/2}(2\pi |n|y)\, e^{2\pi i n x}.
\]
Here:
\begin{itemize}
  \item $\delta_{\mathfrak{a}\mathfrak{b}}$ is the Kronecker delta.
  \item $\varphi_{\mathfrak{a}\mathfrak{b}}(s)$ is the scattering matrix entry.
  \item $\rho_{\mathfrak{a}\mathfrak{b}}(n,s)$ are Fourier coefficients.
  \item $K_{\nu}$ is the modified Bessel function of the second kind.
\end{itemize}

\medskip
\noindent
\textbf{Scattering matrix.}
The collection of coefficients $\varphi_{\mathfrak{a}\mathfrak{b}}(s)$
forms the scattering matrix $\Phi(s)$.
It is unitary on the critical line $\Re(s)=1/2$ and satisfies
$\Phi(s)\Phi(1-s)=I$.
Its determinant admits the functional equation
\[
  \det \Phi(s)\,\det \Phi(1-s) = 1.
\]

\medskip
\noindent
\textbf{Maass–Selberg relations.}
Let $E_{\mathfrak{a}}(z,s)$ and $E_{\mathfrak{b}}(z,\bar{s})$ be Eisenstein
series attached to cusps $\mathfrak{a},\mathfrak{b}$.
For $\Re(s)=1/2$,
\[
  \langle \Lambda^{Y} E_{\mathfrak{a}}(\cdot,s),\,
          \Lambda^{Y} E_{\mathfrak{b}}(\cdot,\bar{s}) \rangle
  = \delta_{\mathfrak{a}\mathfrak{b}} \,\log Y
    + \Re\!\left(\frac{\varphi'_{\mathfrak{a}\mathfrak{b}}}{\varphi_{\mathfrak{a}\mathfrak{b}}}(s)\right)
    + O(Y^{-1}),
\]
where the error arises from tail integrals as in Lemma~\ref{lem:tail-integral}.
This relation identifies the logarithmic divergence of Eisenstein inner products
with the derivative of the scattering matrix.

\begin{remark}
The formula is exact in the limit $Y\to\infty$ and forms the analytic bridge
between cusp geometry and spectral contributions of the continuous spectrum.
\end{remark}

\medskip
\noindent
\textbf{Parabolic contribution in the trace formula.}
Inserting Eisenstein series into the pre-trace identity and applying the
Maass–Selberg relations yields a parabolic term of the form
\[
  G_{\mathrm{para}}(\lambda,\eta)
  = \sum_{\mathfrak{a},\mathfrak{b}}
    \int_{-\infty}^{\infty}
    h_{\lambda,\eta}(r)\,
    \frac{\varphi'_{\mathfrak{a}\mathfrak{b}}}{\varphi_{\mathfrak{a}\mathfrak{b}}}
      \!\left(\tfrac{1}{2}+ir\right) dr,
\]
where $h_{\lambda,\eta}$ is the spectral test function.
This term reflects the influence of the continuous spectrum
and contributes to error terms of size $O(\lambda^{1-\delta_{1}})$,
with $\delta_{1}$ depending on the spectral gap.

\medskip
\noindent
\textbf{Effect of smoothed truncation.}
Replacing $\Lambda^{Y}$ by $\Lambda^{Y}_{\mathrm{sm}}$ modifies the inner product
by an error $O(Y^{-1})$,
which can be absorbed into the cusp error analysis.
Thus the Maass–Selberg relations hold equally with smoothed truncation,
with constants depending only on cusp geometry.

\medskip
\noindent
\textbf{Forward links.}
\begin{itemize}
  \item Chapter~3: pre-trace identity with cusp terms requires Maass–Selberg.
  \item Chapter~6: parabolic orbital integrals are analyzed using scattering data.
  \item Chapter~7: synthesis of spectral and geometric sides includes
        $G_{\mathrm{para}}(\lambda,\eta)$ explicitly.
\end{itemize}

\medskip
\noindent
\textbf{Audit of Block 4.}
\begin{itemize}
  \item[(B13)] Definition of Eisenstein series recalled.
  \item[(B14)] Fourier expansion at cusps written in detail.
  \item[(B15)] Scattering matrix properties recorded.
  \item[(B16)] Maass–Selberg relations stated with proof sketch.
  \item[(B17)] Parabolic contribution formula derived.
  \item[(B18)] Effect of smoothed truncation noted.
  \item[(B19)] Forward links to later chapters documented.
\end{itemize}

\medskip
\noindent
\textbf{Conclusion.}
Block~4 has introduced Eisenstein series, their Fourier expansions,
and the scattering matrix, culminating in the Maass–Selberg relations.
These yield the parabolic contribution in the trace formula.
Together with Blocks~1–3,
this completes the analytic foundation for handling cusp geometry
and continuous spectrum in the localized trace formula.


% Selberg/Harish–Chandra transform: test functions and radial kernels.
% --- Selberg transform and radial kernels (Chapter 2 block) ---

We now review the Selberg transform, which connects radial kernels on
the hyperbolic plane with multipliers acting on the spectral side.
This device is indispensable for formulating and analyzing the trace formula.

\medskip
\noindent\textbf{Radial kernels.}
Let $k:\mathbb{R}_{\ge0}\to\mathbb{C}$ be a smooth compactly supported function,
and define a $\mathrm{PSL}_{2}(\mathbb{R})$–invariant kernel on $\mathbb{H}$ by
\[
  K(z,w) = k\!\big(d(z,w)\big).
\]
Such kernels are central objects in harmonic analysis on $\mathbb{H}$.
They descend to automorphic kernels on $M=\Gamma\backslash\mathbb{H}$ via
\[
  K^{M}(z,w) = \sum_{\gamma\in\Gamma} k\!\big(d(z,\gamma w)\big).
\]
This series converges absolutely for compactly supported $k$ and defines a bounded
operator on $L^{2}(M)$.

\medskip
\noindent\textbf{Spectral action.}
The kernel $K$ acts diagonally on eigenfunctions.
For $\phi$ satisfying $\Delta\phi=\big(\tfrac14+t^{2}\big)\phi$ one has
\[
  \int_{M} K^{M}(z,w)\,\phi(w)\,d\mu(w)
  \;=\; h(t)\,\phi(z),
\]
where $h(t)$ is the \emph{Selberg transform} of $k$.
Thus $K^{M}$ corresponds to the spectral multiplier $h(t)$.

\medskip
\noindent\textbf{Definition of the transform.}
Given $k$, define
\[
  h(t) = \int_{0}^{\infty} k(r)\,\varphi_{t}(r)\,\sinh r\,dr,
\]
where
\[
  \varphi_{t}(r) = \frac{\sin(t r)}{t\sinh r}.
\]
The function $h(t)$ is even, real for real $t$, and rapidly decaying when $k$
is smooth and compactly supported.
The inversion formula states that
\[
  k(r) = \frac{1}{2\pi}\int_{-\infty}^{\infty} h(t)\,\varphi_{t}(r)\,t\tanh(\pi t)\,dt.
\]

\medskip
\noindent\textbf{Analytic properties.}
The transform $h(t)$ extends holomorphically to $t\in\mathbb{C}$
with exponential type determined by the support of $k$.
If $k$ is supported in $[0,R]$, then $h(t)$ grows at most like $e^{R|\,\Im t\,|}$.
This Paley–Wiener type property is essential for localization in the spectral parameter.

\medskip
\noindent\textbf{Relation to spherical functions.}
The kernel $\varphi_{t}(r)$ is the elementary spherical function on $\mathbb{H}$,
satisfying
\[
  \Delta_{z}\,\varphi_{t}(d(z,w)) = \Big(\tfrac14+t^{2}\Big)\varphi_{t}(d(z,w)).
\]
Thus the Selberg transform is nothing but the spherical Fourier transform
for radial functions on $\mathbb{H}$.
This places the trace formula within the general framework of non–Euclidean
harmonic analysis.

\medskip
\noindent\textbf{Automorphic kernels.}
For $f\in C_{c}^{\infty}(\mathbb{R})$ one defines the automorphic kernel
\[
  K_{f}(z,w) = \sum_{\gamma\in\Gamma} k_{f}\!\big(d(z,\gamma w)\big),
\]
where $k_{f}$ is the radial kernel corresponding to spectral multiplier $f(t)$.
Then
\[
  \langle K_{f}\phi,\phi\rangle = f(t)\,\|\phi\|^{2}
\]
for each eigenfunction $\phi$ of spectral parameter $t$.
This correspondence $f \mapsto k_{f}$ is a bijection between admissible multipliers
and radial kernels.

\medskip
\noindent\textbf{Normalization conventions.}
We adopt the convention that the Plancherel measure on the spectral side is
\[
  d\mu_{\mathrm{pl}}(t) = \frac{1}{2\pi}\,t\tanh(\pi t)\,dt,
\]
so that the inversion formula above holds.
This normalization is standard (cf. Selberg~\cite{Selberg1956}, Hejhal~\cite{Hejhal1983}).

\medskip
\noindent\textbf{Consistency check and forward link.}
We have defined the Selberg transform, its inversion formula, and analytic properties.
We fixed the Plancherel normalization and spherical function conventions.
These elements establish the spectral dictionary needed for Chapter~3,
where the truncated kernel will be introduced and analyzed.


% ---------------------------------------------------------------------
% Chapter Audit
% ---------------------------------------------------------------------
\subsection*{Chapter Audit}

\noindent\textbf{Scope and coverage.}  
This chapter has fixed the full analytic and geometric setting.  
The model $(\mathbb{H},ds^{2})$ and the quotient surface $M=\Gamma\backslash\mathbb{H}$ 
with finite area have been established.  
Cuspidal neighborhoods and truncations $M(Y)$ are defined with explicit formulas 
for volume and boundary length.  
The Laplacian sign convention $\Delta\ge 0$ and its parametrization $\lambda=\tfrac14+r^{2}$ 
have been fixed globally.

\medskip
\noindent\textbf{Spectral conventions.}  
We recorded the decomposition of $L^{2}(M)$ into discrete eigenfunctions 
$\{\varphi_{j}\}$ and Eisenstein series $E_{\mathfrak a}(z,\tfrac12+ir)$,  
normalized with Plancherel measure $dr/(4\pi)$.  
This convention is consistent with classical sources (Selberg~\cite{Selberg1956},  
Hejhal~\cite{Hejhal1983}, Iwaniec~\cite{Iwaniec2002}).  
Spectral gap parameter $\beta$ was introduced as a persistent invariant.

\medskip
\noindent\textbf{Transform conventions.}  
The Selberg/Harish–Chandra transform of radial kernels was defined,  
with inversion, Plancherel identity, and Paley–Wiener theorem.  
Examples (heat kernel, wave kernel, cutoff projectors) confirm consistency.  
All normalizations are compatible with kernel constructions in Chapter~3 
and microlocal parametrices in Chapter~5.

\medskip
\noindent\textbf{Dependencies of constants.}  
Error terms are denoted $O_{\Gamma,\beta}(\cdot)$ whenever dependence on $\Gamma$ and on 
a spectral gap lower bound is present.  
Cusp-dependent constants depend only on cusp widths $(w_{\mathfrak{a}})$ and scaling matrices $(\sigma_{\mathfrak{a}})$.  
No constant introduced here depends on spectral window parameters $(\lambda,\eta)$;  
those appear only in Chapter~4 with the definition of spectral projectors.

\medskip
\noindent\textbf{Forward links.}  
\begin{itemize}[leftmargin=2em]
  \item To Chapter~3: truncated kernels $K_{Y}$ inherit cusp bounds and geometric conventions fixed here.  
  \item To Chapter~4: spectral projectors $P_{\lambda,\eta}$ use the Selberg transform and commutator estimates 
        with smoothed truncation $\Lambda^{Y}_{\mathrm{sm}}$.  
  \item To Chapter~5: microlocal parametrices rely on Sobolev bounds and Plancherel normalizations declared here.  
  \item To Chapter~6: parabolic contributions are expressed using scattering determinants introduced here.  
\end{itemize}

\medskip
\noindent\textbf{Backward links.}  
\begin{itemize}[leftmargin=2em]
  \item From Chapter~1: motivation for localization required explicit cusp control, formalized in this chapter.  
  \item From Executive Summary: explicit remainder bounds demanded precise constant tracking, fixed here.  
  \item From Global Glossary (Chapter~0): notations for Laplacian, cusp widths, and transforms are consistent.  
\end{itemize}

\medskip
\noindent\textbf{Consistency checks.}  
\begin{itemize}[leftmargin=2em]
  \item Laplacian sign and Plancherel normalization match classical standards (Selberg, Iwaniec, Buser).  
  \item Cusp volumes and lengths verified: $\vol(M\setminus M(Y))=\sum w_{\mathfrak{a}}/Y$.  
  \item Injectivity radius lower bound recorded explicitly, consistent with Buser~\cite{Buser1992}.  
  \item Smoothed truncation $\Lambda^{Y}_{\mathrm{sm}}$ bounded in $L^{2}$ and Sobolev norms,  
        ensuring commutator estimates.  
  \item No hidden dependencies on $(\lambda,\eta)$ occur at this stage.  
\end{itemize}

\medskip
\noindent\textbf{Conclusion of Audit.}  
Chapter~2 (Preliminaries) has established:  
\begin{itemize}[leftmargin=2em]
  \item The hyperbolic geometry framework with explicit metrics, measures, kernels, and volume growth.  
  \item The structure of cusps, truncation operators, Eisenstein expansions, scattering matrices, and tail estimates.  
  \item The Selberg/Harish–Chandra transform with inversion, decay, convolution, and Paley–Wiener theory.  
  \item Explicit constants and dependencies on $(\Gamma,\beta,\{w_{\mathfrak{a}}\},\{\sigma_{\mathfrak{a}}\})$.  
\end{itemize}

All chapter goals (G1–G7) have been met, invariants (I1–I6) preserved,  
and forward/backward links documented.  
This audit certifies Chapter~2 as a complete and reliable foundation 
for the analytic developments of Chapters~3–7.

% --- End of Chapter 2 (Preliminaries) ---
