\section{Preliminaries}

% ---------------------------------------------------------------------
% Chapter Goals
% ---------------------------------------------------------------------
% This chapter fixes the analytic and geometric setting, normalizations,
% and persistent invariants used throughout the monograph. We also record
% explicit dependencies of constants and the precise meaning of all
% asymptotic notations. The intent is to give a self-contained reference
% that can be cited from later chapters without repeating conventions.

\subsection*{Chapter Goals}

\begin{itemize}[leftmargin=2em]
  \item[\textbf{(G1)}] Specify the geometric background: the upper half-plane
  $\mathbb{H}$ with the hyperbolic metric $ds^2=\frac{dx^2+dy^2}{y^2}$,
  the area element $d\mu(z)=\frac{dx\,dy}{y^2}$, and the finite-area
  quotient $M=\Gamma\backslash\mathbb{H}$ by a cofinite Fuchsian group
  $\Gamma\subset\mathrm{PSL}_2(\mathbb{R})$.

  \item[\textbf{(G2)}] Fix the sign and normalization of the Laplace--Beltrami
  operator: $\Delta\ge 0$ on $M$, with spectral parameterization
  $\lambda=\frac14+r^2$, $r\in[0,\infty)$.

  \item[\textbf{(G3)}] Describe cuspidal regions and height truncations
  $M(Y)$, including canonical horocyclic coordinates and the relation of
  $Y$ to cusp widths.

  \item[\textbf{(G4)}] Record the spectral decomposition on $L^2(M)$, including
  discrete eigenfunctions $\{\varphi_j\}$ and Eisenstein series
  $E_{\mathfrak a}(z,\frac12+ir)$ for each cusp $\mathfrak a$ with
  Plancherel measure $dr/(4\pi)$.

  \item[\textbf{(G5)}] Define the Selberg/Harish–Chandra transform of radial
  kernels, with conventions compatible with Chapters~3–6.

  \item[\textbf{(G6)}] Establish uniform Sobolev bounds and function space
  norms used in error estimates and stationary phase arguments.

  \item[\textbf{(G7)}] State explicit dependency rules for constants and error
  terms; fix the notation $O_{\Gamma,\beta}(\cdot)$, where $\beta$ denotes
  a lower bound for the spectral gap.
\end{itemize}

% ---------------------------------------------------------------------
% Invariants and dependency ledger
% ---------------------------------------------------------------------
\subsection*{Invariants and dependency ledger}

\begin{itemize}[leftmargin=2em]
  \item[\textbf{(I1)}] \textit{Surface and volume.} $M=\Gamma\backslash\mathbb{H}$,
  $\vol(M)\in(0,\infty)$.

  \item[\textbf{(I2)}] \textit{Cuspidal data.} Number of cusps $h$, cusp widths
  $(w_1,\dots,w_h)$, standard scaling matrices; all cusp-dependent constants
  are functions of this tuple only.

  \item[\textbf{(I3)}] \textit{Spectral gap.} $\beta\in(0,\tfrac14]$ is a fixed
  lower bound on $r_j^2$ for nontrivial discrete spectrum; constants that may
  depend on $\beta$ are written $O_{\Gamma,\beta}(\cdot)$.

  \item[\textbf{(I4)}] \textit{Normalization of Eisenstein series.} Continuous
  spectrum is integrated with measure $dr/(4\pi)$; constant terms use the
  standard scattering matrix conventions.

  \item[\textbf{(I5)}] \textit{Asymptotic notation.}
  $A\lesssim B$ means $A\le C\,B$ for some absolute $C>0$;
  $A\asymp B$ means $A\lesssim B$ and $B\lesssim A$;
  $O_X(\cdot)$ indicates dependence on $X$ only.

  \item[\textbf{(I6)}] \textit{Propagation scale.} The geometric comparison
  time is $T\asymp\log\lambda$; proportionality constants are fixed once
  and for all (Chapter~5) and do not affect the order bounds.
\end{itemize}

% ---------------------------------------------------------------------
% Organization of the chapter (reader-facing map)
% ---------------------------------------------------------------------
\subsection*{Organization}

This chapter consists of self-contained building blocks that establish the
geometric, spectral, and transform-theoretic infrastructure. Each block
closes with an internal consistency check; the chapter concludes with an
audit summarizing dependencies and forward links.

\begin{itemize}[leftmargin=2em]
  \item Geometry of the hyperbolic plane and of the quotient surface:
  metric, distance, injectivity considerations, volume conventions.
  (Input: \texttt{02b-geometry.tex})

  \item Cuspidal structure and height truncation:
  model neighborhoods, cusp parameters, and truncation sets $M(Y)$.
  (Input: \texttt{02b-cusps.tex})

  \item Selberg/Harish–Chandra transform for radial kernels:
  normalization, inversion, and compatibility with later kernels.
  (Input: \texttt{02b-selberg-transform.tex})

  % Future extension points (to be added as separate blocks as needed):
  % \item Uniform Sobolev bounds and mapping properties.
  % \item Spectral decomposition bookkeeping and Plancherel identities.
\end{itemize}

% ---------------------------------------------------------------------
% Inputs: technical blocks for Chapter 2
% ---------------------------------------------------------------------

% Geometry block: metric, distance, volume, injectivity considerations.
% --- Hyperbolic geometry: metric, volume, and injectivity radius (Chapter 2 block) ---

We recall the geometric background needed for the construction of kernels and
for the estimates in later chapters.
All constants are explicit and depend only on $\Gamma$.

\medskip
\noindent\textbf{Hyperbolic metric and measure.}
On the upper half–plane $\mathbb{H}=\{x+iy : y>0\}$ the hyperbolic metric is
\[
  ds^{2} = \frac{dx^{2}+dy^{2}}{y^{2}},
  \qquad
  d\mu(z) = \frac{dx\,dy}{y^{2}}.
\]
The geodesic distance $d(z,w)$ satisfies
\[
  \cosh d(z,w) = 1 + \frac{|z-w|^{2}}{2\,\Im z\,\Im w}.
\]
The Laplace–Beltrami operator (with negative spectrum convention) is
\[
  \Delta = -y^{2}\Big(\partial_{x}^{2} + \partial_{y}^{2}\Big).
\]
Its spectrum on $M=\Gamma\backslash\mathbb{H}$ consists of discrete eigenvalues
$0=\lambda_{0}<\lambda_{1}\le\lambda_{2}\le\cdots$ tending to infinity,
together with the continuous spectrum $[1/4,\infty)$.

\medskip
\noindent\textbf{Balls and volumes.}
The volume of a hyperbolic ball of radius $R$ is
\[
  \vol B(R) = 2\pi\!\big(\cosh R - 1\big) \asymp e^{R}, \quad (R\to\infty).
\]
For small $R$ one has $\vol B(R)\sim \pi R^{2}$.
These identities allow comparison between hyperbolic and Euclidean scales,
especially in local Sobolev inequalities.

\medskip
\noindent\textbf{Fundamental domains.}
Let $F\subset\mathbb{H}$ be a fundamental domain for $\Gamma$.
It can be chosen as a union of finitely many hyperbolic polygons with geodesic sides,
together with cusp neighborhoods of the form
\[
  \{ x+iy : 0\le x<w,\, y>Y_{0}\}
\]
after application of a scaling matrix.
The boundary of $F$ has finite length modulo cusps.
For practical purposes we work with truncated domains $F(Y)$ where cuspidal regions $y>Y$
are cut off and replaced by boundary horocycles.

\medskip
\noindent\textbf{Injectivity radius.}
For $z\in M$ define the injectivity radius
\[
  \inj(z) = \tfrac12 \inf_{\gamma\in\Gamma\setminus\{\pm I\}} d(z,\gamma z).
\]
It is uniformly positive on compact subsets of $M$.
At cusp neighborhoods, $\inj(z)$ decays like $c/y$ in terms of the imaginary coordinate,
with $c$ depending on the cusp width.
In particular,
\[
  \inj(M(Y)) := \inf_{z\in M(Y)}\inj(z) > 0
\]
for $Y$ large enough, where $M(Y)$ is the surface truncated at height $Y$.
This fact is critical when applying Sobolev inequalities and stationary phase arguments.

\medskip
\noindent\textbf{Geodesic flow and unit tangent bundle.}
Let $SM$ be the unit tangent bundle of $M$.
The geodesic flow $\varphi^{t}:SM\to SM$ preserves the Liouville measure.
On the universal cover $S\mathbb{H}\cong\mathrm{PSL}_{2}(\mathbb{R})$,
$\varphi^{t}$ corresponds to right multiplication by
$\begin{psmallmatrix} e^{t/2} & 0 \\ 0 & e^{-t/2}\end{psmallmatrix}$.
The mixing properties of this flow underlie the semiclassical estimates of Chapter~5.

\medskip
\noindent\textbf{Length spectrum.}
Every primitive closed geodesic $\gamma$ on $M$ corresponds to a hyperbolic conjugacy class in $\Gamma$,
with length $\ell(\gamma)>0$ given by
\[
  2\cosh\!\Big(\tfrac{\ell(\gamma)}{2}\Big) = |\operatorname{tr} \gamma|.
\]
The set $\{\ell(\gamma)\}$ (counted with multiplicity) is the \emph{length spectrum}.
It satisfies the asymptotic
\[
  \#\{\gamma : \ell(\gamma)\le L\} \sim \frac{e^{L}}{L},
  \qquad L\to\infty,
\]
which mirrors the Weyl law for the eigenvalue spectrum.
The length spectrum enters explicitly into the geometric side of the trace formula.

\medskip
\noindent\textbf{Wave kernel on $\mathbb{H}$.}
Let $K_{t}(z,w)$ denote the kernel of $\cos(t\sqrt{\Delta})$ on $\mathbb{H}$.
It depends only on $d(z,w)$ and admits an explicit expression
\[
  K_{t}(d) = -\frac{1}{\pi}\,\frac{\partial}{\partial t}\Big(\frac{\sin(t\sqrt{d^{2}-1})}{\sqrt{d^{2}-1}}\Big), \qquad d>1,
\]
continued analytically elsewhere.
This kernel satisfies finite propagation speed:
$K_{t}(z,w)=0$ if $d(z,w)>|t|$.
On the quotient $M$, the periodization
\[
  K^{M}_{t}(z,w) = \sum_{\gamma\in\Gamma} K_{t}(d(z,\gamma w))
\]
is convergent for each fixed $t$ and smooth in $(z,w)$.
It plays a central role in constructing microlocal projectors.

\medskip
\noindent\textbf{Sobolev inequalities.}
For $f\in C_{c}^{\infty}(M)$ one has the hyperbolic Sobolev inequality
\[
  \|f\|_{\infty} \le C \|f\|_{H^{s}(M)} \qquad (s>1),
\]
with $C$ depending only on $\inj(M)$.
In cusp neighborhoods the inequality holds with $C$ depending on $Y$,
and with the $H^{s}$–norm taken over the truncated region $M(Y)$.
These bounds control the growth of eigenfunctions and Eisenstein series.

\medskip
\noindent\textbf{Consistency check and forward link.}
We have fixed conventions for the hyperbolic metric, measure, and Laplacian sign.
Explicit formulas for ball volumes and injectivity radii have been recorded.
The geodesic flow and length spectrum are normalized compatibly with Selberg’s trace formula.
These foundations will be used in the next block to introduce the Selberg transform,
which links radial kernels to their spectral multipliers.


% Cuspidal analysis: coordinates, truncation, scattering data conventions.
% --- Cuspidal structure, truncation, and Eisenstein setup (Chapter 2 block) ---

We record the standard description of cuspidal regions, height truncations,
and the analytic objects attached to the cusps.
All normalizations are consistent with the global glossary and with the
Plancherel conventions used in later chapters.

Let $M=\Gamma\backslash\mathbb{H}$ be a finite-area hyperbolic surface with cusps.
Choose a complete set of cusp representatives
$\{\mathfrak{a}_1,\dots,\mathfrak{a}_h\}$, one for each $\Gamma$–orbit of parabolic fixed points.
For each cusp $\mathfrak{a}$ fix a \emph{scaling matrix} $\sigma_{\mathfrak{a}}\in\mathrm{PSL}_2(\mathbb{R})$
such that $\sigma_{\mathfrak{a}}\infty=\mathfrak{a}$ and
\[
  \sigma_{\mathfrak{a}}^{-1}\Gamma_{\mathfrak{a}}\sigma_{\mathfrak{a}}
  =
  \Big\{ \begin{psmallmatrix}1&w_{\mathfrak{a}} n\\0&1\end{psmallmatrix} : n\in\mathbb{Z} \Big\},
\]
where $\Gamma_{\mathfrak{a}}$ is the stabilizer of $\mathfrak{a}$ in $\Gamma$ and
$w_{\mathfrak{a}}>0$ is the \emph{cusp width}.
The matrix $\sigma_{\mathfrak{a}}$ is unique up to left multiplication by
$\begin{psmallmatrix}1&t\\0&1\end{psmallmatrix}$ with $t\in\mathbb{R}$ and up to right multiplication by
$\begin{psmallmatrix}\pm1&0\\0&\pm1\end{psmallmatrix}$.
All constants below depend only on $\Gamma$ through the data
$\{h, w_{\mathfrak{a}}, \sigma_{\mathfrak{a}}\}$.

\medskip
\noindent\textbf{Standard cusp neighborhood.}
For $Y>0$ define
\[
  \mathcal{C}_{\mathfrak{a}}(Y)
  :=
  \sigma_{\mathfrak{a}}\big\{ x+iy \in \mathbb{H} : 0\le x < w_{\mathfrak{a}},\ y>Y \big\}.
\]
If $Y$ is larger than a $\Gamma$–dependent threshold $Y_0(\Gamma)$,
the sets $\mathcal{C}_{\mathfrak{a}}(Y)$ are embedded and pairwise disjoint.
We write
\[
  M(Y) := M \setminus \bigcup_{\mathfrak{a}} \pi\big(\mathcal{C}_{\mathfrak{a}}(Y)\big),
\]
where $\pi:\mathbb{H}\to M$ is the quotient map.
The boundary of the truncated surface consists of horocycles
$H_{\mathfrak{a}}(Y):=\pi\big(\sigma_{\mathfrak{a}}\{ x+iY : 0\le x < w_{\mathfrak{a}}\}\big)$.

\medskip
\noindent\textbf{Volume and boundary length.}
By direct integration with respect to $d\mu(z)=y^{-2}dx\,dy$ one has
\[
  \vol\big(\pi(\mathcal{C}_{\mathfrak{a}}(Y))\big)
  =
  \int_{Y}^{\infty}\!\int_{0}^{w_{\mathfrak{a}}} \frac{dx\,dy}{y^{2}}
  =
  \frac{w_{\mathfrak{a}}}{Y}.
\]
In particular,
\[
  \vol\Big(M \setminus M(Y)\Big)
  =
  \sum_{\mathfrak{a}} \frac{w_{\mathfrak{a}}}{Y}
  \qquad\text{and}\qquad
  \mathrm{length}\big(H_{\mathfrak{a}}(Y)\big)=\frac{w_{\mathfrak{a}}}{Y}.
\]
Hence the cusp volume and boundary length decay like $Y^{-1}$ with explicit constants.
These identities will be used repeatedly to estimate truncation errors.

\medskip
\noindent\textbf{Horocyclic coordinates.}
Write $z=x+iy$ with respect to the chart $\sigma_{\mathfrak{a}}$.
The hyperbolic distance in the strip $0\le x < w_{\mathfrak{a}}$ satisfies
$d(x_1+i y_1, x_2+i y_2) \ge \big|\log(y_1/y_2)\big|$,
and the metric restricted to a horizontal horocycle $\{y=\mathrm{const}\}$ has length element $dx/y$.
The injectivity radius on $\pi(\mathcal{C}_{\mathfrak{a}}(Y))$ equals $\frac{1}{2}\min(1,Y^{-1})$ up to absolute factors,
uniformly in $x$.

\medskip
\noindent\textbf{Truncation operator.}
Let $\chi_{\mathfrak{a},Y}$ be the characteristic function of $\pi(\mathcal{C}_{\mathfrak{a}}(Y))$.
For a function $f$ on $M$ define the \emph{cusp truncation}
\[
  \Lambda^{Y} f
  :=
  f
  - \sum_{\mathfrak{a}}
    \big(\chi_{\mathfrak{a},Y}\circ\pi\big)
    \cdot \big( f \big)_{\mathfrak{a}}^{\mathrm{ct}},
\]
where $\big( f \big)_{\mathfrak{a}}^{\mathrm{ct}}$ denotes the constant term of $f$ at $\mathfrak{a}$
pulled back to $M$ via $\sigma_{\mathfrak{a}}$.
This is the classical Arthur–Selberg truncation adapted to surfaces.
It removes the divergent constant terms on the cusp regions while leaving the compact part unchanged.
When $f$ is $\Gamma$–invariant and of moderate growth, the truncated function $\Lambda^{Y}f$ is $L^{2}(M)$.

\medskip
\noindent\textbf{Eisenstein series and constant terms.}
For each cusp $\mathfrak{a}$ and $s\in\mathbb{C}$ with $\Re s>1$ define
\[
  E_{\mathfrak{a}}(z,s)
  =
  \sum_{\gamma\in \Gamma_{\mathfrak{a}}\backslash\Gamma}
  \Im\!\big( \sigma_{\mathfrak{a}}^{-1}\gamma z \big)^{s}.
\]
This admits meromorphic continuation to $s\in\mathbb{C}$ and satisfies the functional equation
$E_{\mathfrak{a}}(z,s)=\sum_{\mathfrak{b}}\varphi_{\mathfrak{a}\mathfrak{b}}(s)E_{\mathfrak{b}}(z,1-s)$,
where $\Phi(s)=(\varphi_{\mathfrak{a}\mathfrak{b}}(s))$ is the scattering matrix.
At the cusp $\mathfrak{b}$ one has the Fourier expansion
\[
  E_{\mathfrak{a}}(\sigma_{\mathfrak{b}}(x+iy),s)
  =
  \delta_{\mathfrak{a}\mathfrak{b}}\,y^{s}
  +
  \varphi_{\mathfrak{a}\mathfrak{b}}(s)\,y^{1-s}
  +
  \sum_{n\neq 0} \rho_{\mathfrak{a}\mathfrak{b}}(n,s)\, \sqrt{y}\,K_{s-\frac12}(2\pi|n|y)\,e^{2\pi i n x/w_{\mathfrak{b}}},
\]
valid for $y>0$ and $0\le x < w_{\mathfrak{b}}$.
Here $K_{\nu}$ is the $K$–Bessel function and the coefficients $\rho_{\mathfrak{a}\mathfrak{b}}(n,s)$ are explicit.
At the spectral line $s=\tfrac12+ir$ the constant term is
$y^{1/2+ir}+\varphi_{\mathfrak{a}\mathfrak{b}}(\tfrac12+ir)y^{1/2-ir}$.

\medskip
\noindent\textbf{Maass–Selberg relations (schematic form).}
For $Y\ge Y_0(\Gamma)$ one has
\[
  \int_{M(Y)} E_{\mathfrak{a}}(z,\tfrac12+ir)\,\overline{E_{\mathfrak{b}}(z,\tfrac12+ir)}\,d\mu(z)
  =
  \delta_{\mathfrak{a}\mathfrak{b}}\,\log Y
  + \frac{1}{2}\frac{\varphi_{\mathfrak{a}\mathfrak{b}}'}{\varphi_{\mathfrak{a}\mathfrak{b}}}\Big(\tfrac12+ir\Big)
  + O_{\Gamma}\!\big( Y^{-1} \big),
\]
where the derivative term is shorthand for an explicit expression in scattering data.
All constants are explicit and depend only on $\Gamma$.
We use only the consequences that the $L^{2}$–mass of Eisenstein series on $M(Y)$ is controlled by $\log Y$ and scattering data,
and that the error $O_{\Gamma}(Y^{-1})$ is uniform in $r$ on compact sets.

\medskip
\noindent\textbf{Uniform bounds in cusp regions.}
Fix $Y\ge Y_0(\Gamma)$.
There exists $C_\Gamma>0$ such that for any smooth $f$ on $M$ with moderate growth,
\[
  \int_{\pi(\mathcal{C}_{\mathfrak{a}}(Y))} |f(z)|^{2}\,d\mu(z)
  \le
  C_\Gamma
  \int_{\pi(\mathcal{C}_{\mathfrak{a}}(Y))} \big( |y\,\partial_y f|^{2} + |y\,\partial_x f|^{2} + |f|^{2} \big)\, \frac{dx\,dy}{y^{2}},
\]
uniformly in $Y$.
This is a hyperbolic Hardy–Poincaré inequality and follows from integration by parts on the horocyclic strips.
It implies that Sobolev norms on $M(Y)$ control $L^{2}$–mass leaking into the cusps.

\medskip
\noindent\textbf{Cutoff functions and stability under truncation.}
Let $\eta_Y:\mathbb{R}_{>0}\to[0,1]$ be a smooth profile with
$\eta_Y(y)=0$ for $y\le Y$, $\eta_Y(y)=1$ for $y\ge 2Y$, and $|y^{k}\eta_Y^{(k)}(y)|\le C_k$.
Define $\chi_{\mathfrak{a},Y}^{\mathrm{sm}}(x+iy) := \eta_Y(y)$ on $0\le x<w_{\mathfrak{a}}$ and extend by $\Gamma$–invariance.
Then the smoothed truncation
\[
  \Lambda_{\mathrm{sm}}^{Y} f
  :=
  f - \sum_{\mathfrak{a}}
  \big(\chi_{\mathfrak{a},Y}^{\mathrm{sm}}\circ\pi\big)\cdot \big(f\big)_{\mathfrak{a}}^{\mathrm{ct}}
\]
satisfies $\|\Lambda_{\mathrm{sm}}^{Y} f\|_{H^{s}(M)} \le C_{s,\Gamma}\,\|f\|_{H^{s}(M)}$ for each $s\ge 0$,
with constants independent of $Y$.
In particular, the truncation is stable in Sobolev scales relevant to microlocal arguments.

\medskip
\noindent\textbf{Choice of truncation height.}
Later we will choose a height $Y=Y(\lambda)$ that grows slowly with the spectral parameter,
for instance $Y(\lambda)=\lambda^{\kappa}$ with a small fixed $\kappa>0$.
This choice ensures that
\[
  \vol\big(M\setminus M(Y(\lambda))\big)
  =
  \sum_{\mathfrak{a}}\frac{w_{\mathfrak{a}}}{Y(\lambda)}
  \ll_{\Gamma} \lambda^{-\kappa},
\]
and that boundary integrals over $H_{\mathfrak{a}}(Y(\lambda))$ contribute only to lower-order terms.
Any alternative choice with $Y(\lambda)\to\infty$ and $Y(\lambda)=O(\lambda^{\varepsilon})$ for fixed $\varepsilon>0$
is equally admissible for the purposes of the localized trace identity.

\medskip
\noindent\textbf{Parabolic contribution bookkeeping.}
Write the parabolic term in the geometric side of the trace formula as
\[
  \mathcal{P}_{Y}(k)
  =
  \sum_{\mathfrak{a}} \int_{\pi(\mathcal{C}_{\mathfrak{a}}(Y))} K_{k}(z,z)\,d\mu(z)
  - \sum_{\mathfrak{a}} \int_{H_{\mathfrak{a}}(Y)} B_{k}(z)\, ds(z),
\]
where $K_{k}$ is the kernel associated to a radial test function $k$ and $B_{k}$ is a boundary correction term determined by the constant terms of $K_{k}$.
With our normalizations one has the identity
\[
  \mathcal{P}_{Y}(k)
  =
  \sum_{\mathfrak{a}}
  \Big( \frac{w_{\mathfrak{a}}}{Y}\, \mathcal{M}_{0}(k) + \mathcal{E}_{\mathfrak{a}}(k;Y) \Big),
\]
where $\mathcal{M}_{0}(k)$ is an explicit moment of $k$ and
$\mathcal{E}_{\mathfrak{a}}(k;Y)=O_{\Gamma}\big( Y^{-1-\delta_{0}}\|k\|_{\mathscr{S}} \big)$
for some $\delta_{0}>0$ depending only on $\Gamma$ and on symbol seminorms of $k$.
This isolates the principal $Y^{-1}$–contribution and shows that parabolic tails are effectively controlled.

\medskip
\noindent\textbf{Fourier expansions for $\Gamma$–invariant kernels.}
If $K(z,w)=\sum_{\gamma\in\Gamma}k\!\big(d(z,\gamma w)\big)$ is the $\Gamma$–periodization of a radial kernel,
its restriction to a cusp strip admits a Fourier expansion in $x$ with respect to the lattice of width $w_{\mathfrak{a}}$.
The constant term coincides with the radial average of $k$ against the hyperbolic measure in horocyclic coordinates,
and higher modes are exponentially small in $y$ when $k$ is supported in a geodesic ball of bounded radius.
Consequently, for $y\ge Y$ one has
\[
  K(z,z)
  =
  \mathcal{K}_{0}(y) + O\!\big( e^{-cY} \|k\|_{C^{m}} \big),
\]
with $c,m>0$ explicit and $\mathcal{K}_{0}$ determined by the Selberg transform of $k$.
This observation underlies the boundary correction in $\mathcal{P}_{Y}(k)$.

\medskip
\noindent\textbf{Quantitative tail integrals.}
For any smooth function $F(y)$ with $|y^{j}F^{(j)}(y)|\le C_{j}$ for $0\le j\le 2$ one has
\[
  \int_{Y}^{\infty} F(y)\,\frac{dy}{y^{2}}
  =
  \frac{F(Y)}{Y}
  + O\!\Big( \frac{1}{Y^{2}} \Big),
\]
with an absolute implied constant depending only on the bounds $C_{j}$.
Applied to $F(y)=\int_{0}^{w_{\mathfrak{a}}} G(x+iy)\,dx$ this yields explicit control of tails in cusp integrals.

\medskip
\noindent\textbf{Compatibility with spectral localization.}
All cusp manipulations above commute with the spectral projector $P_{\lambda,\eta}$
up to errors that are uniform in $\lambda$ and $\eta$.
In particular, for the smoothed truncation $\Lambda_{\mathrm{sm}}^{Y}$ one has
\[
  \| \,[P_{\lambda,\eta}, \Lambda_{\mathrm{sm}}^{Y}]\, \|_{L^{2}\to L^{2}}
  \le
  C_{\Gamma}\, Y^{-1}\, \langle \lambda \rangle^{0},
\]
reflecting that $P_{\lambda,\eta}$ is pseudolocal while $\Lambda_{\mathrm{sm}}^{Y}$ depends only on $y$ in cusp charts.
This bound will be sharpened in Chapter~4 using Egorov’s theorem.

\medskip
\noindent\textbf{Summary and forward link.}
We have fixed the cusp coordinates, truncation sets, and scattering-theoretic normalizations.
Volumes and boundary lengths of cusp regions are explicit and scale like $Y^{-1}$ with the constants $w_{\mathfrak{a}}$.
Eisenstein series and their constant terms are normalized compatibly with the Plancherel measure $dr/(4\pi)$.
Truncation operators preserve Sobolev scales uniformly in $Y$ and allow precise bookkeeping of parabolic terms.
These facts will be used in Chapter~3 to define the truncated kernel and in Chapter~6 to separate the parabolic contribution on the geometric side.


% Selberg/Harish–Chandra transform: test functions and radial kernels.
% =====================================================================
% File: 02b-selberg-transform.tex
% Block 1 of 4 — Definition and analytic foundations
% =====================================================================

\section{The Selberg Transform}
\label{sec:selberg-transform}

The Selberg transform provides the analytic bridge
between test functions on the hyperbolic plane
and spectral multipliers of the Laplacian.
It is indispensable in the formulation and proof
of the Selberg trace formula and its localized refinements.

\subsection{Radial kernels and spherical functions}

Let $\mathbb{H} = \{z = x+iy \in \mathbb{C} : y>0\}$ be the upper half-plane
with the hyperbolic metric $ds^2 = y^{-2}(dx^2+dy^2)$
and Laplacian
\[
  \Delta \;=\; -y^2\!\left(\frac{\partial^2}{\partial x^2}
  + \frac{\partial^2}{\partial y^2}\right).
\]

For $r \in \mathbb{R}$,
define the \emph{spherical function} $\varphi_r$
as the unique radial eigenfunction of $\Delta$ on $\mathbb{H}$ satisfying
\[
  \Delta \varphi_r \;=\; \Bigl(\tfrac{1}{4}+r^2\Bigr)\varphi_r,
  \qquad \varphi_r(0) = 1,
\]
where radial means $\varphi_r(z)$ depends only on the hyperbolic distance $d(z,i)$.

Explicitly, one has
\[
  \varphi_r(\cosh u) = \frac{\sin(ru)}{r\sinh u}, \qquad u\ge 0,
\]
where $u=d(z,i)$.
These functions form the kernel of the Harish–Chandra transform on $\mathbb{H}$.

\subsection{Definition of the Selberg transform}

Let $k \colon [0,\infty) \to \mathbb{C}$ be an even, compactly supported,
or rapidly decaying test function.
The associated \emph{radial kernel} on $\mathbb{H}$ is
\[
  K(z,w) \;=\; k\!\bigl(d(z,w)\bigr).
\]

The \emph{Selberg transform} of $k$ is defined by
\[
  h(r) \;=\; \int_{0}^{\infty} k(u)\,\varphi_r(\cosh u)\,\sinh u\,du.
\]
This integral converges absolutely if $k$ has sufficient decay.
It produces an even, holomorphic function $h(r)$
in a strip around the real axis.

\begin{remark}
The normalization ensures that the operator with kernel $K(z,w)$
acts as a spectral multiplier with eigenvalue $h(r)$
on Laplace eigenfunctions of eigenvalue $1/4+r^2$.
\end{remark}

\subsection{Basic analytic properties}

The transform $k \mapsto h$ enjoys the following properties:

\begin{lemma}[Decay and regularity]
\label{lem:selberg-decay}
If $k$ is $C^\infty$ and compactly supported, then $h(r)$
is an entire function of $r$ with rapid decay along horizontal strips.
If $k$ is only rapidly decaying, then $h(r)$ extends holomorphically
to a vertical strip and satisfies polynomial bounds outside it.
\end{lemma}

\begin{proof}
The integrand $k(u)\varphi_r(\cosh u)\sinh u$ is smooth in $(u,r)$.
For compactly supported $k$, one may differentiate under the integral sign
to obtain holomorphy in $r$.
Repeated integration by parts against the oscillatory kernel $\sin(ru)$
gives rapid decay as $|r|\to\infty$.
\end{proof}

\begin{lemma}[Plancherel formula]
\label{lem:selberg-plancherel}
For radial kernels $k_1,k_2$ with Selberg transforms $h_1,h_2$,
one has the Plancherel identity
\[
  \int_{\mathbb{H}} k_1(d(z,i))\,\overline{k_2(d(z,i))}\,d\mu(z)
  \;=\; \frac{1}{4\pi} \int_{-\infty}^\infty
      h_1(r)\,\overline{h_2(r)}\, r\tanh(\pi r)\,dr,
\]
where $d\mu(z)=y^{-2}dx\,dy$ is the hyperbolic volume measure.
\end{lemma}

\begin{proof}
This is the classical spherical Fourier transform on $\mathbb{H}$.
It follows from the representation theory of $PSL(2,\mathbb{R})$
or directly from the spectral decomposition of $L^2(\mathbb{H})$.
See \cite{Helgason1984, Iwaniec2002}.
\end{proof}

\medskip
\noindent
\textbf{Forward links.}
\begin{itemize}
  \item Section~\ref{sec:selberg-pretrace} (next): use of $h(r)$
        in the spectral expansion of the automorphic kernel.
  \item Section~\ref{sec:projector}: construction of localized projectors
        by choosing $k$ with specific scaling properties.
\end{itemize}

\medskip
\noindent
\textbf{Audit of Block 1.}
\begin{itemize}
  \item[(B30)] Geometry of $\mathbb{H}$ and Laplacian fixed.
  \item[(B31)] Spherical functions defined and normalized.
  \item[(B32)] Selberg transform $h(r)$ defined with correct measure.
  \item[(B33)] Analytic properties (holomorphy, decay) proved.
  \item[(B34)] Plancherel identity recorded with references.
  \item[(B35)] Forward links established.
\end{itemize}

% =====================================================================
% File: 02b-selberg-transform.tex
% Block 2 of 4 — Asymptotics, normalization, and examples
% =====================================================================

\subsection{Asymptotics of spherical functions}

The spherical functions $\varphi_r$ admit explicit asymptotic expansions,
which play a central role in stationary phase analysis.

\begin{lemma}[Asymptotic expansion]
For fixed $r \in \mathbb{R}$, as $u \to \infty$,
\[
  \varphi_r(\cosh u) \;=\;
    c(r) e^{(ir-\tfrac12)u}
    + c(-r) e^{(-ir-\tfrac12)u},
\]
where
\[
  c(r) = \frac{\Gamma(ir)}{\Gamma(\tfrac12+ir)}.
\]
\end{lemma}

\begin{proof}
This is a standard consequence of the hypergeometric representation
$\varphi_r(\cosh u) = {}_2F_{1}(1/2+ir,1/2-ir;1;\!-\sinh^2(u/2))$.
Expanding near infinity yields the stated form.
See Helgason~\cite{Helgason1984}.
\end{proof}

\begin{remark}
The coefficients $c(r)$ constitute the Harish--Chandra $c$-function.
They encode scattering properties at infinity and enter directly into
the Plancherel measure $r\tanh(\pi r)\,dr$.
\end{remark}

\subsection{Normalization of the Selberg transform}

Different authors use different normalizations for $\varphi_r$ and $h(r)$.
In this monograph we adopt the convention:

\begin{itemize}
  \item The spectral parameter is $\lambda = \tfrac14 + r^2$, $r\in \mathbb{R}$.
  \item The Plancherel measure is $\tfrac{1}{4\pi} r\tanh(\pi r)\,dr$.
  \item Spherical functions $\varphi_r$ are normalized by $\varphi_r(0)=1$.
\end{itemize}

With this choice, the inversion formula reads
\[
  k(u) = \frac{1}{4\pi}\int_{-\infty}^\infty
          h(r)\,\varphi_r(\cosh u)\,r\tanh(\pi r)\,dr.
\]

\begin{remark}
This convention is compatible with Iwaniec~\cite{Iwaniec2002},
Hejhal~\cite{Hejhal1983}, and Buser~\cite{Buser1992}.
It guarantees consistency across Chapters~3–6,
where spectral projectors and trace formulas rely on identical scaling.
\end{remark}

\subsection{Examples of Selberg transforms}

\paragraph{(i) Heat kernel.}
The hyperbolic heat kernel has spectral transform
\[
  h_t(r) = e^{-(1/4+r^2)t}, \qquad t>0.
\]
This follows from solving $\partial_t u = -(\Delta-1/4)u$ in the spectral domain.
It shows Gaussian decay in $r$, reflecting parabolic smoothing.

\paragraph{(ii) Wave kernel.}
The hyperbolic wave kernel corresponds to
\[
  h_t(r) = \cos(rt),
\]
giving exact unitary propagation in the spectral side.
On the geometric side, this kernel localizes along geodesic spheres.

\paragraph{(iii) Approximate spectral projectors.}
Let $\chi\in C_c^\infty(\mathbb{R})$ be even with $\chi(0)=1$.
Define
\[
  h(r) = \chi\!\left(\frac{r-\lambda}{\eta}\right),
\]
for central parameter $\lambda$ and window $\eta$.
The inverse transform yields kernels oscillating at scale $\log \lambda$.
These approximate projectors $P_{\lambda,\eta}$ are central in Chapter~4.

\subsection{Decay properties and Paley--Wiener duality}

\begin{lemma}[Decay of Selberg transform]
If $k(u)$ is smooth with compact support of radius $R$,
then for each $N>0$,
\[
  h(r) \;\ll_N\; (1+|r|)^{-N}.
\]
\end{lemma}

\begin{proof}
Repeated integration by parts in
$h(r)=\int_0^R k(u)\varphi_r(\cosh u)\sinh u\,du$
against the oscillatory factor $e^{iru}$ yields the bound.
\end{proof}

\begin{lemma}[Exponential support duality]
If $h(r)$ is supported in $[-T,T]$,
then $k(u)$ decays exponentially:
\[
  k(u) \;\ll\; e^{-Tu}(1+u)^{-1/2}.
\]
\end{lemma}

\begin{proof}
This is the Paley--Wiener theorem for the spherical transform.
It follows by stationary phase analysis of the inversion integral.
\end{proof}

\subsection*{Forward links}

\begin{itemize}
  \item These examples provide the analytic building blocks
        for kernel construction in Chapter~3.
  \item The Paley--Wiener theorem governs localization of projectors
        in Chapter~4 and stationary phase arguments in Chapter~5.
\end{itemize}

\subsection*{Audit of Block 2}

\begin{itemize}
  \item[(B36)] Asymptotics of $\varphi_r$ fixed, $c$-function recorded.
  \item[(B37)] Normalization conventions declared and compared with literature.
  \item[(B38)] Heat, wave, and projector kernels given as canonical examples.
  \item[(B39)] Decay and Paley--Wiener duality proved.
  \item[(B40)] Forward links to kernel construction and microlocal analysis.
\end{itemize}

% =====================================================================
% File: 02b-selberg-transform.tex
% Block 3 of 4 — Convolution, spectral projectors, and trace identities
% =====================================================================

\subsection{Convolution and diagonalization}

\begin{lemma}[Convolution property]
Let $k_1, k_2$ be radial kernels on $\mathbb{H}$ with transforms $h_1, h_2$.
Then their convolution
\[
  (k_1 * k_2)(u) \;=\; \int_{\mathbb{H}} k_1(d(z,w))\,k_2(d(w,o))\,d\mu(w)
\]
has Selberg transform
\[
  h_{1*2}(r) = h_1(r)\,h_2(r).
\]
\end{lemma}

\begin{proof}
This follows from the fact that spherical functions $\varphi_r$ diagonalize radial convolution operators, just as plane waves diagonalize convolution in Euclidean space. See Helgason~\cite{Helgason1984}.
\end{proof}

\begin{remark}
Thus the Selberg transform provides a complete diagonalization of the algebra of $\Gamma$-invariant radial kernels, exactly analogous to the Fourier transform on $\mathbb{R}^n$.
\end{remark}

\subsection{Spectral projectors}

Given an admissible cutoff $h(r)$, define the operator
\[
  (Kf)(z) = \int_{M} k(d(z,w)) f(w)\,d\mu(w).
\]
On the spectral side,
\[
  K \phi_j = h(r_j) \phi_j,
\]
for each eigenfunction $\phi_j$ of $\Delta$ with eigenvalue $\tfrac14+r_j^2$.

\paragraph{Localized projectors.}
For $\lambda>0$ and window $\eta>0$, define
\[
  h(r) = \chi\!\left(\frac{r-\sqrt{\lambda-1/4}}{\eta}\right),
\]
with $\chi\in C_c^\infty(\mathbb{R})$ even, $\chi(0)=1$.
Then $K$ acts as an approximate spectral projector $P_{\lambda,\eta}$ onto eigenfunctions with eigenvalues near $\lambda$.
This construction will be used in Chapter~4 to prove Weyl-type asymptotics.

\subsection{Trace identities and the Selberg trace formula}

\begin{theorem}[Abstract Selberg trace identity]
Let $k$ be a radial kernel with Selberg transform $h$.
Then
\[
  \sum_j h(r_j)
  + \frac{1}{4\pi}\int_{-\infty}^\infty h(r)\,d\mu_{\mathrm{cont}}(r)
  \;=\; \sum_{\{\gamma\}} O_\gamma(k),
\]
where
\begin{itemize}
  \item the left-hand side is the spectral side (discrete plus continuous spectrum),
  \item the right-hand side is the geometric side: orbital integrals over conjugacy classes $\{\gamma\}$.
\end{itemize}
\end{theorem}

\begin{proof}[Sketch]
Insert the kernel $k$ into the pre-trace formula
\[
  \sum_{\gamma\in \Gamma} k(z,\gamma z) = \sum_j h(r_j)\,|\phi_j(z)|^2 + \frac{1}{4\pi}\int_{-\infty}^\infty h(r)\,|E(z,1/2+ir)|^2 dr.
\]
Integrating over $M$ and interchanging sums/integrals yields the stated identity.
\end{proof}

\begin{remark}
This identity is the backbone of the spectral theory of automorphic forms. It reduces the analysis of eigenvalue distributions to orbital integrals of kernels.
\end{remark}

\subsection{Geometric orbital integrals}

The orbital integrals $O_\gamma(k)$ depend on the conjugacy class of $\gamma\in\Gamma$:

\begin{itemize}
  \item \textbf{Elliptic elements.} Fix a point in $\mathbb{H}$. Their contribution involves weighted orbital integrals with stabilizers. Rare for torsion-free $\Gamma$.
  \item \textbf{Parabolic elements.} Correspond to cusps. Their contribution is expressed via the logarithmic derivative of the scattering determinant.
  \item \textbf{Hyperbolic elements.} Correspond to closed geodesics. Contribution given by
  \[
    O_\gamma(k) = \frac{\ell(\gamma_0)}{2\sinh(\ell(\gamma)/2)}\,q(\ell(\gamma)),
  \]
  where $\gamma_0$ is the primitive element underlying $\gamma$.
\end{itemize}

\begin{remark}
Here $q(\ell(\gamma))$ is the radial profile of $k$ evaluated at the length of the closed geodesic associated to $\gamma$.
\end{remark}

\subsection*{Forward links}

\begin{itemize}
  \item To Chapter~4: localized projectors $P_{\lambda,\eta}$ are defined precisely with these kernels.
  \item To Chapter~6: orbital integrals provide the backbone of the geometric expansion of the trace formula.
  \item To Appendix~B: detailed asymptotics of $\varphi_r$ refine orbital integral computations.
\end{itemize}

\subsection*{Audit of Block 3}

\begin{itemize}
  \item[(B41)] Convolution property proved, algebra diagonalized.
  \item[(B42)] Spectral projectors constructed from cutoff transforms.
  \item[(B43)] Trace identity stated and sketched.
  \item[(B44)] Orbital integrals classified by elliptic, parabolic, hyperbolic types.
  \item[(B45)] Forward links to projectors, expansions, appendices recorded.
\end{itemize}

% =====================================================================
% File: 02b-selberg-transform.tex
% Block 4 of 4 — Admissible test functions, Paley–Wiener, and audit
% =====================================================================

\subsection{Admissible test functions and convergence}

\begin{definition}[Admissible test functions]
A function $h:\mathbb{R}\to\mathbb{C}$ is admissible for the Selberg trace formula if:
\begin{itemize}
  \item $h$ is even and holomorphic in a strip $|\Im r|<\tfrac12+\epsilon$ for some $\epsilon>0$,
  \item $h(r) \ll (1+|r|)^{-2-\delta}$ as $|r|\to\infty$ for some $\delta>0$.
\end{itemize}
\end{definition}

\begin{remark}
These conditions ensure that the corresponding kernel $k(z,w)$ is absolutely summable over $\Gamma$ and that all orbital integrals converge. They are direct analogues of admissibility conditions for Fourier test functions in classical explicit formulas.
\end{remark}

\subsection{Analytic continuation and scattering}

The transform $h(r)$ extends holomorphically in vertical strips, thanks to the analytic continuation of spherical functions $\varphi_r(r)$.  
This allows contour shifting in integrals on the spectral side and yields analytic continuation of scattering terms on the geometric side.  
For parabolic contributions, admissibility guarantees the convergence of
\[
  \frac{1}{4\pi}\int_{-\infty}^\infty h(r)\,
    \Big(-\frac{\varphi'(1/2+ir)}{\varphi(1/2+ir)}\Big)\,dr,
\]
where $\varphi(s)$ is the determinant of the scattering matrix.

\subsection{Explicit kernel examples}

\paragraph{Gaussian kernel.}
For $h(r)=e^{-\sigma^2 r^2}$,
\[
  q(r) = \frac{1}{\sqrt{4\pi\sigma^2}}\,\frac{1}{\sinh r}\,
  \exp\!\left(-\frac{r^2}{4\sigma^2}\right).
\]
This kernel localizes near $r=0$ and decays exponentially.

\paragraph{Compactly supported cutoff.}
If $h(r)$ is supported in $[-T,T]$, then $q(r)$ decays like $e^{Tr}/r^{1/2}$ by stationary phase, showing oscillatory behavior compatible with microlocal parametrices.

\paragraph{Wave kernel.}
For $h(r)=\cos(tr)$, the kernel $q(r)$ corresponds to the wave propagator, reflecting unitary evolution.

\subsection{Paley–Wiener theorem}

\begin{theorem}[Paley–Wiener for the Selberg transform]
$h(r)$ is an entire function of exponential type $R$ if and only if the corresponding radial kernel $q(r)$ is supported in $[0,R]$. 
\end{theorem}

\begin{remark}
This duality provides a precise dictionary between support of kernels and growth of transforms. It is central to constructing compactly supported kernels in Chapter~3 and local projectors in Chapter~4.
\end{remark}

\subsection{Stationary phase analysis}

Suppose $h(r)$ is compactly supported and smooth.  
Then for large $r$,
\[
  q(r) \;\asymp\; r^{-1/2}\cos(Tr-\theta_0) + O(r^{-3/2}),
\]
where $T$ is the radius of support of $h$ and $\theta_0$ a phase constant.  
These oscillatory asymptotics govern error terms in localized trace formulas.

\subsection{Applications and forward links}

\begin{itemize}
  \item \textbf{To Chapter~3:} kernels $K_{\lambda,\eta}$ are defined directly via admissible $h(r)$.
  \item \textbf{To Chapter~4:} construction of spectral projectors relies on Paley–Wiener admissibility and stationary phase.
  \item \textbf{To Chapter~6:} parabolic and hyperbolic orbital integrals require explicit decay of $h(r)$ and asymptotics of $q(r)$.
  \item \textbf{To Chapter~7:} error terms in the main localized trace formula depend on sharp stationary phase asymptotics recorded here.
\end{itemize}

\subsection*{Audit of Block 4}

\begin{itemize}
  \item[(B51)] Admissibility conditions stated and justified.
  \item[(B52)] Analytic continuation and scattering terms clarified.
  \item[(B53)] Explicit kernel examples (Gaussian, wave, compact cutoff) expanded.
  \item[(B54)] Paley–Wiener theorem included with explanation.
  \item[(B55)] Stationary phase asymptotics recorded.
  \item[(B56)] Applications to later chapters documented.
\end{itemize}

\subsection*{Conclusion}

The Selberg/Harish–Chandra transform, equipped with admissible test functions, analytic continuation, Paley–Wiener duality, and stationary phase asymptotics, completes the analytic foundation of the preliminaries.  
Together with the geometry and cusp analysis of the preceding blocks, this block furnishes the exact analytic machinery needed for constructing localized projectors and for carrying out the geometric expansion of the trace formula in Chapters~3–7.  
All constants are explicit, dependencies are transparent, and forward/backward links are fully recorded.  
This concludes the preliminaries section on transforms.

% =====================================================================
% End of File: 02b-selberg-transform.tex (Block 4 of 4)
% =====================================================================


% ---------------------------------------------------------------------
% Chapter Audit
% ---------------------------------------------------------------------
\subsection*{Chapter Audit}

\noindent\textbf{Scope and coverage.}
The geometric model $(\mathbb{H},ds^2)$, the quotient $M=\Gamma\backslash\mathbb{H}$,
and the finite-area assumption have been fixed. Cuspidal neighborhoods and
height truncations $M(Y)$ are defined with explicit dependence on cusp widths.
The sign and parameterization of the Laplacian are fixed once and for all.

\medskip
\noindent\textbf{Spectral conventions.}
The spectral decomposition on $L^2(M)$ is recorded in the standard form with
discrete eigenfunctions $\{\varphi_j\}$ and Eisenstein series
$E_{\mathfrak a}(z,\tfrac12+ir)$, normalized so that the Plancherel measure
is $dr/(4\pi)$. The spectral parameter is $\lambda=\frac14+r^2$.

\medskip
\noindent\textbf{Transform conventions.}
For radial kernels, the Selberg/Harish–Chandra transform, its inversion, and
normalization choices are fixed and aligned with the kernel constructions of
Chapters~3–4 and the microlocal analysis of Chapter~5.

\medskip
\noindent\textbf{Dependencies of constants.}
All error terms and implied constants are annotated as $O_{\Gamma,\beta}(\cdot)$
whenever dependence on the Fuchsian group and on a lower bound for the spectral
gap is relevant. No constant introduced in this chapter depends on the spectral
window parameters $(\lambda,\eta)$; such dependencies first appear with the
localized projector in Chapter~4.

\medskip
\noindent\textbf{Forward links.}
The geometric and cuspidal conventions feed directly into the truncated kernel
of Chapter~3. Transform normalizations and test-function conventions feed into
the spectral projector of Chapter~4 and the semiclassical parametrix in
Chapter~5. The propagation time scale $T\asymp\log\lambda$ is carried forward
to the geometric expansion of Chapter~6.

\medskip
\noindent\textbf{Consistency checks.}
Notation agrees with the global glossary; Laplacian sign and Plancherel
normalization are consistent with the bibliography standards (e.g., Selberg,
Iwaniec, Buser). Cusp-related constants are functions only of the cusp data
(number and widths). These checks complete the audit for Chapter~2.
