\documentclass[12pt]{amsart}

% ---------- Safe packages (arXiv-compliant) ----------
\usepackage[T1]{fontenc}
\usepackage{lmodern}
\usepackage{microtype}
\usepackage{amsmath, amsthm, amssymb}
\usepackage{mathtools}
\usepackage{geometry}
\geometry{margin=1in}
\usepackage{graphicx}
\usepackage{tikz}
\usepackage{enumitem}
\setlist{nosep}
\usepackage[colorlinks=true,linkcolor=blue,citecolor=teal,urlcolor=magenta]{hyperref}
\usepackage[nameinlink,capitalise]{cleveref}
\usepackage{doi} % keep AFTER hyperref

% ---------- PDF metadata ----------
\pdfinfo{
  /Title (A Localized Trace Formula for the Discrete Cuspidal Spectrum on Finite-Volume Hyperbolic Surfaces)
  /Author (Alexander Stepanovich Kozhukharev)
  /Subject (Spectral Geometry, Trace Formulas, Hyperbolic Surfaces)
  /Keywords (trace formula; hyperbolic surfaces; microlocal analysis; cuspidal spectrum; Weyl law)
}

% ---------- Theorem environments ----------
\numberwithin{equation}{section}
\theoremstyle{plain}
\newtheorem{theorem}{Theorem}[section]
\newtheorem{proposition}[theorem]{Proposition}
\newtheorem{lemma}[theorem]{Lemma}
\newtheorem{corollary}[theorem]{Corollary}
\theoremstyle{definition}
\newtheorem{definition}[theorem]{Definition}
\theoremstyle{remark}
\newtheorem{remark}[theorem]{Remark}

% ---------- Notation ----------
\newcommand{\HH}{\mathbb{H}}
\newcommand{\RR}{\mathbb{R}}
\DeclareMathOperator{\vol}{vol}
\DeclareMathOperator{\supp}{supp}
\newcommand{\injrad}{\mathrm{inj}}
\newcommand{\Lap}{\Delta}
\newcommand{\Tr}{\mathrm{Tr}}
\newcommand{\PSL}{\mathrm{PSL}}
\newcommand{\TR}{\mathsf{T}_R}

% ---------- Title ----------
\title[A Localized Trace Formula]{A Localized Trace Formula for the Discrete Cuspidal Spectrum on Finite-Volume Hyperbolic Surfaces}

\author{Alexander Stepanovich Kozhukharev}
\address{Independent Researcher, Moscow, Russia}

\date{\today}

\begin{document}

% ---------- Abstract ----------
\begin{abstract}
We establish a microlocally localized trace formula for finite-area hyperbolic surfaces $X=\Gamma\backslash\HH$ that isolates the discrete cuspidal spectrum in short frequency windows $[R-R^\theta,R+R^\theta]$ under a height cutoff $y\le Y=R^\beta$, with an identity term involving the effective volume, a geometric term from short closed geodesics, and a power-saving remainder $O(R^{1-\varepsilon(\theta,\beta)})$; the method completely avoids Eisenstein series and yields a windowed Weyl law with constants polynomial in the geometric data of $X$.
\end{abstract}

\keywords{trace formula; hyperbolic surfaces; microlocal analysis; cuspidal spectrum; Weyl law}
\subjclass[2020]{58J50, 35P20; 11F72, 58J40}

\maketitle

\tableofcontents
% === Block 1 sections (included from files) ===
\section{Introduction}\label{sec:intro}

This preprint develops a localized (microlocal) version of the Selberg trace
formula on a finite-area hyperbolic surface $X=\Gamma\backslash\HH$, with an
emphasis on \emph{windowed} counting of spectral data. The localization is
achieved by inserting smooth cutoffs both in configuration and spectral variables,
thus producing a trace distribution $\TR$ adapted to a prescribed microlocal region.

\smallskip
\noindent\textbf{Motivation.}
In spectral counting problems one often seeks information restricted to a
phase-space window (e.g.\ near a geodesic segment or within a fixed angular sector).
Classical trace formulae \cite{selberg1956,hejhal1976} capture global information;
a localized variant aligns naturally with tools of semiclassical and microlocal
analysis \cite{zworski2012,dyatlovzworski2019}, while avoiding the need to invoke
the full scattering theory in the cusps.

\smallskip
\noindent\textbf{Informal statement.}
Let $-\Lap$ be the Laplacian on $X$, with cuspidal eigenvalues
$\lambda_j=\tfrac14+r_j^2$ and $L^2$-normalized eigenfunctions $\psi_j$.
Fix a smooth spatial cutoff $\chi_Y(z):=\chi\!\big(y(z)/Y\big)$ with $Y=R^\beta$,
where $0<\beta<\tfrac12$, and let $h\in\mathcal{S}(\RR)$ be even with
$\supp\widehat{h}\subset[-c_0,c_0]$ for some $c_0>0$. For a spectral center
$R\gg1$ and a window exponent $0<\theta<1$, set
\[
  h_R(t):=h\!\left(\frac{t-R}{R^\theta}\right),
  \qquad
  \TR := \sum_j h_R(r_j)\,\|\chi_Y\psi_j\|_{L^2(X)}^2.
\]
Then for the admissible range
\[
  0<\beta<\tfrac12,
  \qquad
  0<\theta<\tfrac{1+\beta}{2},
\]
we establish a localized trace identity of the schematic form
\[
  \TR \;=\; \underbrace{\mathcal{I}_R(\chi_Y,h)}_{\text{identity / effective volume}}
  \;+\;
  \underbrace{\mathcal{G}_R(\chi_Y,h)}_{\substack{\text{sum over primitive closed}\\\text{geodesics }\ell(\gamma)\le c_0}}
  \;+\;
  O\!\big(R^{\,1-\varepsilon(\theta,\beta)}\big),
\]
with an explicit positive exponent
\[
  \varepsilon(\theta,\beta)=\EpsDef \;>\; 0,
\]
and with constants depending polynomially on the geometric complexity
$C_{\mathrm{geo}}(X):=m+\injrad(X_{\mathrm{core}})^{-1}$ and on finitely many
$C^k$-seminorms of $\chi$ and $h$.
The identity term $\mathcal{I}_R$ features the \emph{effective volume}
$\int_X \chi_Y^2\,d\mu$, which admits a cusp-expansion with a shape-constant
$\kappa_\chi$, while $\mathcal{G}_R$ is a geometric sum over primitive closed
geodesics with $\ell(\gamma)\le c_0$ determined by $\supp\widehat{h}$.
As a consequence, one obtains a windowed Weyl law with power-saving error.

\smallskip
\noindent\textbf{Contributions (Block 0 skeleton).}
\begin{itemize}
  \item A microlocally localized trace identity intertwining the spectral window
        $h_R$ and the spatial truncation $\chi_Y$, stated with explicit admissible
        ranges $0<\beta<\tfrac12$ and $0<\theta<\tfrac{1+\beta}{2}$;
  \item An identity term controlled by the effective volume
        $\vol_{\mathrm{eff}}(Y)=\int_X \chi_Y^2\,d\mu$ and its cusp asymptotics
        (with a shape-constant $\kappa_\chi$);
  \item A geometric term $\mathcal{G}_R$ summing over \emph{primitive} closed
        geodesics with $\ell(\gamma)\le c_0$ (the support bound for $\widehat{h}$);
  \item A power-saving remainder $O(R^{1-\EpsDef})$, with constants polynomial in
        $C_{\mathrm{geo}}(X)$ and in finitely many seminorms of $\chi,h$;
  \item A windowed Weyl law as an immediate corollary.
\end{itemize}

\smallskip
\noindent\textbf{Organization.}
\Cref{sec:preliminaries} records geometric and analytic preliminaries,
cutoff conventions and transform normalizations. The kernel and projector
localization appear in \S\S\ref{sec:kernel}--\ref{sec:projector}, while
microlocal and geometric contributions are outlined in
\S\ref{sec:microlocal}--\ref{sec:geometric}. Full proofs are deferred to later
blocks; Block~0 contains the minimal skeleton needed for cross-referencing.

\section{Preliminaries}\label{sec:prelim}

\subsection{Geometry and notation.}
Let $X=\Gamma\backslash\HH$ be a finite-area hyperbolic surface with $m$ cusps.
We write $z=x+iy$ on $\HH$, use the hyperbolic measure $d\mu=y^{-2}\,dx\,dy$, and take the (positive) Laplacian to be $-\Lap$.
Denote normalized cuspidal eigenpairs by $(-\Lap)\psi_j=\lambda_j\psi_j$ with $\lambda_j=\tfrac14+r_j^2$ and $\|\psi_j\|_{L^2(X)}=1$.
The continuous spectrum is treated via Eisenstein series but will be \emph{suppressed in Block~0} (only cuspidal contributions are kept).

We set
\[
X_{\mathrm{core}}:=X\setminus\{\text{cuspidal ends}\},\qquad
\injrad(X_{\mathrm{core}}):=\inf_{z\in X_{\mathrm{core}}}\injrad(z),
\]
and use the geometric size parameter
\[
C_{\mathrm{geo}}(X):=m+\injrad(X_{\mathrm{core}})^{-1},
\]
which controls polynomially all implicit constants below.

\subsection{Cutoffs, windows, and parameter regime.}
Fix $\chi\in C_c^\infty([0,\infty))$ such that $\chi\equiv1$ on $[0,1]$ and $\supp\chi\subset[0,4)$.
For a height scale $Y>0$ define the spatial cutoff
\[
\chi_Y(z):=\chi\!\big(y(z)/Y\big).
\]
Throughout Block~0 we couple $Y$ to the spectral scale via
\[
Y=R^\beta,\qquad R\gg1,\quad 0<\beta<\tfrac12.
\]

Let $h\in\mathcal{S}(\RR)$ be even with compactly supported Fourier transform.
We fix a constant $c_0\in\big(0,\tfrac{\log 2}{2}\big)$ and assume
\[
\supp \widehat{h}\subset[-c_0,c_0].
\]
For a window exponent $0<\theta<1$ we localize in frequency by
\[
h_R(t):=h\!\left(\frac{t-R}{R^\theta}\right),
\]
so $h_R$ selects the spectral window $[R-R^\theta,R+R^\theta]$ centered at $R$ with width $R^\theta$.
In the main estimates we will work in the admissible range
\[
0<\beta<\tfrac12,\qquad 0<\theta<\tfrac{1+\beta}{2},
\]
which will be seen to be optimal at the level of the error exponent.

\begin{definition}[Localized trace]\label{def:TR}
The \emph{localized trace distribution} is
\[
  \TR := \sum_j h_R(r_j)\,\|\chi_Y\psi_j\|_{L^2(X)}^2.
\]
\emph{Remark.} In later sections we prove a decomposition of the form
\[
\TR \;=\; \mathcal{I}_R(\chi_Y,h)\;+\;\mathcal{G}_R(\chi_Y,h)\;+\;O\!\big(R^{1-\varepsilon(\theta,\beta)}\big),
\]
where $\mathcal{I}_R$ is the identity contribution, $\mathcal{G}_R$ is a geometric sum over short closed geodesics (with the \emph{same} $c_0$ as above), and $\varepsilon(\theta,\beta)>0$ on the stated admissible region.
\end{definition}

\begin{lemma}[Windowed Plancherel: skeleton]\label{lem:planch}
Let $h\in\mathcal{S}(\RR)$ be even with $\supp\widehat{h}\subset[-c_0,c_0]$.
Then
\[
  \sum_j h_R(r_j)\,\|\chi_Y\psi_j\|_{L^2(X)}^2
  \;=\; \int_X \chi_Y(z)\,K_R(z,z)\,d\mu(z) \;+\; O_N(R^{-N})\quad\forall N,
\]
where $K_R$ is the Schwartz kernel of the spectral multiplier $h_R(\sqrt{-\Lap})$.
The remainder $O_N(R^{-N})$ depends polynomially on $C_{\mathrm{geo}}(X)$ and on finitely many seminorms of $h$ and $\chi$.
\end{lemma}

\begin{remark}[Effective volume and normalizations]
The effective volume is defined by
\[
\vol_{\mathrm{eff}}(Y):=\int_X \chi_Y^2\,d\mu,
\]
and satisfies (see Appendix~\ref{app:effvol})
\[
\vol_{\mathrm{eff}}(Y)=\vol(X)-\frac{m}{Y}\,\kappa_\chi+O(mY^{-2}),
\qquad
\kappa_\chi:=\int_1^\infty (1-\chi(u)^2)\,u^{-2}\,du\in(0,\tfrac12].
\]
Our Fourier transform and Plancherel conventions match those used in \S\ref{sec:kernel} and \S\ref{sec:proj-comm}; in particular the Plancherel measure is $\tfrac{1}{2\pi}\,r\tanh(\pi r)\,dr$.
\end{remark}
```0


\section{Introduction}
This preprint presents the localized trace formula and its windowed Weyl consequences. Full details and proofs are developed in subsequent sections/blocks.

\section{Notation}
Let $X=\Gamma\backslash\HH$ be a finite-area hyperbolic surface with $m$ cusps; $d\mu=y^{-2}\,dx\,dy$, $-\Lap$ has cuspidal eigenvalues $\lambda_j=\tfrac14+r_j^2$ and $L^2$-normalized eigenfunctions $\psi_j$.
We use a smooth cutoff $\chi\in C_c^\infty([0,\infty))$, $\chi=1$ on $[0,1]$, and set $\chi_Y(z)=\chi(y(z)/Y)$ with $Y=R^\beta$.
Let $h\in\mathcal{S}(\RR)$ be even with $\supp \widehat{h}\subset[-c_0,c_0]$, and $h_R(t)=h((t-R)/R^\theta)$.
The localized trace is
\begin{equation}\label{eq:TR-def}
\TR := \sum_j h_R(r_j)\,\|\chi_Y\psi_j\|_{L^2(X)}^2.
\end{equation}

% (Minimal lemma/theorem statements kept for Block 0; detailed proofs follow in later blocks.)
\section{Main statements}
\begin{theorem}\label{thm:main}
For $0<\beta<\tfrac12$ and $0<\theta<\tfrac{1+\beta}{2}$,
\[
\TR = \mathcal{I}_R(\chi_Y,h)+\mathcal{G}_R(\chi_Y,h)+O(R^{1-\varepsilon(\theta,\beta)}),
\]
with $\varepsilon(\theta,\beta)=\min\{\theta,1-\theta+\beta,\tfrac12,1-2\theta+\beta\}-\delta$ for any $\delta>0$, where the implicit constants depend polynomially on $m+\injrad(X_{\mathrm{core}})^{-1}$ and on finitely many seminorms of $\chi$ and $h$.
\end{theorem}

\begin{remark}
The admissible region is $0<\beta<\tfrac12$ and $0<\theta<\tfrac{1+\beta}{2}$; on the boundary $1-2\theta+\beta=0$ and the power-saving vanishes.
\end{remark}

% ===========================
% Acknowledgments & Data note
% ===========================
\section*{Acknowledgments}
The author thanks colleagues for helpful discussions. Any remaining errors are the author's.

\section*{Data availability}
Not applicable.

% ===========================
% References (manual for Block 0; will switch to .bib later)
% ===========================
\begin{thebibliography}{99}

\bibitem{selberg1956}
A.~Selberg,
\textit{Harmonic analysis and discontinuous groups},
J. Indian Math. Soc. \textbf{20} (1956), 47--87.

\bibitem{hejhal1976}
D.~A.~Hejhal,
\textit{The Selberg Trace Formula for $\mathrm{PSL}(2,\mathbb{R})$}, Vol.~I,
Lecture Notes in Mathematics \textbf{548}, Springer, 1976.
ISBN: 978-3-540-07727-7, \doi{10.1007/BFb0074437}

\bibitem{hejhal1983}
D.~A.~Hejhal,
\textit{The Selberg Trace Formula for $\mathrm{PSL}(2,\mathbb{R})$}, Vol.~II,
Lecture Notes in Mathematics \textbf{1001}, Springer, 1983.
\doi{10.1007/BFb0061300}

\bibitem{mueller1983}
W.~M\"uller,
\textit{Spectral theory for Riemannian manifolds with cusps},
J. Differential Geom. \textbf{18} (1983), 575--598.
\doi{10.4310/jdg/1214437785}

\bibitem{iwaniec1995}
H.~Iwaniec, P.~Sarnak,
\textit{$L^\infty$ norms of eigenfunctions on arithmetic surfaces},
Ann. of Math. \textbf{141} (1995), 301--320.
\doi{10.2307/2118520}

\bibitem{buser1992}
P.~Buser,
\textit{Geometry and Spectra of Compact Riemann Surfaces},
Birkh\"auser, 1992.
ISBN: 978-0-8176-3404-7, \doi{10.1007/978-1-4684-9172-2}

\bibitem{zworski2012}
M.~Zworski,
\textit{Semiclassical Analysis},
Graduate Studies in Mathematics \textbf{138}, AMS, 2012.
\doi{10.1090/gsm/138}

\bibitem{dyatlovzworski2019}
S.~Dyatlov, M.~Zworski,
\textit{Mathematical Theory of Scattering Resonances},
University Lecture Series \textbf{200}, AMS, 2019.
\doi{10.1090/ulect/200}

\bibitem{chazarain1974}
J.~Chazarain,
\textit{Formule de Poisson pour les vari\'et\'es riemanniennes},
Invent. Math. \textbf{24} (1974), 65--82.
MR0350064, Zbl 0284.35058, \doi{10.1007/BF01418762}

\end{thebibliography}

\end{document}
