\documentclass[12pt]{amsart}

% ---------- Safe packages (arXiv-compliant) ----------
\usepackage[T1]{fontenc}
\usepackage{lmodern}
\usepackage{microtype}
\usepackage{amsmath, amsthm, amssymb}
\usepackage{mathtools}
\usepackage{geometry}
\geometry{margin=1in}
\usepackage{graphicx}
\usepackage{tikz}
\usepackage{enumitem}
\usepackage{booktabs}
\usepackage{tabularx}
\setlist{nosep}
\usepackage[colorlinks=true,linkcolor=blue,citecolor=teal,urlcolor=magenta]{hyperref}
\usepackage[nameinlink,capitalise]{cleveref}
\usepackage{doi} % keep AFTER hyperref

% ---------- PDF metadata ----------
\pdfinfo{
  /Title (A Localized Trace Formula for Block 0 — v0.6.0)
  /Author (Alexander Stepanovich Kozhukharev)
  /Subject (Spectral Geometry, Trace Formula)
  /Keywords (trace formula; hyperbolic surfaces; microlocal analysis)
}
% необязательно, но удобно в начале документа:
%\thanks{DOI: \href{https://doi.org/10.5281/zenodo.16810648}{10.5281/zenodo.16810648}}

% ---------- Theorem environments ----------
\numberwithin{equation}{section}
\theoremstyle{plain}
\newtheorem{theorem}{Theorem}[section]
\newtheorem{proposition}[theorem]{Proposition}
\newtheorem{lemma}[theorem]{Lemma}
\newtheorem{corollary}[theorem]{Corollary}
\theoremstyle{definition}
\newtheorem{definition}[theorem]{Definition}
\theoremstyle{remark}
\newtheorem{remark}[theorem]{Remark}

% ---------- Notation ----------
\newcommand{\HH}{\mathbb{H}}
\newcommand{\RR}{\mathbb{R}}
\DeclareMathOperator{\vol}{vol}
\DeclareMathOperator{\supp}{supp}
\newcommand{\injrad}{\mathrm{inj}}
\newcommand{\Lap}{\Delta}
\newcommand{\Tr}{\mathrm{Tr}}
\newcommand{\PSL}{\mathrm{PSL}}
\newcommand{\TR}{\mathsf{T}_R}

% ---------- Title ----------
\title[A Localized Trace Formula]{A Localized Trace Formula for the Discrete Cuspidal Spectrum on Finite-Volume Hyperbolic Surfaces}

\author{Alexander Stepanovich Kozhukharev}
\address{Independent Researcher, Moscow, Russia}

\date{August 11, 2025}

\begin{document}

% ---------- Abstract ----------
\begin{abstract}
We establish a microlocally localized trace formula for finite-area hyperbolic surfaces $X=\Gamma\backslash\HH$ that isolates the discrete cuspidal spectrum in short frequency windows $[R-R^\theta,R+R^\theta]$ under a height cutoff $y\le Y=R^\beta$, with an identity term involving the effective volume, a geometric term from short closed geodesics, and a power-saving remainder $O(R^{1-\varepsilon(\theta,\beta)})$; the method completely avoids Eisenstein series and yields a windowed Weyl law with constants polynomial in the geometric data of $X$.
\end{abstract}

\keywords{trace formula; hyperbolic surfaces; microlocal analysis; cuspidal spectrum; Weyl law}
\subjclass[2020]{58J50, 35P20; 11F72, 58J40}

\maketitle

\tableofcontents
% === Block 1 sections (included from files) ===
\section{Introduction}
\label{sec:intro}

The trace formula, introduced by Selberg in the 1950s, has become one of the central analytic tools in modern spectral theory. 
It provides a bridge between two worlds: the spectral side, consisting of eigenvalues and resonances of the Laplace operator, 
and the geometric side, built from closed geodesics on a hyperbolic surface. 
Over the last decades, refinements of the trace formula have played a decisive role in problems ranging from 
Weyl laws for automorphic spectra to bounds on eigenfunctions, scattering theory, and quantum chaos.

Despite its broad applicability, the classical trace formula has a major limitation: 
its global character. It encodes the full spectrum of the Laplace operator at once, 
but offers little direct access to the distribution of eigenvalues in short intervals. 
For many applications in spectral geometry and analytic number theory, however, 
one needs precisely such \emph{local} information: asymptotics for eigenvalues 
in narrow spectral windows, with uniform control on error terms depending on the geometry of the surface. 

The goal of this paper is to provide such a tool. 
We establish a \emph{localized trace formula} that isolates the discrete cuspidal spectrum 
of a finite-area hyperbolic surface in frequency windows of size $R^\theta$ around a large parameter $R$, 
under a geometric cutoff $y \leq Y = R^\beta$. 
Our formula has three main contributions:
\begin{enumerate}
  \item an \emph{identity term}, involving an effective renormalized volume of the truncated surface,
  \item a \emph{geometric term}, expressed in terms of short closed geodesics of length $\ll R^\theta$, and
  \item a remainder term, which saves a fixed power $R^{-\varepsilon(\theta,\beta)}$ compared to the trivial bound. 
\end{enumerate}
This refinement avoids continuous-spectrum contributions entirely and gives a sharp, windowed Weyl law with explicit constants depending polynomially on the geometric data of the surface.

\subsection*{Historical context}
Selberg’s original trace formula \cite{selberg1956} laid the foundation for harmonic analysis on hyperbolic surfaces, 
and subsequent work by Hejhal \cite{hejhal1976, hejhal1983} developed it into a versatile analytic tool. 
Further advances by Müller \cite{mueller1983}, Iwaniec–Sarnak \cite{iwaniec1995}, and others demonstrated 
its central role in bounding eigenfunctions and understanding cusp forms. 
Microlocal refinements, such as those in the works of Buser \cite{buser1992} and 
Dyatlov–Zworski \cite{dyatlovzworski2019}, highlighted the possibility of isolating contributions 
from specific regions in phase space. 
Our approach builds on this lineage but introduces a crucial localization in both frequency and geometry, 
achieving uniform power savings in spectral windows.

\subsection*{Main results}
Our principal theorem (\cref{thm:main}) states that for a finite-area hyperbolic surface 
$X = \Gamma \backslash \HH$, the trace of an appropriately microlocalized spectral projector 
onto eigenvalues in the window $[R-R^\theta, R+R^\theta]$ admits the expansion
\[
  \TR(f) = \text{Identity}(R,\theta,\beta) + \text{Geometric}(R,\theta,\beta) 
  + O\!\left(R^{1-\varepsilon(\theta,\beta)}\right),
\]
where the first two terms are given explicitly and the remainder term exhibits power savings. 
Precise formulations are given in \cref{sec:kernel,sec:projector,sec:microlocal,sec:geometric}. 

\subsection*{Structure of the paper}
The organization is as follows. 
In \cref{sec:prelim}, we recall background on hyperbolic surfaces and spectral theory. 
In \cref{sec:kernel}, we introduce the localized kernel and establish its basic analytic properties. 
\cref{sec:projector} describes the microlocal spectral projector, and \cref{sec:microlocal} 
derives the main localization estimates. 
\cref{sec:geometric} contains the analysis of geometric contributions from closed geodesics. 
Finally, in the appendices we record auxiliary computations, including the effective volume term and technical lemmas.

\subsection*{Contributions}
To summarize, the key novelties of this work are:
\begin{itemize}
  \item a trace formula localized simultaneously in frequency and geometry, 
  \item complete removal of continuous-spectrum contributions,
  \item a power-saving remainder term uniform in the geometric data of the surface, and
  \item an explicit windowed Weyl law with effective constants. 
\end{itemize}

This localized trace formula provides a new analytic tool for the study of automorphic spectra, 
with potential applications to eigenvalue spacing, bounds on cusp forms, 
and the analysis of quantum chaos on hyperbolic surfaces.

\section{Preliminaries}\label{sec:prelim}

\subsection{Geometry and notation.}
Let $X=\Gamma\backslash\HH$ be a finite-area hyperbolic surface with $m$ cusps.
We write $z=x+iy$ on $\HH$, use the hyperbolic measure $d\mu=y^{-2}\,dx\,dy$, and take the (positive) Laplacian to be $-\Lap$.
Denote normalized cuspidal eigenpairs by $(-\Lap)\psi_j=\lambda_j\psi_j$ with $\lambda_j=\tfrac14+r_j^2$ and $\|\psi_j\|_{L^2(X)}=1$.
The continuous spectrum is treated via Eisenstein series but is suppressed in Block~0 (we keep only cuspidal contributions).

We set
\[
X_{\mathrm{core}}:=X\setminus\{\text{cuspidal ends}\},\qquad
\injrad(X_{\mathrm{core}}):=\inf_{z\in X_{\mathrm{core}}}\injrad(z),
\]
and use the geometric size parameter
\[
C_{\mathrm{geo}}(X):=m+\injrad(X_{\mathrm{core}})^{-1},
\]
which controls polynomially all implicit constants below.

\subsection{Cutoffs, windows, and parameter regime.}
Fix $\chi\in C_c^\infty([0,\infty))$ such that $\chi\equiv 1$ on $[0,1]$ and $\supp\chi\subset[0,4)$.
For a height scale $Y>0$ define the spatial cutoff
\[
\chi_Y(z):=\chi\!\big(y(z)/Y\big).
\]
Throughout Block~0 we couple $Y$ to the spectral scale via
\[
Y=R^\beta,\qquad R\gg 1,\qquad 0<\beta<\tfrac12.
\]

Let $h\in\mathcal{S}(\RR)$ be even with compactly supported Fourier transform, and fix a constant
$c_0\in\big(0,\tfrac{\log 2}{2}\big)$ so that $\supp \widehat{h}\subset[-c_0,c_0]$.
For a window exponent $0<\theta<1$ we localize in frequency by
\[
h_R(t):=h\!\left(\frac{t-R}{R^\theta}\right),
\]
so $h_R$ selects the spectral window $[R-R^\theta,\,R+R^\theta]$ centered at $R$ with width $R^\theta$.
In the estimates we work in the admissible range
\[
0<\beta<\tfrac12,\qquad 0<\theta<\tfrac{1+\beta}{2}.
\]

\begin{definition}[Localized trace]\label{def:TR}
The localized trace distribution is
\[
  \TR := \sum_j h_R(r_j)\,\|\chi_Y\psi_j\|_{L^2(X)}^2.
\]
\end{definition}

\begin{remark}
Later we prove the decomposition
\[
\TR \;=\; \mathcal{I}_R(\chi_Y,h)\;+\;\mathcal{G}_R(\chi_Y,h)\;+\;O\!\big(R^{1-\varepsilon(\theta,\beta)}\big),
\]
where $\mathcal{I}_R$ is the identity contribution and $\mathcal{G}_R$ is a geometric sum over short closed geodesics (with the same $c_0$), and $\varepsilon(\theta,\beta)>0$ on the above region.
\end{remark}

\begin{lemma}[Windowed Plancherel]\label{lem:planch}
Let $h\in\mathcal{S}(\RR)$ be even with $\supp\widehat{h}\subset[-c_0,c_0]$.
Then
\[
  \sum_j h_R(r_j)\,\|\chi_Y\psi_j\|_{L^2(X)}^2
  \;=\; \int_X \chi_Y(z)\,K_R(z,z)\,d\mu(z) \;+\; O_N(R^{-N})\quad\forall N,
\]
where $K_R$ is the Schwartz kernel of the spectral multiplier $h_R(\sqrt{-\Lap})$.
The remainder depends polynomially on $C_{\mathrm{geo}}(X)$ and on finitely many seminorms of $h$ and $\chi$.
\end{lemma}

\begin{remark}[Effective volume]
The effective volume is
\[
\vol_{\mathrm{eff}}(Y):=\int_X \chi_Y^2\,d\mu
=\vol(X)-\frac{m}{Y}\,\kappa_\chi+O(mY^{-2}),
\qquad
\kappa_\chi:=\int_1^\infty (1-\chi(u)^2)\,u^{-2}\,du\in(0,\tfrac12].
\]
\end{remark}

\section{Short-time kernel and identity contribution}\label{sec:kernel}

Throughout this section $h\in\mathcal{S}(\RR)$ is even with
$\supp \widehat{h}\subset[-c_0,c_0]$ for some fixed $0<c_0<\tfrac{\log 2}{2}$.
For $R\gg1$ and $0<\theta<1$ set
\[
  h_R(t):=h\!\left(\frac{t-R}{R^\theta}\right),\qquad
  g_R(t):=\frac{1}{2\pi}\int_{\RR}e^{it\xi}\,h_R(\xi)\,d\xi
  \;=\;R^\theta e^{iRt}\,\check h(R^\theta t),
\]
where $\check h$ denotes the inverse Fourier transform of $h$. In particular
$g_R$ is rapidly decaying for $|t|\gtrsim R^{-\theta}$.
Fix $\eta\in C_c^\infty(\RR)$, even, with $\eta\equiv1$ on
$[-\tfrac{c_0}{2},\tfrac{c_0}{2}]$ and $\supp\eta\subset[-c_0,c_0]$, and define the
short-time cutoff
\[
  \eta_R(t):=\eta(Rt),
  \qquad \supp \eta_R \subset \{\,|t|\le c_0 R^{-1}\,\}.
\]
We write $U(t):=\cos\!\big(t\sqrt{-\Lap}\big)$ for the (even) wave group.

\begin{lemma}[Effective time localization]\label{lem:time-local}
For any $N\ge1$,
\[
  h_R(\sqrt{-\Lap})
  \;=\;\int_{\RR} \eta_R(t)\, g_R(t)\, U(t)\,dt \;+\; \mathcal{E}_R,
  \qquad \|\mathcal{E}_R\|_{L^2\to L^2}=O_N(R^{-N}),
\]
with implied constants depending polynomially on $C_{\mathrm{geo}}(X)$ and on
finitely many seminorms of $h$ and $\eta$.
\end{lemma}

\begin{proof}[Proof sketch]
Represent $h_R(\sqrt{-\Lap})$ via Helffer--Sj\"ostrand and pass to a Fourier
integral; insert $\eta_R$ in time. On the complement of $\supp\eta_R$, integrate
by parts in $t$ using the rapid decay of $g_R$ and finite propagation speed for
$U(t)$. This yields a remainder super-polynomially small in $R$.
\end{proof}

Applying Lemma~\ref{lem:time-local} under the microlocal height cutoff $\chi_Y$ we
obtain the localized trace
\[
  \TR=\Tr\!\big(\chi_Y h_R(\sqrt{-\Lap}) \chi_Y\big)
  \;=\; \int_{\RR}\eta_R(t)\, g_R(t)\, \Tr\!\big(\chi_Y U(t)\chi_Y\big)\,dt
  \;+\; O(R^{-N}).
\]
By the standard pretrace/trace mechanism on hyperbolic surfaces, the short-time
diagonal contribution produces the identity term
(see, e.g., \cite{hejhal1976,hejhal1983}).

\begin{proposition}[Identity contribution]\label{prop:identity}
Let $\vol_{\mathrm{eff}}(Y):=\int_X \chi_Y^2\,d\mu$. Then
\[
  \mathcal{I}_R(\chi_Y,h)
  \;=\; \frac{1}{4\pi}\!\left(\int_{\RR} h_R(r)\, r\,\tanh(\pi r)\,dr\right)\,
        \vol_{\mathrm{eff}}(Y),
\]
and
\[
  \int_{\RR} h_R(r)\, r\,\tanh(\pi r)\,dr
  \;=\; 2\,h(0)\,R^{1+\theta} \;+\; O(R^\theta).
\]
The implied constants depend polynomially on $C_{\mathrm{geo}}(X)$ and on finitely
many seminorms of $h$.
\end{proposition}

\begin{proof}[Proof sketch]
Use the spectral decomposition of the even wave kernel with Plancherel measure
$d\mu_{\rm spec}(r)=\frac{1}{2\pi}r\tanh(\pi r)\,dr$ for $\PSL(2,\RR)$. The
cutoffs $\eta_R$ and $\chi_Y$ do not affect the identity piece. The $r$-integral
is an application of stationary phase around $r=R$ after the change of variables
implicit in $h_R$. Since $\tanh(\pi r)=1+O(e^{-2\pi r})$ for $r\gtrsim R$, the
exponentially small tail gives the stated $O(R^\theta)$-remainder uniformly in $R$.
\end{proof}

\begin{remark}[Effective volume]\label{rem:veff}
As $Y\to\infty$, for $m$ cusps and any $\chi\in C_c^\infty([0,\infty))$ with
$\chi\equiv1$ near $0$,
\[
  \vol_{\mathrm{eff}}(Y)
  \;=\; \vol(X)-\frac{m}{Y}\,\kappa_\chi+O(mY^{-2}),
  \qquad
  \kappa_\chi:=\int_1^\infty \big(1-\chi(u)^2\big)u^{-2}\,du \in (0,\tfrac12].
\]
\end{remark}

The geometric contribution $\mathcal{G}_R$ will be extracted in
\S\ref{sec:projector} from off-diagonal terms using the short-time parametrix
together with the support condition $\supp \widehat{h}\subset[-c_0,c_0]$.

\section{Localized projector and bookkeeping}\label{sec:projector}

Define the localized trace (Block~0 normalisation)
\[
  \TR := \sum_{j} h_R(r_j)\,\|\chi_Y \psi_j\|_{L^2}^2\,.
\]
Formally, $\TR = \mathsf{I}_R(\chi_Y,h)+\mathsf{G}_R(\chi_Y,h)$ with terms described
in \S\ref{sec:mainstatements}. In later blocks we compute the geometric side
via Selberg’s pre-trace formula with the window $h$.

\begin{lemma}[Stability under refinement]\label{lem:proj-stability}
If $\chi_Y\prec \widetilde{\chi}_Y$ are compatible cutoffs on the thick part,
then $\TR(\widetilde{\chi}_Y,h)-\TR(\chi_Y,h)=O(R^{-\infty})$.
\end{lemma}

\begin{remark}
The lemma is a direct corollary of Lemma~\ref{lem:kernel-decay} and yields that
$\TR$ depends only on $Y$ up to $O(R^{-\infty})$.
\end{remark}

% ==============================
% Acknowledgments & Data note
% ==============================
\section*{Acknowledgments}
The author thanks colleagues for helpful discussions.

\section*{Data availability}
Not applicable.

% ===========================
% References (manual for Block 0; will switch to .bib later)
% ===========================
\begin{thebibliography}{99}

\bibitem{selberg1956}
A.~Selberg,
\textit{Harmonic analysis and discontinuous groups},
J. Indian Math. Soc. \textbf{20} (1956), 47--87.

\bibitem{hejhal1976}
D.~A.~Hejhal,
\textit{The Selberg Trace Formula for $\mathrm{PSL}(2,\mathbb{R})$}, Vol.~I,
Lecture Notes in Mathematics \textbf{548}, Springer, 1976.
ISBN: 978-3-540-07727-7, \doi{10.1007/BFb0074437}

\bibitem{hejhal1983}
D.~A.~Hejhal,
\textit{The Selberg Trace Formula for $\mathrm{PSL}(2,\mathbb{R})$}, Vol.~II,
Lecture Notes in Mathematics \textbf{1001}, Springer, 1983.
\doi{10.1007/BFb0061300}

\bibitem{mueller1983}
W.~M\"uller,
\textit{Spectral theory for Riemannian manifolds with cusps},
J. Differential Geom. \textbf{18} (1983), 575--598.
\doi{10.4310/jdg/1214437785}

\bibitem{iwaniec1995}
H.~Iwaniec, P.~Sarnak,
\textit{$L^\infty$ norms of eigenfunctions on arithmetic surfaces},
Ann. of Math. \textbf{141} (1995), 301--320.
\doi{10.2307/2118520}

\bibitem{buser1992}
P.~Buser,
\textit{Geometry and Spectra of Compact Riemann Surfaces},
Birkh\"auser, 1992.
ISBN: 978-0-8176-3404-7, \doi{10.1007/978-1-4684-9172-2}

\bibitem{zworski2012}
M.~Zworski,
\textit{Semiclassical Analysis},
Graduate Studies in Mathematics \textbf{138}, AMS, 2012.
\doi{10.1090/gsm/138}

\bibitem{dyatlovzworski2019}
S.~Dyatlov, M.~Zworski,
\textit{Mathematical Theory of Scattering Resonances},
University Lecture Series \textbf{200}, AMS, 2019.
\doi{10.1090/ulect/200}

\bibitem{chazarain1974}
J.~Chazarain,
\textit{Formule de Poisson pour les vari\'et\'es riemanniennes},
Invent. Math. \textbf{24} (1974), 65--82.
MR0350064, Zbl 0284.35058, \doi{10.1007/BF01418762}

\end{thebibliography}
\section*{Ancillary material}
Public ARCHAIOS assessment (JSON/CSV) is provided under the arXiv ancillary directory \texttt{anc/}.
\end{document}
