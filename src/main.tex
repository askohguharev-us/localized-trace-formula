\documentclass[12pt]{amsart}

% ---------- Safe packages (arXiv-compliant) ----------
\usepackage[T1]{fontenc}
\usepackage{lmodern}
\usepackage{microtype}
\usepackage{amsmath, amsthm, amssymb}
\usepackage{mathtools}
\usepackage{geometry}
\geometry{margin=1in}
\usepackage{graphicx}
\usepackage{tikz}
\usepackage{enumitem}
\setlist{nosep}
\usepackage[colorlinks=true,linkcolor=blue,citecolor=teal,urlcolor=magenta]{hyperref}
\usepackage[nameinlink,capitalise]{cleveref}
\usepackage{doi} % keep AFTER hyperref

% ---------- PDF metadata ----------
\pdfinfo{
  /Title (A Localized Trace Formula for the Discrete Cuspidal Spectrum on Finite-Volume Hyperbolic Surfaces)
  /Author (Alexander Stepanovich Kozhukharev)
  /Subject (Spectral Geometry, Trace Formulas, Hyperbolic Surfaces)
  /Keywords (trace formula; hyperbolic surfaces; microlocal analysis; cuspidal spectrum; Weyl law)
}

% ---------- Theorem environments ----------
\numberwithin{equation}{section}
\theoremstyle{plain}
\newtheorem{theorem}{Theorem}[section]
\newtheorem{proposition}[theorem]{Proposition}
\newtheorem{lemma}[theorem]{Lemma}
\newtheorem{corollary}[theorem]{Corollary}
\theoremstyle{definition}
\newtheorem{definition}[theorem]{Definition}
\theoremstyle{remark}
\newtheorem{remark}[theorem]{Remark}

% ---------- Notation ----------
\newcommand{\HH}{\mathbb{H}}
\newcommand{\RR}{\mathbb{R}}
\DeclareMathOperator{\vol}{vol}
\DeclareMathOperator{\supp}{supp}
\newcommand{\injrad}{\mathrm{inj}}
\newcommand{\Lap}{\Delta}
\newcommand{\Tr}{\mathrm{Tr}}
\newcommand{\PSL}{\mathrm{PSL}}
\newcommand{\TR}{\mathsf{T}_R}

% ---------- Title ----------
\title[A Localized Trace Formula]{A Localized Trace Formula for the Discrete Cuspidal Spectrum on Finite-Volume Hyperbolic Surfaces}

\author{Alexander Stepanovich Kozhukharev}
\address{Independent Researcher, Moscow, Russia}

\date{\today}

\begin{document}

% ---------- Abstract ----------
\begin{abstract}
We establish a microlocally localized trace formula for finite-area hyperbolic surfaces $X=\Gamma\backslash\HH$ that isolates the discrete cuspidal spectrum in short frequency windows $[R-R^\theta,R+R^\theta]$ under a height cutoff $y\le Y=R^\beta$, with an identity term involving the effective volume, a geometric term from short closed geodesics, and a power-saving remainder $O(R^{1-\varepsilon(\theta,\beta)})$; the method completely avoids Eisenstein series and yields a windowed Weyl law with constants polynomial in the geometric data of $X$.
\end{abstract}

\keywords{trace formula; hyperbolic surfaces; microlocal analysis; cuspidal spectrum; Weyl law}
\subjclass[2020]{58J50, 35P20; 11F72, 58J40}

\maketitle

\tableofcontents

% ===========================
% Introduction (skeleton only; main content to be filled in Blocks 1+)
% ===========================
\section{Introduction}
This preprint presents the localized trace formula and its windowed Weyl consequences. Full details and proofs are developed in subsequent sections/blocks.

\section{Notation}
Let $X=\Gamma\backslash\HH$ be a finite-area hyperbolic surface with $m$ cusps; $d\mu=y^{-2}\,dx\,dy$, $-\Lap$ has cuspidal eigenvalues $\lambda_j=\tfrac14+r_j^2$ and $L^2$-normalized eigenfunctions $\psi_j$.
We use a smooth cutoff $\chi\in C_c^\infty([0,\infty))$, $\chi=1$ on $[0,1]$, and set $\chi_Y(z)=\chi(y(z)/Y)$ with $Y=R^\beta$.
Let $h\in\mathcal{S}(\RR)$ be even with $\supp \widehat{h}\subset[-c_0,c_0]$, and $h_R(t)=h((t-R)/R^\theta)$.
The localized trace is
\begin{equation}\label{eq:TR-def}
\TR := \sum_j h_R(r_j)\,\|\chi_Y\psi_j\|_{L^2(X)}^2.
\end{equation}

% (Minimal lemma/theorem statements kept for Block 0; detailed proofs follow in later blocks.)
\section{Main statements}
\begin{theorem}\label{thm:main}
For $0<\beta<\tfrac12$ and $0<\theta<\tfrac{1+\beta}{2}$,
\[
\TR = \mathcal{I}_R(\chi_Y,h)+\mathcal{G}_R(\chi_Y,h)+O(R^{1-\varepsilon(\theta,\beta)}),
\]
with $\varepsilon(\theta,\beta)=\min\{\theta,1-\theta+\beta,\tfrac12,1-2\theta+\beta\}-\delta$ for any $\delta>0$, where the implicit constants depend polynomially on $m+\injrad(X_{\mathrm{core}})^{-1}$ and on finitely many seminorms of $\chi$ and $h$.
\end{theorem}

\begin{remark}
The admissible region is $0<\beta<\tfrac12$ and $0<\theta<\tfrac{1+\beta}{2}$; on the boundary $1-2\theta+\beta=0$ and the power-saving vanishes.
\end{remark}

% ===========================
% Acknowledgments & Data note
% ===========================
\section*{Acknowledgments}
The author thanks colleagues for helpful discussions. Any remaining errors are the author's.

\section*{Data availability}
Not applicable.

% ===========================
% References (manual for Block 0; will switch to .bib later)
% ===========================
\begin{thebibliography}{99}

\bibitem{selberg1956}
A.~Selberg,
\textit{Harmonic analysis and discontinuous groups},
J. Indian Math. Soc. \textbf{20} (1956), 47--87.

\bibitem{hejhal1976}
D.~A.~Hejhal,
\textit{The Selberg Trace Formula for $\mathrm{PSL}(2,\mathbb{R})$}, Vol.~I,
Lecture Notes in Mathematics \textbf{548}, Springer, 1976.
ISBN: 978-3-540-07727-7, \doi{10.1007/BFb0074437}

\bibitem{hejhal1983}
D.~A.~Hejhal,
\textit{The Selberg Trace Formula for $\mathrm{PSL}(2,\mathbb{R})$}, Vol.~II,
Lecture Notes in Mathematics \textbf{1001}, Springer, 1983.
\doi{10.1007/BFb0061300}

\bibitem{mueller1983}
W.~M\"uller,
\textit{Spectral theory for Riemannian manifolds with cusps},
J. Differential Geom. \textbf{18} (1983), 575--598.
\doi{10.4310/jdg/1214437785}

\bibitem{iwaniec1995}
H.~Iwaniec, P.~Sarnak,
\textit{$L^\infty$ norms of eigenfunctions on arithmetic surfaces},
Ann. of Math. \textbf{141} (1995), 301--320.
\doi{10.2307/2118520}

\bibitem{buser1992}
P.~Buser,
\textit{Geometry and Spectra of Compact Riemann Surfaces},
Birkh\"auser, 1992.
ISBN: 978-0-8176-3404-7, \doi{10.1007/978-1-4684-9172-2}

\bibitem{zworski2012}
M.~Zworski,
\textit{Semiclassical Analysis},
Graduate Studies in Mathematics \textbf{138}, AMS, 2012.
\doi{10.1090/gsm/138}

\bibitem{dyatlovzworski2019}
S.~Dyatlov, M.~Zworski,
\textit{Mathematical Theory of Scattering Resonances},
University Lecture Series \textbf{200}, AMS, 2019.
\doi{10.1090/ulect/200}

\bibitem{chazarain1974}
J.~Chazarain,
\textit{Formule de Poisson pour les vari\'et\'es riemanniennes},
Invent. Math. \textbf{24} (1974), 65--82.
MR0350064, Zbl 0284.35058, \doi{10.1007/BF01418762}

\end{thebibliography}

\end{document}
