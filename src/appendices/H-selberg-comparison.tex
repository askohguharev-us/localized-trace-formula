% H-selberg-comparison.tex
\appendix
\section*{Appendix H. Comparison with the Selberg Trace Formula}
\addcontentsline{toc}{section}{Appendix H. Comparison with the Selberg Trace Formula}
\label{appH:root}

\noindent\textbf{Purpose of Appendix H.}
We align the localized trace identity proved in the main text with the classical
Selberg trace formula on finite-area hyperbolic surfaces. The goal is to match term-by-term
the identity, parabolic, and hyperbolic contributions, and to quantify the remainder produced
by localization and smoothing.

\subsection*{H.1. Statement of the classical formula}
\label{appH:selberg-statement}
\noindent
We recall the Selberg trace formula in the normalizations of \cite[Ch.~11]{Iwaniec2002}
for even test functions with suitable decay and record the identity/parabolic/hyperbolic terms.

\subsection*{H.2. Localization and smoothing: interface terms}
\noindent
Using the geometric constants of Appendix~A and the microlocal calculus of Appendix~C,
we express the localized trace as the Selberg side plus explicit boundary counterterms
depending on the smoothing profile.

\subsection*{H.3. Matching of hyperbolic contributions}
\noindent
We show that the localized orbital integrals over conjugacy classes agree with the Selberg
hyperbolic sum up to an error controlled by the support scale and the injectivity data.

\subsection*{H.4. Bessel kernels and spectral measure}
\noindent
We match the Bessel-$K_{ir}$ representation used in Appendix~B with the spectral measure
normalization of the Selberg formula; this identifies the spectral density and fixes remaining constants.

\subsection*{H.5. Consolidated comparison and remainder}
\label{appH:conclusion}
\noindent
We collect the identities to obtain a consolidated comparison theorem stating that
the localized trace equals the Selberg trace plus explicit boundary/smoothing remainders,
uniform in the truncation parameter $Y$.

\bigskip
\noindent\textbf{References.}
Normalizations and statements follow \cite{Iwaniec2002,Hejhal1983,Helgason}.
