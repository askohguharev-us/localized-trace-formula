\section{Introduction}

We establish a localized trace formula for finite-area hyperbolic
surfaces $X=\Gamma\backslash\HH$ which isolates the discrete cuspidal
spectrum in short spectral windows $[R-R^\theta,\,R+R^\theta]$
under a geometric height cutoff $y\le Y=R^\beta$. The identity term
is expressed via an effective volume, the geometric contribution
comes from closed geodesics of length $\ll R^\theta$, and the
remainder enjoys a power saving $O(R^{1-\varepsilon(\theta,\beta)})$.
All constants are polynomial in the geometric data of~$X$.

The proof is fully microlocal: we construct cutoff projectors and
windowed kernels and control their contributions without invoking
continuous-spectrum expansions. This yields a \emph{windowed Weyl law}
for the cuspidal spectrum with explicit exponents.

\subsection*{Method and contributions}
\begin{itemize}
  \item Construction of frequency-windowed test operators and height
        truncations compatible with the cuspidal subspace.
  \item Kernel bounds uniform in $(R,\theta,\beta)$ with explicit
        transport of singularities and geometric propagation.
  \item A localized trace identity whose spectral side contains only
        cuspidal eigenvalues in the chosen window, and whose geometric
        side involves short closed geodesics and an effective-volume term.
  \item A power-saving remainder $O(R^{1-\varepsilon(\theta,\beta)})$
        with $\varepsilon(\theta,\beta)>0$ on a nonempty region of
        parameters.
\end{itemize}

\subsection*{Historical background}
Selberg’s trace formula initiated the spectral–geometric relation
for hyperbolic surfaces~\cite{selberg1956}; analytic foundations and
implementations were developed in \cite{hejhal1976,hejhal1983},
while manifolds with cusps were treated in~\cite{mueller1983}.
Geometric and spectral preliminaries are presented in~\cite{buser1992}.
Microlocal and semiclassical tools appear systematically in
\cite{zworski2012,dyatlovzworski2019}; for propagation phenomena see
also~\cite{chazarain1974}. The present paper introduces a \emph{localized}
framework adapted to short spectral windows and effective geometric
cutoffs, producing polynomial dependence on geometric data and a
power-saving error term for the cuspidal contribution.
