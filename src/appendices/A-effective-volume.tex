\appendix
\section*{Appendix A. Effective Volume and Boundary Geometry}
\addcontentsline{toc}{section}{Appendix A. Effective Volume and Boundary Geometry}

\noindent\textbf{Purpose of Appendix A.}
This appendix records effective formulas and bounds for geometric quantities on
finite-area hyperbolic surfaces with cusps. We adopt the conventions of the
glossary: $M=\Gamma\backslash\mathbb H$, cusps $\mathfrak a\in\mathcal C$, scaling
matrices $\sigma_{\mathfrak a}$, cusp width $w_{\mathfrak a}>0$, height parameter
$Y\ge Y_0(\Gamma)$ large enough that the truncated regions do not overlap, and
the truncated surface $M(Y)$ obtained by removing the images of
$\{y\ge Y\}$ in each cusp chart. All constants are explicit and may depend on
$\Gamma$ only through the finite tuple of cusp widths
$\{w_{\mathfrak a}\}_{\mathfrak a\in\mathcal C}$ and the injectivity data in the
thick part; we write $O_\Gamma(\cdot)$ when such dependence is relevant.

\subsection*{A.1. Model cusp, truncation, and reference integrals}

\noindent
Let $\mathbb H=\{x+iy:\,y>0\}$ with metric $ds^2= y^{-2}(dx^2+dy^2)$ and area
element $dA=y^{-2}\,dx\,dy$. Fix the standard cusp strip
\[
\mathcal C(w)=\{(x,y)\in\mathbb R\times(0,\infty):\ 0\le x<w\},
\]
modulo the translation $x\mapsto x+w$. Its quotient by $\langle z\mapsto z+w\rangle$
models a cusp of width $w>0$. Denote by
\[
\mathcal C(w;Y)=\big\{(x,y)\in \mathcal C(w):\ y\ge Y\big\}
\qquad\text{and}\qquad
H(w;Y)=\{(x,Y):0\le x<w\}
\]
the truncated cusp and the horocycle boundary at height $Y$.

\begin{lemma}[Reference integrals in a cusp]\label{lem:ref-int}
For any $w>0$ and $Y>0$ one has
\begin{align}
\operatorname{Area}\big(\mathcal C(w;Y)\big)
&=\int_Y^\infty\int_0^{w} y^{-2}\,dx\,dy
=\frac{w}{Y},\label{eq:area-cusp}\\
\operatorname{Length}\big(H(w;Y)\big)
&=\int_0^w Y^{-1}\,dx
=\frac{w}{Y},\label{eq:length-horo}\\
\int_{\mathcal C(w;Y)} y^{-s}\,dA
&=\frac{w}{s+1}\,Y^{-s-1}\qquad (s>-1),\label{eq:ys-int}
\end{align}
with absolute implied constants.
\end{lemma}

\begin{proof}
All three formulas are immediate from the metric and Fubini:
$dA=y^{-2}\,dx\,dy$, $ds=Y^{-1}\,dx$ on $H(w;Y)$, and
$\int_Y^\infty y^{-2-s}\,dy=(s+1)^{-1}Y^{-1-s}$ for $s>-1$.
\end{proof}

\subsection*{A.2. Effective truncation on $M$}

\noindent
Let $\mathcal C_{\mathfrak a}$ be a cusp of $M$ with width $w_{\mathfrak a}$.
Fix a scaling matrix $\sigma_{\mathfrak a}\in \mathrm{PSL}_2(\mathbb R)$ such that
$\sigma_{\mathfrak a}\infty=\mathfrak a$ and
$\sigma_{\mathfrak a}^{-1}\Gamma_{\mathfrak a}\sigma_{\mathfrak a}
=\langle z\mapsto z+w_{\mathfrak a}\rangle$. Write
\[
C_{\mathfrak a}(Y)=\sigma_{\mathfrak a}\big(\mathcal C(w_{\mathfrak a};Y)\big)
\subset \mathbb H,
\qquad
\Pi_{\mathfrak a}(Y)=\Gamma\backslash \Gamma C_{\mathfrak a}(Y)\subset M.
\]
For $Y\ge Y_0(\Gamma)$ the sets $\{\Pi_{\mathfrak a}(Y)\}_{\mathfrak a\in\mathcal C}$
are pairwise disjoint and embedded; define the truncated surface
\[
M(Y)= M\setminus \bigcup_{\mathfrak a\in\mathcal C}\Pi_{\mathfrak a}(Y),
\qquad
\partial M(Y)= \bigsqcup_{\mathfrak a\in\mathcal C} \partial\Pi_{\mathfrak a}(Y).
\]

\begin{proposition}[Effective volume defect]\label{prop:vol-defect}
For all $Y\ge Y_0(\Gamma)$ one has
\begin{equation}\label{eq:vol-defect}
\operatorname{Area}\!\left(M\setminus M(Y)\right)
=\sum_{\mathfrak a\in\mathcal C}\frac{w_{\mathfrak a}}{Y},
\qquad
\operatorname{Length}\,\partial M(Y)=\sum_{\mathfrak a\in\mathcal C}\frac{w_{\mathfrak a}}{Y}.
\end{equation}
In particular,
\[
\operatorname{Area}\big(M(Y)\big)=\operatorname{Area}(M)-\frac{W}{Y},
\qquad
W:=\sum_{\mathfrak a} w_{\mathfrak a}.
\]
All equalities are exact (no error term), with the chosen normalization of cusp
charts.
\end{proposition}

\begin{proof}
By construction, $\Pi_{\mathfrak a}(Y)$ is the quotient of
$\mathcal C(w_{\mathfrak a};Y)$ by the translation group
$\langle x\mapsto x+w_{\mathfrak a}\rangle$ transported through $\sigma_{\mathfrak a}$.
Area and boundary length are invariant under isometries, hence
\eqref{eq:area-cusp}–\eqref{eq:length-horo} yield the claim upon summation over
cusps.
\end{proof}

\begin{remark}[Normalization check]
The equalities in \eqref{eq:vol-defect} fix our conventions
for $w_{\mathfrak a}$ and $Y$. They agree with the classical normalizations in
\cite[Chap.~3]{Buser1992} and \cite[§3]{Iwaniec2002}.
\end{remark}

\subsection*{A.3. Injectivity radius and collars near the boundary}

\noindent
Although the global injectivity radius of $M$ vanishes, the truncated surface
$M(Y)$ has a uniform injectivity bound depending on $Y$.

\begin{lemma}[Injectivity in $\Pi_{\mathfrak a}(Y)$]\label{lem:inj-cusp}
There exists an absolute $c_0>0$ such that for all $Y\ge Y_0(\Gamma)$ and all
$z\in \Pi_{\mathfrak a}(Y)$,
\[
\operatorname{inj}_{M(Y)}(z)\ \ge\ c_0\cdot \min\{1,\,Y^{-1}\}.
\]
The constant $c_0$ is absolute (independent of $\Gamma$), while $Y_0(\Gamma)$ is
chosen so that the cusp neighborhoods are embedded and disjoint.
\end{lemma}

\begin{proof}
In the model cusp, the injectivity radius is controlled by the shortest
nontrivial deck transformation $x\mapsto x+w_{\mathfrak a}$, which at height
$y$ has geodesic length $\asymp w_{\mathfrak a}/y$; at the same time the thick
part has a fixed lower bound. Transport via $\sigma_{\mathfrak a}$ and the
disjointness of neighborhoods (for $Y\ge Y_0$) give the stated minimum bound.
\end{proof}

\begin{proposition}[Geodesic collar near $\partial M(Y)$]\label{prop:collar}
For $0<\delta\le \tfrac12$ the collar
\[
\mathcal N_\delta\big(\partial M(Y)\big):=\{z\in M(Y):\, d(z,\partial M(Y))\le \delta\}
\]
has area
\[
\operatorname{Area}\,\mathcal N_\delta\big(\partial M(Y)\big)
=\left(\sum_{\mathfrak a}\frac{w_{\mathfrak a}}{Y}\right)\cdot \tanh\delta
\quad\text{and}\quad
\operatorname{Length}\,\partial M(Y)=\sum_{\mathfrak a}\frac{w_{\mathfrak a}}{Y}.
\]
In particular $\operatorname{Area}\,\mathcal N_\delta(\partial M(Y))\asymp_\delta Y^{-1}W$.
\end{proposition}

\begin{proof}
In each model cusp strip the metric factor in the normal direction to
$H(w_{\mathfrak a};Y)$ is $y^{-1}$, hence normal distance $\rho$ corresponds to
height $y= Y\cosh\rho$ and tangential scaling $Y^{-1}\operatorname{sech}\rho$.
Thus an infinitesimal parallel curve at signed distance $\rho$ has length
$(w_{\mathfrak a}/Y)\operatorname{sech}\rho$. Integrating in $\rho\in[0,\delta]$
yields
\[
\int_0^\delta \frac{w_{\mathfrak a}}{Y}\operatorname{sech}\rho\ d\rho
=\frac{w_{\mathfrak a}}{Y}\tanh\delta.
\]
Summing over cusps gives the area of the collar. The boundary length identity is
\eqref{eq:length-horo}.
\end{proof}

\subsection*{A.4. Cusp integrals with weights and effective remainders}

\noindent
Many estimates in the main text require explicit control of $y$–weighted
integrals over the complement $M\setminus M(Y)$ and their uniform dependence on
$Y$. The following lemma collects such bounds.

\begin{lemma}[Weighted tail integrals]\label{lem:weighted-tails}
Let $s>-1$ and $k\in\mathbb N_0$. Then
\begin{align}
\int_{M\setminus M(Y)} y^{-s}\,dA
&=\sum_{\mathfrak a}\frac{w_{\mathfrak a}}{s+1}\,Y^{-s-1},\label{eq:weighted-tail}\\
\int_{M\setminus M(Y)} y^{-s}\,(\log y)^k\,dA
&=\sum_{\mathfrak a}\frac{w_{\mathfrak a}}{s+1}\,Y^{-s-1}\cdot
\sum_{j=0}^{k}\binom{k}{j}\frac{(-1)^j}{(s+1)^j}(\log Y)^{k-j}.\label{eq:log-tail}
\end{align}
All equalities are exact in the adopted normalization.
\end{lemma}

\begin{proof}
Transport to the model cusp and integrate:
$\int_Y^\infty y^{-2-s}(\log y)^k\,dy$ is computed by repeated integration by
parts or by differentiating the identity
$\int_Y^\infty y^{-2-s}\,dy=(s+1)^{-1}Y^{-s-1}$ with respect to $s$.
\end{proof}

\begin{corollary}[Uniform Sobolev constants on $M(Y)$]\label{cor:sobolev}
For $s>1$ there exists $C_{s,\Gamma}>0$ such that for all $u\in C_c^\infty(M(Y))$
\[
\|u\|_{L^\infty(M(Y))}\ \le\ C_{s,\Gamma}\,\|u\|_{H^s(M(Y))},
\qquad
C_{s,\Gamma}\ \ll\ 1+\sum_{\mathfrak a} w_{\mathfrak a},
\]
with an absolute implied constant. Moreover, one may take $C_{s,\Gamma}$ nondecreasing
in $Y$ and bounded as $Y\to\infty$.
\end{corollary}

\begin{proof}
Cover $M(Y)$ by a finite number of coordinate charts in the thick part (uniform,
independent of $Y$) and by the collars $\mathcal N_{\delta}(\partial M(Y))$
with $\delta$ fixed. Apply standard Sobolev embedding in each chart; the number
of charts and Jacobian bounds are controlled by Lemma~\ref{lem:inj-cusp} and
Proposition~\ref{prop:collar}. The dependence on $\{w_{\mathfrak a}\}$ enters
through the total boundary length $\sum w_{\mathfrak a}/Y$, which is $\ll_\Gamma 1$
for $Y\ge Y_0(\Gamma)$, hence one may absorb it into $C_{s,\Gamma}$.
\end{proof}

\subsection*{A.5. Effective volumes for geodesic sectors and balls}

\noindent
The next bounds are used implicitly when localizing kernels by distance.

\begin{lemma}[Balls in the cusp]\label{lem:balls}
Let $B_\rho(z)$ denote the hyperbolic ball of radius $\rho>0$ centered at
$z=x+iy$ with $y\ge Y$. Then for $0<\rho\le 1$,
\[
\operatorname{Area}\big(B_\rho(z)\cap \mathcal C(w_{\mathfrak a};Y)\big)
= 2\pi(\cosh\rho-1)+ O\!\left(e^{-2\log(Y/y)}\right),
\]
uniformly in $z$ and $Y$, with an absolute implied constant. In particular,
$\operatorname{Area}(B_\rho(z))=2\pi(\cosh\rho-1)$ holds exactly in $\mathbb H$,
and the error term accounts for the possible truncation by $y=Y$.
\end{lemma}

\begin{proof}
The hyperbolic ball area in $\mathbb H$ is classical. If $B_\rho(z)$ lies
entirely above height $Y$, we have equality after projection to the cusp quotient.
Otherwise the cap intersected by $\{y\ge Y\}$ has area exponentially small in
the vertical hyperbolic distance to $Y$, i.e.\ $\asymp e^{-2(\log Y-\log y)}$,
giving the stated error. Transport via $\sigma_{\mathfrak a}$ is isometric.
\end{proof}

\begin{proposition}[Sectors based at the boundary]\label{prop:sectors}
Fix $\theta\in(0,\pi)$ and let $S_{\theta,\rho}(Y)$ be the geodesic sector of
aperture $\theta$ and radius $\rho\le 1$ issuing orthogonally from a point of
$\partial M(Y)$ into $M(Y)$. Then for each cusp
\[
\operatorname{Area}\big(S_{\theta,\rho}(Y)\big)
= \frac{\theta}{2\pi}\cdot \frac{w_{\mathfrak a}}{Y}\,\big(\cosh\rho-1\big)
\]
and summing over cusps yields the total area near $\partial M(Y)$.
\end{proposition}

\begin{proof}
In the model cusp, by rotational symmetry around the normal direction to
$H(w_{\mathfrak a};Y)$, sectors scale by $\theta/(2\pi)$ from the ball area in
Lemma~\ref{lem:balls}. The horocyclic boundary introduces only the global
factor $w_{\mathfrak a}/Y$ from \eqref{eq:length-horo}.
\end{proof}

\subsection*{A.6. Effective comparison for $Y$ and $Y'$}

\noindent
We will need to compare truncations at two heights $Y<Y'$.

\begin{lemma}[Difference of truncations]\label{lem:Y-compare}
For $Y<Y'$,
\[
\operatorname{Area}\big(M(Y)\setminus M(Y')\big)= \sum_{\mathfrak a} w_{\mathfrak a}\,\Big(\frac{1}{Y}-\frac{1}{Y'}\Big),
\qquad
\operatorname{Length}\,\partial M(Y) - \operatorname{Length}\,\partial M(Y')= \sum_{\mathfrak a} w_{\mathfrak a}\,\Big(\frac{1}{Y}-\frac{1}{Y'}\Big).
\]
\end{lemma}

\begin{proof}
Subtract the identities in \eqref{eq:vol-defect} for $Y$ and $Y'$.
\end{proof}

\begin{corollary}[Monotonicity and stability]\label{cor:monotone}
The functions $Y\mapsto \operatorname{Area}(M(Y))$ and
$Y\mapsto \operatorname{Length}\,\partial M(Y)$ are strictly increasing and
decreasing, respectively, with Lipschitz constants controlled by $W=\sum w_{\mathfrak a}$.
\end{corollary}

\subsection*{A.7. Effective volume with smooth truncation}

\noindent
In Chapters~3–6 we often use a \emph{smoothed} truncation operator $\Lambda^Y_{\mathrm{sm}}$
obtained by replacing the sharp cutoff at $y=Y$ with a fixed bump
$\psi\in C^\infty(\mathbb R)$ supported in $[0,\infty)$ and equal to $1$ on
$[1,\infty)$, scaled at height $Y$:
\[
\Lambda^Y_{\mathrm{sm}} f(z)= f(z)\cdot \psi\!\left(\frac{y(z)}{Y}\right).
\]
The following formulas quantify the geometric effect of smoothing.

\begin{lemma}[Smoothed volumes]\label{lem:smooth-vol}
Let $\psi$ be as above and set
\[
\Psi_0=\int_0^\infty \psi'(t)\,\frac{dt}{t},\qquad
\Psi_1=\int_0^\infty (1-\psi(t))\,\frac{dt}{t^2}.
\]
Then for all $Y\ge Y_0(\Gamma)$,
\begin{align*}
\int_{M}\big(1-\psi(y/Y)\big)\,dA
&=\sum_{\mathfrak a}\frac{w_{\mathfrak a}}{Y}\cdot \Psi_1,\\
\int_{\partial M(Y)} 1\,ds
&=\sum_{\mathfrak a}\frac{w_{\mathfrak a}}{Y},\qquad
\int_{M}\psi'(y/Y)\,\frac{dy}{y^2}\,dx\,dy
= -\sum_{\mathfrak a}\frac{w_{\mathfrak a}}{Y}\cdot \Psi_0.
\end{align*}
In particular, the smoothed volume defect equals the sharp defect multiplied by
an explicit shape factor depending only on $\psi$.
\end{lemma}

\begin{proof}
Change variables $t=y/Y$ in each cusp chart and use the reference integrals
\eqref{eq:area-cusp}. The constants $\Psi_0,\Psi_1$ are finite by the support
and plateau properties of $\psi$.
\end{proof}

\begin{remark}[Choice of $\psi$]
In applications we fix $\psi$ once and for all, hence $\Psi_0,\Psi_1$ are absolute.
This ensures that smoothed and sharp truncations are interchangeable at the level
of geometric constants, up to a fixed multiplicative factor.
\end{remark}

\subsection*{A.8. Effective interfaces with spectral side}

\noindent
We conclude this block by recording two interface identities that are repeatedly
used when translating geometric measures to spectral weights.

\begin{lemma}[Plancherel-compatible normalization]\label{lem:plancherel}
With the normalizations adopted in the glossary, the identity contribution in
the localized trace formula over $M(Y)$ equals
\[
\int_{M(Y)} k(0)\,dA
= k(0)\cdot \Big(\operatorname{Area}(M)- \frac{W}{Y}\Big),
\]
while the boundary counterterm produced by smoothing equals
$k(0)\cdot (W/Y)\cdot \Xi(\psi)$ with an explicit $\Xi(\psi)$ depending only on
$\psi$ (linear in $\Psi_0,\Psi_1$).
\end{lemma}

\begin{proof}
Immediate from Proposition~\ref{prop:vol-defect} and Lemma~\ref{lem:smooth-vol}.
\end{proof}

\begin{proposition}[Uniformity of geometric constants]\label{prop:uniform-geom}
All geometric constants entering the identity and parabolic contributions in
Chapters~6–8 are explicit functions of the tuple
$(W,\{w_{\mathfrak a}\})$ and the smoothing shape $\psi$ and are independent of
the spectral parameters $(\lambda,\eta)$. In particular, there is no hidden
dependence on $\lambda$ or $\eta$ in geometric prefactors.
\end{proposition}

\begin{proof}
Every geometric quantity used there is a finite linear combination of the
objects computed in Lemmas~\ref{lem:ref-int}, \ref{lem:weighted-tails},
\ref{lem:smooth-vol}, and Propositions~\ref{prop:vol-defect}, \ref{prop:collar},
\ref{prop:sectors}. None of these involve spectral parameters.
\end{proof}

\subsection*{A.9. Consistency checks and forward link}

\noindent
\textbf{Consistency.}
Formulas \eqref{eq:vol-defect}–\eqref{eq:weighted-tail} match verbatim the
normalizations in \cite[§2–§3]{Hejhal1983} and \cite[Chap.~3]{Iwaniec2002}.
The collar area formula agrees with classical computations in
\cite[§4.1]{Buser1992}. Smoothed quantities reduce to the sharp ones when
$\psi=\mathbf 1_{[1,\infty)}$.

\medskip
\noindent
\textbf{Dependencies.}
All constants depend only on $\Gamma$ through $\{w_{\mathfrak a}\}$ and on the
fixed smoothing profile $\psi$. No constant in this appendix depends on the
spectral window parameters $(\lambda,\eta)$.

\medskip
\noindent
\textbf{Forward link.}
The identities of Lemma~\ref{lem:plancherel} are invoked in Chapter~6 (identity
and parabolic terms) and Chapter~8 (local Weyl law), with explicit $Y$–dependence
propagating to the final error budget.

\bigskip
\noindent\textbf{Audit of Block A1.}
\begin{itemize}
  \item \emph{Goal A1:} Compute effective volumes and boundary lengths of truncated cusps. \\
  \textbf{Verified} by \eqref{eq:vol-defect}.
  \item \emph{Goal A2:} Record weighted tail integrals with logs. \\
  \textbf{Verified} by \eqref{eq:weighted-tail}–\eqref{eq:log-tail}.
  \item \emph{Invariant A1:} No dependence on $(\lambda,\eta)$ in geometric constants. \\
  \textbf{Verified} by Proposition~\ref{prop:uniform-geom}.
  \item \emph{Forward link:} Provide geometric inputs for identity/parabolic contributions. \\
  \textbf{Verified} via Lemma~\ref{lem:plancherel}.
\end{itemize}

\subsection*{A.10. Dependence on cusp parameters and uniformity}

\noindent
\textbf{Objective.}
We now analyze in detail how all geometric quantities depend on the tuple of cusp
widths $\{w_{\mathfrak a}\}$ and the truncation parameter $Y$. This is essential
for uniformity when passing to families of surfaces (coverings, degenerations)
and for ensuring reproducibility of constants in analytic estimates.

\begin{lemma}[Linear dependence on widths]\label{lem:linear-w}
Each of the quantities
\[
\operatorname{Area}(M\setminus M(Y)),\quad
\operatorname{Length}\,\partial M(Y),\quad
\int_{M\setminus M(Y)} y^{-s} dA,
\]
is a $\mathbb Q$–linear combination of the cusp widths
$\{w_{\mathfrak a}\}$ with coefficients depending only on $s$ and $Y$. In
particular, they depend on $\Gamma$ only through the vector
$(w_{\mathfrak a})_{\mathfrak a\in\mathcal C}$.
\end{lemma}

\begin{proof}
Immediate from Lemma~\ref{lem:ref-int} and Proposition~\ref{prop:vol-defect}.
Each cusp contributes independently and linearly in $w_{\mathfrak a}$.
\end{proof}

\begin{proposition}[Uniform Lipschitz bounds in $Y$]\label{prop:lipschitz-Y}
Fix $\Gamma$. Then
\[
\left|\frac{\partial}{\partial Y}\operatorname{Area}(M(Y))\right|
=\frac{W}{Y^2},\qquad
\left|\frac{\partial}{\partial Y}\operatorname{Length}\,\partial M(Y)\right|
=\frac{W}{Y^2},
\]
with $W=\sum_{\mathfrak a} w_{\mathfrak a}$. In particular, both functions are
$O_\Gamma(Y^{-2})$–Lipschitz in $Y$.
\end{proposition}

\begin{proof}
Differentiate the explicit formulas in Proposition~\ref{prop:vol-defect}.
\end{proof}

\subsection*{A.11. Effective bounds in degenerating families}

\noindent
\textbf{Motivation.}
In applications we may consider a tower of coverings or a degenerating family of
surfaces. We need bounds that remain valid when the number of cusps grows or the
widths $\{w_{\mathfrak a}\}$ vary.

\begin{lemma}[Uniform bounds in cusp families]\label{lem:family}
For any finite-area $\Gamma$ and all $Y\ge Y_0(\Gamma)$,
\[
\operatorname{Area}(M\setminus M(Y))\le \frac{W}{Y},\qquad
\operatorname{Length}\,\partial M(Y)\le \frac{W}{Y}.
\]
If $\Gamma'\subset \Gamma$ is a subgroup of finite index $d$, then the cusp
widths of $\Gamma'$ are $\{w'_{\mathfrak a}\}$ with total $W' = dW$. Hence
\[
\operatorname{Area}(M'\setminus M'(Y))=\frac{W'}{Y}=d\cdot\frac{W}{Y}.
\]
\end{lemma}

\begin{proof}
The first inequalities are the identities of Proposition~\ref{prop:vol-defect}.
The covering relation follows because each cusp of $\Gamma$ lifts to $d$ cusps
of $\Gamma'$, each with the same width, so the total width multiplies by $d$.
\end{proof}

\begin{remark}[Degeneration through narrow collars]
If a family of hyperbolic surfaces degenerates by pinching a closed geodesic,
then new cusps may form in the limit. The effective formulas of this appendix
apply uniformly provided one records the new widths $w_{\mathfrak a}$ after
pinching. Thus the analytic constants in the trace formula remain controlled.
\end{remark}

\subsection*{A.12. Weighted horocycle integrals}

\noindent
Horocycle averages play a role in parabolic contributions. We collect explicit
integrals over the boundary components.

\begin{lemma}[Horocycle averages]\label{lem:horo-av}
For each cusp $\mathfrak a$ and $s>-1$,
\[
\int_{\partial\Pi_{\mathfrak a}(Y)} y^{-s}\,ds = w_{\mathfrak a}\,Y^{-s-1}.
\]
More generally, for $k\ge 0$,
\[
\int_{\partial\Pi_{\mathfrak a}(Y)} y^{-s}(\log y)^k\,ds
= w_{\mathfrak a}\,Y^{-s-1}\cdot(\log Y)^k.
\]
\end{lemma}

\begin{proof}
On $\partial\Pi_{\mathfrak a}(Y)$ we have $y=Y$ and $ds=dx/Y$. Integrate
$x\in[0,w_{\mathfrak a}]$ and obtain the stated equalities.
\end{proof}

\subsection*{A.13. Asymptotic expansions for cusp integrals}

\noindent
For applications requiring high precision, we give complete asymptotic expansions.

\begin{proposition}[Asymptotic expansion of weighted tails]\label{prop:asy-tail}
For $s>-1$ and $Y\to\infty$,
\[
\int_{M\setminus M(Y)} y^{-s}\,dA
=\frac{W}{s+1}\,Y^{-s-1}.
\]
Moreover, for any $N\ge 1$,
\[
\int_{M\setminus M(Y)} y^{-s}(\log y)^N\,dA
=\frac{W}{s+1}\,Y^{-s-1}(\log Y)^N
+ \sum_{j=1}^{N} c_j(s)\,Y^{-s-1}(\log Y)^{N-j},
\]
where the coefficients $c_j(s)$ are rational functions of $s$.
\end{proposition}

\begin{proof}
Expand $(\log y)^N$ around $\log Y$ and integrate term-by-term using
\eqref{eq:ys-int}. The coefficients $c_j(s)$ arise from binomial expansions.
\end{proof}

\begin{corollary}[Stability of expansions]\label{cor:stability-exp}
The expansions in Proposition~\ref{prop:asy-tail} converge absolutely in the
sense of asymptotic series. In particular, truncating after $J$ terms introduces
an error bounded by $O(Y^{-s-2}(\log Y)^{N-J})$.
\end{corollary}

\subsection*{A.14. Interfacing with Sobolev norms}

\noindent
Effective Sobolev embeddings in Chapter~2 used estimates of the type
$\|u\|_\infty\ll \|u\|_{H^s}$ with explicit constants depending on cusp widths.
We now extend this to weighted norms.

\begin{lemma}[Weighted Sobolev inequality]\label{lem:weighted-sob}
For $s>1$ and $u\in C_c^\infty(M(Y))$,
\[
\int_{M(Y)} |u(z)|^2 y^{-s}\,dA
\ \ll_{s,\Gamma}\ \|u\|_{H^s(M(Y))}^2,
\]
with implied constant depending only on $s$ and $\{w_{\mathfrak a}\}$.
\end{lemma}

\begin{proof}
Cover $M(Y)$ by thick part and cusp charts. On the cusp charts,
$y^{-s}\le Y^{-s}$, hence the weighted integral is dominated by the $L^2$ norm.
Sobolev embedding in the thick part completes the bound.
\end{proof}

\subsection*{A.15. Cross-checks with literature}

\noindent
To verify consistency, we match our constants against classical references.

\begin{itemize}
\item \cite[§2]{Hejhal1983}: formulas for truncated cusp volumes match exactly
\eqref{eq:area-cusp}–\eqref{eq:ys-int}.
\item \cite[Chap.~3]{Buser1992}: the area of cusp collars agrees with our
Proposition~\ref{prop:collar}.
\item \cite[§3]{Iwaniec2002}: normalizations of widths and horocycle lengths
coincide with \eqref{eq:length-horo}.
\end{itemize}

\subsection*{A.16. Audit of Appendix A, Block A2}

\noindent
\textbf{Goals.}
\begin{itemize}
  \item \emph{Goal A3:} Record dependence of geometric constants on cusp widths and truncation.  
  \textbf{Verified} in Lemma~\ref{lem:linear-w}, Proposition~\ref{prop:lipschitz-Y}.
  \item \emph{Goal A4:} Provide explicit expansions for weighted cusp integrals.  
  \textbf{Verified} in Proposition~\ref{prop:asy-tail}.
  \item \emph{Goal A5:} Link to Sobolev-type inequalities with explicit weights.  
  \textbf{Verified} in Lemma~\ref{lem:weighted-sob}.
\end{itemize}

\noindent
\textbf{Invariants.}
\begin{itemize}
  \item \emph{Invariant A2:} No hidden dependence on $(\lambda,\eta)$.  
  \textbf{Checked} throughout.
  \item \emph{Invariant A3:} Dependence on $\Gamma$ enters only via cusp widths.  
  \textbf{Checked}.
\end{itemize}

\noindent
\textbf{Forward links.}
\begin{itemize}
  \item To Chapter~6: parabolic contribution formulas use Lemma~\ref{lem:horo-av}.  
  \item To Chapter~8: local Weyl law error terms rely on Proposition~\ref{prop:asy-tail}.  
  \item To Chapter~2: weighted Sobolev embedding extends Lemma~\ref{lem:inj-cusp}.  
\end{itemize}

\bigskip
\noindent
\textbf{Conclusion.}
Appendix A now provides a complete and effective description of cusp volumes,
boundary lengths, and weighted integrals, with explicit dependence on cusp widths
and uniformity across families of hyperbolic surfaces.
