% File: src/appendices/A-effvol.tex
\appendix
\section{Effective Volume Computations}
\label{app:effvol}

In this appendix we collect the detailed computations for effective
volume integrals and normalizations which underpin the main
asymptotic results of the localized trace formula.

\subsection{Hyperbolic volume element}
Let $X = \Gamma \backslash \HH$ be a finite-area hyperbolic surface.
The hyperbolic metric is
\[
ds^2 = y^{-2}(dx^2 + dy^2), \qquad z = x + iy \in \HH,
\]
with volume form
\[
d\vol(z) = y^{-2}\,dx\,dy.
\]
The volume of the truncated domain
\[
\mathcal{F}_Y = \{ z \in \mathcal{F} : \Im z \le Y \},
\]
where $\mathcal{F}$ is a fundamental domain, is
\[
\vol(\mathcal{F}_Y) = \vol(\mathcal{F}) - \frac{\#\{\text{cusps}\}}{Y}.
\]

\subsection{Kernel normalization integrals}
Recall the kernel
\[
K_R^Y(z,w) = \sum_{\gamma \in \Gamma} k_R^Y(z,\gamma w),
\]
with cutoff $\chi_Y$. Its $L^1$-mass is given by
\[
\int_X K_R^Y(z,z)\,d\vol(z).
\]
Unfolding and using Selberg’s pre-trace identity yields
\[
\int_X K_R^Y(z,z)\,d\vol(z) = \vol(\mathcal{F}_Y) \cdot H_R(0),
\]
where $H_R(0)$ is the Fourier transform of the test function at $0$.
This contributes the leading term in the identity component.

\subsection{Asymptotics with cusp cutoff}
For cusp regions, introduce cutoff $\chi_Y$ with $Y = R^\beta$. Then
\[
\int_{\Im z > Y} K_R^Y(z,z)\,d\vol(z) \ll Y^{-1} R^C,
\]
for some constant $C>0$. This term is negligible for admissible $(\theta,\beta)$
since $Y^{-1} = R^{-\beta}$ dominates polynomial error bounds.

\subsection{Stationary phase contributions}
To refine, write the kernel in oscillatory integral form:
\[
K_R^Y(z,z) = \int_{\RR} e^{iR\phi(z,\xi)} a_R(z,\xi)\,d\xi.
\]
Stationary points correspond to geodesic loops of length $0$ (identity contribution).
A stationary phase estimate yields
\[
K_R^Y(z,z) \sim \frac{R}{2\pi}\,a_R(z,\xi_0),
\]
uniformly in $z$. Integrating gives
\[
\int_X K_R^Y(z,z)\,d\vol(z) \sim \frac{R}{2\pi}\vol(X).
\]

\subsection{Effective constants}
Explicit constants appear as follows:
\begin{itemize}
\item The hyperbolic volume $\vol(X)$ enters linearly.
\item The number of cusps contributes $O(R^{-\beta})$ terms.
\item Sobolev embeddings introduce polynomial factors in genus $g$.
\end{itemize}

\subsection{Summary of Appendix~\ref{app:effvol}}
We have:
\[
\int_X K_R^Y(z,z)\,d\vol(z) \;=\; \vol(X)\cdot H_R(0) \;+\; O(R^{1-\varepsilon}),
\]
with $\varepsilon(\theta,\beta)>0$ from admissibility conditions.
This validates the effective normalization constants in the identity
contribution of the trace formula.
