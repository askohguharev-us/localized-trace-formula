% ===== Appendix A (file: src/appendices/A-effvol.tex) =====
\section{Effective volume normalizations}
\label{app:effvol}

This appendix fixes the volume and Fourier–Plancherel normalizations used
throughout the note.  We keep the setup minimal so that it is compatible
with the arXiv-compliant preamble of the main text.

\subsection*{A.1. Measure on $X$ and $T^*X$}
Let $(X,g)$ be a smooth $n$–dimensional Riemannian manifold.
The Riemannian volume on $X$ is denoted by $\mathrm{d}V_g$.
On the cotangent bundle $T^*X$ we use the Liouville (symplectic) measure
$\mathrm{d}\mu_{\mathrm{Liou}}$, i.e.
\[
  \mathrm{d}\mu_{\mathrm{Liou}}
  = \frac{\omega^n}{n!},
\]
where $\omega$ is the canonical symplectic form on $T^*X$.
For $x\in X$ and $r>0$ we write
$B_X(x,r)$ for the geodesic ball in $X$ and
$B^*_x(r) \subset T^*_xX$ for the Euclidean ball in the cotangent fiber.
The product box has volume
\[
  \mathrm{Vol}\big(B_X(x,r)\times B^*_x(R)\big)
  = \int_{B_X(x,r)}\!\!\int_{B^*_x(R)} \mathrm{d}\mu_{\mathrm{Liou}}.
\]

\subsection*{A.2. Fourier transform}
For $f\in \mathcal{S}(\mathbb{R}^n)$ we use the unitary convention
\[
  \widehat{f}(\xi)
  = \int_{\mathbb{R}^n} f(x)\,e^{-2\pi i\, x\cdot \xi}\, \mathrm{d}x,
  \qquad
  f(x)
  = \int_{\mathbb{R}^n} \widehat{f}(\xi)\,e^{2\pi i\, x\cdot \xi}\, \mathrm{d}\xi.
\]
With this choice, Plancherel holds in the form
$\|f\|_{L^2(\mathbb{R}^n)}=\|\widehat{f}\|_{L^2(\mathbb{R}^n)}$.

\subsection*{A.3. Effective local volume}
For a measurable $E\subset T^*X$ we set
\[
  \mathrm{vol}_{\mathrm{eff}}(E)
  := \frac{1}{\mathrm{Vol}(X)} \int_E \mathrm{d}\mu_{\mathrm{Liou}}
  \quad \text{when $\mathrm{Vol}(X)<\infty$.}
\]
If $\mathrm{Vol}(X)=\infty$, we use a tempered exhaustion
$X=\bigcup_{j} \Omega_j$ with $\mathrm{Vol}(\Omega_j)\to\infty$ and define
\[
  \mathrm{vol}_{\mathrm{eff}}(E)
  := \lim_{j\to\infty}
     \frac{1}{\mathrm{Vol}(\Omega_j)}
     \int_{E\cap T^*\!\Omega_j} \mathrm{d}\mu_{\mathrm{Liou}},
\]
whenever the limit exists.  This normalization is the one implicitly used
in the localized trace identities of the main text and matches the
microlocal counting interpretation for compactly supported test symbols.

\subsection*{A.4. A scaling sanity check}
Let $a\in C_c^\infty(T^*X)$ and, for $\lambda>0$, define
$a_\lambda(x,\xi)=a(x,\xi/\lambda)$.
Then
\[
  \int_{T^*X} a_\lambda\,\mathrm{d}\mu_{\mathrm{Liou}}
  = \lambda^n \int_{T^*X} a\,\mathrm{d}\mu_{\mathrm{Liou}},
\]
so that $\mathrm{vol}_{\mathrm{eff}}$ scales like $\lambda^n$ in the
covariable, consistent with the principal symbol calculus used in the
localized trace formula.
