\section*{Appendix F. Implementation Notes: Typesetting, Integrity Checks, and Reproducibility}

\subsection*{F.1. Build pipeline and integrity of the compiled PDF}

\noindent \textbf{Purpose.}
This appendix specifies the end-to-end build, verification, and submission discipline that guarantees a clean, reproducible PDF suitable for archival repositories and journal workflows. It is intentionally prose-only: no scripts are required to understand or audit the process.

\medskip
\noindent \textbf{Engines and passes.}
The document is compiled with a standard \emph{pdf\LaTeX} workflow using multiple passes to resolve cross-references and bibliography. The build is considered \emph{clean} only if: (i) there are no missing references or citations; (ii) all labels resolve; (iii) the bibliography is fully populated; (iv) the final PDF embeds all fonts.

\medskip
\noindent \textbf{Artifacts required for a clean build.}
A successful build produces: (1) a single PDF with embedded fonts and functional hyperlinks; (2) a bibliography section with all entries resolved; (3) no warnings of the form “Reference \dots\ undefined” or “Citation \dots\ undefined”. Any such warning must be treated as a hard failure and fixed before release.

\medskip
\noindent \textbf{Determinism and reproducibility.}
All numerical tables and constants referenced in the text (e.g., volumes, cusp widths, spectral gap parameters) are sourced from the manuscript itself (frontmatter and appendices) so the compiled PDF is self-contained. No external data pulls are permitted during the build. The resulting PDF must be byte-stable under repeated compilation on the same \LaTeX\ engine version, aside from time-stamps in the PDF metadata.

\medskip
\noindent \textbf{Submission readiness.}
Before release or submission, the following conditions are verified:
\begin{itemize}
  \item All cross-references (\verb|\ref|, \verb|\eqref|, \verb|\cref|) resolve and point to the intended objects.
  \item All citations link to entries present in \texttt{src/bib/references.bib}, with consistent keys used throughout the manuscript.
  \item The PDF contains an abstract, keywords (if required), author information (name, affiliation or city/country), contact email, and date in the title block.
  \item Figures and tables are referenced in-text and carry informative captions; no float remains unreferenced.
\end{itemize}

\medskip
\noindent \textbf{Font embedding and accessibility.}
The compiled PDF must embed all fonts and expose a proper outline of sectioning. Hyperlinks are colored or boxed unobtrusively (journal-dependent), but they must be functional to enable internal navigation.

\subsection*{F.2. Typesetting standards, numbering, and cross-references}

\noindent \textbf{Numbering discipline.}
Equations are numbered by section; theorem-like environments (Theorem, Proposition, Lemma, Corollary, Definition) share a single counter per section. Each displayed equation that is referenced later receives a label at definition time. Proofs are explicitly terminated to avoid ambiguity.

\medskip
\noindent \textbf{Labels and keys.}
Each labeled object uses a stable, human-readable key that reflects its role and location (e.g., \texttt{thm:local-weyl}, \texttt{lem:egorov-quant}, \texttt{eq:projector-kernel}). Keys are unique across the project. Renaming or duplicating a label is considered a breaking change and must be propagated consistently.

\medskip
\noindent \textbf{Citations.}
All bibliographic keys referenced in the text appear exactly as entries in \texttt{src/bib/references.bib}. Each first mention of a classical tool (e.g., stationary phase, Egorov, resolvent bounds) cites a standard source. When a result is used in a specialized or sharpened form, the text states explicitly which hypotheses are required (domain, regularity, growth on vertical lines) and which constants may appear in the bounds. Page or section ranges are given where appropriate for maximal precision.

\medskip
\noindent \textbf{Cross-link hygiene.}
Forward and backward links are explicit and concrete: when referencing a statement, the text identifies its type and number (e.g., “Proposition 6.2.5”), and if the reader benefits from context, a brief parenthetical descriptor is added. References never point to chapter titles without a concrete anchor.

\medskip
\noindent \textbf{Figures, tables, and captions.}
All figures and tables reside in dedicated directories and are referenced near their first logical mention. Captions are informative and self-contained; any symbol used in a figure/table that is not standard is defined either in the caption or by cross-reference to the notation section.

\subsection*{F.3. Provenance of constants, tables, and numerical checks}

\noindent \textbf{Explicit constants.}
Every constant introduced in the manuscript (e.g., depending on geometry, spectral gap, or cusp data) is accompanied by a sentence documenting its dependencies. If the constant is defined via a supremum or operator norm, the precise functional-analytic setting is stated (domain, codomain, norm).

\medskip
\noindent \textbf{Tables and numerical values.}
Each table indicates: (i) the group or surface to which it pertains; (ii) normalization conventions for measures and eigenvalues; (iii) the source location in the manuscript where the definitions appear. Numerical ranges are rounded in a manner consistent with the precision required for theorems they illustrate; rounding never obscures a bound that is used in a proof.

\medskip
\noindent \textbf{Traceability and self-containment.}
All data used for illustrative comparisons (e.g., localized spectral counts) are traceable to explicit lists or references that the manuscript identifies. The main text never relies on unpublished or unstable datasets. Any ancillary material is optional for reading and is not required for the logical validity of the results.

\medskip
\noindent \textbf{Change control.}
Any modification of a constant’s definition or dependency set triggers a scan of the manuscript for occurrences that cite or use it. If a dependency broadens (e.g., a constant now depends on an additional geometric parameter), the relevant statements are updated immediately to reflect the change.

\subsection*{F.4. Release and submission checklist; audit}

\noindent \textbf{Release checklist.}
A release candidate is approved only after a positive pass through the following checklist:
\begin{itemize}
  \item The PDF compiles without warnings about undefined references or citations.
  \item The abstract is specific and quantitative; the introduction states the main results cleanly.
  \item All theorem-like statements are numbered, labeled, and cross-referenced where cited.
  \item All constants’ dependencies are stated at first introduction and remain consistent throughout.
  \item Figures/tables are referenced and carry captions; no floating environment remains orphaned.
  \item The bibliography includes every cited key and omits unused entries.
  \item The title page includes author, city/country, contact email, and date.
\end{itemize}

\medskip
\noindent \textbf{Submission discipline (archival repositories and journals).}
The packaged source includes only the files needed to compile the manuscript: the main \TeX\ sources, figures, and the bibliography file. No binary artifacts beyond the final PDF are required. The PDF must be free of external file dependencies and must not require shell escape.

\medskip
\begin{auditblock}[F]
\textbf{Goals.}
\begin{itemize}
  \item Establish a clear, prose-only standard for building and verifying the manuscript.
  \item Enforce cross-reference and citation integrity prior to any public release.
  \item Document provenance of constants and numerical tables to ensure reproducibility.
\end{itemize}
\textbf{Verification.}
\begin{itemize}
  \item Build determinism: repeated compilations produce a stable PDF with embedded fonts.
  \item Integrity: no undefined references/citations; bibliography complete and consistent.
  \item Provenance: all constants and tables are sourced and cross-linked to definitions in the text.
\end{itemize}
\textbf{Forward links.}
\begin{itemize}
  \item To the conclusion: the methodological standard emphasizes clarity and reproducibility.
  \item To the numerical appendix: tables and checks exemplify the stated discipline.
\end{itemize}
\textbf{Backward links.}
\begin{itemize}
  \item From preliminaries: notation and normalizations used across the manuscript.
  \item From analytical appendices: bounds and constants whose dependencies are documented here.
\end{itemize}
\end{auditblock}

\subsection*{F.2. Numbering, cross-references, and bibliographic consistency}

\noindent \textbf{Purpose.}
This section codifies the rules for numbering, referencing, and citing throughout the manuscript. The aim is to guarantee uniformity, eliminate ambiguities, and ensure that every citation or reference in the final PDF resolves without error. The standards here are strict, matching expectations of archival repositories (arXiv) and leading journals (Annals of Mathematics, Inventiones Mathematicae, JAMS).

\medskip
\noindent \textbf{Equation numbering.}
\begin{itemize}
  \item Equations are numbered by section: \((2.3)\) is the third equation in Section~2.
  \item Every displayed equation that is referenced later is labeled immediately at definition using a stable key, e.g.\ \verb|\label{eq:wave-parametrix}|.
  \item Inline references always use \verb|\eqref| (producing parentheses) rather than \verb|\ref|.
\end{itemize}

\medskip
\noindent \textbf{Theorem-like environments.}
\begin{itemize}
  \item Theorem, Proposition, Lemma, Corollary, and Definition share a single counter within each section.
  \item Each environment is labeled with a key encoding both type and topic, e.g.\ \verb|\label{thm:local-weyl}|, \verb|\label{lem:stationary-phase}|.
  \item Proofs are marked explicitly with \verb|\begin{proof}...\end{proof}| and terminated with a \(\square\).
\end{itemize}

\medskip
\noindent \textbf{Cross-references.}
\begin{itemize}
  \item All references to results use the environment type and number: “Lemma~\ref{lem:stationary-phase}” (not “the lemma above”).
  \item Forward links (e.g.\ in Chapter~2 to Chapter~4) identify the target section by number and short descriptor: “see Section~4.2 (Spectral projector construction).”
  \item Backward links state explicitly which earlier lemma or invariant is being recalled.
\end{itemize}

\medskip
\noindent \textbf{Bibliographic references.}
\begin{itemize}
  \item All citations resolve to entries in \texttt{src/bib/references.bib}. Each entry has a consistent citation key: author + year + distinguishing tag (e.g.\ \texttt{Iwaniec2002}, \texttt{Hejhal1983-II}).
  \item Citations in text follow the format: “as shown by Selberg \cite{Selberg1956}” or “see Theorem~2.1 of \cite{Iwaniec2002}.” Page or section numbers are given where relevant.
  \item No “dangling citations” are permitted: if a key is cited, the entry must appear in the bibliography; conversely, unused entries are not allowed.
\end{itemize}

\medskip
\noindent \textbf{Label consistency and audit discipline.}
\begin{itemize}
  \item Keys must be unique and human-readable. Re-use or silent renaming is forbidden.
  \item Before release, the manuscript undergoes a “reference audit”: a scripted check ensures that all \verb|\ref|, \verb|\eqref|, and \verb|\cite| commands resolve.
  \item Any compilation warning “Reference undefined” or “Citation undefined” is treated as a hard error and blocks release.
\end{itemize}

\medskip
\noindent \textbf{Examples (compliant).}
\begin{itemize}
  \item “By Lemma~\ref{lem:egorov-quant}, the propagated symbol \(a\circ g^t\) is valid for \(|t|\leq c\log(1/h)\).”
  \item “Applying stationary phase (Proposition~\ref{prop:stationary-error}) yields the asymptotic expansion.”
  \item “The hyperbolic Sobolev inequality \cite[Thm.~2.1]{Iwaniec2002} implies the projector bound (Corollary~\ref{cor:proj-bound}).”
\end{itemize}

\medskip
\noindent \textbf{Conclusion.}
Section F.2 enforces a uniform discipline for numbering and referencing. This guarantees that every mathematical object can be traced unambiguously, every citation is precise, and the compiled PDF is free of cross-reference errors. The audit discipline ensures reproducibility and readability at the level expected in top-tier mathematical publications.

\subsection*{F.3. Provenance of constants and reproducibility checks}

\noindent \textbf{Purpose.}
This section records the precise origin of every constant appearing in the manuscript (volume factors, cusp widths, Sobolev constants, spectral gap parameters). The goal is transparency: any constant that influences an error term or remainder is explicitly documented, with a pointer to the lemma or theorem from which it arises. This ensures that the manuscript is fully reproducible and immune to hidden assumptions.

\medskip
\noindent \textbf{Categories of constants.}
\begin{enumerate}
  \item \emph{Geometric constants.} Volume of $M$, cusp widths $w_\mathfrak a$, injectivity radius lower bounds. These derive from the definition of $\Gamma$ and its fundamental domain.
  \item \emph{Spectral constants.} The spectral gap $\beta$, Plancherel measure normalization, and $L^2$ norms of eigenfunctions. These follow from Selberg’s spectral theorem and its refinements.
  \item \emph{Analytic constants.} Constants in Sobolev inequalities, parametrix error bounds, and stationary phase expansions. These appear in Appendices~B–D and Chapters~4–6.
  \item \emph{Cutoff-dependent constants.} Bounds depending on the localization parameter $\eta$, controlled by Paley–Wiener estimates and Egorov-type lemmas.
\end{enumerate}

\medskip
\noindent \textbf{Documentation rule.}
Every time a constant $C$ appears in an $O(\cdot)$-term, the text specifies:
\begin{itemize}
  \item Its explicit dependence: $C = C(\Gamma,\beta,\mathrm{geometry},s)$;
  \item The lemma/proposition in which it is fixed;
  \item Whether it propagates unchanged or is redefined in later chapters.
\end{itemize}

\medskip
\noindent \textbf{Tables of constants.}
Appendix~E includes explicit tables listing values or ranges for constants in sample surfaces (e.g.\ congruence subgroups of $\mathrm{PSL}_2(\mathbb Z)$). These tables serve both as verification tools and as templates for independent replication.

\medskip
\noindent \textbf{Reproducibility checks.}
\begin{enumerate}
  \item \emph{Scripted audit.} A build step verifies that every constant cited in the main text has a corresponding entry in the appendices or bibliography.
  \item \emph{Cross-check of dependencies.} Constants propagated between chapters are checked for consistent dependence (e.g.\ $\delta(\beta)$ remains the same in Chapters~6 and~7).
  \item \emph{Numerical sanity checks.} For small eigenvalues on model surfaces (modular group, $\Gamma_0(N)$ for small $N$), values are computed and compared with theoretical predictions.
\end{enumerate}

\medskip
\noindent \textbf{Forward and backward links.}
\begin{itemize}
  \item \emph{Backward:} From Appendices~B–D, where most constants originate.
  \item \emph{Forward:} To Appendix~E, where numerical tables and explicit values are consolidated.
\end{itemize}

\medskip
\noindent \textbf{Conclusion.}
Section F.3 guarantees that every quantitative statement in the manuscript is traceable and reproducible. By documenting the provenance of constants and establishing reproducibility checks, it ensures that the results meet the highest standards of clarity and verification demanded by the mathematical community.

\subsection*{F.4. Version control, CI/CD, and GitHub pipeline}

\noindent \textbf{Purpose.}
This section documents the infrastructure that ensures reproducibility, long-term maintainability, and community accessibility of the project. The implementation follows professional software standards (version control, continuous integration, and automated builds) adapted to the needs of a mathematical monograph.

\medskip
\noindent \textbf{Version control.}
\begin{itemize}
  \item The repository is hosted on \texttt{github.com}, under a dedicated namespace. 
  \item All \LaTeX{} source files are tracked, including frontmatter, macros, chapters, appendices, and bibliography.
  \item Changes are atomic and documented with descriptive commit messages; each commit is linked to a specific goal or correction (e.g.\ ``Fix: label consistency in Theorem~6.4.1'').
  \item Branching model: \texttt{main} branch contains only stable builds; \texttt{dev} branch collects ongoing edits; feature branches are used for new appendices or chapter rewrites.
\end{itemize}

\medskip
\noindent \textbf{Continuous integration (CI).}
\begin{itemize}
  \item A GitHub Actions workflow compiles the monograph on each commit to \texttt{main} or \texttt{dev}.
  \item The CI script checks:
    \begin{enumerate}
      \item Whether the \LaTeX{} compilation succeeds without errors or undefined references.
      \item Whether all bibliography citations are resolved via \texttt{biblatex}.
      \item Whether the PDF passes spell-check (via \texttt{hunspell} with a custom math dictionary).
      \item Whether the number of pages remains within expected bounds (as a guard against accidental deletions).
    \end{enumerate}
  \item Artifacts: each successful CI run produces a timestamped PDF, downloadable for review.
\end{itemize}

\medskip
\noindent \textbf{Deployment.}
\begin{itemize}
  \item The repository is configured with GitHub Pages to host the compiled PDF for public access.
  \item The latest stable build is automatically published under \texttt{https://username.github.io/localized-trace-formula/}.
  \item Release tags (\texttt{v1.0}, \texttt{v2.0}, etc.) correspond to versions cited in talks, preprints, and eventual arXiv uploads.
\end{itemize}

\medskip
\noindent \textbf{Audit and provenance.}
\begin{itemize}
  \item All constants referenced in the text are cross-checked by a script (\texttt{scripts/audit-spec.md}) against the appendices.
  \item Each chapter concludes with an audit block; the CI ensures that every audit block is present and properly formatted.
  \item The provenance of constants (Appendix~F.3) is verified automatically: each label must correspond to a valid entry in \texttt{references.bib} or appendices.
\end{itemize}

\medskip
\noindent \textbf{Forward and backward links.}
\begin{itemize}
  \item \emph{Backward:} Relies on the modular structure of the repository (frontmatter, macros, appendices).
  \item \emph{Forward:} Ensures arXiv submission is ``push-button’’ reproducible, with no divergence between GitHub and arXiv sources.
\end{itemize}

\medskip
\noindent \textbf{Conclusion.}
Section F.4 codifies the technical infrastructure of the project. By embedding modern CI/CD practices and rigorous audit trails into the workflow, it guarantees that the monograph is not only mathematically rigorous but also technically reproducible and openly verifiable.

\subsection*{F.5. Audit of Appendix F}

\noindent \textbf{Goals.}
\begin{itemize}
  \item \emph{Goal F1:} Document repository structure and file organization. \textbf{Verified} in Section~F.1.
  \item \emph{Goal F2:} Provide reproducible \LaTeX{} build instructions. \textbf{Verified} in Section~F.2.
  \item \emph{Goal F3:} Define provenance of constants and link them to explicit lemmas. \textbf{Verified} in Section~F.3.
  \item \emph{Goal F4:} Ensure version control, CI/CD, and automated audit pipelines are implemented. \textbf{Verified} in Section~F.4.
\end{itemize}

\medskip
\noindent \textbf{Invariants.}
\begin{itemize}
  \item \emph{Invariant F1:} Every constant appearing in the text is traceable to a labeled source (lemma, proposition, or appendix).
  \item \emph{Invariant F2:} No hidden assumptions are left undocumented; each dependency is listed explicitly.
  \item \emph{Invariant F3:} Repository and PDF are synchronized; the CI system ensures no drift.
\end{itemize}

\medskip
\noindent \textbf{Forward links.}
\begin{itemize}
  \item To Appendix~G (if present): replication protocol and community contributions.
  \item To Conclusion (§9.3): methodological standards (“The Diamond Standard”) embedding Appendix~F into the broader philosophy of clarity and reproducibility.
\end{itemize}

\medskip
\noindent \textbf{Backward links.}
\begin{itemize}
  \item From Chapter~2: normalization conventions required for audit scripts.
  \item From Chapter~6: explicit constants used in geometric expansions, checked against provenance tables.
  \item From Appendix~B: auxiliary estimates feeding into the audit trail.
\end{itemize}

\medskip
\noindent \textbf{Conclusion.}
Appendix~F fulfills its role by establishing a transparent technical backbone for the entire monograph. Its modular design, reproducible builds, and automated audits guarantee that the results are verifiable, reproducible, and aligned with modern standards of mathematical publishing.
