\section*{Appendix F. Reproducibility and Implementation Standard}
\addcontentsline{toc}{section}{Appendix F. Reproducibility and Implementation Standard}

\noindent
This appendix specifies the semantic structure, labeling discipline, deterministic build requirements,
automated verification criteria, and archival audit format for the monograph. All statements below are
methodological (non-mathematical) and \emph{testable}. “CI” denotes Continuous Integration.

\medskip
\noindent\textbf{Date:} 2025-09-09 \qquad \textbf{Contact:} \texttt{askohguharev@yandex.ru}

%%%%%%%%%%%%%%%%%%%%%%%%%%%%%%%%%%%%%%%%%%%%%%%%%%%%%%%%%%%%%%%%%%%%%%%%%%%%%%%%
% F.0 Scope and Normative Paths
%%%%%%%%%%%%%%%%%%%%%%%%%%%%%%%%%%%%%%%%%%%%%%%%%%%%%%%%%%%%%%%%%%%%%%%%%%%%%%%%

\subsection*{F.0. Scope and Normative Paths}

\noindent
The following paths are normative and must exist in the repository. A path listed here is a \emph{contract}:
its presence, purpose, and semantic contents are enforced by CI and by the audit in \S F.7.

\begin{itemize}
  \item \texttt{src/main.tex} — master file; includes frontmatter, sections, appendices, bibliography.
  \item \texttt{src/frontmatter/00-executive-summary.tex} — executive summary (results first).
  \item \texttt{src/frontmatter/00-readers-roadmap.tex} — roadmap (structure and dependencies).
  \item \texttt{src/frontmatter/00-notation-glossary.tex} — notation, normalizations, asymptotic conventions.
  \item \texttt{src/sections/01-introduction.tex} — introduction (history, motivation, main theorems).
  \item \texttt{src/sections/02-preliminaries/02-preliminaries.tex} — preliminaries (global).
  \item \texttt{src/sections/02-preliminaries/blocks/02b-geometry.tex} — geometric conventions.
  \item \texttt{src/sections/02-preliminaries/blocks/02b-cusps.tex} — cusp structure, truncation, Eisenstein data.
  \item \texttt{src/sections/02-preliminaries/blocks/02b-selberg-transform.tex} — Selberg transform, spherical kernel normalizations.
  \item \texttt{src/sections/03-kernel.tex} — truncated kernel construction (global).
  \item \texttt{src/sections/04-projector.tex} — spectral projector $P_{\lambda,\eta}$ (approximate idempotence and orthogonality).
  \item \texttt{src/sections/05-microlocal.tex} — semiclassical parametrix, Egorov, stationary phase, error hierarchy.
  \item \texttt{src/sections/06-geometric.tex} — identity/geodesic/parabolic expansions, assembly.
  \item \texttt{src/sections/07-main-results.tex} — localized trace formula and quantitative local Weyl law.
  \item \texttt{src/sections/08-applications.tex} — Weyl windows, variance bounds, QUE-scale consequences.
  \item \texttt{src/conclusion/09-conclusion.tex} — conclusion and perspectives (bridges, error-budget map).
  \item \texttt{src/appendices/A-effective-volume.tex} — effective volume and truncation geometry.
  \item \texttt{src/appendices/B-auxiliary-estimates.tex} — analytic and geometric auxiliary estimates.
  \item \texttt{src/appendices/C-bridges-roadmap.tex} — bridges and roadmap (safe forwards, no hidden keys).
  \item \texttt{src/appendices/D-tauberian.tex} — extended Tauberian estimates (quantitative forms).
  \item \texttt{src/appendices/E-verification-tables.tex} — verification tables (explicit constants, dependencies).
  \item \texttt{src/appendices/F-reproducibility-standard.tex} — this appendix (methodology and checks).
  \item \texttt{src/macros/environments.tex} — theorem/lemma/proposition/corollary environments.
  \item \texttt{src/macros/operators.tex} — operators, function spaces, asymptotic symbols.
  \item \texttt{src/macros/house-style.tex} — house style (equation numbering, references, typography).
  \item \texttt{bib/references.bib} — \emph{single} bibliography file; source of truth.
  \item \texttt{figures/}, \texttt{tables/} — assets with stable labels and captions.
\end{itemize}

\noindent\textbf{Invariant F0.1 (Path reality).} Every path above exists at build time; missing paths cause CI failure.\\
\textbf{Invariant F0.2 (Single source).} There is exactly one \texttt{.bib} file: \texttt{bib/references.bib}.\\
\textbf{Invariant F0.3 (No mixed files).} A file carries one semantic unit (chapter or appendix). No mixing.

%%%%%%%%%%%%%%%%%%%%%%%%%%%%%%%%%%%%%%%%%%%%%%%%%%%%%%%%%%%%%%%%%%%%%%%%%%%%%%%%
% F.1 Labeling Discipline
%%%%%%%%%%%%%%%%%%%%%%%%%%%%%%%%%%%%%%%%%%%%%%%%%%%%%%%%%%%%%%%%%%%%%%%%%%%%%%%%

\subsection*{F.1. Labeling Discipline (Semantic and Stable)}

\noindent
Labels are semantic, globally unique, and stable once introduced. Prefixes:

\begin{center}
\begin{tabular}{ll}
\texttt{sec:} & sections/subsections, e.g. \texttt{sec:05:parametrix}, \texttt{sec:06:geodesic-assembly} \\
\texttt{thm:} & theorems, e.g. \texttt{thm:07:localized-trace}, \texttt{thm:08:local-weyl} \\
\texttt{lem:} & lemmas, e.g. \texttt{lem:05:egorov}, \texttt{lem:06:geodesic-cutoff} \\
\texttt{prop:} & propositions, e.g. \texttt{prop:04:approx-idempotence}, \texttt{prop:06:parabolic-regularization} \\
\texttt{cor:} & corollaries, e.g. \texttt{cor:05:stationary-phase} \\
\texttt{eq:} & equations, e.g. \texttt{eq:trace-localized}, \texttt{eq:weyl-window} \\
\texttt{fig:}, \texttt{tab:} & figures/tables with stable captions \\
\texttt{app:} & appendix sections, e.g. \texttt{app:B:oscillatory}, \texttt{app:D:ikehara} \\
\end{tabular}
\end{center}

\noindent
\textbf{Rules.}
\begin{enumerate}
  \item \emph{Uniqueness:} No duplicate label keys across the project.
  \item \emph{Stability:} Published labels are immutable; renaming requires synchronized updates.
  \item \emph{Granularity:} Cite-only equations carry \texttt{eq:...}; otherwise they remain unnumbered.
  \item \emph{Reference type:} Use \verb|\eqref| for equations; \verb|\ref| for theorems, lemmas, sections, figures, tables.
\end{enumerate}

\noindent\textbf{Invariant F1.1 (Resolvable refs).} Every \verb|\ref|/\verb|\eqref| resolves at build time.\\
\textbf{Invariant F1.2 (No orphans).} Every \verb|\label| is referenced at least once outside its defining environment.\\
\textbf{Invariant F1.3 (Counter discipline).} \verb|\numberwithin{equation}{section}| holds; equation indices reset per section.

%%%%%%%%%%%%%%%%%%%%%%%%%%%%%%%%%%%%%%%%%%%%%%%%%%%%%%%%%%%%%%%%%%%%%%%%%%%%%%%%
% F.2 Deterministic Build
%%%%%%%%%%%%%%%%%%%%%%%%%%%%%%%%%%%%%%%%%%%%%%%%%%%%%%%%%%%%%%%%%%%%%%%%%%%%%%%%

\subsection*{F.2. Deterministic Build (Engine, Passes, Parity)}

\noindent
Builds are deterministic: same engine, same major package versions, same outputs (up to timestamps).

\begin{itemize}
  \item \textbf{Engine:} TeX~Live~2025 (fixed); \texttt{pdflatex} $\to$ \texttt{bibtex} $\to$ multi-pass \texttt{pdflatex} until labels/citations stabilize.
  \item \textbf{Parity:} Cross-platform runs (Linux, macOS) must agree on pass/fail and on checksum of the final PDF artifact.
  \item \textbf{Artifacts:} PDF and logs are archived with SHA-256.
\end{itemize}

\noindent\textbf{Invariant F2.1 (Engine immutability).} The engine and major packages are pinned per release.\\
\textbf{Invariant F2.2 (Cross parity).} Linux and macOS builds are identical in status and checksum.\\
\textbf{Invariant F2.3 (Multi-pass closure).} The build completes with no “undefined reference/citation” warnings.

%%%%%%%%%%%%%%%%%%%%%%%%%%%%%%%%%%%%%%%%%%%%%%%%%%%%%%%%%%%%%%%%%%%%%%%%%%%%%%%%
% F.3 CI Pass/Fail Specification
%%%%%%%%%%%%%%%%%%%%%%%%%%%%%%%%%%%%%%%%%%%%%%%%%%%%%%%%%%%%%%%%%%%%%%%%%%%%%%%%

\subsection*{F.3. CI Pass/Fail Specification (Objective Patterns)}

\noindent
CI enforces pass/fail by log patterns. In particular:

\paragraph{Critical (must be zero).}
\begin{itemize}
  \item “\texttt{LaTeX Warning: Reference ... undefined}”
  \item “\texttt{LaTeX Warning: Citation ... undefined}”
  \item “\texttt{Label ... multiply defined}”
  \item “\texttt{BibTeX failed}” or missing \texttt{.bbl}
\end{itemize}

\paragraph{Warnings (must be zero).}
\begin{itemize}
  \item “\texttt{Underfull \hbox in display math}”
  \item Any “\texttt{Overfull \hbox}” exceeding 5pt
\end{itemize}

\paragraph{Cosmetic allowance.}
\begin{itemize}
  \item Overfull boxes $\le 5$pt are counted; total count $E_o \le 5$ per full build.
\end{itemize}

\noindent\textbf{Invariant F3.1 (Zero tolerance).} Critical and warning budgets are zero: $E_c=0$, $E_w=0$.\\
\textbf{Invariant F3.2 (Tight cosmetics).} Cosmetic budget $E_o\le 5$.

%%%%%%%%%%%%%%%%%%%%%%%%%%%%%%%%%%%%%%%%%%%%%%%%%%%%%%%%%%%%%%%%%%%%%%%%%%%%%%%%
% F.4 Concrete Verification Tasks
%%%%%%%%%%%%%%%%%%%%%%%%%%%%%%%%%%%%%%%%%%%%%%%%%%%%%%%%%%%%%%%%%%%%%%%%%%%%%%%%

\subsection*{F.4. Concrete Verification Tasks (What is Checked)}

\paragraph{F4.a — Labels and Refs.} 
Enumerate all \verb|\label| keys from \texttt{.aux}; enumerate all \verb|\ref|/\verb|\eqref| targets; 
check surjectivity both ways (no undefined refs; no orphans).

\paragraph{F4.b — Equations.}
Verify per-section numbering; any cited equation has an \texttt{eq:...} label; no duplicate equation labels.

\paragraph{F4.c — Bibliography.}
Every \verb|\cite{key}| matches an entry in \texttt{bib/references.bib}; \texttt{.bbl} is present and complete.

\paragraph{F4.d — Cross-platform parity.}
Status and checksum coincide between Linux and macOS.

\paragraph{F4.e — Artifacts.}
Archive PDF and logs; record SHA-256; store counts $(E_c,E_w,E_o)$.

\noindent\textbf{Invariant F4.1 (Objective audit).} CI decides pass/fail solely by these measurable results.

%%%%%%%%%%%%%%%%%%%%%%%%%%%%%%%%%%%%%%%%%%%%%%%%%%%%%%%%%%%%%%%%%%%%%%%%%%%%%%%%
% F.5 Semantic Link Registry
%%%%%%%%%%%%%%%%%%%%%%%%%%%%%%%%%%%%%%%%%%%%%%%%%%%%%%%%%%%%%%%%%%%%%%%%%%%%%%%%

\subsection*{F.5. Semantic Link Registry (Forward/Backward, Named)}

\noindent
The registry lists \emph{expected} semantic targets (file and label names). 
This appendix does not use \verb|\ref| here to avoid hard failures if labels are renamed;
instead it declares the contract that CI validates.

\paragraph{Forward (from Appendix F to main text).}
\begin{itemize}
  \item \texttt{src/sections/05-microlocal.tex}: \texttt{sec:05:parametrix}, \texttt{lem:05:egorov}, \texttt{cor:05:stationary-phase}.
  \item \texttt{src/sections/06-geometric.tex}: \texttt{sec:06:geodesic}, \texttt{lem:06:geodesic-cutoff}, \texttt{prop:06:parabolic-regularization}, \texttt{thm:06:assembly}.
  \item \texttt{src/sections/07-main-results.tex}: \texttt{thm:07:localized-trace}, \texttt{cor:07:error-hierarchy}.
  \item \texttt{src/sections/08-applications.tex}: \texttt{thm:08:local-weyl}, \texttt{prop:08:variance}, \texttt{cor:08:que-scale}.
\end{itemize}

\paragraph{Backward (from main text to Appendix F).}
\begin{itemize}
  \item \texttt{src/conclusion/09-conclusion.tex}: pointer to this appendix as the reproducibility standard.
  \item \texttt{src/appendices/B-auxiliary-estimates.tex}: cross-reference to F.3 budgets and F.4 checks.
  \item \texttt{src/frontmatter/00-readers-roadmap.tex}: reference to this appendix in the methodology line.
\end{itemize}

\noindent\textbf{Invariant F5.1 (Named links).} Links are named by file paths and label keys, not by ordinals like “Chapter N”.

%%%%%%%%%%%%%%%%%%%%%%%%%%%%%%%%%%%%%%%%%%%%%%%%%%%%%%%%%%%%%%%%%%%%%%%%%%%%%%%%
% F.6 Error Budget: Definitions and Measurement
%%%%%%%%%%%%%%%%%%%%%%%%%%%%%%%%%%%%%%%%%%%%%%%%%%%%%%%%%%%%%%%%%%%%%%%%%%%%%%%%

\subsection*{F.6. Error Budget (Definitions and Measurement)}

\noindent
Budget vector $(E_c,E_w,E_o)$ is computed from \texttt{.log} by exact patterns:
\begin{itemize}
  \item \textbf{Critical ($E_c$):} undefined reference/citation; multiply defined label; BibTeX failure.
  \item \textbf{Warnings ($E_w$):} underfull in displays; overfull $>5$pt.
  \item \textbf{Cosmetic ($E_o$):} overfull $\le 5$pt (counted).
\end{itemize}
\noindent\textbf{Invariant F6.1 (Objective patterns).} Only explicit pattern counts determine the status.

%%%%%%%%%%%%%%%%%%%%%%%%%%%%%%%%%%%%%%%%%%%%%%%%%%%%%%%%%%%%%%%%%%%%%%%%%%%%%%%%
% F.7 Audit Record (Archival)
%%%%%%%%%%%%%%%%%%%%%%%%%%%%%%%%%%%%%%%%%%%%%%%%%%%%%%%%%%%%%%%%%%%%%%%%%%%%%%%%

\subsection*{F.7. Audit Record (Archival Metadata and Counters)}

\noindent
Each passing build archives an audit record with:

\begin{center}
\begin{tabular}{ll}
\textbf{Field} & \textbf{Meaning} \\
\hline
Commit & VCS hash (source identity) \\
Engine & TeX engine and version (e.g., pdfTeX/TeX Live 2025) \\
Passes & \# of \LaTeX/\texttt{bibtex} passes to stabilize references \\
Labels & total labels; orphan count (must be $0$) \\
Refs & total refs; undefined count (must be $0$) \\
Citations & total cites; undefined count (must be $0$) \\
Eq. counters & per-section resets verified (yes/no) \\
Overfull (cosmetic) & count $\le 5$ \\
Warnings & non-cosmetic warnings (must be $0$) \\
Checksum & SHA-256 of PDF artifact \\
\end{tabular}
\end{center}

\noindent\textbf{Invariant F7.1 (Completeness).} A build without a complete audit record is not “passing”.

%%%%%%%%%%%%%%%%%%%%%%%%%%%%%%%%%%%%%%%%%%%%%%%%%%%%%%%%%%%%%%%%%%%%%%%%%%%%%%%%
% F.8 Responsibilities and Change Control
%%%%%%%%%%%%%%%%%%%%%%%%%%%%%%%%%%%%%%%%%%%%%%%%%%%%%%%%%%%%%%%%%%%%%%%%%%%%%%%%

\subsection*{F.8. Responsibilities and Change Control}

\paragraph{Labels.} Introducing, renaming, or deleting label keys must be atomic with synchronized reference updates; otherwise CI fails on unresolved refs.

\paragraph{Bibliography.} A new \texttt{.bib} entry must be cited in the same or a subsequent commit; stale entries are removed periodically.

\paragraph{Macros and environments.} All theorem and macro changes live in \texttt{src/macros/}; chapter files contain mathematical content only.

\noindent\textbf{Invariant F8.1 (Atomic refactoring).} Protected branches never contain half-consistent states.

%%%%%%%%%%%%%%%%%%%%%%%%%%%%%%%%%%%%%%%%%%%%%%%%%%%%%%%%%%%%%%%%%%%%%%%%%%%%%%%%
% F.9 Minimal Worked Examples (Concrete, Non-Executable)
%%%%%%%%%%%%%%%%%%%%%%%%%%%%%%%%%%%%%%%%%%%%%%%%%%%%%%%%%%%%%%%%%%%%%%%%%%%%%%%%

\subsection*{F.9. Minimal Worked Examples (Concrete Targets)}

\noindent
The following named items \emph{must} exist in the indicated files. They are exemplary; CI validates existence
(and uniqueness) of these label keys. This list is intentionally concrete to avoid abstraction drift.

\paragraph{File: \texttt{src/sections/05-microlocal.tex}}
\begin{itemize}
  \item \texttt{sec:05:parametrix} — section heading for the semiclassical parametrix (small times).
  \item \texttt{lem:05:egorov} — Egorov lemma (quantitative form with time scale $c\log(1/h)$).
  \item \texttt{cor:05:stationary-phase} — stationary phase bound for oscillatory kernels.
  \item \texttt{eq:05:parametrix-kernel} — display equation for the kernel expansion with phase/amplitude.
\end{itemize}

\paragraph{File: \texttt{src/sections/06-geometric.tex}}
\begin{itemize}
  \item \texttt{sec:06:geodesic} — section heading for geodesic classes and summation.
  \item \texttt{lem:06:geodesic-cutoff} — lemma controlling contributions of long/short geodesics.
  \item \texttt{prop:06:parabolic-regularization} — regularization of parabolic terms via truncation $M(Y)$.
  \item \texttt{thm:06:assembly} — assembly theorem combining identity/geodesic/parabolic contributions.
  \item \texttt{eq:06:geodesic-weight} — weight/amplitude for a primitive geodesic in the localized window.
\end{itemize}

\paragraph{File: \texttt{src/sections/07-main-results.tex}}
\begin{itemize}
  \item \texttt{thm:07:localized-trace} — localized trace formula with explicit error $O(\lambda^{-\delta})$.
  \item \texttt{cor:07:error-hierarchy} — corollary summarizing error tiers and parameter dependencies.
  \item \texttt{eq:07:trace-identity} — identity term with explicit constant (volume contribution).
\end{itemize}

\paragraph{File: \texttt{src/sections/08-applications.tex}}
\begin{itemize}
  \item \texttt{thm:08:local-weyl} — quantitative local Weyl law on window $[\lambda-\eta,\lambda+\eta]$.
  \item \texttt{prop:08:variance} — variance bound for Fourier coefficients (depth aspect).
  \item \texttt{cor:08:que-scale} — corollary on quantum ergodicity scale (no hidden QUE keys).
  \item \texttt{eq:08:weyl-window} — main-term $\asymp \mathrm{vol}(M)\,\lambda\,\eta$ with explicit constant.
\end{itemize}

\paragraph{File: \texttt{src/appendices/B-auxiliary-estimates.tex}}
\begin{itemize}
  \item \texttt{app:B:oscillatory} — section on oscillatory integrals (quantitative stationary phase).
  \item \texttt{lem:B:paley-wiener} — Paley–Wiener decay for cutoff transforms.
  \item \texttt{lem:B:resolvent} — resolvent kernel bound with spectral parameter.
\end{itemize}

\paragraph{File: \texttt{src/appendices/D-tauberian.tex}}
\begin{itemize}
  \item \texttt{app:D:ikehara} — Ikehara-type lemma (effective form and conditions).
  \item \texttt{lem:D:laplace-tauberian} — Laplace Tauberian bound (explicit growth line).
\end{itemize}

\noindent\textbf{Invariant F9.1 (Concrete seeds).} These named items are seeds for CI validation; missing any is a build failure.

%%%%%%%%%%%%%%%%%%%%%%%%%%%%%%%%%%%%%%%%%%%%%%%%%%%%%%%%%%%%%%%%%%%%%%%%%%%%%%%%
% F.10 Global Closure
%%%%%%%%%%%%%%%%%%%%%%%%%%%%%%%%%%%%%%%%%%%%%%%%%%%%%%%%%%%%%%%%%%%%%%%%%%%%%%%%

\subsection*{F.10. Global Closure}

\noindent
The standards above are minimal and testable. They avoid numeric placeholders, enforce semantic labels,
define deterministic builds, and make CI pass/fail a matter of measurable facts. The methodological contract
is therefore auditable: any inconsistency produces a concrete failing pattern and prevents archival status.

\medskip
\noindent\textbf{Audit summary (illustrative template).}
\begin{itemize}
  \item Commit: \texttt{<hash>} \quad Engine: \texttt{pdfTeX (TeX Live 2025)} \quad Passes: 4
  \item Labels: 600 (orphans 0) \quad Refs: 780 (undefined 0) \quad Citations: 214 (undefined 0)
  \item Eq. counters: verified \quad Overfull (cosmetic): 3 \quad Warnings: 0
  \item Checksum: \texttt{<sha256>}
\end{itemize}

\noindent
Upon satisfying the invariants and checks in \S\S F.0–F.9, this monograph attains the reproducibility and
implementation standard expected of a high-level mathematical submission. All claims in the main text that
depend on methodology (not mathematics) are now tethered to explicit, verifiable conditions.
