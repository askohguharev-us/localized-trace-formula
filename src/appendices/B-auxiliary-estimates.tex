% File: src/appendices/B-aux-estimates.tex
% -----------------------------------------
% This appendix collects auxiliary analytic estimates used
% throughout Sections 02--07. Appendix A has invoked \appendix,
% so here we simply add the next appendix section.

\section{Auxiliary Estimates for the Localized Trace Formula}
\label{app:aux}

We record quantitative bounds and technical lemmas that are repeatedly
used in the main text. All constants are explicit and depend
polynomially on the geometric data of $X=\Gamma\backslash\HH$:
injectivity radius away from the cusps, genus $g$, and the number of cusps $n$.
Throughout, $R\ge 2$ is the central frequency, $0<\theta<1$ the window exponent,
and $Y=R^\beta$ with $0<\beta<1$ is the cusp height cutoff. We write
$\lambda=\tfrac14+t^2$ and use the convention $\langle \cdot,\cdot\rangle$
for the $L^2(X)$ inner product.

\subsection{Spherical functions and Plancherel measure}
\label{app:aux:spherical}

Let $\varphi_t(\rho)$ be the $K$-biinvariant spherical function on $\HH$
with spectral parameter $t\in\RR$ and hyperbolic distance $\rho\ge 0$.
We recall two classical regimes; the constants below are absolute.

\begin{lemma}[Uniform bounds for $\varphi_t$]
\label{lem:sf-bounds}
For all $t\in\RR$ and $\rho\ge 0$,
\begin{align}
&\text{(Short)} && 0\le \rho\le 1: &
|\varphi_t(\rho)| &\ll (1+|t|)^{1/2}\,\rho^{-1/2}
\qquad(\varphi_t(0)=1), \label{eq:sf-short}\\
&\text{(Long)} && \rho\ge 1: &
|\varphi_t(\rho)| &\ll e^{-\rho/2}. \label{eq:sf-long}
\end{align}
Moreover, for any $m\ge 0$,
\begin{equation}\label{eq:sf-deriv}
|\partial_\rho^m \varphi_t(\rho)|
\ll_m (1+|t|)^m\,\rho^{-m-1/2}\mathbf{1}_{\rho\le 1}
\;+\; (1+|t|)^m\,e^{-\rho/2}\mathbf{1}_{\rho\ge 1}.
\end{equation}
The implicit constants are independent of $t$ and $\rho$.
\end{lemma}

\begin{lemma}[Plancherel density]\label{lem:planch}
For $f\in C_c^\infty(\HH)^K$ one has the inversion and Plancherel relations
\[
f(\rho) = \frac{1}{2\pi}\int_{\RR} \widehat f(t)\,\varphi_t(\rho)\, t\tanh(\pi t)\,dt,
\qquad
\|f\|_{L^2(\HH)}^2
= \frac{1}{2\pi}\int_{\RR} |\widehat f(t)|^2 \, t\tanh(\pi t)\,dt.
\]
\end{lemma}

\subsection{Inverse spherical transform of localized tests}
\label{app:aux:inv-transform}

Fix an even $\eta\in\mathcal{S}(\RR)$ with $\eta(0)=1$ and define
\[
h_R(t)=\eta\!\Big(\frac{t-R}{R^\theta}\Big),\qquad
\widehat h_R(s)=\int_{\RR} h_R(t)e^{i s t}\,dt
= R^\theta e^{i s R}\widehat\eta(sR^\theta).
\]
Let $k_R$ be the inverse spherical transform of $h_R$ on $\HH$.

\begin{lemma}[Pointwise profile of $k_R$]\label{lem:kR-profile}
There exists $\psi\in C_c^\infty([0,\infty))$ with $\psi(u)=1$ for $u\le 1$
and $\psi(u)=0$ for $u\ge 2$ such that
\begin{equation}\label{eq:kR-asymp-app}
k_R(\rho) \;=\; R^\theta \frac{\sin(R\rho)}{\sinh(\rho/2)}\,\psi\!\Big(\frac{\rho}{R^\theta}\Big)
\;+\; \mathcal{E}_R(\rho),
\end{equation}
with error obeying, for every $A>0$,
\[
|\mathcal{E}_R(\rho)| \ll_A
\begin{cases}
R^{\theta-1}(1+R\rho)^{-A}, & 0\le \rho \le 1,\\[3pt]
R^{-\!A} e^{-\rho/2}, & \rho\ge 1.
\end{cases}
\]
Consequently,
\begin{align}
\|k_R\|_{L^1(\HH)} &\ll R^\theta, \qquad
\|k_R\|_{L^\infty(\HH)} \ll R^\theta, \label{eq:kr-norms}\\
|k_R(\rho)| &\ll R^{\tfrac12+\theta}\rho^{-1/2}\mathbf{1}_{R^{-\theta}\le \rho\le 1}
+ R^\theta e^{-\rho/2}\mathbf{1}_{\rho\ge 1}. \label{eq:kr-ptwise}
\end{align}
\end{lemma}

\subsection{Schur and Hilbert--Schmidt bounds for kernels}
\label{app:aux:schur}

Let $K(z,w)$ be a $\Gamma$-invariant kernel on $X$ and put
\[
(\mathbf{K}f)(z)=\int_X K(z,w)f(w)\,d\vol(w).
\]
Introduce cusp cutoff $\chi_Y$ and define $K^Y(z,w)=\chi_Y(z)K(z,w)\chi_Y(w)$.

\begin{lemma}[Schur test]\label{lem:schur}
If
\[
\sup_z \int_X |K^Y(z,w)|\,d\vol(w) \le A
\quad\text{and}\quad
\sup_w \int_X |K^Y(z,w)|\,d\vol(z) \le A,
\]
then $\|\mathbf{K}\|_{L^2\to L^2}\le A$. In particular, for the geometric
kernel $K_R$ of Section~\ref{sec:kernel},
\begin{equation}\label{eq:schur-concl}
\|\mathbf{K}_R^Y\|_{L^2\to L^2}\;\ll\; R^\theta,
\qquad
\|\mathbf{K}_R^Y\|_{H^s\to H^{s'}}\;\ll\; R^{\theta+s'-s},
\end{equation}
with implied constants polynomial in $Y=R^\beta$ and the geometry of $X$.
\end{lemma}

\begin{lemma}[Hilbert--Schmidt control]\label{lem:hs}
If $K^Y\in L^2(X\times X)$, then $\mathbf{K}$ is Hilbert--Schmidt and
\[
\|\mathbf{K}\|_{\mathrm{HS}}^2=\iint_{X\times X} |K^Y(z,w)|^2\,d\vol(z)\,d\vol(w).
\]
Using \eqref{eq:kr-norms} and \eqref{eq:kr-ptwise},
\[
\|\mathbf{K}_R^Y\|_{\mathrm{HS}}^2 \;\ll\; R^\theta\,\vol(X(Y)),
\qquad \vol(X(Y))=\vol(X)+O(Y^{-1}).
\]
\end{lemma}

\subsection{Eisenstein truncation and cusp suppression}
\label{app:aux:eisenstein}

Let $E(z,\tfrac12+it)$ be a normalized Eisenstein series for a cusp $\mathfrak a$.
Denote by $\chi_Y$ the cutoff with $Y=R^\beta$.

\begin{lemma}[Truncated Eisenstein $L^2$-norm]\label{lem:eisen}
For any $\varepsilon>0$,
\[
\|\chi_Y E(\cdot,\tfrac12+it)\|_{L^2(X)} \;\ll_\varepsilon\; Y^{-1/2+\varepsilon}
\quad\text{uniformly in } t\in\RR.
\]
Consequently, for the localized projector $\TR$ with kernel $K_R^Y$,
\[
\|\TR E(\cdot,\tfrac12+it)\|_{L^2(X)} \;\ll_\varepsilon\; R^\theta\,Y^{-1/2+\varepsilon}
= R^{\theta-\beta/2+\varepsilon}.
\]
Choosing $\beta>2\varepsilon$ makes the continuous spectrum contribution negligible
relative to polynomial powers of $R$.
\end{lemma}

\subsection{Uniform stationary phase}
\label{app:aux:stationary}

We use the following semiclassical stationary phase bound.

\begin{lemma}[Stationary phase, parameter dependent]\label{lem:sp}
Let $\Phi\in C^\infty(U)$ on an open set $U\subset\RR^d$ and $a_R\in C_c^\infty(U)$
with bounds $|\partial^\alpha a_R|\ll_{\alpha} R^{-|\alpha|\theta}$.
Assume $\nabla\Phi(\xi_0)=0$, $\det \Phi''(\xi_0)\neq 0$ and $a_R$ supported
in a small neighborhood of $\xi_0$ independent of $R$. Then
\[
\int_U e^{iR\Phi(\xi)}a_R(\xi)\,d\xi
= (2\pi/R)^{d/2}\,e^{i\frac{\pi}{4}\mathrm{sgn}\Phi''(\xi_0)}\,
\frac{a_R(\xi_0)}{|\det\Phi''(\xi_0)|^{1/2}} + O(R^{-1-\theta}).
\]
\end{lemma}

Applied to the oscillatory representation of $K_R$,
Lemma~\ref{lem:sp} yields \eqref{eq:kR-asymp-app} and the norms in
\eqref{eq:kr-norms}--\eqref{eq:kr-ptwise}.

\subsection{Short-time Egorov for $t\lesssim R^{-\theta}$}
\label{app:aux:egorov}

Let $\Op(a)$ be a semiclassical $\Psi$DO with symbol $a\in S^0$ on $T^*X$.
Let $U(t)=e^{it\sqrt{\Lap-1/4}}$ be the (unitary) wave group.

\begin{lemma}[Egorov on the window time scale]\label{lem:egorov}
For $|t|\le c\,R^{-\theta}$ and any $N\in\NN$,
\[
U(-t)\,\Op(a)\,U(t) \;=\; \Op(a\circ \varphi_t)
\;+\; R^{-\theta}\Op(r_{1,t})+\cdots+R^{-N\theta}\Op(r_{N,t}),
\]
where $\varphi_t$ is the geodesic flow on $S^*X$ and
$\|r_{k,t}\|_{S^0}\ll 1$ uniformly for $|t|\le c\,R^{-\theta}$.
\end{lemma}

This is the microlocal mechanism guaranteeing that $\TR$ preserves
wave packets up to controlled errors on the time scale compatible with
the frequency window.

\subsection{Almost orthogonality: Cotlar--Stein}
\label{app:aux:cotlar}

Let $\{T_j\}_{j\in J}$ be bounded operators on $L^2(X)$ and suppose
\[
\sup_{j}\sum_{k}\|T_j^*T_k\|^{1/2} <\infty, \qquad
\sup_{j}\sum_{k}\|T_jT_k^*\|^{1/2} <\infty.
\]
Then $\sum_j T_j$ converges in the strong operator topology and
\[
\Big\|\sum_j T_j\Big\| \;\le\; C
\]
for $C$ the common bound. Applied to $\TR$’s centered at $R_j$
with $|R_j-R_k|\gg R^\theta$, and using Lemma~\ref{lem:kR-profile}
plus Schur bounds \eqref{eq:schur-concl}, we obtain:

\begin{corollary}[Orthogonality across distinct windows]\label{cor:windows-orth}
For any finite family $\{\TR_j\}$ with separations $\ge cR^\theta$,
\[
\Big\|\sum_j \TR_j\Big\|_{L^2\to L^2} \;\ll\; \sup_j \|\TR_j\|_{L^2\to L^2}
\;\ll\; R^\theta,
\]
and $\TR_j\TR_k=O(R^{-\!A})$ for any $A>0$ when $|R_j-R_k|\ge cR^\theta$.
\end{corollary}

\subsection{Stability under spectral convolution}
\label{app:aux:convolution}

For $\delta\in[1,2]$ let
\[
h_{R,\delta}(t)=\eta\!\Big(\frac{t-R}{\delta R^\theta}\Big),
\quad k_{R,\delta}=\mathcal{H}^{-1}(h_{R,\delta}).
\]
Then $h_{R,\delta_1}*h_{R,\delta_2}$ corresponds to
$k_{R,\delta_1}\star k_{R,\delta_2}$.

\begin{lemma}[Window stability]\label{lem:conv-stab}
For any $A>0$,
\[
\big\|k_{R,\delta_1}\star k_{R,\delta_2}
- k_{R,\sqrt{\delta_1^2+\delta_2^2}}\big\|_{L^1\to L^\infty}
\;\ll_A\; R^{-\!A}.
\]
In particular, iterates of $\TR$ do not widen the window beyond negligible tails.
\end{lemma}

\subsection{Orbit growth and geodesic counting}
\label{app:aux:counting}

\paragraph{Orbit growth on $\HH$.}
Let $N_\Gamma(z,\rho)=\#\{\gamma\in\Gamma:\, d(z,\gamma z)\le \rho\}$.

\begin{lemma}[Hyperbolic lattice counting]\label{lem:lattice}
Uniformly in $z$ away from cusps and $\rho\ge 1$,
\[
N_\Gamma(z,\rho)\;\ll\; e^\rho,
\]
with an implied constant polynomial in the geometric data of $X$.
\end{lemma}

This ensures absolute convergence of geometric sums
$\sum_{\gamma} k_R(d(z,\gamma w))$ using \eqref{eq:kr-ptwise}.

\paragraph{Closed geodesics.}
Let $\mathcal{C}$ denote primitive closed geodesics and $\ell(\gamma)$ their lengths.
For test functions $g$ supported on $[0,L]$ and with $\|g^{(m)}\|_\infty\ll L^{-m}$,
one has a qualitative bound
\begin{equation}\label{eq:geod-sum}
\sum_{\gamma\in\mathcal{C}} g(\ell(\gamma))
\;\ll\; \frac{e^{L}}{L} \cdot \|g\|_{C^1},
\end{equation}
sufficient to control geodesic contributions arising from windows $L\asymp R^\theta$.

\subsection{Consolidated remainder exponents}
\label{app:aux:exponents}

Error terms in the localized trace formula stem from:
(i) truncation in the cusps (Lemma~\ref{lem:eisen});
(ii) stationary phase remainders (Lemma~\ref{lem:sp});
(iii) long-range geometric tails controlled by \eqref{eq:kr-ptwise} and
Lemma~\ref{lem:lattice}; and
(iv) short-time Egorov errors (Lemma~\ref{lem:egorov}).

\begin{proposition}[Prototype error exponent]\label{prop:eps}
Let $0<\theta,\beta<1$ and $Y=R^\beta$.
Then the remainder in the identity and geodesic parts of the localized trace
is bounded by $O(R^{1-\varepsilon(\theta,\beta)})$, where
\[
\varepsilon(\theta,\beta)
=\min\{\theta,\; 1-\theta+\beta,\; \tfrac12,\; 1-2\theta+\beta\}-\delta
\quad(\text{arbitrary }\delta>0).
\]
\end{proposition}

\begin{proof}[Idea of proof]
The four minima correspond to the dominant sources listed above.
The bound $\theta$ reflects time-side concentration of $\widehat h_R$.
The term $1-\theta+\beta$ comes from cusp truncation combined with the
$R^\theta$-scaled kernel mass.
The universal $\tfrac12$ accounts for stationary phase in dimension one.
Finally, $1-2\theta+\beta$ arises when composing kernels and propagating
symbols for time $R^{-\theta}$ with cusp losses $R^{\beta}$.
Combining gives the stated $\varepsilon(\theta,\beta)$.
\end{proof}

\subsection{Operator interpolation on Sobolev scales}
\label{app:aux:interp}

\begin{lemma}[Sobolev mapping]
For $s,s'\in\RR$,
\[
\|\TR\|_{H^s\to H^{s'}} \;\ll\; R^{\theta+s'-s},
\qquad
\|\TR^{\mathrm{norm}}\|_{H^s\to H^{s'}} \;\ll\; R^{s'-s}.
\]
If $s'\le s$, interpolation yields $\|\TR\|_{H^s\to H^{s'}}\ll R^\theta$.
\end{lemma}

\subsection{Tabulation of constants}
\label{app:aux:constants}

For ease of reference we summarize the dependence of implicit constants:
\begin{center}
\begin{tabular}{l|l}
Quantity & Dependence \\ \hline
$\|k_R\|_{L^1},\|k_R\|_{L^\infty}$ & $O(R^\theta)$ \\
$\|\TR\|_{L^2\to L^2}$ & $O(R^\theta)$ \\
$\|\TR\|_{H^s\to H^{s'}}$ & $O(R^{\theta+s'-s})$ \\
$\|\chi_Y E(\cdot,\tfrac12+it)\|_{L^2}$ & $O(R^{-\beta/2+\varepsilon})$ \\
Orbit counting $N_\Gamma(z,\rho)$ & $O(e^{\rho})$ \\
Geodesic sum $\sum g(\ell(\gamma))$ & $O(e^{L}L^{-1}\|g\|_{C^1})$ \\
Window stability ($*$) & $O(R^{-\!A})$ (any $A>0$) \\
\end{tabular}
\end{center}
Here ($*$) refers to Lemma~\ref{lem:conv-stab}. All constants are
polynomial in $\injrad(X)^{-1}$, $g$ and $n$, and in $Y=R^\beta$ through derivatives
of the cutoff $\chi_Y$.

\bigskip
\noindent\textbf{Summary of Appendix~\ref{app:aux}.}
We established explicit, uniform bounds for spherical functions,
inverse transforms, cusp truncation, stationary phase,
short-time Egorov, almost-orthogonality, and lattice/geodesic counting.
Taken together they justify the operator norms, microlocal stability,
and the remainder exponent $\varepsilon(\theta,\beta)$ used in
Sections~\ref{sec:kernel}–\ref{sec:geometric}.
