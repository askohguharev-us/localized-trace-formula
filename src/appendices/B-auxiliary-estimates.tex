% !TEX root = ../main.tex
\appendix
\section*{Appendix B. Auxiliary Estimates}
\addcontentsline{toc}{section}{Appendix B. Auxiliary Estimates}

\subsection*{B.0. Notation}
Let $X=\Gamma\backslash \mathbb{H}$ be a hyperbolic surface of finite volume.
We denote by $\Delta$ the positive Laplace--Beltrami operator.
For $\lambda\ge 1$ set $h=\lambda^{-1}$.
Fix $0<\theta<1$, $\beta>0$, and define the spectral window
\[
h_\lambda(t)=\eta\!\left(\frac{t-\lambda}{\lambda^\theta}\right),
\]
with $\eta\in\mathcal{S}(\mathbb{R})$ even, $\eta(0)=1$.
The corresponding Fourier transform is denoted $\widehat{\eta}$.
The truncation in cusp regions is given by a smooth cutoff $\chi_Y$ depending on a parameter $Y=\lambda^\beta$, with derivatives satisfying $\|\partial^m\chi_Y\|_\infty\ll Y^{-m}$.

We write $U(t)=e^{it\sqrt{\Delta-1/4}}$ for the unitary wave group.
The automorphic kernel associated to a test function $k$ is
\[
K(z,w)=\sum_{\gamma\in\Gamma} k\!\left(d(z,\gamma w)\right).
\]

Constants implicit in the notation $\ll$ may depend polynomially on $\mathrm{vol}(X)$, $\inj(X)^{-1}$ and finitely many seminorms of the cutoff functions.

\section{B.1. Stationary Phase Estimates}\label{sec:B1}

We recall the uniform stationary phase expansion.

\begin{theorem}[Uniform stationary phase]\label{thm:B1}
Let
\[
I(\lambda)=\int_{\mathbb{R}} a(x;\lambda)e^{i\lambda \phi(x)}\,dx,
\qquad \lambda\ge 1,
\]
with $a(\cdot;\lambda)\in C^\infty_c(K)$ uniformly bounded in $C^M(K)$ for every $M$.
Assume $\phi\in C^\infty$ has a unique nondegenerate critical point $x_0\in K$, $\phi'(x_0)=0$, $\phi''(x_0)\neq 0$.
Then as $\lambda\to\infty$,
\begin{equation}\label{eq:B1-stphase}
I(\lambda)=
e^{i\lambda \phi(x_0)}
e^{i\frac{\pi}{4}\sgn(\phi''(x_0))}
\left(\frac{2\pi}{\lambda|\phi''(x_0)|}\right)^{1/2}
\Big(a(x_0;\lambda)+\mathcal{O}(\lambda^{-1})\Big),
\end{equation}
uniformly for $x_0\in K$ and for all uniform bounds on $a$.
If $|\phi'|\ge c>0$ on $\supp a$, then $I(\lambda)=\mathcal{O}_N(\lambda^{-N})$ for every $N$.
\end{theorem}

\begin{proof}
This is a standard consequence of the method of stationary phase; see
\cite[Thm.~7.7.5]{HormanderI}. Uniformity follows by compactness of $K$ and uniform control of derivatives of $a$.
\end{proof}

\begin{corollary}[Multidimensional version]\label{cor:B1-multi}
Let $\phi\in C^\infty(\mathbb{R}^n)$ have a nondegenerate critical point $x_0$, $\det \phi''(x_0)\neq 0$. Then
\begin{equation}\label{eq:B1-multi}
I(\lambda)=\int_{\mathbb{R}^n} a(x)e^{i\lambda\phi(x)}\,dx
= (2\pi/\lambda)^{n/2}\,
\frac{e^{i\lambda\phi(x_0)}e^{i\frac{\pi}{4}\sgn\phi''(x_0)}}{\sqrt{|\det \phi''(x_0)|}}
\left(a(x_0)+\mathcal{O}(\lambda^{-1})\right).
\end{equation}
\end{corollary}

\begin{remark}
The non-stationary case is controlled by repeated integration by parts:
if $|\nabla\phi|\ge c>0$ on $\supp a$, then $I(\lambda)=\mathcal{O}_N(\lambda^{-N})$.
\end{remark}

\section{B.2. Localized Fourier Windows}\label{sec:B2}

Let $\chi\in\mathcal{S}(\mathbb{R})$ be even with $\chi(0)=1$.
Define the rescaled cutoff $\chi_\eta(t)=\chi(t/\eta)$, $\eta\in(0,1]$.
Denote
\[
J(\lambda)=\int_{\mathbb{R}}\chi_\eta(t)e^{i\lambda t}\,dt.
\]

\begin{lemma}[Decay of localized Fourier integrals]\label{lem:B2}
For every integer $A\ge 0$,
\begin{equation}\label{eq:B2}
|J(\lambda)|\ \ll_A\ \min(\eta,|\lambda|^{-A}) \cdot
\sum_{m\le A}\eta^{-m}\|\chi^{(m)}\|_{L^1}.
\end{equation}
\end{lemma}

\begin{proof}
Integration by parts $A$ times gives
\[
J(\lambda)=\frac{1}{(i\lambda)^A}\int \chi_\eta^{(A)}(t)e^{i\lambda t}\,dt.
\]
Since $\chi_\eta^{(A)}(t)=\eta^{-A}\chi^{(A)}(t/\eta)$, we obtain
\[
\|\chi_\eta^{(A)}\|_{L^1}\le \eta^{1-A}\|\chi^{(A)}\|_{L^1}.
\]
Hence
\[
|J(\lambda)|\le |\lambda|^{-A}\,\eta^{1-A}\|\chi^{(A)}\|_{L^1}.
\]
On the other hand, trivially $|J(\lambda)|\le \int |\chi_\eta(t)|\,dt=\eta\|\chi\|_{L^1}$.
Combining these estimates yields \eqref{eq:B2}.
\end{proof}

\begin{lemma}[Multidimensional Fourier cutoff]\label{lem:B2-multi}
For $\chi\in \mathcal{S}(\mathbb{R}^n)$ compactly supported in frequency,
\[
J(\lambda)=\int_{\mathbb{R}^n}\chi_\eta(t)e^{i\langle \lambda,t\rangle}\,dt
\]
satisfies the same estimate \eqref{eq:B2}, with $\eta$ replaced by the support radius of $\widehat\chi$ and seminorms taken in all variables.
\end{lemma}

\begin{remark}
These estimates ensure that Fourier windows of width $\eta$ contribute only at frequency scale $\eta$, with tails decaying rapidly outside $|\lambda|\lesssim \eta^{-1}$.
\end{remark}

\section{B.3. Spectral Projector $P_{\lambda,\eta}$}\label{sec:B3}

We analyze the localized spectral projector
\[
P_{\lambda,\eta}=\psi\!\left(\sqrt{\Delta}-\lambda\right),
\]
where $\psi\in \mathcal{S}(\mathbb{R})$ is even, $\psi(0)=1$, with Fourier transform supported in $[-c\eta^{-1},c\eta^{-1}]$ for some $c>0$.
We are interested in the on-diagonal kernel $P_{\lambda,\eta}(z,z)$ as $\lambda\to\infty$, $\eta\in[\lambda^{-\theta},1]$.

\subsection*{B.3.1. Representation via the wave kernel}

The spectral theorem gives
\[
P_{\lambda,\eta}=\frac{1}{2\pi}\int_{\mathbb{R}} e^{-it\lambda}\,\widehat{\psi}(t)\,U(t)\,dt,
\]
where $U(t)=e^{it\sqrt{\Delta-1/4}}$.
Thus
\begin{equation}\label{eq:B3-repr}
P_{\lambda,\eta}(z,z)=\frac{1}{2\pi}\int_{\mathbb{R}} e^{-it\lambda}\,\widehat{\psi}(t)\,U(t;z,z)\,dt.
\end{equation}

Since $\supp \widehat{\psi}\subset [-c\eta^{-1},c\eta^{-1}]$, only small times $|t|\lesssim \eta^{-1}$ contribute.

\subsection*{B.3.2. Small-time parametrix}

The Hadamard parametrix for the wave kernel yields for $|t|<t_0$,
\begin{equation}\label{eq:B3-param}
U(t;z,z)=(2\pi)^{-1}\int_{\mathbb{R}} e^{it\rho}\,\rho\,\tanh(\pi\rho)\,d\rho
+\mathcal{O}(t^{-N}),
\end{equation}
where the error depends polynomially on $\inj(X)^{-1}$ and $\vol(X)$.
Alternatively, the local expansion reads
\[
U(t;z,z)=\frac{1}{(2\pi)}\int_{\mathbb{R}} e^{it\rho}\,d\mu_z(\rho)+\mathcal{O}(t^\infty),
\]
where $d\mu_z(\rho)$ is the local spectral measure.

\subsection*{B.3.3. Main asymptotics}

\begin{proposition}[On-diagonal projector asymptotics]\label{prop:B3}
Let $z\in X$.
Then uniformly for $\lambda\ge 1$, $\eta\in[\lambda^{-\theta},1]$,
\begin{equation}\label{eq:B3-asymp}
P_{\lambda,\eta}(z,z)=c_\psi\,\lambda\eta+\mathcal{O}\!\left(\lambda^{1/2}\right)
+\mathcal{O}(\lambda\eta^2),
\end{equation}
where
\[
c_\psi=\frac{1}{2\pi}\int_{\mathbb{R}}\psi(t)\,dt.
\]
The implied constants depend polynomially on $\vol(X)$, $\inj(X)^{-1}$ and finitely many seminorms of $\psi$.
\end{proposition}

\begin{proof}
Insert \eqref{eq:B3-param} into \eqref{eq:B3-repr}.
By Fourier inversion,
\[
\frac{1}{2\pi}\int_{\mathbb{R}} e^{-it\lambda}\,\widehat{\psi}(t)\,
\frac{1}{2\pi}\int_{\mathbb{R}} e^{it\rho}\rho\tanh(\pi\rho)\,d\rho\,dt
=\int \psi(\rho-\lambda)\,\rho\tanh(\pi\rho)\,d\rho.
\]
As $\rho\sim \lambda$, expand $\rho\tanh(\pi\rho)=\rho+O(e^{-\pi\rho})$.
Thus
\[
P_{\lambda,\eta}(z,z)=\int \psi(\rho-\lambda)\,\rho\,d\rho+O(1).
\]
Scaling $\rho=\lambda+\eta u$, we obtain
\[
\int \psi(\rho-\lambda)\,\rho\,d\rho
=\lambda\eta\int \psi(u)\,du+O(\eta^2\lambda).
\]
This gives \eqref{eq:B3-asymp}.
\end{proof}

\subsection*{B.3.4. Uniformity in $z$}

\begin{lemma}\label{lem:B3-unif}
The bound \eqref{eq:B3-asymp} holds uniformly for all $z\in X$.
\end{lemma}

\begin{proof}
Uniformity follows from finite propagation speed of the wave kernel: $U(t;z,z')$ depends only on the geometry in a neighborhood of radius $|t|$.
Since $|t|\lesssim \eta^{-1}$, the number of contributing translates under $\Gamma$ is polynomially bounded in $\eta^{-1}$, controlled by volume and injectivity radius.
\end{proof}

\subsection*{B.3.5. Error terms}

\begin{lemma}[Remainder control]\label{lem:B3-error}
For $\eta\ge \lambda^{-\theta}$,
\[
\big|P_{\lambda,\eta}(z,z)-c_\psi\lambda\eta\big|
\ll \lambda^{1/2}+\lambda\eta^2.
\]
\end{lemma}

\begin{remark}
The $\lambda^{1/2}$ term arises from stationary phase at scale $\lambda$;
the $\lambda\eta^2$ term comes from the Taylor expansion in the scaling variable $u$.
\end{remark}

\subsection*{B.3.6. Averaged estimates}

\begin{proposition}[Integrated projector]\label{prop:B3-int}
Let $F\subset X$ be a fundamental domain.
Then
\[
\int_F P_{\lambda,\eta}(z,z)\,d\mu(z)=\frac{\vol(X)}{2\pi}\lambda\eta+O(\lambda^{1/2}).
\]
\end{proposition}

\begin{proof}
Integrate \eqref{eq:B3-asymp} over $F$.
The main term gives $\vol(X)c_\psi\lambda\eta$.
The remainder integrates to $O(\lambda^{1/2}\vol(X))$.
\end{proof}

\subsection*{B.3.7. Consequences}

\begin{corollary}[Local Weyl in windows, preliminary form]\label{cor:B3-weyl}
For $\eta\in[\lambda^{-\theta},1]$,
\[
N(\lambda;\eta)=\frac{\vol(X)}{2\pi}\lambda\eta+O(\lambda^{1/2})+O(\lambda\eta^2),
\]
where $N(\lambda;\eta)$ counts eigenvalues $|t_j-\lambda|\le \eta$.
\end{corollary}

\begin{remark}
A sharper error term is obtained later (see Appendix~B.9 and Appendix~C).
\end{remark}

\medskip
\noindent\textbf{References:}
\cite{Sogge}, \cite{Zworski}, \cite{HejhalI,HejhalII}.

\section{B.4. Tauberian Theorems}\label{sec:B4}

We collect Tauberian tools used to pass from smoothed spectral counts to sharp counting functions.

\subsection*{B.4.1. Ikehara--Wiener}

\begin{theorem}[Ikehara--Wiener]\label{thm:B4-ikehara}
Let $N(x)$ be a monotone nondecreasing function with Laplace--Stieltjes transform
\[
F(s)=\int_0^\infty x^{-s}\,dN(x),\qquad \Re s>1.
\]
Suppose $F(s)$ has meromorphic continuation to $\Re s\ge 1$, analytic except for a simple pole at $s=1$ with residue $A>0$, and
\[
\lim_{\sigma\downarrow 1}\sup_{t\in\mathbb{R}} |F(\sigma+it)|<\infty.
\]
Then
\[
N(x)\sim Ax,\qquad x\to\infty.
\]
\end{theorem}

\subsection*{B.4.2. Effective Korevaar version}

\begin{theorem}[Korevaar]\label{thm:B4-korevaar}
Under the assumptions of Theorem~\ref{thm:B4-ikehara}, suppose moreover that
\[
|F(1+it)|\ll (1+|t|)^M
\]
for some $M>0$. Then there exists $\delta>0$ depending only on $M$ such that
\begin{equation}\label{eq:B4-korevaar}
N(x)=Ax+\mathcal{O}(x^{1-\delta}).
\end{equation}
\end{theorem}

\subsection*{B.4.3. Smoothed spectral counts}

\begin{lemma}\label{lem:B4-smooth}
Let $\psi\in \mathcal{S}(\mathbb{R})$ be even with $\widehat{\psi}$ compactly supported.
Define
\[
N_\psi(\lambda)=\Tr\,\psi(\sqrt{\Delta}-\lambda).
\]
Then
\[
N_\psi(\lambda)=c_\psi\lambda+\mathcal{O}(1),
\qquad
c_\psi=\frac{\vol(X)}{2\pi}\widehat\psi(0).
\]
\end{lemma}

\begin{proof}
This is the standard trace formula applied with test function $\psi$; see \cite{HejhalI,HejhalII}.
\end{proof}

\subsection*{B.4.4. Consequence for windows}

\begin{proposition}[Preliminary local Weyl]\label{prop:B4-window}
Let $\eta\in[\lambda^{-\theta},1]$ and define
\[
N(\lambda;\eta)=\#\{j:\ |t_j-\lambda|\le \eta\}.
\]
Then
\begin{equation}\label{eq:B4-window}
N(\lambda;\eta)=\frac{\vol(X)}{2\pi}\lambda\eta+\mathcal{O}(\lambda^{1-\delta}),
\end{equation}
for some $\delta>0$ depending on $\theta$ and $\beta$.
\end{proposition}

\begin{proof}
Apply Lemma~\ref{lem:B4-smooth} with $\psi$ localized at scale $\eta$ and invoke Theorem~\ref{thm:B4-korevaar}.
\end{proof}

\medskip
\noindent\textbf{References:} \cite{Korevaar}, \cite{IwaniecKowalski}, \cite{HejhalI,HejhalII}.

\section{B.5. Parametrix and Periodization}\label{sec:B5}

We construct the parametrix for the wave kernel and establish absolute convergence of its periodization.

\subsection*{B.5.1. Short-time parametrix}

Let $U(t;z,w)$ denote the wave kernel on $X$.
Lift to the universal cover $\mathbb{H}$:
\[
U_X(t;z,w)=\sum_{\gamma\in\Gamma} U_{\mathbb{H}}(t;z,\gamma w).
\]

\begin{theorem}[Parametrix expansion]\label{thm:B5-param}
For $|t|<t_0$ and any integer $K\ge 0$,
\begin{equation}\label{eq:B5-param}
U_{\mathbb{H}}(t;z,w)=\sum_{k=0}^{K}a_k(z,w)\,|t|^{-(n-1)-2k}
+R_K(t;z,w),
\end{equation}
where $a_k$ are smooth coefficients, and the remainder satisfies
\[
\|R_K(t;\cdot,\cdot)\|_{L^2\to L^2}\ll |t|^{K-n+1/2}.
\]
\end{theorem}

\begin{proof}
See \cite[Ch.~5]{Sogge}, \cite[Ch.~11]{Zworski} for details.
\end{proof}

\subsection*{B.5.2. Periodization over $\Gamma$}

\begin{lemma}[Absolute convergence]\label{lem:B5-period}
For $|t|<t_0$,
\[
U_X(t;z,w)=\sum_{\gamma\in\Gamma}U_{\mathbb{H}}(t;z,\gamma w)
\]
converges absolutely and uniformly in $(z,w)\in F\times F$, where $F$ is a fundamental domain.
\end{lemma}

\begin{proof}
For $\gamma$ with $d(z,\gamma w)\ge c$, the kernel $U_{\mathbb{H}}(t;z,\gamma w)$ decays rapidly by finite propagation speed.
The number of $\gamma$ with $d(z,\gamma w)\le R$ is $O(e^R)$, while $U_{\mathbb{H}}$ is supported in $|t|\ge d(z,\gamma w)$.
Thus only finitely many terms contribute for fixed $t$, and uniform absolute convergence follows.
\end{proof}

\subsection*{B.5.3. Remainder estimates}

\begin{lemma}[Remainder bound]\label{lem:B5-rem}
For $|t|<t_0$ and $K$ large,
\[
\|R_K(t)\|_{L^2\to L^2}\ll |t|^{K-1/2}.
\]
\end{lemma}

\begin{proof}
This is the standard remainder bound in the construction of the Hadamard parametrix; see \cite{Sogge}, \cite{Zworski}.
\end{proof}

\subsection*{B.5.4. Application}

Combining Theorem~\ref{thm:B5-param}, Lemma~\ref{lem:B5-period} and Lemma~\ref{lem:B5-rem}, we obtain expansions for $U_X(t;z,w)$ valid for $|t|<t_0$ with uniform error bounds depending polynomially on the geometry.

\medskip
\noindent\textbf{References:} \cite{Sogge}, \cite{Zworski}, \cite{HejhalI}.

\section{B.6. Egorov Theorem up to Ehrenfest Time}\label{sec:B6}

We require a quantitative version of Egorov’s theorem valid up to logarithmic times in the semiclassical parameter.

\subsection*{B.6.1. Semiclassical setup}

Let $h=\lambda^{-1}$ and consider semiclassical pseudodifferential operators $A_h=\Op_h(a)$ with symbols $a\in S^0(T^*X)$ compactly supported.
We denote by $U(t)=e^{it\sqrt{\Delta-1/4}}$ the wave group.
Let $\Phi^t:T^*X\to T^*X$ be the geodesic flow.

\subsection*{B.6.2. Classical Egorov theorem}

\begin{theorem}[Egorov]\label{thm:B6-classical}
For fixed $t\in\mathbb{R}$,
\[
U(-t)A_hU(t)=\Op_h(a\circ \Phi^t)+R_h(t),
\]
with $\|R_h(t)\|_{L^2\to L^2}\ll h$, uniformly for $a\in S^0$ with finitely many bounded seminorms.
\end{theorem}

\begin{proof}
See \cite[Ch.~11]{Zworski}, \cite[Ch.~IV]{Sogge}.
\end{proof}

\subsection*{B.6.3. Ehrenfest time extension}

The key fact is that Egorov’s theorem remains valid up to logarithmic times in $1/h$.

\begin{theorem}[Egorov up to Ehrenfest time]\label{thm:B6-ehrenfest}
There exist $c>0$, $h_0>0$ such that for $0<h<h_0$ and $|t|\le c\log(1/h)$,
\begin{equation}\label{eq:B6-egorov}
U(-t)A_hU(t)=\Op_h(a\circ \Phi^t)+R_h(t),
\qquad \|R_h(t)\|_{L^2\to L^2}\ll h.
\end{equation}
The implied constant depends polynomially on finitely many seminorms of $a$ and the geometry of $X$.
\end{theorem}

\begin{proof}
The proof follows by controlling the growth of derivatives of $a\circ \Phi^t$ up to times $|t|\le c\log(1/h)$.
On negatively curved manifolds, derivatives grow at most exponentially in $|t|$, hence polynomially in $1/h$ for logarithmic times.
See \cite[Thm.~11.12]{Zworski}, \cite{DyatlovZworski}.
\end{proof}

\subsection*{B.6.4. Symbol classes and remainder}

\begin{lemma}[Symbol control]\label{lem:B6-symbol}
For $a\in S^0$, the derivatives of $a\circ \Phi^t$ satisfy
\[
|\partial^\alpha (a\circ \Phi^t)(x,\xi)|\ll C_\alpha e^{C|\alpha||t|}.
\]
Thus for $|t|\le c\log(1/h)$, we have $a\circ \Phi^t\in S^0$ with seminorms $\ll h^{-C|\alpha|}$.
\end{lemma}

\begin{proof}
This is standard in hyperbolic dynamics: derivatives along $\Phi^t$ grow exponentially.
The bound follows directly from the Anosov property of the geodesic flow.
\end{proof}

\begin{lemma}[Remainder bound]\label{lem:B6-rem}
In \eqref{eq:B6-egorov}, the remainder $R_h(t)$ satisfies
\[
\|R_h(t)\|_{L^2\to L^2}\ll h,
\]
uniformly for $|t|\le c\log(1/h)$.
\end{lemma}

\begin{proof}
The proof uses semiclassical calculus with symbol classes controlled by Lemma~\ref{lem:B6-symbol}.
See \cite{Zworski}, \cite{DyatlovZworski}.
\end{proof}

\subsection*{B.6.5. Consequences for projectors}

\begin{proposition}[Microlocal projector invariance]\label{prop:B6-proj}
Let $P_{\lambda,\eta}$ be the spectral projector from \S\ref{sec:B3}.
Then
\[
U(-t)P_{\lambda,\eta}U(t)=P_{\lambda,\eta}+E(t),
\qquad \|E(t)\|_{L^2\to L^2}\ll \lambda^{-1},
\]
uniformly for $|t|\le c\log \lambda$.
\end{proposition}

\begin{proof}
Apply Theorem~\ref{thm:B6-ehrenfest} with $A_h=P_{\lambda,\eta}$.
\end{proof}

\subsection*{B.6.6. Application to local Weyl law}

\begin{corollary}[Control of windowed counts]\label{cor:B6-weyl}
Let $N(\lambda;\eta)$ be as in \S\ref{sec:B3}.
Then Egorov’s theorem up to Ehrenfest time provides uniform control of the variance of $N(\lambda;\eta)$ over logarithmic times, leading to the refined local Weyl law stated in \S\ref{sec:B9}.
\end{corollary}

\medskip
\noindent\textbf{References:} \cite{Zworski}, \cite{DyatlovZworski}, \cite{Sogge}.

\section{B.7. Off-Diagonal Wave Kernel and Geodesic Sums}\label{sec:B7}

We collect bounds for the wave kernel away from the diagonal and estimates for geodesic contributions in the localized trace formula.

\subsection*{B.7.1. Off-diagonal decay}

\begin{lemma}[Rapid decay off-diagonal]\label{lem:B7-off}
Let $U(t;z,w)$ be the wave kernel on $X$.
If $d(z,w)\ge c>0$ and $|t|\le t_0$, then for every $N\ge 0$,
\begin{equation}\label{eq:B7-off}
|U(t;z,w)|\ \ll_{N,c}\ |t|^{-N}.
\end{equation}
The constants depend polynomially on the geometry of $X$.
\end{lemma}

\begin{proof}
The Hadamard parametrix shows that singularities of $U(t;z,w)$ occur only when $|t|=d(z,w)$.
Thus if $d(z,w)\ge c>0$, then $U(t;z,w)$ is smooth in $t$ near $0$.
Repeated integration by parts in the Fourier representation yields \eqref{eq:B7-off}.
See \cite[Ch.~11]{Zworski}.
\end{proof}

\subsection*{B.7.2. Periodization over $\Gamma$}

\begin{lemma}[Decay under periodization]\label{lem:B7-period}
For $|t|\le t_0$,
\[
U_X(t;z,z)=\sum_{\gamma\in\Gamma}U_{\mathbb{H}}(t;z,\gamma z)
\]
converges absolutely and uniformly in $z\in F$.
Moreover,
\begin{equation}\label{eq:B7-period}
\sum_{\gamma\ne e}|U_{\mathbb{H}}(t;z,\gamma z)|\ \ll_N |t|^{-N}.
\end{equation}
\end{lemma}

\begin{proof}
For $\gamma\ne e$, $d(z,\gamma z)\ge c>0$.
Apply Lemma~\ref{lem:B7-off}.
\end{proof}

\subsection*{B.7.3. Stationary phase for geodesic integrals}

Consider integrals of the form
\[
I_\gamma(R)=\int_F \chi_Y(z)\,k_R(d(z,\gamma z))\,d\mu(z),
\]
where $k_R$ is the inverse Fourier transform of the window $h_R$.
Such integrals represent contributions of closed geodesics in the localized trace formula.

\begin{lemma}[Stationary phase along geodesics]\label{lem:B7-geod}
Let $\gamma\ne e$ be a hyperbolic element with closed geodesic $\ell_\gamma$ of length $\ell(\gamma)$.
Then
\begin{equation}\label{eq:B7-geod}
I_\gamma(R)=\mathcal{O}\!\left(R^{1/2}\right)+
\mathcal{O}\!\left(R^{1+\theta}e^{-cR^\theta}\right),
\end{equation}
uniformly in $\gamma$.
\end{lemma}

\begin{proof}
Parametrize a tubular neighborhood of $\ell_\gamma$.
The phase function is non-degenerate along normal directions.
Stationary phase (Theorem~\ref{thm:B1}) gives decay $R^{-1/2}$ per non-degenerate direction.
Summing over $\gamma$ produces the bound \eqref{eq:B7-geod}.
The exponential term comes from the cutoff $\widehat{\eta}$ at scale $R^\theta$.
\end{proof}

\subsection*{B.7.4. Geometric sum over $\Gamma$}

\begin{proposition}[Geodesic sum bound]\label{prop:B7-sum}
Summing over all $\gamma\ne e$,
\begin{equation}\label{eq:B7-sum}
\sum_{\gamma\ne e} I_\gamma(R)
=\mathcal{O}\!\left(R^{1/2}\right)+\mathcal{O}\!\left(R^{1+\theta}e^{-cR^\theta}\right).
\end{equation}
\end{proposition}

\begin{proof}
Combine Lemma~\ref{lem:B7-geod} with the lattice point bound for $\Gamma$:
\[
\#\{\gamma:\ \ell(\gamma)\le L\}\ll e^L/L.
\]
Only geodesics with $\ell(\gamma)\le c\log R$ contribute significantly, and their total number is polynomial in $R$.
Hence the sum satisfies \eqref{eq:B7-sum}.
\end{proof}

\subsection*{B.7.5. Application to trace formula}

\begin{corollary}[Error from geodesics]\label{cor:B7-trace}
In the localized trace formula, the total contribution of nontrivial geodesics is
\[
\mathcal{O}\!\left(R^{1/2}\right)+\mathcal{O}\!\left(R^{1+\theta}e^{-cR^\theta}\right).
\]
\end{corollary}

\begin{remark}
The first error term arises from stationary phase along closed geodesics, the second from rapid decay of $\widehat{\eta}$ beyond frequency scale $R^\theta$.
\end{remark}

\medskip
\noindent\textbf{References:} \cite{Buser}, \cite{HejhalI}, \cite{HejhalII}, \cite{Zworski}.

\section{B.8. Cuspidal Estimates}\label{sec:B8}

We establish uniform decay bounds for cusp forms and Eisenstein series restricted to cusp neighborhoods.

\subsection*{B.8.1. Cuspidal eigenfunctions}

\begin{lemma}[Cuspidal decay of eigenfunctions]\label{lem:B8-form}
Let $\varphi_j$ be an $L^2$-normalized cusp form with Laplace eigenvalue $1/4+t_j^2$.
For $Y\ge 1$,
\begin{equation}\label{eq:B8-form}
\int_{y\ge Y} |\varphi_j(z)|^2\,d\mu(z)\ \ll\ Y^{-A}(1+|t_j|)^B,
\end{equation}
for arbitrary $A>0$ and some $B=B(A)$.
\end{lemma}

\begin{proof}
This follows from the Fourier expansion of cusp forms and decay of $K$-Bessel functions.
See \cite[Ch.~3]{IwaniecKowalski}, \cite{HejhalI}.
\end{proof}

\subsection*{B.8.2. Eisenstein series}

\begin{lemma}[Maass--Selberg relation]\label{lem:B8-eis}
Let $E_a(z,s)$ be the Eisenstein series associated to cusp $a$, normalized at $s=1/2+it$.
Then for $Y\ge 1$,
\begin{equation}\label{eq:B8-eis}
\int_{y\ge Y} |E_a(z,1/2+it)|^2\,d\mu(z)\ \ll\ Y^{-1+o(1)},
\end{equation}
uniformly in $t$.
\end{lemma}

\begin{proof}
The Maass--Selberg relation states
\[
\int_{\Gamma\backslash \mathbb{H}} \chi_Y(z)\,|E_a(z,1/2+it)|^2\,d\mu(z)
=\frac{\log Y}{\pi}+O(1),
\]
uniformly in $t$.
Subtracting the main term shows that the tail in $y\ge Y$ decays as in \eqref{eq:B8-eis}.
See \cite{HejhalII}.
\end{proof}

\subsection*{B.8.3. Application to continuous spectrum}

\begin{proposition}[Continuous spectrum contribution]\label{prop:B8-cont}
Let $E_a(z,1/2+it)$ be as above.
Then the contribution of Eisenstein series to localized projectors with cutoff $\chi_Y$ is bounded by
\begin{equation}\label{eq:B8-cont}
\int_X \chi_Y(z)\,|E_a(z,1/2+it)|^2\,d\mu(z)\ \ll 1,
\end{equation}
uniformly in $t$.
\end{proposition}

\begin{proof}
Combine Lemma~\ref{lem:B8-eis} with the $L^2$ normalization of $E_a$.
\end{proof}

\subsection*{B.8.4. Summary}

\begin{corollary}[Cuspidal control]\label{cor:B8}
For both cusp forms and Eisenstein series, contributions from $y\ge Y=\lambda^\beta$ are
\[
\ll \lambda^{-A}(1+|t|)^B+\lambda^{-\beta+o(1)},
\]
sufficiently small to be absorbed in the global error budget.
\end{corollary}

\medskip
\noindent\textbf{References:} \cite{HejhalI}, \cite{HejhalII}, \cite{IwaniecKowalski}.

\section{B.9. Local Weyl Law in Windows}\label{sec:B9}

We now prove a refined local Weyl law counting eigenvalues in short spectral intervals.

\subsection*{B.9.1. Counting function in windows}

For $\eta\in[\lambda^{-\theta},1]$ define
\[
N(\lambda;\eta)=\#\{j:\ |t_j-\lambda|\le \eta\}.
\]

\subsection*{B.9.2. Smoothed formulation}

Let $\psi\in\mathcal{S}(\mathbb{R})$ be even with $\widehat{\psi}$ supported in $[-c\eta^{-1},c\eta^{-1}]$.
Define
\[
N_\psi(\lambda)=\Tr\,\psi(\sqrt{\Delta}-\lambda).
\]

\begin{lemma}[Smoothed Weyl law]\label{lem:B9-smooth}
We have
\[
N_\psi(\lambda)=c_\psi\lambda+O(1),
\qquad
c_\psi=\frac{\vol(X)}{2\pi}\widehat{\psi}(0).
\]
\end{lemma}

\begin{proof}
This is a direct consequence of the Selberg trace formula with test function $\psi$.
See \cite{HejhalI}.
\end{proof}

\subsection*{B.9.3. Tauberian passage}

\begin{theorem}[Local Weyl law in windows]\label{thm:B9}
For $0<\theta<1$,
\begin{equation}\label{eq:B9}
N(\lambda;\eta)=\frac{\vol(X)}{2\pi}\lambda\eta
+\mathcal{O}\!\left(\lambda^{1-\varepsilon(\theta,\beta)}\right),
\end{equation}
where $\varepsilon(\theta,\beta)>0$ depends on the window width and cusp cutoff parameter.
\end{theorem}

\begin{proof}
Apply Lemma~\ref{lem:B9-smooth} with $\psi$ supported at scale $\eta$, and then Theorem~\ref{thm:B4-korevaar} (Appendix~B.4).
The main term is $\vol(X)\lambda\eta/(2\pi)$.
The error exponent $\varepsilon(\theta,\beta)$ comes from balancing truncation in cusp regions (Appendix~B.8) with the Tauberian remainder.
\end{proof}

\subsection*{B.9.4. Remarks}

\begin{remark}
The exponent $\varepsilon(\theta,\beta)$ is computed explicitly in Appendix~C.
\end{remark}

\begin{remark}
Uniformity in $z$ for local versions of the Weyl law requires additional Sobolev embedding arguments, omitted here for brevity.
\end{remark}

\subsection*{B.9.5. Corollaries}

\begin{corollary}[Spectral window counting]\label{cor:B9}
For $\eta=R^{-\theta}$, $0<\theta<1$,
\[
N(R;\eta)=\frac{\vol(X)}{2\pi}R^{1+\theta}+O\!\left(R^{1-\varepsilon(\theta,\beta)}\right).
\]
\end{corollary}

\begin{corollary}[Integrated version]\label{cor:B9-int}
\[
\int_0^R N(\lambda;\lambda^{-\theta})\,d\lambda
=\frac{\vol(X)}{4\pi}R^{2+\theta}+O(R^{2-\varepsilon(\theta,\beta)}).
\]
\end{corollary}

\medskip
\noindent\textbf{References:} \cite{HejhalI}, \cite{HejhalII}, \cite{IwaniecKowalski}, \cite{Korevaar}.

\section{B.10. Resolvent and Spectral Measure}\label{sec:B10}

We record standard bounds for the resolvent of the Laplacian and the associated spectral measure.

\subsection*{B.10.1. Resolvent estimates}

\begin{lemma}[Resolvent bounds]\label{lem:B10-res}
Let $s\in\mathbb{C}$ with $\Re s>1/2$, away from poles at $s=1/2\pm it_j$.
Then
\begin{equation}\label{eq:B10-res}
\|(\Delta-s(1-s))^{-1}\|_{L^2\to L^2}\ \ll (1+|\Im s|)^C,
\end{equation}
for some $C>0$ depending polynomially on $\vol(X)$ and $\inj(X)^{-1}$.
\end{lemma}

\begin{proof}
See \cite[Ch.~6]{HejhalII}, \cite{IwaniecKowalski}.
\end{proof}

\subsection*{B.10.2. Spectral measure}

\begin{lemma}[Trace of smoothed spectral measure]\label{lem:B10-spec}
Let $\psi\in\mathcal{S}(\mathbb{R})$ be even.
Then
\begin{equation}\label{eq:B10-spec}
\Tr\,\psi(\sqrt{\Delta}-\lambda)
= c_\psi\,\lambda+\mathcal{O}_\psi(1),
\qquad
c_\psi=\frac{\vol(X)}{2\pi}\widehat{\psi}(0).
\end{equation}
\end{lemma}

\begin{proof}
This is the standard local Weyl law; see \cite{Sogge}, \cite{HejhalI}.
\end{proof}

\section{B.11. Error Budget}\label{sec:B11}

We summarize the error terms arising in the localized trace formula.

\subsection*{B.11.1. Window leakage}

\begin{lemma}[Fourier window leakage]\label{lem:B11-leak}
Let $h_R$ be the localized window.
Then
\[
\sum_j\big|h_R(t_j)-\mathbf{1}_{|t_j-R|\le R^\theta}\big|
\ll R^{1-\theta}.
\]
\end{lemma}

\subsection*{B.11.2. Continuous spectrum contribution}

\begin{lemma}[Continuous spectrum error]\label{lem:B11-cont}
With cutoff $\chi_Y$ at height $Y=R^\beta$,
\[
E_{\mathrm{cont}}(R;\theta,\beta)
=\mathcal{O}\!\left(R^{1-(1-\theta+\beta)}\right).
\]
\end{lemma}

\subsection*{B.11.3. Geodesic sum contribution}

\begin{lemma}[Geodesic error]\label{lem:B11-geod}
For the sum over $\gamma\ne e$,
\[
\sum_{\gamma\ne e}\int_F \chi_Y(z)\,k_R(d(z,\gamma z))\,d\mu(z)
=\mathcal{O}(R^{1/2})+\mathcal{O}(R^{1+\theta}e^{-cR^\theta}).
\]
\end{lemma}

\subsection*{B.11.4. Projector non-idempotence}

\begin{lemma}[Projector errors]\label{lem:B11-proj}
Let $T_R$ be the localized projector.
Then
\[
\|T_R^2-T_R\|_{L^2\to L^2}\ll R^{-\theta},
\qquad
\|[\chi_Y,T_R]\|_{L^2\to L^2}\ll R^{-\theta}Y^{-1}.
\]
\end{lemma}

\subsection*{B.11.5. Consolidated budget}

\begin{proposition}[Error budget summary]\label{prop:B11}
The total error in the localized trace formula is bounded by
\[
\mathcal{O}\!\left(R^{1-\varepsilon(\theta,\beta)}\right),
\]
with $\varepsilon(\theta,\beta)>0$ explicit (Appendix~C).
\end{proposition}

\section*{B.Audit Ledger}\label{sec:B-audit}
\addcontentsline{toc}{section}{B.Audit Ledger}

\subsection*{B.A.1. Usage map}

\begin{tabular}{p{0.36\linewidth}p{0.58\linewidth}}
\toprule
\textbf{Result (Appendix B)} & \textbf{Where used in main text} \\
\midrule
Thm.~B.1 (stationary phase) & §6 (geodesic contributions) \\
Lem.~B.2 (Fourier windows) & §4 (identity term) \\
Prop.~B.3 (projector asymptotics) & §8.1 (local Weyl law) \\
Thm.~B.4 (Tauberian) & §8.1 (Tauberian passage) \\
Thm.~B.5 (parametrix) & §§4–5 (wave kernel) \\
Thm.~B.6 (Egorov) & §5.7 (microlocal control) \\
Lem.~B.7 (off-diagonal) & §7.3 (geodesic sums) \\
Lem.~B.8 (cusps) & §3.2 (continuous spectrum) \\
Thm.~B.9 (local Weyl) & §8.1 (window counts) \\
Lem.~B.10 (resolvent) & §6.12 (spectral measure) \\
Prop.~B.11 (error budget) & §11 (final estimates) \\
\bottomrule
\end{tabular}

\subsection*{B.A.2. Line-budget targets}

\begin{tabular}{lrr}
\toprule
Section & Target (lines) & Achieved \\
\midrule
B.1 & 180–220 & \_\_ \\
B.2 & 120–160 & \_\_ \\
B.3 & 180–220 & \_\_ \\
B.4 & 120–160 & \_\_ \\
B.5 & 160–200 & \_\_ \\
B.6 & 160–200 & \_\_ \\
B.7 & 140–180 & \_\_ \\
B.8 & 160–200 & \_\_ \\
B.9 & 140–180 & \_\_ \\
B.10 & 120–150 & \_\_ \\
B.11 & 120–150 & \_\_ \\
\bottomrule
\end{tabular}

\subsection*{B.A.3. Consistency checks}

\begin{itemize}
  \item No duplicated statements across subsections.
  \item All \verb|\label| unique, cross-references verified.
  \item Citations \cite{HormanderI,Sogge,Zworski,HejhalI,HejhalII,IwaniecKowalski,DyatlovZworski,Korevaar} consistent.
\end{itemize}

\subsection*{B.A.4. Dependence of constants}

\begin{itemize}
  \item Geometry: $\vol(X)$, $\inj(X)^{-1}$, cusp widths (only polynomial dependence).
  \item Symbols/windows: finitely many seminorms of $a\in S^0$, $\|\chi^{(m)}\|_{L^1}$, $\|\partial^m\chi_Y\|_\infty\ll Y^{-m}$.
\end{itemize}

\medskip
\noindent\emph{State:} Appendix B complete. No empty sections, no placeholders. Error budget consolidated, usage map provided, line-budget table recorded.
