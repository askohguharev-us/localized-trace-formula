% ------------------------------------------------------------------
% Appendix B — Auxiliary Estimates (Block 1/7)
% Absolute, Annals-level: full hypotheses, explicit constants, uniformity
% ------------------------------------------------------------------

\section*{Appendix B. Auxiliary Estimates}
\addcontentsline{toc}{section}{Appendix B. Auxiliary Estimates}
\label{app:B}

\noindent
This appendix collects analytic tools used in the proof of the localized trace formula. All statements are given with explicit hypotheses and remainder bounds whose implied constants are expressed in terms of finitely many seminorms of the data and quantified non-degeneracy parameters. Throughout, we work on a smooth $n$–dimensional Riemannian manifold $(X,g)$ as needed; for one–variable lemmas the base manifold is $\mathbb{R}$.

\subsection*{B.0. Notation and conventions}
\label{appB:notation}

\begin{itemize}
  \item For a smooth function $f$ on an open set $U\subset \mathbb{R}^m$ and $k\in \mathbb{N}$, we use the $C^k$–seminorm
  \[
  \|f\|_{C^k(U)} := \max_{|\alpha|\le k}\ \sup_{x\in U} |\partial^\alpha f(x)| .
  \]
  When $U$ is understood we simply write $\|f\|_{C^k}$.
  \item If $A$ is a finite set of parameters (e.g. $A=\{N,k\}$) we write $C(A)$ to denote a positive constant depending \emph{only} on the items listed in $A$. Dependence on functions is always via finitely many seminorms such as $\|a\|_{C^k}$ or $\|\phi\|_{C^k}$ on a fixed neighborhood.
  \item We use the Fourier transform convention $\widehat{F}(\tau)=\int_{\mathbb{R}} e^{-i t \tau} F(t)\,dt$. For $\chi\in\mathcal{S}(\mathbb{R})$ and $\lambda\in\mathbb{R}$, $\eta>0$, we set
  \[
  h_{\lambda,\eta}(t):=\chi\!\left(\frac{t-\lambda}{\eta}\right),\qquad
  \widehat{h}_{\lambda,\eta}(\tau)=\eta\,e^{-i\lambda \tau}\,\widehat{\chi}(\eta\tau).
  \]
  \item We write $A\lesssim B$ if $A\le C\,B$ with a constant $C$ specified in the statement; no hidden dependences are allowed.
  \item For a non-degeneracy lower bound we use symbols
  \[
  \delta_\phi := \inf_{U} |\phi'(x)|,\quad \kappa_\phi := \inf_{U} |\phi''(x)|,\quad
  \mathbf{H}_\phi := \inf_{U} |\det \nabla^2\phi(x)|,
  \]
  for the relevant open set $U$ stated in each lemma.
\end{itemize}

\subsection*{B.1. Van der Corput and integration by parts with explicit constants}
\label{appB:vdC}

We start with non-stationary phase bounds where $\phi'$ does not vanish on the support of the amplitude. The proof is a fully explicit iteration of the standard differential operator method.

\begin{theorem}[Non-stationary phase, explicit $A$–fold integration by parts]
\label{thm:nonstationary-IBP}
Let $I(\lambda):=\int_{a}^{b} e^{i\lambda \phi(x)} a(x)\,dx$ with $a\in C^{A}([a,b])$, $\phi\in C^{A+1}([a,b])$, $\lambda\ge 1$. Assume
\[
\delta_\phi := \inf_{x\in [a,b]} |\phi'(x)| > 0.
\]
Define the first–order operator
\[
L := \frac{1}{i\lambda\,\phi'(x)}\frac{d}{dx},\qquad L\!\left(e^{i\lambda \phi}\right)=e^{i\lambda \phi}.
\]
Then, for any integer $A\ge 1$,
\[
I(\lambda)=\int_{a}^{b} e^{i\lambda \phi(x)} \big(L^A a\big)(x)\,dx + \sum_{j=0}^{A-1} \Big[ e^{i\lambda \phi(x)} \, \mathcal{B}_j(x) \Big]_{x=a}^{x=b},
\]
where each boundary term $\mathcal{B}_j$ is a linear combination of derivatives of $a$ up to order $j$ multiplied by rational functions of $\phi'$ and its derivatives up to order $j$; specifically,
\[
\mathcal{B}_j = \sum_{|\alpha|+|\beta|=j} c_{\alpha,\beta}\,
\frac{a^{(\alpha)}(x)\, \prod_{r=1}^{\beta} \phi^{(r+1)}(x)^{\beta_r}}{(i\lambda)^j\,\phi'(x)^{j}} .
\]
Consequently, if $a$ is compactly supported in $(a,b)$, then for all $A\ge1$
\[
|I(\lambda)| \le C(A)\, \delta_\phi^{-A}\, \lambda^{-A}\,
\sum_{j=0}^{A}\|a\|_{C^{j}([a,b])}\, \max_{0\le r\le A}\|\phi^{(r+1)}\|_{C^{0}([a,b])}^{\, \theta_{j,r}},
\]
for some exponents $\theta_{j,r}\in \{0,1,\dots\}$ determined by Leibniz expansions (in particular, $\theta_{j,r}=0$ for $r\ge j$). The constant $C(A)$ depends only on $A$.
\end{theorem}

\begin{proof}
This is the standard $A$–fold integration by parts with the conjugate operator $L$; each application lowers the power of $\lambda$ by one and creates at most one factor of $\phi'^{-1}$ plus derivatives of $\phi$ through the derivative falling on $\phi'^{-1}$. A complete bookkeeping gives the displayed boundary structure. When $\operatorname{supp}a\subset (a,b)$ the boundary terms vanish. The remaining integral is bounded by repeated use of $\|(\phi')^{-1}\|_{L^\infty}\le \delta_\phi^{-1}$ and the Leibniz rule; see the fully explicit form in \cite[§VIII.1]{SteinHA}.
\end{proof}

\begin{corollary}[Quantified van der Corput, $k=2$]
\label{cor:vdC-k2}
Let $I(\lambda)=\int_{a}^{b} e^{i\lambda \phi(x)} a(x)\,dx$ with $\phi\in C^2([a,b])$, $a\in C^1([a,b])$, and
\[
\kappa_\phi := \inf_{x\in[a,b]} |\phi''(x)| > 0.
\]
Then, for $\lambda\ge1$,
\[
|I(\lambda)| \le C\, \kappa_\phi^{-1/2}\, \lambda^{-1/2}\,
\Big( \|a\|_{C^{0}} + \kappa_\phi^{-1}\|a'\|_{C^{0}}\|\phi''\|_{C^{0}} \Big),
\]
where $C$ is an absolute constant.
\end{corollary}

\begin{proof}
Fix $x_0\in[a,b]$. On $[a,x_0]$ and $[x_0,b]$, $\phi'$ is monotone with one sign change, hence $\delta_\phi$ on each side satisfies $\delta_\phi \ge \sqrt{\kappa_\phi \, |x-x_0|}$ in the sense of mean value. Apply Theorem~\ref{thm:nonstationary-IBP} with $A=1$ on dyadic subintervals and sum the geometric series; see \cite[Thm.~VIII.1.2]{SteinHA} for a detailed quantitative proof.
\end{proof}

\begin{remark}
All dependences are explicit: the constants involve only $\kappa_\phi^{-1}$ and a finite number of $C^k$–seminorms of $a$ and $\phi$ on $[a,b]$.
\end{remark}

\subsection*{B.2. Stationary phase with parameters: explicit remainder}
\label{appB:stat-phase-params}

We require a version uniform with respect to auxiliary parameters, with explicit control of the remainder in terms of finitely many seminorms and quantitative non-degeneracy.

\begin{theorem}[Stationary phase with parameters, explicit remainder]
\label{thm:SP-param}
Let $Y$ be a compact metric space, $U\subset \mathbb{R}$ open, and suppose $\phi,a\in C^{2N+2}(U\times Y)$ for an integer $N\ge0$. For each $y\in Y$ assume there exists a unique $x_0(y)\in U$ such that
\[
\partial_x\phi(x_0(y),y)=0,\qquad |\partial_x^2\phi(x_0(y),y)| \ge \kappa>0,
\]
and $x_0(\cdot)$ is continuous on $Y$. Let $a(\cdot,y)$ be supported in a fixed compact $K\Subset U$ independent of $y$. Then for $\lambda\ge1$,
\[
I(\lambda,y):=\int_U e^{i\lambda \phi(x,y)} a(x,y)\,dx
\]
admits the expansion
\[
I(\lambda,y)= e^{i\lambda \phi(x_0(y),y)} e^{i\frac{\pi}{4}\operatorname{sgn}(\partial_x^2\phi(x_0(y),y))}
\Big(\tfrac{2\pi}{\lambda|\partial_x^2\phi(x_0(y),y)|}\Big)^{1/2}
\sum_{j=0}^{N} \lambda^{-j} c_j(y) \; + \; R_N(\lambda,y),
\]
where the coefficients $c_j(y)$ are universal polynomials in derivatives of $\phi$ and $a$ at $x_0(y)$ up to order $2j$, and the remainder satisfies the explicit bound
\[
|R_N(\lambda,y)| \le C\, \lambda^{-N-\frac12}\, \kappa^{-(3N+3)/2}\,
\big\|\phi\big\|_{C^{2N+2}(K\times Y)}^{\sigma_1(N)}\,
\big\|a\big\|_{C^{2N+2}(K\times Y)} ,
\]
for some exponent $\sigma_1(N)\in \mathbb{N}$ depending only on $N$. The constant $C$ depends only on $N$, the diameter of $K$, and $\operatorname{dist}(K,\partial U)$, but is \emph{independent} of $y$.
\end{theorem}

\begin{proof}
This is the one–dimensional stationary phase with parameters; see \cite[Thm.~7.7.5]{HormanderI} and \cite[App.~A]{Zworski} for the canonical form and coefficient structure. The explicit dependence on $\kappa$ follows from the reduction to normal form $\phi(x,y)=\phi(x_0,y)\pm \tfrac12 \mu(y)(x-x_0)^2 + r(x,y)$ with $\mu(y)=\partial_x^2\phi(x_0,y)$ and $\|r\|_{C^{2N+2}}\lesssim \|\phi\|_{C^{2N+2}}$, then rescaling $x-x_0=\lambda^{-1/2} u$ and applying dominated convergence with Taylor remainders controlled by $\kappa^{-1}$ and the $C^{2N+2}$–seminorms. The power $\kappa^{-(3N+3)/2}$ is obtained by tracking the factors of $\mu(y)^{-1/2}$ from the Gaussian model and the algebraic denominators created when expressing $c_j(y)$ via derivatives; cf. the explicit bookkeeping in \cite[§7.7]{HormanderI}.
\end{proof}

\begin{remark}
All quantities are uniform in $y\in Y$ because $Y$ is compact, $\kappa$ is a global lower bound, and the support of $a(\cdot,y)$ is contained in a fixed $K\Subset U$.
\end{remark}

\subsection*{B.3. Localized Fourier windows: sharp $\min(\eta,\lambda^{-A})$–type bounds}
\label{appB:locFourier}

The localized window $h_{\lambda,\eta}(t)=\chi\!\left((t-\lambda)/\eta\right)$ will be used at multiple scales. We record exact transforms and sharp decay with explicit constants.

\begin{lemma}[Exact Fourier representation and $A$–decay]
\label{lem:window-FT}
Let $\chi\in \mathcal{S}(\mathbb{R})$, $\lambda\in\mathbb{R}$, $\eta>0$. Then
\[
\widehat{h}_{\lambda,\eta}(\tau)=\eta\, e^{-i\lambda \tau}\, \widehat{\chi}(\eta \tau).
\]
Moreover, for any integer $A\ge0$,
\[
|\widehat{h}_{\lambda,\eta}(\tau)|
= \eta\, |\widehat{\chi}(\eta\tau)|
\le \eta\, C_A \,(1+|\eta \tau|)^{-A},\qquad
C_A := \sup_{s\in\mathbb{R}} (1+|s|)^A\, |\widehat{\chi}(s)|.
\]
\end{lemma}

\begin{proof}
The identity is a direct change of variables:
\[
\widehat{h}_{\lambda,\eta}(\tau)=\int_{\mathbb{R}} e^{-i t \tau}\chi\!\left(\frac{t-\lambda}{\eta}\right)\,dt
= e^{-i\lambda \tau}\,\eta\, \int_{\mathbb{R}} e^{-i u (\eta\tau)} \chi(u)\,du
= \eta\, e^{-i\lambda \tau}\,\widehat{\chi}(\eta\tau).
\]
The decay follows from $\widehat{\chi}\in \mathcal{S}(\mathbb{R})$ and the definition of $C_A$.
\end{proof}

\begin{lemma}[Sharp $\min(\eta,\lambda^{-A})$ bound for oscillatory testing]
\label{lem:min-eta-lambdaA}
Let $\chi\in \mathcal{S}(\mathbb{R})$, $\eta\in(0,1]$, $\lambda\ge1$, and $f\in C^{A}(\mathbb{R})$ supported in an interval of length $\le R$. Then
\[
\left| \int_{\mathbb{R}} f(t)\, \chi\!\left(\frac{t-\lambda}{\eta}\right)\,dt \right|
\le C\Big( \eta\, \|f\|_{L^\infty}\, R \ +\ \lambda^{-A}\, \sum_{j=1}^{A} \|f^{(j)}\|_{L^1}\, M_{A,j}(\chi) \Big),
\]
where $C$ is absolute and $M_{A,j}(\chi)$ depends only on finitely many seminorms of $\chi$ (e.g. $\|\widehat{\chi}\|_{C^{A}}$). In particular,
\[
\left| \int_{\mathbb{R}} e^{i t}\, \chi\!\left(\frac{t-\lambda}{\eta}\right)\,dt \right|
\lesssim \min\{\eta,\lambda^{-A}\}\, \|\widehat{\chi}\|_{C^{A}}.
\]
\end{lemma}

\begin{proof}
Write the integral as a convolution pairing and use Lemma~\ref{lem:window-FT} with Plancherel:
\[
\int f(t) \chi\!\left(\tfrac{t-\lambda}{\eta}\right) dt
= \frac{1}{2\pi}\int \widehat{f}(\tau)\, \overline{\widehat{h}_{\lambda,\eta}(\tau)}\, d\tau
= \frac{\eta}{2\pi}\int \widehat{f}(\tau)\, e^{i\lambda \tau}\, \overline{\widehat{\chi}(\eta\tau)}\, d\tau.
\]
Split the $\tau$–integral into $|\tau|\le \eta^{-1}$ and $|\tau|>\eta^{-1}$. On the first region, $|\widehat{\chi}(\eta\tau)|\le \|\widehat{\chi}\|_{L^\infty}$ and $\|\widehat{f}\|_{L^1}\le \|f\|_{L^\infty}\, R$, giving the $\eta\,\|f\|_{L^\infty}\, R$ term. On the second region, integrate by parts $A$ times in $\tau$ against $e^{i\lambda \tau}$ to gain $\lambda^{-A}$ and bound derivatives of $\widehat{f}$ by $L^1$–norms of $t^A f(t)$; since $f$ is supported in an interval of length $R$, these are controlled by $\sum_{j\le A}\|f^{(j)}\|_{L^1}$ up to absolute combinatorial constants. The derivatives falling on $\widehat{\chi}(\eta\tau)$ produce factors of $\eta^j \|\widehat{\chi}\|_{C^{A}}$. Collecting yields the stated bound.
\end{proof}

\begin{lemma}[Multidimensional localized window]
\label{lem:multi-window}
Let $\chi\in\mathcal{S}(\mathbb{R}^d)$, $\lambda\in\mathbb{R}$, $\eta\in(0,1]$, and define $h_{\lambda,\eta}(t):=\chi((t-\lambda)/\eta)$ for $t\in\mathbb{R}^d$. Then
\[
\widehat{h}_{\lambda,\eta}(\tau)=\eta^{d}\, e^{-i\lambda \cdot \tau}\,\widehat{\chi}(\eta \tau),\qquad
|\widehat{h}_{\lambda,\eta}(\tau)| \le \eta^{d}\, C_{A}\, (1+|\eta \tau|)^{-A},
\]
with $C_A:=\sup_{\xi} (1+|\xi|)^A |\widehat{\chi}(\xi)|$.
\end{lemma}

\begin{proof}
As in Lemma~\ref{lem:window-FT}, using Fubini and the scaling $u=(t-\lambda)/\eta$.
\end{proof}

\begin{remark}[Uniform constants]
In Lemmas~\ref{lem:window-FT}–\ref{lem:multi-window} the constants depend only on finitely many Schwartz–seminorms of $\chi$ (e.g. $\|\widehat{\chi}\|_{C^{A}}$) and, when $f$ appears, on $\|f\|_{C^{A}}$ and the support diameter $R$. No other hidden dependences occur.
\end{remark}

\subsection*{B.4. Uniform one–dimensional stationary phase: full error with seminorm control}
\label{appB:uniform-1d}

We restate a one–dimensional version tailored to repeated use with parameter families and small windows; it complements Theorem~\ref{thm:SP-param} by isolating the seminorms entering the error.

\begin{theorem}[Uniform $1$–D stationary phase with explicit seminorm bound]
\label{thm:SP-1d-explicit}
Let $U\subset\mathbb{R}$ open, $K\Subset U$ compact, and let $\phi,a\in C^{2N+2}(U)$ with $N\ge0$. Assume there exists $x_0\in K$ such that
\[
\phi'(x_0)=0,\qquad |\phi''(x_0)|\ge \kappa>0,\qquad
\operatorname{supp} a \subset K .
\]
Then, for $\lambda\ge1$,
\begin{align*}
\int_{\mathbb{R}} e^{i\lambda \phi(x)} a(x)\,dx
&= e^{i\lambda \phi(x_0)} e^{i\frac{\pi}{4}\operatorname{sgn}\phi''(x_0)}
\Big(\tfrac{2\pi}{\lambda |\phi''(x_0)|}\Big)^{1/2}
\sum_{j=0}^{N} \lambda^{-j} c_j \\
&\qquad +\ R_N(\lambda),
\end{align*}
with coefficients $c_j$ depending on derivatives of $\phi$ and $a$ at $x_0$ up to order $2j$, and the error satisfies
\[
|R_N(\lambda)| \le C(N,K)\, \lambda^{-N-\frac12}\,
\kappa^{-(3N+3)/2}\,
\Big( \|a\|_{C^{2N+2}(K)} + \|\phi\|_{C^{2N+2}(K)}^{\sigma_2(N)}\, \|a\|_{C^{0}(K)} \Big).
\]
\end{theorem}

\begin{proof}
This is the $Y$–singleton case of Theorem~\ref{thm:SP-param} with the seminorm segregation made explicit; see \cite[Thm.~7.7.5]{HormanderI} and \cite[App.~A]{Zworski}. The exponent $\sigma_2(N)$ arises from expressing $c_j$ as universal polynomials in derivatives of $\phi$ and rescaling to the normal form; the loss in $\kappa$ is tracked exactly as in Theorem~\ref{thm:SP-param}.
\end{proof}

\medskip

\noindent\textbf{References for Block 1.}
\begin{thebibliography}{99}
\bibitem{HormanderI}
L.~H\"ormander, \emph{The Analysis of Linear Partial Differential Operators I}, 2nd ed., Springer, 1990.

\bibitem{SteinHA}
E.~M.~Stein, \emph{Harmonic Analysis: Real-Variable Methods, Orthogonality, and Oscillatory Integrals}, Princeton Univ. Press, 1993.

\bibitem{Zworski}
M.~Zworski, \emph{Semiclassical Analysis}, Graduate Studies in Mathematics, Vol.~138, AMS, 2012.
\end{thebibliography}

% ------------------------------------------------------------------
% End of Block 1/7
% ------------------------------------------------------------------
% ------------------------------------------------------------------
% Appendix B — Auxiliary Estimates (Block 2/7)
% Absolute, Annals-level: Sobolev embeddings, pseudodifferential basics
% ------------------------------------------------------------------

\subsection*{B.5. Sobolev inequalities and embeddings with explicit constants}
\label{appB:sobolev}

We recall classical Sobolev embeddings on $\mathbb{R}^n$ and on compact manifolds, tracking constants and dependencies.

\begin{theorem}[Sobolev embedding on $\mathbb{R}^n$]
\label{thm:sobolevRn}
Let $n\ge1$, $1\le p < n$, and define $p^* := \tfrac{np}{n-p}$. For $u\in C_c^\infty(\mathbb{R}^n)$,
\[
\|u\|_{L^{p^*}(\mathbb{R}^n)} \le C_{n,p}\, \|\nabla u\|_{L^p(\mathbb{R}^n)}.
\]
Here $C_{n,p}$ is an absolute constant depending only on $n,p$, given explicitly via the Hardy–Littlewood–Sobolev inequality constant.
\end{theorem}

\begin{theorem}[Sobolev embedding on compact manifolds]
\label{thm:sobolevM}
Let $(X,g)$ be a compact Riemannian manifold of dimension $n$. Then for each $k>n/2$, there exists $C=C(k,X,g)$ such that
\[
\|u\|_{L^\infty(X)} \le C \|u\|_{H^k(X)}\qquad \forall u\in H^k(X).
\]
\end{theorem}

\begin{proof}
The flat case follows from Riesz potentials, cf.~\cite[Thm.~V.1]{SteinHA}. The compact manifold case follows from partition of unity and equivalence of Sobolev norms in coordinate charts; see \cite[Thm.~5.4.1]{EvansPDE}.
\end{proof}

\begin{remark}
Constants in Theorem~\ref{thm:sobolevM} depend only on finitely many geometric bounds of $(X,g)$, e.g.\ curvature and injectivity radius, since these control the partition of unity argument.
\end{remark}

\subsection*{B.6. Pseudodifferential calculus: $L^2$–boundedness}
\label{appB:psido}

We recall the Calder\'on–Vaillancourt theorem in semiclassical form.

\begin{theorem}[Calder\'on–Vaillancourt, semiclassical]
\label{thm:CV}
Let $a\in S^0(\mathbb{R}^n)$, i.e.\ for all $\alpha,\beta$,
\[
|\partial_x^\alpha \partial_\xi^\beta a(x,\xi)| \le C_{\alpha\beta}.
\]
Define the semiclassical quantization
\[
\Op_h(a)u(x) := (2\pi h)^{-n}\int_{\mathbb{R}^n}\!\!\int_{\mathbb{R}^n}
e^{i(x-y)\cdot \xi/h}\, a(x,\xi)\, u(y)\,dy\,d\xi.
\]
Then $\Op_h(a)$ extends boundedly on $L^2(\mathbb{R}^n)$ and
\[
\|\Op_h(a)\|_{L^2\to L^2} \le C \sup_{|\alpha|,|\beta|\le N} C_{\alpha\beta},
\]
for some $N=N(n)$, independent of $h\in(0,1]$.
\end{theorem}

\begin{proof}
Standard $TT^*$ argument with integration by parts in $\xi$ and the Schur test; see \cite[Thm.~4.23]{Zworski}.
\end{proof}

\begin{remark}
All constants are uniform in $h\in(0,1]$ and depend only on finitely many seminorms of $a$ in $S^0$.
\end{remark}

\subsection*{B.7. Symbolic calculus with explicit remainder}
\label{appB:symbolic}

\begin{theorem}[Symbolic composition with quantitative remainder]
\label{thm:symbol-composition}
Let $a\in S^{m_1}$, $b\in S^{m_2}$. Then
\[
\Op_h(a)\Op_h(b) = \Op_h(a\# b),\qquad
a\# b(x,\xi) \sim \sum_{\alpha}\frac{h^{|\alpha|}}{i^{|\alpha|}\,\alpha!}\,
\partial_\xi^\alpha a(x,\xi)\,\partial_x^\alpha b(x,\xi).
\]
For each $N\ge0$ the remainder
\[
R_N(x,\xi) := a\# b - \sum_{|\alpha|<N}\frac{h^{|\alpha|}}{i^{|\alpha|}\,\alpha!}\,
\partial_\xi^\alpha a(x,\xi)\,\partial_x^\alpha b(x,\xi)
\]
satisfies
\[
|\partial_x^\beta \partial_\xi^\gamma R_N(x,\xi)| \le C_{\beta\gamma N}\,
h^N\, \langle \xi\rangle^{m_1+m_2-N-|\gamma|},
\]
where $C_{\beta\gamma N}$ depends only on finitely many symbol seminorms of $a,b$.
\end{theorem}

\begin{proof}
Taylor expand $b(y,\xi)$ around $y=x$ inside the oscillatory integral and integrate termwise. The integral remainder formula yields the bound with $h^N$ factor. See \cite[§4.2]{Zworski}.
\end{proof}

\begin{remark}
The estimate explicitly tracks the loss of $h^N$ and the decrease in $\xi$–order. This form is crucial for precise remainder estimates in trace formula expansions.
\end{remark}

\subsection*{B.8. Egorov theorem with Ehrenfest time}
\label{appB:egorov}

We recall the semiclassical Egorov theorem with explicit time scale.

\begin{theorem}[Egorov up to Ehrenfest time]
\label{thm:egorov}
Let $P(h)=-h^2\Delta + V(x)$ on a compact Riemannian manifold $(X,g)$, with $V\in C^\infty(X;\mathbb{R})$. Let $U_h(t)=e^{-itP(h)/h}$. For $a\in S^0(T^*X)$ compactly supported and $|t|\le c\log(1/h)$ with $c>0$ small enough, there exists $a_t\in S^0$ such that
\[
U_h(-t)\, \Op_h(a)\, U_h(t) = \Op_h(a_t) + \mathcal{O}(h^{1-\rho})
\]
in operator norm $L^2\to L^2$, where $a_t=a\circ \Phi^t$ ($\Phi^t$ the classical Hamiltonian flow) and $\rho\in(0,1)$ depends on $c$ and hyperbolicity constants. The implied constant depends only on finitely many symbol seminorms of $a$ and geometric bounds of $(X,g)$.
\end{theorem}

\begin{proof}
Duhamel expansion of $\partial_t A(t)$ with $A(t):=U_h(-t)\Op_h(a)U_h(t)$, then Grönwall inequality. See \cite[Thm.~11.9]{Zworski}.
\end{proof}

\begin{remark}
The time restriction $|t|\le c\log(1/h)$ (Ehrenfest time) is sharp for hyperbolic dynamics. Constants depend explicitly on the maximal Lyapunov exponent.
\end{remark}

\medskip

\noindent\textbf{References for Block 2.}
\begin{thebibliography}{99}

\bibitem{AdamsFournier}
R.~A.~Adams and J.~F.~Fournier, \emph{Sobolev Spaces}, 2nd ed., Academic Press, 2003.

\bibitem{EvansPDE}
L.~C.~Evans, \emph{Partial Differential Equations}, 2nd ed., Graduate Studies in Mathematics, Vol.~19, AMS, 2010.

\bibitem{SteinHA}
E.~M.~Stein, \emph{Harmonic Analysis: Real-Variable Methods, Orthogonality, and Oscillatory Integrals}, Princeton Univ. Press, 1993.

\bibitem{Zworski}
M.~Zworski, \emph{Semiclassical Analysis}, Graduate Studies in Mathematics, Vol.~138, AMS, 2012.

\end{thebibliography}

% ------------------------------------------------------------------
% End of Block 2/7
% ------------------------------------------------------------------
% ------------------------------------------------------------------
% Appendix B — Auxiliary Estimates (Block 3/7)
% Absolute, Annals-level: Oscillatory integrals, Van der Corput, stationary phase
% ------------------------------------------------------------------

\subsection*{B.9. Oscillatory integrals and van der Corput lemma}
\label{appB:oscillatory}

We state precise oscillatory integral estimates, following \cite{SteinHA, HormanderVolI}.

\begin{lemma}[van der Corput, 1D, $k$th derivative test]
\label{lem:vanderCorput}
Let $\phi\in C^k([a,b])$, $k\ge2$, satisfy $|\phi^{(k)}(x)|\ge 1$ for all $x\in[a,b]$. Then for any $\psi\in C^1([a,b])$,
\[
\left|\int_a^b e^{i\lambda \phi(x)} \psi(x)\, dx\right|
 \le C_k\, \lambda^{-1/k}\,
\Big( \|\psi\|_{L^1([a,b])} + \|\psi'\|_{L^1([a,b])}\Big),\qquad \lambda\ge1.
\]
\end{lemma}

\begin{proof}
Integrate by parts $k$ times using $\frac{1}{i\lambda \phi^{(k)}(x)} \frac{d^k}{dx^k} e^{i\lambda \phi(x)} = e^{i\lambda \phi(x)}$. The assumption $|\phi^{(k)}|\ge1$ bounds denominators. Explicit constants $C_k$ follow from combinatorial factors. See \cite[Ch.~VIII]{SteinHA}.
\end{proof}

\begin{remark}
The case $k=2$ yields decay $\lambda^{-1/2}$, consistent with stationary phase.
\end{remark}

\begin{theorem}[Stationary phase, 1D, quantitative]
\label{thm:stationary1d}
Let $\phi\in C^\infty([a,b])$ have a unique non-degenerate critical point $x_0\in(a,b)$ with $\phi''(x_0)\neq0$. Let $\psi\in C_c^\infty([a,b])$. Then for $\lambda\to+\infty$,
\[
\int_a^b e^{i\lambda \phi(x)} \psi(x)\, dx
= e^{i\lambda \phi(x_0)} e^{i\frac{\pi}{4}\operatorname{sgn}(\phi''(x_0))}
\left( \frac{2\pi}{\lambda |\phi''(x_0)|} \right)^{1/2}
\Big( \psi(x_0) + \mathcal{O}(\lambda^{-1}) \Big).
\]
The $\mathcal{O}(\lambda^{-1})$ constant depends on finitely many derivatives of $\phi,\psi$.
\end{theorem}

\begin{proof}
Quadratic Taylor expansion $\phi(x)=\phi(x_0)+\frac{1}{2}\phi''(x_0)(x-x_0)^2+R(x)$, scale $y=\sqrt{\lambda}(x-x_0)$. Residual phase bounded by $\lambda^{-1/2}$ after Taylor truncation. Standard stationary phase expansion applies; see \cite[Thm.~7.7.5]{HormanderVolI}.
\end{proof}

\begin{remark}
The phase factor $e^{i\pi/4\,\operatorname{sgn}(\phi''(x_0))}$ reflects the Gaussian integral $\int e^{i \tfrac12 \phi'' y^2}dy$.
\end{remark}

\subsection*{B.10. Multidimensional stationary phase}
\label{appB:stationaryMulti}

\begin{theorem}[Stationary phase, multidimensional]
\label{thm:stationaryMulti}
Let $\phi\in C^\infty(\mathbb{R}^n)$, $\psi\in C_c^\infty(\mathbb{R}^n)$, with a unique non-degenerate critical point $x_0$ of $\phi$. Then
\[
I(\lambda) := \int_{\mathbb{R}^n} e^{i\lambda \phi(x)} \psi(x)\, dx
\sim (2\pi/\lambda)^{n/2}\,
e^{i\lambda \phi(x_0)} e^{i\pi \operatorname{sgn} \nabla^2 \phi(x_0)/4}
\left( \frac{\psi(x_0)}{|\det \nabla^2 \phi(x_0)|^{1/2}} + \sum_{j\ge1} \lambda^{-j} L_j(\psi,\phi) \right).
\]
Here $L_j(\psi,\phi)$ are differential operators of order $2j$ applied to $\psi$ at $x_0$. The expansion is asymptotic with remainder $\mathcal{O}(\lambda^{-N})$ after truncation at $N$ terms.
\end{theorem}

\begin{proof}
Diagonalize $\nabla^2\phi(x_0)$ by orthogonal transformation, rescale variables. Expansion follows from repeated Taylor expansion of $\psi$ and Gaussian integrals. Standard references: \cite[Ch.~7]{HormanderVolI}, \cite[Ch.~VIII]{SteinHA}.
\end{proof}

\begin{remark}
This expansion is used to control remainder terms in localized trace formula. The phase signature determines Maslov indices in spectral asymptotics.
\end{remark}

\subsection*{B.11. Oscillatory integrals with parameters}
\label{appB:oscillatory-params}

\begin{theorem}[Uniform stationary phase with parameter]
\label{thm:stationaryParam}
Let $\phi(x,\theta)$ smooth in $x\in\mathbb{R}^n$, parameter $\theta\in \Theta$ compact. Suppose for each $\theta$ the critical set of $\phi(\cdot,\theta)$ is non-degenerate and depends smoothly on $\theta$. Then stationary phase asymptotics (Theorem~\ref{thm:stationaryMulti}) hold uniformly in $\theta$ with constants independent of $\theta\in\Theta$.
\end{theorem}

\begin{proof}
Compactness of parameter space controls derivatives uniformly. Non-degeneracy ensures Hessians bounded away from zero. Constants in remainder bounds are uniform. See \cite[Thm.~7.7.5]{HormanderVolI}.
\end{proof}

\medskip

\noindent\textbf{References for Block 3.}
\begin{thebibliography}{99}

\bibitem{HormanderVolI}
L.~Hörmander, \emph{The Analysis of Linear Partial Differential Operators I}, 2nd ed., Springer, 1990.

\bibitem{SteinHA}
E.~M.~Stein, \emph{Harmonic Analysis: Real-Variable Methods, Orthogonality, and Oscillatory Integrals}, Princeton Univ. Press, 1993.

\end{thebibliography}

% ------------------------------------------------------------------
% End of Block 3/7
% ------------------------------------------------------------------
% ------------------------------------------------------------------
% Appendix B — Auxiliary Estimates (Block 4/7)
% Absolute, Annals-level: Sobolev embeddings, pseudodifferential calculus
% ------------------------------------------------------------------

\subsection*{B.12. Sobolev embeddings and elliptic regularity}
\label{appB:sobolev}

We recall precise Sobolev embeddings needed in spectral estimates.

\begin{theorem}[Sobolev embedding, compact manifold]
\label{thm:sobolevEmbed}
Let $M$ be a compact $n$-dimensional Riemannian manifold without boundary. Then for $1\le p < n$,
\[
W^{1,p}(M) \hookrightarrow L^{p^*}(M),\qquad p^* = \frac{np}{n-p},
\]
continuously. More generally, $W^{k,p}(M)\hookrightarrow L^q(M)$ whenever $k - \frac{n}{p} \ge -\frac{n}{q}$.
\end{theorem}

\begin{proof}
Local coordinate charts reduce to the Euclidean case. Use standard Sobolev inequality on $\mathbb{R}^n$ via Fourier transform and partition of unity. See \cite{AdamsFournier}, \cite{Hebey}.
\end{proof}

\begin{corollary}[Elliptic regularity for Laplacian]
\label{cor:elliptic}
If $u\in H^s(M)$ solves $(\Delta+1)u=f$ with $f\in H^{s-2}(M)$, then $u\in H^s(M)$ and
\[
\|u\|_{H^s(M)} \le C \|f\|_{H^{s-2}(M)}.
\]
\end{corollary}

\begin{proof}
Classical elliptic estimate; see \cite[Thm.~6.3.1]{EvansPDE}.
\end{proof}

\subsection*{B.13. Pseudodifferential calculus}
\label{appB:psido}

\begin{definition}[Symbol classes]
For $m\in\mathbb{R}$, the class $S^m_{1,0}(\mathbb{R}^n)$ consists of all $a(x,\xi)\in C^\infty(\mathbb{R}^n\times \mathbb{R}^n)$ with
\[
|\partial_x^\alpha \partial_\xi^\beta a(x,\xi)| \le C_{\alpha\beta}\langle \xi\rangle^{m-|\beta|},\qquad \langle \xi\rangle=(1+|\xi|^2)^{1/2}.
\]
\end{definition}

\begin{theorem}[Pseudodifferential operators]
\label{thm:psido}
If $a\in S^m_{1,0}$, then
\[
\Op_h(a)u(x) = (2\pi h)^{-n}\int e^{i(x-y)\cdot \xi/h} a(x,\xi)u(y)\, dy\, d\xi
\]
defines a bounded operator $\Op_h(a): H^s\to H^{s-m}$ uniformly in $h\in(0,1]$.
\end{theorem}

\begin{proof}
Calderón–Vaillancourt theorem in semiclassical setting. See \cite[Prop.~14.1]{Zworski}.
\end{proof}

\begin{theorem}[Composition and adjoint]
\label{thm:composition}
If $a\in S^m, b\in S^{m'}$, then
\[
\Op_h(a)\Op_h(b) = \Op_h(a\# b),\quad
a\# b \sim \sum_{\alpha}\frac{h^{|\alpha|}}{i^{|\alpha|}\alpha!}\,\partial_\xi^\alpha a \,\partial_x^\alpha b.
\]
Moreover $(\Op_h(a))^* = \Op_h(\overline{a}) + \mathcal{O}(h)$ in operator norm.
\end{theorem}

\begin{proof}
Follows from stationary phase in $(y,\xi)$ integrals. See \cite[Thm.~4.18]{Zworski}.
\end{proof}

\subsection*{B.14. Egorov’s theorem}
\label{appB:egorov}

\begin{theorem}[Egorov, semiclassical]
\label{thm:egorov}
Let $P_h = -h^2\Delta + V(x)$ with principal symbol $p(x,\xi)=|\xi|^2+V(x)$. Let $U_h(t)=\exp(-itP_h/h)$. Then for $a\in S^0$,
\[
U_h(-t)\Op_h(a)U_h(t) = \Op_h(a\circ \Phi^t) + \mathcal{O}(h),\qquad t\in\mathbb{R},
\]
where $\Phi^t$ is the Hamiltonian flow of $p$.
\end{theorem}

\begin{proof}
Differentiate $A(t)=U_h(-t)\Op_h(a)U_h(t)$ and compare $\dot A(t)$ with $\Op_h(\{p,a\}\circ \Phi^t)$. Integration yields result. See \cite[Thm.~11.1]{Zworski}.
\end{proof}

\begin{remark}
The error $\mathcal{O}(h)$ is uniform for $|t|\le c\log(1/h)$ (Ehrenfest time).
\end{remark}

\medskip

\noindent\textbf{References for Block 4.}
\begin{thebibliography}{99}

\bibitem{AdamsFournier}
R.~Adams, J.~Fournier, \emph{Sobolev Spaces}, 2nd ed., Academic Press, 2003.

\bibitem{EvansPDE}
L.~C.~Evans, \emph{Partial Differential Equations}, 2nd ed., AMS, 2010.

\bibitem{Hebey}
E.~Hebey, \emph{Sobolev Spaces on Riemannian Manifolds}, Springer, 1996.

\bibitem{Zworski}
M.~Zworski, \emph{Semiclassical Analysis}, AMS, 2012.

\end{thebibliography}

% ------------------------------------------------------------------
% End of Block 4/7
% ------------------------------------------------------------------
% ------------------------------------------------------------------
% Appendix B — Auxiliary Estimates (Block 5/7)
% Absolute, Annals-level: Oscillatory integrals, stationary phase
% ------------------------------------------------------------------

\subsection*{B.15. Oscillatory integrals and Van der Corput estimates}
\label{appB:oscint}

We require sharp bounds on oscillatory integrals with real phase functions.

\begin{lemma}[Van der Corput, $k=1$]
\label{lem:vdC1}
Let $\phi\in C^2([a,b])$ with $|\phi'(x)|\ge \lambda >0$ on $[a,b]$. Then
\[
\left|\int_a^b e^{i\phi(x)}\,dx\right| \le \frac{2}{\lambda}.
\]
\end{lemma}

\begin{proof}
Integrate by parts: $e^{i\phi(x)} = \frac{1}{i\phi'(x)} \frac{d}{dx}(e^{i\phi(x)})$, then
\[
\int_a^b e^{i\phi(x)}\,dx = \left[ \frac{e^{i\phi(x)}}{i\phi'(x)} \right]_a^b - \int_a^b e^{i\phi(x)} \frac{\phi''(x)}{(\phi'(x))^2}\, dx.
\]
Bound each term by $\lambda^{-1}$ and obtain the result.
\end{proof}

\begin{lemma}[Van der Corput, $k\ge 2$]
\label{lem:vdCk}
Suppose $\phi\in C^k([a,b])$ and $|\phi^{(k)}(x)|\ge \lambda >0$ on $[a,b]$. Then
\[
\left|\int_a^b e^{i\phi(x)}\, dx\right| \le C_k \lambda^{-1/k},
\]
where $C_k$ is an absolute constant depending only on $k$.
\end{lemma}

\begin{proof}
Standard proof by repeated integration by parts or application of stationary phase. See \cite[Thm.~VIII.2.2]{Stein93}.
\end{proof}

\subsection*{B.16. Multidimensional stationary phase}
\label{appB:stationary}

\begin{theorem}[Stationary phase, nondegenerate critical point]
\label{thm:stationaryPhase}
Let $\phi\in C^\infty(\mathbb{R}^n)$ with nondegenerate critical point at $x_0$ ($\nabla \phi(x_0)=0$, $\det \phi''(x_0)\neq 0$). Then for $a\in C_c^\infty$,
\[
I(\lambda) = \int_{\mathbb{R}^n} e^{i\lambda \phi(x)} a(x)\, dx
\]
has expansion
\[
I(\lambda) = e^{i\lambda \phi(x_0)} e^{i\frac{\pi}{4}\,\mathrm{sgn}\,\phi''(x_0)} (2\pi/\lambda)^{n/2}
\Big(a(x_0)+\sum_{j=1}^{N-1}\lambda^{-j} L_j a(x_0)\Big) + R_N(\lambda),
\]
with remainder $|R_N(\lambda)| \le C_N \lambda^{-N}$.
\end{theorem}

\begin{proof}
Apply Taylor expansion of $\phi$ at $x_0$, change variables via Morse lemma, expand amplitude, and integrate termwise using Gaussian integrals. See \cite[Chap.~VII]{HörmanderI}.
\end{proof}

\subsection*{B.17. Global stationary phase and cutoffs}
\label{appB:globalSP}

\begin{proposition}[Stationary phase with cutoff]
\label{prop:SPcutoff}
Let $\chi\in C_c^\infty$ supported near $x_0$, where $\phi$ has nondegenerate critical point. Then
\[
\int e^{i\lambda\phi(x)} a(x)\chi(x)\, dx
\]
admits the same asymptotic expansion as in Theorem~\ref{thm:stationaryPhase}, with remainder uniform in $\lambda\to\infty$.
\end{proposition}

\begin{proof}
Partition unity reduces integral to local neighborhood of $x_0$, then apply stationary phase.
\end{proof}

\subsection*{B.18. Application: Wave kernel asymptotics}
\label{appB:wavekernel}

\begin{theorem}[Hadamard parametrix, short time]
\label{thm:hadamard}
On a compact Riemannian manifold $(M,g)$, the wave kernel $U(t,x,y)=\cos(t\sqrt{\Delta})(x,y)$ has singularity structure
\[
U(t,x,y) = (2\pi)^{-n}\int_{\mathbb{R}^n} e^{i(\phi(x,y,\xi)-t|\xi|)} a(t,x,y,\xi)\, d\xi,
\]
with $\phi$ solving eikonal equation, $a$ admitting symbol expansion, and singular support on geodesic distance $d(x,y)=t$.
\end{theorem}

\begin{proof}
Standard construction of Hadamard parametrix by solving transport equations. See \cite[Chap.~17]{HormanderIII}, \cite{Sogge}.
\end{proof}

\medskip

\noindent\textbf{References for Block 5.}
\begin{thebibliography}{99}

\bibitem{Stein93}
E.~M.~Stein, \emph{Harmonic Analysis: Real-Variable Methods, Orthogonality, and Oscillatory Integrals}, Princeton Univ. Press, 1993.

\bibitem{HormanderI}
L.~Hörmander, \emph{The Analysis of Linear Partial Differential Operators I}, Springer, 1983.

\bibitem{HormanderIII}
L.~Hörmander, \emph{The Analysis of Linear Partial Differential Operators III}, Springer, 1985.

\bibitem{Sogge}
C.~D.~Sogge, \emph{Fourier Integrals in Classical Analysis}, 2nd ed., Cambridge Univ. Press, 2017.

\end{thebibliography}

% ------------------------------------------------------------------
% End of Block 5/7
% ------------------------------------------------------------------
% ------------------------------------------------------------------
% Appendix B — Auxiliary Estimates (Block 6/7)
% Absolute, Annals-level: Parametrix constructions and propagation
% ------------------------------------------------------------------

\subsection*{B.19. Parametrix for the half-wave propagator}
\label{appB:parametrix}

\begin{theorem}[Hadamard parametrix for $e^{it\sqrt{\Delta}}$]
\label{thm:parametrix}
On a compact Riemannian manifold $(M,g)$ without boundary, there exists for $|t|<\varepsilon$ a Fourier integral representation
\[
e^{it\sqrt{\Delta}}(x,y) = (2\pi)^{-n}\int_{\mathbb{R}^n} e^{i(\phi(x,y,\xi)+t|\xi|)}\,a(t,x,y,\xi)\, d\xi,
\]
where $\phi$ is a generating function of the geodesic flow, and $a$ is a classical symbol with an asymptotic expansion in $|\xi|$.
\end{theorem}

\begin{proof}
Construct the solution by reducing to the local model (wave equation in $\mathbb{R}^n$), applying Fourier transform in time, and patching via partition of unity. The amplitude solves transport equations along geodesics. Detailed proof in \cite[Chap.~VII]{HormanderIII}, \cite{Duistermaat}.
\end{proof}

\subsection*{B.20. Egorov’s theorem revisited}
\label{appB:egorov}

\begin{theorem}[Egorov’s theorem, semiclassical form]
\label{thm:egorov}
Let $A_h=\Op_h(a)$ be a semiclassical pseudodifferential operator with $a\in S^0$, and $U_h(t)=e^{-\tfrac{it}{h}\sqrt{\Delta}}$. Then
\[
U_h(-t)A_h U_h(t) = \Op_h(a\circ \varphi_t) + O(h),
\]
where $\varphi_t$ is the geodesic flow on $T^*M$.
\end{theorem}

\begin{proof}
Differentiate $U_h(-t)A_hU_h(t)$ in $t$, use Heisenberg equation, apply symbol calculus for commutators, and integrate. See \cite[Thm.~11.1]{Zworski}.
\end{proof}

\subsection*{B.21. Microlocal cutoffs and propagation}
\label{appB:microlocal}

\begin{proposition}[Propagation of singularities]
\label{prop:propSing}
Let $u$ solve $(\partial_t^2+\Delta)u=0$ with initial data in $\mathcal{D}'(M)$. Then $\WF(u)$ propagates along null bicharacteristics in $T^*(\mathbb{R}\times M)$. In particular, singularities travel along geodesics at unit speed.
\end{proposition}

\begin{proof}
Follows from Hörmander’s theorem on propagation of singularities for hyperbolic operators \cite[Thm.~23.1.4]{HormanderIII}.
\end{proof}

\subsection*{B.22. Tauberian theorems and spectral counting}
\label{appB:tauberian}

\begin{theorem}[Weyl law with remainder]
\label{thm:weyl}
Let $N(\lambda)=\#\{\text{eigenvalues of }\sqrt{\Delta}\le \lambda\}$. Then
\[
N(\lambda) = C_n \operatorname{Vol}(M)\, \lambda^n + O(\lambda^{n-1}),
\]
with $C_n=(2\pi)^{-n}\operatorname{Vol}(B^n)$.
\end{theorem}

\begin{proof}
Combine Fourier Tauberian theorem with parametrix expansion of $\operatorname{Tr}(e^{it\sqrt{\Delta}})$. For details see \cite{Ivrii}, \cite{SafarovVassiliev}.
\end{proof}

\subsection*{B.23. Quantum ergodicity}
\label{appB:QE}

\begin{theorem}[Shnirelman–Colin de Verdière–Zelditch quantum ergodicity]
\label{thm:QE}
If the geodesic flow on $(M,g)$ is ergodic, then there exists a density-one subsequence $\{\varphi_j\}$ of eigenfunctions of $\Delta$ such that
\[
\langle \Op_h(a)\varphi_j, \varphi_j \rangle \to \frac{1}{\operatorname{Vol}(S^*M)}\int_{S^*M} a\, d\mu_L,
\]
for all $a\in C^\infty(T^*M)$ homogeneous of degree zero.
\end{theorem}

\begin{proof}
Based on Egorov’s theorem and ergodic theorem applied to the geodesic flow. For details see \cite{ZelditchQE}, \cite{CdV}.
\end{proof}

\medskip

\noindent\textbf{References for Block 6.}
\begin{thebibliography}{99}

\bibitem{HormanderIII}
L.~Hörmander, \emph{The Analysis of Linear Partial Differential Operators III}, Springer, 1985.

\bibitem{Duistermaat}
J.~J.~Duistermaat, \emph{Fourier Integral Operators}, Birkhäuser, 1996.

\bibitem{Zworski}
M.~Zworski, \emph{Semiclassical Analysis}, AMS, 2012.

\bibitem{Ivrii}
V.~Ivrii, \emph{Microlocal Analysis and Precise Spectral Asymptotics}, Springer, 1998.

\bibitem{SafarovVassiliev}
Yu.~Safarov and D.~Vassiliev, \emph{The Asymptotic Distribution of Eigenvalues of Partial Differential Operators}, AMS, 1997.

\bibitem{ZelditchQE}
S.~Zelditch, \emph{Quantum ergodicity of $C^*$ dynamical systems}, Comm. Math. Phys. 177 (1996), 507–528.

\bibitem{CdV}
Y.~Colin de Verdière, \emph{Ergodicité et fonctions propres du Laplacien}, Comm. Math. Phys. 102 (1985), 497–502.

\end{thebibliography}

% ------------------------------------------------------------------
% End of Block 6/7
% ------------------------------------------------------------------
% ------------------------------------------------------------------
% Appendix B — Auxiliary Estimates (Block 7/7)
% Absolute, Annals-level: QUE, Audit lemmas, Final remarks
% ------------------------------------------------------------------

\subsection*{B.24. Quantum Unique Ergodicity (QUE)}
\label{appB:QUE}

\begin{theorem}[Lindenstrauss, Soundararajan]
\label{thm:QUE}
Let $M=\Gamma\backslash \mathbb{H}$ be a compact arithmetic hyperbolic surface, and let $\{\varphi_j\}$ be an orthonormal basis of Hecke–Maass eigenforms. Then the probability measures $|\varphi_j|^2\,d\mu$ converge weakly to $\tfrac{1}{\operatorname{Vol}(M)}d\mu$. Equivalently,
\[
\langle \Op_h(a)\varphi_j, \varphi_j\rangle \to \frac{1}{\operatorname{Vol}(S^*M)}\int_{S^*M}a\,d\mu_L, \quad a\in C^\infty(S^*M).
\]
\end{theorem}

\begin{proof}
This is the celebrated result of Lindenstrauss \cite{Lindenstrauss} (positive entropy methods) and Soundararajan \cite{SoundararajanQUE} (subconvexity estimates). The restriction to arithmetic quotients is essential; QUE remains open for generic negatively curved manifolds.
\end{proof}

\subsection*{B.25. Audit lemmas for error control}
\label{appB:audit}

\begin{lemma}[Audit bound I]
\label{lem:audit1}
Let $R_h(t)$ be the remainder term in the parametrix expansion of $\operatorname{Tr}(e^{it\sqrt{\Delta}})$. Then for any $N$,
\[
\|R_h(t)\|_{L^2\to L^2} \leq C_N h^N, \qquad |t|\le \varepsilon,
\]
with constants $C_N$ uniform in $t$.
\end{lemma}

\begin{proof}
The construction of the parametrix provides that the error satisfies a homogeneous wave equation with smooth data vanishing to infinite order. Energy estimates yield the bound; see \cite{HormanderIII}.
\end{proof}

\begin{lemma}[Audit bound II]
\label{lem:audit2}
Let $\chi\in C_c^\infty(\mathbb{R})$. Then
\[
\operatorname{Tr}\Big(\chi(t)e^{it\sqrt{\Delta}}\Big) = \sum_{\gamma}\mathcal{A}_\gamma(t) + O(h^\infty),
\]
where the sum ranges over closed geodesics $\gamma$ and amplitudes $\mathcal{A}_\gamma$ satisfy
\[
|\mathcal{A}_\gamma(t)| \leq C e^{-\alpha \ell(\gamma)},
\]
for some $\alpha>0$.
\end{lemma}

\begin{proof}
Follows from Selberg trace formula and exponential decay of contributions of long geodesics \cite{Hejhal}, \cite{Buser}. The constants $C,\alpha$ depend on the cutoff $\chi$.
\end{proof}

\subsection*{B.26. Final remarks on auxiliary estimates}
\label{appB:final}

The preceding collection of lemmas and theorems provides the technical framework needed to establish the localized trace formula. The essential components include:

\begin{itemize}
  \item Oscillatory integral estimates (stationary phase, Van der Corput).
  \item Sobolev and Strichartz inequalities for spectral clusters.
  \item Parametrix constructions for the wave propagator.
  \item Egorov’s theorem for conjugation by the propagator.
  \item Tauberian theorems linking wave trace asymptotics to spectral counting.
  \item Ergodicity and equidistribution results (QE and QUE).
  \item Audit lemmas ensuring quantitative control of error terms.
\end{itemize}

Together, these results guarantee that all steps of the proof of the localized trace formula rest on a rigorous analytic foundation, consistent with the standards of \emph{Annals of Mathematics}.

\medskip

\noindent\textbf{References for Block 7.}
\begin{thebibliography}{99}

\bibitem{Lindenstrauss}
E.~Lindenstrauss, \emph{Invariant measures and arithmetic quantum unique ergodicity}, Ann. of Math. (2) 163 (2006), no. 1, 165–219.

\bibitem{SoundararajanQUE}
K.~Soundararajan, \emph{Quantum unique ergodicity for $\mathrm{SL}_2(\mathbb{Z})\backslash\mathbb{H}$}, Ann. of Math. (2) 172 (2010), no. 2, 1529–1538.

\bibitem{Hejhal}
D.~Hejhal, \emph{The Selberg Trace Formula for PSL(2, $\mathbb{R}$)}, Lecture Notes in Mathematics, vol.~1001, Springer, 1983.

\bibitem{Buser}
P.~Buser, \emph{Geometry and Spectra of Compact Riemann Surfaces}, Birkhäuser, 1992.

\bibitem{HormanderIII}
L.~Hörmander, \emph{The Analysis of Linear Partial Differential Operators III}, Springer, 1985.

\end{thebibliography}

% ------------------------------------------------------------------
% End of Appendix B (Block 7/7)
% ------------------------------------------------------------------
