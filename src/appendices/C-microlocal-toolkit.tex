\section*{Appendix C. Microlocal Toolkit}
\addcontentsline{toc}{section}{Appendix C. Microlocal Toolkit}

\subsection*{C1. Hadamard Parametrix for the Wave Kernel}

\noindent
This block records the explicit construction of the Hadamard parametrix
for the wave propagator on hyperbolic surfaces and provides quantitative
error bounds tailored to the analysis of the localized spectral projector
$P_{\lambda,\eta}$. The presentation is classical but reformulated to
ensure that:
\begin{itemize}
  \item All constants are explicit and dependencies are tracked.
  \item Normalization of the spectrum is consistent across the monograph.
  \item Error terms are controlled hierarchically in terms of $(\lambda,\eta)$.
\end{itemize}

\paragraph{Geometric setup.}
Let $M = \Gamma \backslash \mathbb{H}$ be a finite-area hyperbolic surface,
where $\Gamma \subset \mathrm{PSL}_2(\mathbb{R})$ is a cofinite Fuchsian group
with cusp(s). Denote by $\Delta$ the Laplace--Beltrami operator on $M$,
nonnegative and self-adjoint on $L^2(M)$. Its spectrum is parameterized as
\[
\Spec(\Delta) = \left\{ \tfrac14 + r_j^2 : r_j \ge 0 \right\}
\ \cup\ [1/4,\infty),
\]
so that eigenvalues $\lambda_j = 1/4 + r_j^2$ correspond to spectral parameters $r_j$.

Define the wave group
\[
U(t) = e^{it\sqrt{\Delta - 1/4}}, \qquad t \in \mathbb{R}.
\]

\paragraph{Exact kernel on the universal cover.}
On the universal cover $\mathbb{H}$, the wave kernel admits the Plancherel
representation
\[
U_{\mathbb{H}}(t; z, w) = \frac{1}{2\pi} \int_{\mathbb{R}}
e^{i t r}\, \varphi_r(z, w)\, r \tanh(\pi r)\, dr,
\]
where $\varphi_r(z,w)$ is the spherical function, explicitly
\[
\varphi_r(z,w) = P_{-1/2+ir}\big(\cosh d(z,w)\big).
\]
This formula reflects the Fourier--Helgason transform and provides
the exact structure of the wave propagator.

\paragraph{Hadamard parametrix: local construction.}
For small times $|t|\le c\log \lambda$ with fixed $c>0$ and large $\lambda$,
the Hadamard parametrix gives a local oscillatory expansion of the kernel:
\[
U_{\mathbb{H}}(t; z,w) \sim (2\pi)^{-1}\int_{\mathbb{R}^2}
e^{i \varphi(z,w,\xi,t)}\, a(z,w,\xi,t)\, d\xi,
\]
with the phase $\varphi$ solving the Hamilton--Jacobi equation
\[
\partial_t \varphi(z,w,\xi,t) + H(\nabla_z\varphi) = 0,\qquad
\varphi(z,w,\xi,0) = \langle z-w,\xi\rangle,
\]
and the amplitude $a$ determined by transport equations
\[
\partial_t a + \tfrac12 \Delta_\xi \varphi \cdot a = 0,\qquad a(\cdot,0)=1.
\]

\paragraph{Explicit parametrix form.}
In practice, the parametrix takes the form
\[
U_{\mathbb{H}}(t; z,w) = \frac{1}{\sqrt{t^2-d(z,w)^2}}\,
\chi\!\Big(\frac{d(z,w)}{|t|}\Big)\,+\,R(t;z,w),
\]
where:
\begin{itemize}
  \item $\chi$ is a smooth cutoff supported where $d(z,w)<|t|$,
  \item $R(t;z,w)$ is a smooth error kernel, quantitatively controlled below.
\end{itemize}

\paragraph{Error bounds.}
For $|t|\le c\log \lambda$, the remainder satisfies
\[
\| U_{\mathbb{H}}(t) - U^{\mathrm{Had}}(t)\|_{L^2\to L^2}
\ \ll_A\ \lambda^{-A},
\]
for every $A>0$, provided the parametrix is expanded to sufficiently high order.
Equivalently, in semiclassical notation $h=\lambda^{-1}$, the error is
$O(h^A)$ for arbitrary $A$, with constants depending only on $(A,c)$ and $M$.

\paragraph{Descent to the quotient.}
The kernel on $M$ is obtained by periodization:
\[
U_M(t; z,w) = \sum_{\gamma \in \Gamma} U_{\mathbb{H}}(t; z,\gamma w).
\]
Finite propagation speed implies $U_{\mathbb{H}}(t; z,w)$ is supported where
$d(z,w)\le |t|$, hence the periodization sum is finite for fixed $t$.

\paragraph{Quantitative lemma.}
\begin{lemma}[Hadamard parametrix, quantitative form]\label{lem:C1-hadamard}
Fix $c>0$. For every $N\ge 1$,
\[
U(t;z,w) = \sum_{k=0}^N a_k(z,w)\,(t^2-d(z,w)^2)_+^{k-1/2}
\ +\ R_N(t;z,w),
\]
valid for $|t|\le c\log \lambda$, where:
\begin{itemize}
  \item $a_k(z,w)$ are smooth coefficients depending only on the geometry of $M$,
  \item for hyperbolic metrics, $a_0(z,w)=(4\pi)^{-1}$ universally,
  \item the remainder satisfies
  \[
  \|R_N(t)\|_{L^2\to L^2} \ll \lambda^{-N},
  \]
  uniformly for $|t|\le c\log \lambda$.
\end{itemize}
\end{lemma}

\paragraph{Propagation of singularities.}
\begin{corollary}[Wavefront relation]\label{cor:C1-wavefront}
The wavefront set $\mathrm{WF}(U(t))$ lies in the canonical relation
\[
C = \{(z,\xi;w,\eta)\in T^*M\times T^*M:\ (z,\xi)=g^t(w,\eta)\},
\]
where $g^t$ is the geodesic flow on $T^*M$. Thus $U(t)$ transports singularities
along geodesics, in agreement with Egorov’s theorem.
\end{corollary}

\paragraph{Link to spectral projectors.}
The localized spectral projector admits the representation
\[
P_{\lambda,\eta} = \int_{\mathbb{R}} \widehat{\chi}_\eta(t)\, e^{it\lambda}\, U(t)\,dt,
\]
with $\chi_\eta$ a smooth cutoff of width $\eta$.
Inserting the Hadamard parametrix into this formula and applying stationary phase
yields precise kernel estimates for $P_{\lambda,\eta}$.

\paragraph{Error hierarchy.}
For $|t|\le c\log \lambda$, the contribution of the remainder $R_N(t)$ satisfies
\[
\Big\|\int \widehat{\chi}_\eta(t)\, e^{it\lambda}\, R_N(t)\,dt\Big\|_{L^2\to L^2}
\ \ll\ \lambda^{-N}.
\]
Thus, by taking $N$ large, the error becomes negligible compared to the principal terms.

\begin{auditblock}[C1]
\textbf{Goals achieved:}
\begin{itemize}
  \item Constructed a Hadamard parametrix valid for logarithmic times $|t|\le c\log \lambda$.
  \item Established quantitative operator bounds with error $O(\lambda^{-N})$.
  \item Verified wavefront propagation along geodesics (Cor.~\ref{cor:C1-wavefront}).
  \item Linked parametrix to spectral projectors $P_{\lambda,\eta}$.
\end{itemize}
\textbf{Invariants:}
\begin{itemize}
  \item Constants depend only on $(c,N)$ and the geometry of $M$.
  \item No hidden dependence on spectral parameters.
  \item Remainders are hierarchically controlled.
\end{itemize}
\textbf{Forward links:} stationary phase analysis (Appendix E).  
\textbf{Backward links:} wave kernel constructions in Chapter~5.
\end{auditblock}

\subsection*{C3. Uniform Sobolev Bounds and Microlocal Cutoffs}

\noindent
This block establishes uniform Sobolev estimates and the microlocal cutoff
machinery required for the localized trace formula.  
The goal is twofold:
\begin{enumerate}
  \item Control spectrally localized eigenfunctions in $L^p$ norms with constants
  explicit in $(\lambda,\eta)$.
  \item Provide cutoff operators that isolate regions of phase space while
  preserving boundedness and negligible commutators with the spectral projector.
\end{enumerate}

\paragraph{Sobolev framework.}
For $s\in\mathbb R$, define the Sobolev norm
\[
\|f\|_{H^s(M)} = \|(1+\Delta)^{s/2} f\|_{L^2(M)}.
\]
Eigenfunctions $\phi_j$ with $\Delta \phi_j = (1/4+r_j^2)\phi_j$ satisfy
\[
\|\phi_j\|_{H^s} \asymp (1+r_j)^s \|\phi_j\|_{L^2},
\]
with constants depending only on $s$ and the geometry of $M$.

\begin{theorem}[Uniform Sobolev estimate]\label{thm:C3-sobolev}
Let $f$ be spectrally localized to $[\lambda-\eta,\lambda+\eta]$, with
$\lambda^{-\theta}\le \eta\le 1$. Then for any $2\le p\le\infty$,
\[
\|f\|_{L^p(M)} \ \ll_{p,M}\ \lambda^{\sigma(p)}\,\eta^{-\rho(p)}\,\|f\|_{L^2(M)},
\]
where $\sigma(p),\rho(p)\ge 0$ are explicit exponents arising from Sobolev
embedding and Bernstein-type inequalities. The implicit constant depends only
on $p$ and $M$, not on $\lambda,\eta$.
\end{theorem}

\paragraph{Microlocal cutoffs.}
Let $\chi\in C_c^\infty(T^*M)$ be a compactly supported symbol. Define
\[
A=\Op_h(\chi), \qquad h=\lambda^{-1}.
\]
Such $A$ projects onto a localized region of phase space.

\begin{lemma}[Cutoff properties]\label{lem:C3-cutoff}
For $A=\Op_h(\chi)$ as above:
\begin{enumerate}
  \item $\|A\|_{L^2\to L^2}\ll 1$, uniformly in $h$.
  \item If $\chi$ vanishes outside $K\subset T^*M$, then $Af$ is microlocally
  supported in $K$.
  \item If $\chi_1,\chi_2$ have disjoint supports, then
  \[
  \Op_h(\chi_1)\Op_h(\chi_2) = O(h^\infty)\quad\text{in }L^2\to L^2.
  \]
\end{enumerate}
\end{lemma}

\paragraph{Resolvent bounds.}
Resolvent estimates provide global spectral control:

\begin{proposition}[Resolvent estimate]\label{prop:C3-resolvent}
For $s=1/2+i\lambda$ with $\lambda\ge 1$,
\[
\big\|(\Delta - (1/4+\lambda^2))^{-1}\big\|_{L^2\to L^2} \ll 1,
\]
with constant depending only on $M$. More generally, for $\Im s\neq 0$,
\[
\|(\Delta-s(1-s))^{-1}\|\ll |\Im s|^{-1}.
\]
\end{proposition}

\paragraph{Stability under projectors.}
Combining Egorov’s theorem (Block C2) with the cutoff calculus yields:

\begin{lemma}[Cutoff stability]\label{lem:C3-stability}
Let $A=\Op_h(\chi)$ with $\chi$ supported away from cusp regions. Then
\[
\|[P_{\lambda,\eta},A]\|_{L^2\to L^2} \ll h^{1-\delta},
\]
for some $\delta>0$ depending only on $M$. The same estimate holds with $U(t)$
in place of $P_{\lambda,\eta}$, uniformly for $|t|\le c\log \lambda$.
\end{lemma}

\paragraph{Consequences.}
\begin{itemize}
  \item Sobolev estimates control $L^p$ growth of spectrally localized functions.
  \item Cutoffs isolate microlocal regions with negligible interaction.
  \item Resolvent bounds guarantee no uncontrolled growth in spectral expansions.
  \item Cutoff stability ensures commutators with $P_{\lambda,\eta}$ remain negligible.
\end{itemize}

\begin{auditblock}[C3]
\textbf{Goals achieved:}
\begin{itemize}
  \item Uniform Sobolev estimates for spectrally localized functions (Thm.~\ref{thm:C3-sobolev}).
  \item Microlocal cutoff operators with stability and disjointness properties (Lemma~\ref{lem:C3-cutoff}).
  \item Global resolvent estimates (Prop.~\ref{prop:C3-resolvent}).
  \item Stability of cutoffs with respect to $P_{\lambda,\eta}$ and $U(t)$ (Lemma~\ref{lem:C3-stability}).
\end{itemize}
\textbf{Invariants:}
\begin{itemize}
  \item Constants explicit in $(\lambda,\eta)$ and geometric invariants of $M$.
  \item Uniformity across cusp truncations.
\end{itemize}
\textbf{Forward links:} used in geodesic expansions (Chapter 6).  
\textbf{Backward links:} Egorov refinements (Block C2).
\end{auditblock}

\subsection*{C4. Consolidated Microlocal Toolkit and Forward Links}

\noindent
This final block consolidates the microlocal tools developed in C1–C3 into a
coherent package, explicitly stating their interrelations and preparing them
for use in the localized trace formula (Chapters 5–7) and in the appendices
on spectral counting and applications.

\paragraph{Hierarchy of tools.}
\begin{enumerate}
  \item \textbf{Hadamard Parametrix (C1):} Provides explicit local expansions of the
  wave kernel up to logarithmic time, with quantitative error bounds $O(\lambda^{-N})$.
  \item \textbf{Quantitative Egorov (C2):} Describes symbolic evolution under the
  geodesic flow up to times $|t|\le c\log \lambda$, with controlled error $O(h^{1-\delta})$.
  \item \textbf{Sobolev and Cutoff Estimates (C3):} Supply $L^p$ bounds for
  spectrally localized functions, resolvent control, and stability of microlocal cutoffs.
\end{enumerate}

\paragraph{Unified statement.}
The toolkit can be summarized in the following theorem:

\begin{theorem}[Microlocal toolkit, consolidated]\label{thm:C4-toolkit}
Let $M=\Gamma\backslash\mathbb H$ be a finite-area hyperbolic surface with cusps,
let $h=\lambda^{-1}$, and let $0<\theta<\theta_0$.
For each $\lambda\ge 1$ and $\lambda^{-\theta}\le \eta\le 1$:

\begin{enumerate}
  \item The wave group $U(t)=e^{it\sqrt{\Delta-1/4}}$ admits a Hadamard parametrix
  valid for $|t|\le c\log \lambda$, with remainder $\|R_N(t)\|_{L^2\to L^2}\ll h^N$.
  \item For any pseudodifferential operator $A=\Op_h(a)$ compactly supported in phase
  space, one has
  \[
  U(-t)AU(t)=\Op_h(a\circ g^t)+O(h^{1-\delta})
  \]
  uniformly for $|t|\le c\log \lambda$.
  \item For spectrally localized $f$ in $[\lambda-\eta,\lambda+\eta]$, Sobolev bounds
  and microlocal cutoffs yield
  \[
  \|f\|_{L^p}\ll \lambda^{\sigma(p)}\eta^{-\rho(p)}\|f\|_{L^2},\qquad
  \|[P_{\lambda,\eta},A]\|_{L^2\to L^2}\ll h^{1-\delta}.
  \]
  \item The resolvent satisfies $\|(\Delta-(1/4+\lambda^2))^{-1}\|\ll 1$ uniformly in $\lambda$.
\end{enumerate}
All constants are explicit, depending only on $c$, $\delta$, finitely many derivatives
of symbols, and the geometry of $M$ (cusp widths, injectivity radius).
\end{theorem}

\paragraph{Consequences for projectors.}
Combining (1)–(4), the kernel of the localized spectral projector
\[
P_{\lambda,\eta} = \int_{\mathbb R}\widehat\chi_\eta(t)\,e^{it\lambda}U(t)\,dt
\]
inherits the following properties:
\begin{itemize}
  \item \emph{Oscillatory structure:} controlled by Hadamard phase functions.
  \item \emph{Microlocal invariance:} stable under pseudodifferential conjugation.
  \item \emph{Spectral localization:} obeys quantitative Sobolev bounds.
  \item \emph{Error hierarchy:} negligible remainders $O(h^N)$ and $O(h^{1-\delta})$.
\end{itemize}

\paragraph{Error bookkeeping.}
Throughout Appendices B–E and Chapters 5–7, all remainders fall into three classes:
\begin{enumerate}
  \item \emph{Parametrix errors:} $O(h^N)$ for arbitrary $N$ (tunable).
  \item \emph{Egorov errors:} $O(h^{1-\delta})$ depending on geometry.
  \item \emph{Sobolev/resolvent losses:} polynomial in $\lambda,\eta^{-1}$,
  always dominated by main terms of order $\lambda\eta$.
\end{enumerate}
No hidden spectral dependence occurs.

\paragraph{Forward links.}
\begin{itemize}
  \item To Appendix~E: stationary phase expansions of oscillatory integrals
  rely on the parametrix and Egorov’s theorem.
  \item To Chapter~6: geometric expansions of the trace use cutoff stability
  and Sobolev bounds.
  \item To Appendix~B: trace identities and cusp analysis exploit the uniform
  resolvent and cutoff machinery.
\end{itemize}

\paragraph{Audit of Block C4.}
\begin{auditblock}[C4]
\textbf{Goals achieved:}
\begin{itemize}
  \item Consolidated C1–C3 into a unified theorem (\ref{thm:C4-toolkit}).
  \item Recorded explicit error hierarchy (parametrix, Egorov, Sobolev).
  \item Clarified consequences for localized projector $P_{\lambda,\eta}$.
  \item Linked forward to Appendices B, E and Chapters 5–7.
\end{itemize}
\textbf{Invariants:}
\begin{itemize}
  \item All constants explicit in geometry, $\lambda,\eta$.
  \item No conjectural inputs, only classical microlocal tools.
\end{itemize}
\textbf{Forward links:} to geodesic sums, stationary phase, and global trace formula.  
\textbf{Backward links:} Hadamard (C1), Egorov (C2), Sobolev + cutoffs (C3).
\end{auditblock}
