\appendix
\section*{Appendix G. Normalizations and Conventions}
\addcontentsline{toc}{section}{Appendix G. Normalizations and Conventions}
\label{appG:root}

\noindent\textbf{Purpose of Appendix G.}  
This appendix establishes the complete set of normalizations used throughout the monograph.  
It serves as the single source of truth for:
\begin{itemize}
    \item Measures on $\mathbb{H}$ and $\mathrm{PSL}_2(\mathbb{R})$.
    \item Haar measure conventions and Iwasawa decomposition.
    \item Fourier and spherical transforms.
    \item Pseudodifferential operator quantization.
    \item Kernel and trace-class sign conventions.
    \item Reference identities (Plancherel, stationary phase, Bessel asymptotics).
\end{itemize}
Every constant, sign, and scaling factor appearing in Appendices~A--F and H is consistent with this registry.  
No local redefinitions are permitted; deviations must reference \cref{appG:root} explicitly.  

\bigskip
\subsection*{G.1. Measures and Haar normalizations}
\label{appG:measures}

\paragraph{Hyperbolic plane.}  
We write $\mathbb{H}=\{z=x+iy:y>0\}$ with Riemannian metric
\[
ds^2 = y^{-2}(dx^2+dy^2).
\]
The associated hyperbolic area measure is
\[
dA(z) = y^{-2}\, dx\, dy,
\]
and the geodesic distance is denoted $\dist(z,w)$.  
On the horizontal boundary $y=Y$, the induced line measure is
\[
ds = Y^{-1}\, dx.
\]

\paragraph{Haar measure on $\mathrm{PSL}_2(\mathbb{R})$.}  
We fix the left Haar measure $dg$ consistent with Helgason~\cite[Ch.~I]{Helgason} and Iwaniec--Sarnak~\cite[§7.2]{Iwaniec2002}.  
In Iwasawa coordinates $g=kan$ with
\[
k=\begin{pmatrix}
\cos\theta & \sin\theta \\
-\sin\theta & \cos\theta
\end{pmatrix},\quad
a=\begin{pmatrix}
y^{1/2} & 0 \\
0 & y^{-1/2}
\end{pmatrix},\quad
n=\begin{pmatrix}
1 & x \\
0 & 1
\end{pmatrix},
\]
the Haar density is
\[
dg = \frac{1}{2\pi}\, d\theta \,\frac{dy}{y^2}\, dx.
\]

\paragraph{Quotient surfaces.}  
On $M=\Gamma\backslash\mathbb{H}$, integrals are always taken with respect to $dA$.  
If $M(Y)$ denotes the truncated surface (cutting at $y=Y$ in each cusp), then the volume satisfies
\[
\vol(M(Y)) = \vol(M) - O(Y^{-1}).
\]

\paragraph{Consistency check.}  
This normalization guarantees that the Selberg transform and Plancherel formula align with the classical convention:
\[
\int_{\PSL_2(\mathbb R)} f(g)\, dg
= \int_0^\infty\int_\mathbb R\int_0^{2\pi} f\!\left(kan\right)\,\frac{1}{2\pi}\,d\theta \,\frac{dy}{y^2}\, dx.
\]

\bigskip
\noindent\textbf{Audit of Block G.1.}
\begin{itemize}
    \item \emph{Goal G1:} Fix the hyperbolic area measure and geodesic conventions. \textbf{Verified}.
    \item \emph{Goal G2:} Specify Haar measure in full detail, including Iwasawa densities. \textbf{Verified}.
    \item \emph{Invariant G1:} All group integrals use $dg$ as defined above. No alternative normalization allowed.
    \item \emph{Forward links:} Used in Appendix~A (volume estimates), Appendix~C (Sobolev norms), and Chapter~6 (geometric expansion).
    \item \emph{Backward links:} Derived from standard sources \cite{Helgason,Iwaniec2002}.
\end{itemize}

\bigskip
\noindent\textbf{Conclusion.}  
Block G.1 eliminates all ambiguity about measure conventions, securing consistency between analytic, geometric, and spectral formulas throughout the monograph.

\subsection*{G.2. Fourier and Spectral Transforms}
\label{appG:fourier}

\paragraph{Euclidean Fourier transform.}  
On $\mathbb{R}$ we adopt the convention
\[
\widehat{f}(\xi) = \int_{\mathbb{R}} f(t)\, e^{-2\pi i t \xi}\, dt,
\qquad
f(t) = \int_{\mathbb{R}} \widehat{f}(\xi)\, e^{2\pi i t \xi}\, d\xi.
\]
This choice ensures Plancherel’s identity
\[
\|f\|_{L^2(\mathbb{R})} = \|\widehat{f}\|_{L^2(\mathbb{R})},
\]
with no additional normalization constants.

\paragraph{Hyperbolic spherical transform.}  
For radial kernels $k(\dist(z,w))$ on $\mathbb{H}$, we employ the Selberg/Harish–Chandra transform:
\[
h(r) = \int_{-\infty}^\infty k(\cosh^{-1}(1+u^2/2))\, \cos(r u)\, du,
\]
consistent with \cite[Ch.~IV]{Helgason}.  
The inversion formula reads
\[
k(\cosh d) = \frac{1}{2\pi} \int_{-\infty}^\infty h(r)\, \varphi_r(d)\, r \tanh(\pi r)\, dr,
\]
where $\varphi_r$ denotes the zonal spherical function.  

\paragraph{Fourier–Helgason transform.}  
For $f\in C_c^\infty(\mathbb{H})$, the Fourier–Helgason transform is
\[
\mathcal{F}(f)(r,\theta) = \int_{\mathbb{H}} f(z)\, E(z,\tfrac{1}{2}+ir,\theta)\, dA(z),
\]
where $E(z,s,\theta)$ is the Eisenstein plane wave.  
Plancherel measure is normalized as
\[
d\mu(r) = \frac{1}{4\pi}\, r \tanh(\pi r)\, dr.
\]

\paragraph{Bessel kernels.}  
All Bessel functions are normalized as
\[
K_{ir}(y) = K_{-ir}(y),
\]
ensuring parity symmetry. Their asymptotics are recorded in \cref{appG:refidentities}.

\paragraph{Spectral decomposition on $M=\Gamma\backslash \mathbb H$.}  
Every $f\in L^2(M)$ decomposes as
\[
f = \sum_j \langle f, \varphi_j\rangle \varphi_j
+ \sum_{\mathfrak{a}} \frac{1}{4\pi} \int_{-\infty}^\infty 
   \langle f, E_{\mathfrak{a}}(\cdot,\tfrac{1}{2}+ir)\rangle
   E_{\mathfrak{a}}(\cdot,\tfrac{1}{2}+ir)\, dr,
\]
where $\{\varphi_j\}$ are $L^2$-normalized eigenfunctions and $E_{\mathfrak{a}}$ are Eisenstein series normalized with spectral measure $dr/(4\pi)$.  

\paragraph{Spectral parameterization.}  
We write
\[
\lambda_j = \tfrac{1}{4} + r_j^2, \quad r_j\ge 0,
\]
so that the continuous spectrum starts at $1/4$, aligning with Selberg’s convention.  

\bigskip
\noindent\textbf{Audit of Block G.2.}
\begin{itemize}
    \item \emph{Goal G3:} Fix Fourier conventions (Euclidean and hyperbolic). \textbf{Verified}.
    \item \emph{Goal G4:} Normalize spherical and Fourier–Helgason transforms with precise Plancherel factors. \textbf{Verified}.
    \item \emph{Goal G5:} Ensure compatibility of spectral decomposition on $M$ with Chapters~2–7. \textbf{Verified}.
    \item \emph{Invariant G2:} All Fourier transforms use $2\pi$ in the exponential; all spectral decompositions use measure $dr/(4\pi)$.  
    \item \emph{Forward links:} Applied in Chapter~5 (microlocal analysis), Chapter~6 (geometric expansion), Appendix~D (Tauberian estimates).  
    \item \emph{Backward links:} Derived from Helgason~\cite{Helgason}, Iwaniec~\cite{Iwaniec2002}, and consistent with Appendix~B (stationary phase).  
\end{itemize}

\bigskip
\noindent\textbf{Conclusion.}  
Block G.2 secures the Fourier and spectral conventions. All later analytic and microlocal computations rely on these normalizations, ensuring that constants, transforms, and Plancherel measures remain transparent and reproducible.

\subsection*{G.3. Operator Classes and Quantization}
\label{appG:pdo}

\paragraph{Quantization scheme.}  
We adopt \emph{Weyl quantization} for pseudodifferential operators.  
Given a symbol $a(x,\xi;h)$, the operator $\Op_h^w(a)$ acts as
\[
(\Op_h^w(a)u)(x) = \frac{1}{(2\pi h)^n} \int_{\mathbb{R}^n}\int_{\mathbb{R}^n}
   e^{i(x-y)\cdot \xi /h} \, a\!\left(\frac{x+y}{2}, \xi;h\right) u(y)\, dy\, d\xi.
\]

\paragraph{Symbol classes.}  
We work with H\"ormander classes $S^m_{\rho,\delta}(\mathbb{R}^n)$ with parameters $(\rho,\delta)=(1,0)$:
\[
a \in S^m_{1,0} \iff
\forall \alpha,\beta \; \sup_{x,\xi}
(1+|\xi|)^{-m+|\beta|} |\partial_x^\alpha \partial_\xi^\beta a(x,\xi)| < \infty.
\]
This choice provides stability under composition and aligns with the semiclassical scaling adopted throughout the monograph.

\paragraph{Boundedness.}  
By the Calderón–Vaillancourt theorem, for $a\in S^0_{1,0}$, 
\[
\|\Op_h^w(a)\|_{L^2 \to L^2} \le C \sup_{|\alpha|+|\beta|\le N}
\|\partial_x^\alpha \partial_\xi^\beta a\|_\infty,
\]
with $C,N$ independent of $h$. This guarantees $L^2$-boundedness of zero-order pseudodifferential operators.

\paragraph{Composition formula.}  
For $a\in S^{m_1}_{1,0}$ and $b\in S^{m_2}_{1,0}$, one has
\[
\Op_h^w(a)\Op_h^w(b) = \Op_h^w(a\# b),
\]
where
\[
a\# b \sim \sum_{k=0}^\infty \frac{1}{k!}\left(\frac{ih}{2}\sigma(D_x,D_\xi;D_y,D_\eta)\right)^k
a(x,\xi)b(y,\eta)\Big|_{x=y,\;\xi=\eta}.
\]
Here $\sigma(D_x,D_\xi;D_y,D_\eta)$ denotes the standard symplectic form of differential operators.

\paragraph{Microlocal windows and partitions.}  
Cutoffs $\chi \in C_c^\infty(T^*M)$ are introduced to localize operators in phase space.  
Partitions of unity subordinate to a fixed atlas (see Appendix~A) guarantee that local quantizations patch consistently to a global operator on $M=\Gamma\backslash \mathbb{H}$.

\paragraph{Wave propagators.}  
The semiclassical propagator $e^{it\sqrt{\Delta-1/4}/h}$ is approximated by a Fourier integral operator with phase given by the geodesic flow and amplitude belonging to symbol class $S^0_{1,0}$.  
This convention ensures consistency with Egorov’s theorem and stationary phase expansions.

\bigskip
\noindent\textbf{Audit of Block G.3.}
\begin{itemize}
    \item \emph{Goal G6:} Establish pseudodifferential quantization framework. \textbf{Verified}.
    \item \emph{Goal G7:} Guarantee boundedness and symbolic calculus properties. \textbf{Verified}.
    \item \emph{Invariant G3:} All operators are quantized in Weyl form with $(\rho,\delta)=(1,0)$.  
    \item \emph{Forward links:} Chapter~5 (microlocal analysis), Appendix~B (stationary phase expansions), Chapter~7 (main theorems).  
    \item \emph{Backward links:} Appendix~A (atlas and partitions of unity), standard references \cite{Zworski, Hormander}.  
\end{itemize}

\bigskip
\noindent\textbf{Conclusion.}  
Block G.3 fixes the quantization scheme, symbol classes, and microlocal conventions. These form the analytic machinery for semiclassical propagation, stationary phase, and microlocalization used throughout the monograph.

\subsection*{G.4. Kernels, Traces, and Test Functions}
\label{appG:kernels}

\paragraph{Test functions.}  
All test functions $k\in C_c^\infty(\mathbb{R})$ used in spectral and trace formulae are normalized to be \emph{even}, i.e.
\[
k(t) = k(-t), \qquad \forall t \in \mathbb{R}.
\]
This symmetry ensures compatibility with spectral decompositions of the Laplace operator, whose eigenvalues appear symmetrically as $r_j$ and $-r_j$ in the spectral parameterization $\lambda_j = 1/4 + r_j^2$.

\paragraph{Wave kernel conventions.}  
We adopt the convention
\[
U(t) = \cos\!\big(t\sqrt{\Delta - 1/4}\big),
\]
so that
\[
U(t) = \frac{1}{2}\big( e^{i t \sqrt{\Delta - 1/4}} + e^{-i t \sqrt{\Delta - 1/4}} \big).
\]
This definition guarantees self-adjointness on $L^2(M)$ and consistency with the unitary normalization chosen in Appendix~G.2.  

\paragraph{Distributional pairings.}  
For Schwartz functions $f,g \in \mathcal{S}(\mathbb{R})$, the distributional pairing is defined as
\[
\langle f, g \rangle = \int_{\mathbb{R}} f(t)\, g(t)\, dt,
\]
extended by duality to tempered distributions. In particular, the pairing conventions fix the sign rules in trace identities, preventing ambiguity in oscillatory integrals.

\paragraph{Traces of operators.}  
For a trace-class operator $A$ on $L^2(M)$ with integral kernel $K_A(z,w)$, we define
\[
\operatorname{tr}(A) = \int_M K_A(z,z)\, dA(z),
\]
where $dA(z)$ is the hyperbolic area measure (Appendix~G.1).  
If $A$ is not trace-class but admits a regularized kernel (e.g. wave group $U(t)$ with compact support in $t$), the trace is interpreted distributionally in accordance with \cite{Sogge}.

\paragraph{Parametrix kernels.}  
The Hadamard parametrix and microlocal kernels appearing in Appendices~B--C are normalized so that their leading coefficients agree with the universal Euclidean or hyperbolic model:
\[
K_{\text{Had}}(t;z,w) \sim \frac{1}{\sqrt{t^2 - d(z,w)^2}} \, \chi\!\left(\tfrac{d(z,w)}{t}\right),
\]
where $\chi$ is the cutoff introduced in Appendix~C. This normalization ensures that error terms are properly scaled by powers of the semiclassical parameter $h=\lambda^{-1}$.

\paragraph{Stationary phase factors.}  
Oscillatory integrals arising from Fourier inversion are normalized so that the stationary phase expansion takes the standard form
\[
\int_{\mathbb{R}^n} e^{i \varphi(x)/h} a(x)\, dx
   \sim \sum_{k=0}^\infty h^k L_k(a,\varphi),
\]
with phase factor $e^{i\pi \operatorname{sgn}(\varphi'')/4}$ included.  
These conventions eliminate sign discrepancies across different chapters.

\bigskip
\noindent\textbf{Audit of Block G.4.}
\begin{itemize}
  \item \emph{Goal G8:} Fix normalization of test functions, kernels, and traces. \textbf{Verified}.
  \item \emph{Goal G9:} Guarantee compatibility of distributional pairings and wave kernels. \textbf{Verified}.
  \item \emph{Invariant G4:} All test functions are even; wave kernel defined as $\cos(t\sqrt{\Delta-1/4})$.  
  \item \emph{Forward links:} Appendix~B (stationary phase expansions), Appendix~C (Hadamard parametrix), Chapter~7 (trace formula).  
  \item \emph{Backward links:} Appendix~G.1 (measures), Appendix~G.2 (spectral transforms).  
\end{itemize}

\bigskip
\noindent\textbf{Conclusion.}  
Block G.4 establishes the conventions for kernels, traces, and test functions. These choices enforce absolute consistency across all analytic manipulations and prevent sign or normalization ambiguities in trace formula derivations.

\subsection*{G.5. Reference Identities}
\label{appG:identities}

\paragraph{Purpose.}  
This block collects a catalogue of standard analytic identities that are repeatedly invoked throughout the monograph.  
By recording them here once and for all, we eliminate duplication, guarantee uniform sign conventions, and provide a transparent audit trail for all constants.  
Each identity is tied to authoritative references and aligned with the notational framework of Appendices~A--F.

\medskip

\paragraph{Plancherel identity (Euclidean).}  
For $f \in L^2(\mathbb{R})$ with Fourier transform as defined in Appendix~G.2:
\[
\int_{\mathbb{R}} |f(t)|^2\, dt
  = \int_{\mathbb{R}} |\widehat{f}(\xi)|^2\, d\xi.
\]

\paragraph{Spherical Fourier transform on $\mathbb{H}$.}  
For $f \in C_c^\infty(\mathbb{H})$, the Helgason–Fourier transform is
\[
\widetilde{f}(r,\theta)
   = \int_{\mathbb{H}} f(z)\, e^{(1/2+ir)\langle z,\theta\rangle}\, dA(z),
\]
with inversion formula and Plancherel measure
\[
\|f\|_{L^2(\mathbb{H})}^2
   = \frac{1}{4\pi} \int_{-\infty}^\infty \int_{0}^{2\pi}
     |\widetilde{f}(r,\theta)|^2 \, r \tanh(\pi r)\, d\theta\, dr.
\]

\paragraph{Bessel asymptotics.}  
For fixed $\nu$ and large argument $x$:
\[
K_{i r}(x) \sim \sqrt{\frac{\pi}{2x}} e^{-x}, \qquad
J_\nu(x) \sim \sqrt{\frac{2}{\pi x}} \cos\!\left(x - \tfrac{\pi \nu}{2} - \tfrac{\pi}{4}\right).
\]
These asymptotics are used in Appendix~B (stationary phase) and Appendix~C (Hadamard parametrix).

\paragraph{Stationary phase signature factor.}  
If $\varphi''(x_0)\neq 0$ and $a\in C_c^\infty$,
\[
\int_{\mathbb{R}} e^{i\varphi(x)/h} a(x)\, dx
  \sim e^{i \varphi(x_0)/h}
       e^{i\pi \,\sgn(\varphi''(x_0))/4}
       \sum_{k=0}^\infty h^k L_k(a,\varphi),
\]
as $h\to 0^+$, where $L_k$ are explicit differential operators.  
The signature factor $e^{i\pi \sgn(\varphi'')/4}$ is fixed globally in this appendix.

\paragraph{Trace formula constants.}  
In the Selberg trace formula, the contribution of the identity conjugacy class is normalized as
\[
\frac{\vol(M)}{4\pi} \int_{-\infty}^\infty h(r)\, r \tanh(\pi r)\, dr,
\]
where $h$ is the even test function from Appendix~G.4.  
This ensures compatibility with the Plancherel identity above.

\bigskip
\noindent\textbf{Audit of Block G.5.}
\begin{itemize}
  \item \emph{Goal G10:} Consolidate reference identities to prevent duplication. \textbf{Verified}.
  \item \emph{Goal G11:} Fix global conventions for Fourier, Bessel, and stationary phase identities. \textbf{Verified}.
  \item \emph{Invariant G5:} Every analytic identity appears in exactly one location (Appendix~G.5).  
  \item \emph{Forward links:} Appendix~B (stationary phase), Appendix~C (microlocal analysis), Chapter~6 (geometric side), Chapter~7 (trace formula).  
  \item \emph{Backward links:} Appendix~G.2 (Fourier transforms), Appendix~G.4 (test functions and kernels).  
\end{itemize}

\bigskip
\noindent\textbf{Global Audit of Appendix G.}

\paragraph{Goals recap.}
\begin{itemize}
  \item \emph{Goal G1:} Fix Haar and hyperbolic measures.  
  \item \emph{Goal G2:} Standardize Fourier and spectral transforms.  
  \item \emph{Goal G3:} Define pseudodifferential quantization and operator classes.  
  \item \emph{Goal G4:} Normalize kernels, traces, and test functions.  
  \item \emph{Goal G5:} Record reference identities.  
\end{itemize}

\paragraph{Verification.}
\begin{itemize}
  \item \textbf{V(G1):} Verified in Block G.1 via explicit Haar measures.  
  \item \textbf{V(G2):} Verified in Block G.2 with Plancherel measure and Fourier conventions.  
  \item \textbf{V(G3):} Verified in Block G.3 with Weyl quantization.  
  \item \textbf{V(G4):} Verified in Block G.4 with test function symmetries and trace definitions.  
  \item \textbf{V(G5):} Verified in Block G.5 by consolidation of analytic identities.  
\end{itemize}

\paragraph{Invariants.}
\begin{itemize}
  \item \emph{Invariant G1:} All integrals use $dA = y^{-2} dx dy$.  
  \item \emph{Invariant G2:} Fourier transform fixed with $2\pi$ normalization.  
  \item \emph{Invariant G3:} Operator quantization $\Op_h^w$ with $(\rho,\delta)=(1,0)$.  
  \item \emph{Invariant G4:} Wave kernel $U(t)=\cos(t\sqrt{\Delta-1/4})$.  
  \item \emph{Invariant G5:} Reference identities centralized here.  
\end{itemize}

\paragraph{Conclusion.}  
Appendix~G provides the global \emph{normalization contract} of the monograph.  
Every measure, transform, operator, and kernel is fixed once and for all.  
Forward and backward audits confirm that all analytic constants across Appendices A–F and H are traceable to this appendix.  
This guarantees absolute consistency and eliminates ambiguity in the entire work.
