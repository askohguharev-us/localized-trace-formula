% ======================================================================
% File: src/frontmatter/00-executive-summary.tex
% ======================================================================

\section{Executive Summary}

This monograph develops a localized Selberg trace formula for finite-area
hyperbolic surfaces with cusps and derives quantitative spectral consequences
with \emph{effective} constants. The construction proceeds via a microlocal
wave approach and a band-limited functional-calculus projector acting on the
full $L^2$-space, including both discrete and continuous spectrum, with all
dependencies on geometric and spectral data explicitly recorded.

\medskip
\noindent\textbf{Historical context.}
From Selberg’s trace formula \cite{Selberg1956} through
Duistermaat–Guillemin \cite{DG1975} and Colin de Verdière \cite{CdV1980},
trace methods have linked spectrum and closed geodesics. For global counting
functions, sharp remainders are typically limited to logarithmic savings
(e.g.\ $O(\lambda/\log\lambda)$) or sub-power improvements \cite{Ivrii2016,Berger2019},
reflecting singularities in the length spectrum. We show that \emph{microlocal
localization} at frequency scale $\eta\ge \lambda^{-\theta}$ smooths the
relevant oscillatory integrals and yields genuine power savings for the local
counting problem. Conceptually, microlocal windows of width $\eta$ reduce the
impact of singularities of the length spectrum in frequency space and permit
stationary-phase/Egorov estimates with power-saving remainders.

\medskip
\noindent\textbf{Localized spectral projector (definition and domain).}
Let $\Delta$ denote the Laplacian on $X=\Gamma\backslash\mathbb{H}$ with
continuous spectrum $\sigma_{\mathrm{cont}}(\Delta)=[1/4,\infty)$ and discrete
eigenvalues $\lambda_j=\tfrac14+t_j^2$ (counted with multiplicities). Fix an
effectively determined $\lambda_0\ge 1$ (depending only on $X$). Let
$\Phi\in C^\infty_0([-1,1])$ be real-valued with $\int\Phi=1$. For
$\lambda\ge\lambda_0$ and $\eta\in[\lambda^{-\theta},1]$ set
\[
  \phi_\eta(t)\coloneqq \Phi\!\left(\frac{t-\lambda}{\eta}\right),\qquad
  P_{\lambda,\eta}\coloneqq \phi_\eta\!\left(\sqrt{\Delta-\tfrac14}\right),
\]
where $\sqrt{\Delta-\tfrac14}$ is defined by the spectral theorem for the
self-adjoint operator $\Delta$. Then $P_{\lambda,\eta}$ is a bounded
self-adjoint operator on $L^2(X)$ acting simultaneously on discrete and
continuous spectrum. Its trace is given by the standard spectral decomposition:
\begin{equation}\label{eq:trace-decomp}
  \Tr(P_{\lambda,\eta})
  \;=\; \sum_j \phi_\eta(t_j)
        \;+\; \frac{1}{4\pi}\int_{-\infty}^{\infty}
        \phi_\eta(t)\,\frac{\varphi'}{\varphi}\!\left(\tfrac12+it\right)\,dt
        \;+\; \mathcal{C}_{\mathrm{sing}}(\phi_\eta),
\end{equation}
where $\varphi(s)$ is the scattering determinant and
$\mathcal{C}_{\mathrm{sing}}(\phi_\eta)$ collects the explicit singular/constant
terms. Throughout, the implicit constants in $O(\cdot)$ depend only on $\Gamma$
(via $\kappa$, $\{w_i\}$, $r_{\mathrm{inj}}$), on the fixed profile $\Phi$, and
on the choice of $\theta<\theta_0$, and are independent of $\lambda$ and $\eta$.
We write $O_M((\eta\lambda)^{-M})$ for bounds valid for each fixed $M>0$, with
constants depending on $M$, on $\Phi$ (via uniform bounds on derivatives), and
on the geometric data of $X$.

\medskip
\noindent\textbf{Main results.}
\begin{theorem}[Localized trace formula]\label{thm:localized}
Let $X=\Gamma\backslash\mathbb{H}$ be of finite area with $\kappa$ cusps of
widths $w_i$ and injectivity radius $r_{\mathrm{inj}}>0$. There exists an
\emph{effectively computable} threshold
\[
  \theta_0 \;=\;
  \frac{c_{\mathrm{geom}}\cdot c_{\mathrm{moll}}}{\kappa \max_i w_i}\,
  \min\{1,\,r_{\mathrm{inj}}\},
\]
where $c_{\mathrm{geom}},c_{\mathrm{moll}}>0$ are explicit constants depending
only on the wave parametrix and the fixed Fejér-type profile $\Phi$. Assume
that $X$ has a spectral gap $\beta_\Gamma>0$. Then for every
$0<\theta<\theta_0$ and $\eta\in[\lambda^{-\theta},1]$ one has
\[
  \Tr(P_{\lambda,\eta})
  \;=\; \mathcal{I}_{\lambda,\eta} \,+\, \mathcal{G}_{\lambda,\eta}
        \,+\, \mathcal{P}_{\lambda,\eta}
        \;+\; O\!\left(\lambda^{1-\delta}\right),
\]
\emph{uniformly in $\eta$}. Here $\mathcal{I}$ (identity), $\mathcal{G}$
(hyperbolic) and $\mathcal{P}$ (parabolic/Eisenstein) are given by explicit
oscillatory amplitudes; the effective exponent $\delta>0$ depends only on
$\beta_\Gamma$ and on the geometric data of $X$. If only $\beta_\Gamma\ge 0$ is
known, the stated identity holds with $\delta=0$.
\end{theorem}

\begin{theorem}[Quantitative local Weyl law]\label{thm:local-weyl}
Let $N(\lambda,\eta)$ be the number of discrete eigenvalues in
$[\lambda-\eta,\lambda+\eta]$ (counted with multiplicity). Assume
$\beta_\Gamma>0$. Then, as $\lambda\to\infty$,
\[
  N(\lambda,\eta)
  \;=\; \frac{\vol(X)}{2\pi}\,\lambda\,\eta \;+\; O\!\left(\lambda^{1-\delta}\right),
\]
with the same $\delta>0$ as in Theorem~\ref{thm:localized}, \emph{uniformly for}
$\eta\in[\lambda^{-\theta},1]$. The coefficient $\vol(X)/(2\pi)$ corresponds to
the window of length $2\eta$. If only $\beta_\Gamma\ge 0$ is known, the formula
holds with $\delta=0$.
\end{theorem}

\noindent\emph{Uniform relative error.} In the regime above,
\[
  \frac{O(\lambda^{1-\delta})}{(\vol(X)/2\pi)\lambda\eta}
  \;=\; O\!\bigl(\lambda^{-\delta+\theta}\bigr).
\]
Thus choosing any fixed $\theta<\delta$ gives relative $o(1)$-error, uniformly
for $\eta\in[\lambda^{-\theta},1]$.

\medskip
\noindent\textbf{Effectivity and the role of the spectral gap.}
The remainder exponent satisfies an effective inequality
\[
  \delta \;\ge\; c_0(\kappa,\{w_i\},r_{\mathrm{inj}})\cdot \beta_\Gamma,
\]
where $c_0$ is an explicit function of geometric constants arising from the
stationary-phase bounds, Egorov-type transport, and Maass–Selberg estimates.
Appendix~J provides closed formulas/upper bounds for these constants and a
step-by-step evaluation procedure. For arithmetic surfaces, known lower bounds
for $\beta_\Gamma$ (e.g.\ Kim–Sarnak for $PSL(2,\mathbb{Z})$) yield concrete
numerical exponents.

\medskip
\noindent\textbf{Concrete example: $PSL(2,\mathbb{Z})\backslash\mathbb{H}$.}
For the modular surface, $\kappa=1$, $w_1=1$, $r_{\mathrm{inj}}\ge c_0'>0$ and
$\beta_\Gamma\ge 25/64$. Combining this with the explicit
$c_0(\kappa=1,w_1=1,r_{\mathrm{inj}}\ge c_0')$ from Appendix~J and our fixed
profile $\Phi$ yields
\[
  \delta \;\ge\; \frac{1}{64},\qquad
  N(\lambda,\eta) \;=\; \frac{\vol}{2\pi}\lambda\eta \;+\; O\!\left(\lambda^{63/64}\right),
\]
uniformly for $\eta\in[\lambda^{-\theta},1]$ and any $\theta<\theta_0$.

\medskip
\noindent\textbf{Applications (selected).}
\begin{itemize}
  \item Variance bounds for Fourier coefficients of Hecke–Maass forms in short
  spectral windows, with effective power-saving remainders.
  \item Quantitative statements in quantum chaos (QUE, delocalization, scarring),
  where microlocal windows improve error terms.
  \item \emph{Prime Geodesic Theorem (conditional).} Assuming $\beta_\Gamma>0$,
  the localized framework supplies all ingredients (test functions, uniform error
  control, effective constants) to carry out the standard derivation of
  \(
    \pi_\Gamma(X) = \mathrm{Li}(X) + O\!\left(X^{1-\delta}\right)
  \)
  with computable constants. A full implementation is outside the present scope;
  the algorithmic blueprint is recorded within our framework.
\end{itemize}

\medskip
\noindent\textbf{Error budget and notation.}
All error contributions are separated and bounded with explicit constants
depending only on geometric data and $\beta_\Gamma$:
\[
  \text{mollifier error} \;\ll\; C_1\,\lambda^{-\theta},\quad
  \text{hyperbolic truncation} \;\ll\; C_2\,\lambda^{1-\delta},\quad
  \text{Eisenstein term} \;\ll\; C_3\,\lambda^{1-\delta}.
\]
We use $O_M((\eta\lambda)^{-M})$ to denote bounds holding for each fixed $M>0$,
with constants depending on $M$, on the fixed $\Phi$ (via uniform derivative
bounds), and on the geometric data of $X$.

\medskip
\noindent\textbf{Methodological standards.}
Statements are accompanied by explicit hypotheses, constant tracking, and
forward/backward links to proofs. All constructive thresholds (e.g.\ $\theta_0$)
are displayed in the main text, with detailed derivations and algorithms in
Appendix~J. The implicit constants in all $O(\cdot)$-notations depend only on
$\Gamma$, the fixed profile $\Phi$, and the chosen $\theta<\theta_0$, but not on
$\lambda$ or $\eta$.

% ======================================================================
% End of 00-executive-summary.tex
% ======================================================================
