% ======================================================================
% File: src/frontmatter/00-readers-roadmap.tex
% ======================================================================

\section{Reader’s Roadmap}
\label{sec:readers-roadmap}

This section provides a structured roadmap for navigating the monograph.
The aim is to clarify the logical progression, to highlight the dependencies
between chapters, and to ensure that both expert readers and graduate-level
students can orient themselves efficiently.
We emphasize three layers: (i) orientation (high-level goals), 
(ii) audit trail (explicit dependencies), and 
(iii) navigation (forward/backward links).

\medskip
\noindent\textbf{Orientation.}
The monograph develops a localized Selberg trace formalism with microlocal
windows, yielding sharp local Weyl laws and effective error budgets.
The technical core is the construction of band-limited parametrices for the
wave group and the quantification of error propagation through spectral and
geometric expansions.
The roadmap makes explicit how each component (notation, kernels,
parametrices, constants) supports the final theorems on eigenvalue
counting and uniform asymptotics.

% ----------------------------------------------------------------------
\subsection*{Layer I: Chapter Overview}

\begin{description}
  \item[Chapter 1: Introduction.]
  States the main theorems, their context in spectral geometry, and the scope
  of the results. Outlines prior work and emphasizes novelty (cf.\ Executive Summary).
  
  \item[Chapter 2: Preliminaries.]
  Establishes background on hyperbolic geometry, Fuchsian groups,
  Laplacians, and Eisenstein series. Sets analytic and geometric conventions.
  
  \item[Chapter 3: Spectral and Geometric Tools.]
  Reviews the Selberg trace formula, spherical transforms,
  and functional calculus, with proofs deferred to appendices.
  
  \item[Chapter 4: Complexity Framework.]
  Introduces quantitative geometry (injectivity radius, systole, thick–thin decomposition)
  and prepares microlocal tools for parametrix estimates.
  
  \item[Chapter 5: Microlocal Analysis.]
  Develops semiclassical calculus, band-limited parametrices, and analytic windows $h_\eta$.
  This chapter forms the analytic core: Egorov transport, stationary phase, and error bounds.
  
  \item[Chapter 6: Trace Expansion.]
  Assembles contributions of discrete spectrum, Eisenstein series, hyperbolic geodesics,
  and parabolic terms. Establishes localized trace identities.
  
  \item[Chapter 7: Local Weyl Law.]
  Derives uniform asymptotics for localized eigenvalue counts $N(\lambda,\eta)$.
  Power-saving error terms are proved using the parametrix bounds from Chapter~5
  and error budgets from Chapter~6.
  
  \item[Chapter 8: Error Budgets and Constants.]
  Collects constants, dependency ledgers, and quantitative bounds.
  Ensures all hidden constants are auditable and cross-referenced to Appendix~J.
  
  \item[Chapter 9: Applications and Extensions.]
  Illustrates applications: numerical verification strategies,
  extensions to congruence subgroups, and open problems.
  
  \item[Appendices A–J.]
  Provide proofs of parametrix constructions, stationary phase lemmas,
  spherical transform identities, and complete audit trails of constants.
\end{description}

% ----------------------------------------------------------------------
\subsection*{Layer II: Audit Trail and Dependencies}

\begin{itemize}
  \item \textbf{Notation and Glossary (\S\ref{sec:notation-glossary}).}
  Fixed once for all; every subsequent theorem relies on these conventions.
  
  \item \textbf{Parametrix constants (Ch.~5, App.~A).}
  Feed into error estimates in Chapters~6--7.
  
  \item \textbf{Scattering data.}
  Enter through $(\sigma'/\sigma)(s)$ in trace expansions (Ch.~6).
  
  \item \textbf{Spectral gap $\beta_\Gamma$.}
  Determines whether a power-saving $\delta>0$ is achieved (Ch.~7, \S G).
  
  \item \textbf{Constants $c_{\mathrm{geom}},c_{\mathrm{moll}},C_{\mathrm{Eg}},C_{\mathrm{stat}},C_{\mathrm{MS}}$.}
  All dependencies explicitly traced in Appendix~J; referenced in Ch.~8.
  
  \item \textbf{Forward/backward links.}
  Each chapter states the inputs it requires (backward) and the outputs
  it delivers (forward), summarized in the navigation map below.
\end{itemize}

% ----------------------------------------------------------------------
\subsection*{Layer III: Navigation Map}

\begin{center}
\begin{tabular}{|c|c|c|}
\hline
\textbf{Chapter} & \textbf{Inputs} & \textbf{Outputs} \\
\hline
1. Introduction & --- & Main theorems, novelty \\
\hline
2. Preliminaries & Notation (A--C) & Geometric background, operators \\
\hline
3. Tools & (2), Notation (E--J) & Trace identities, transforms \\
\hline
4. Complexity & (2--3) & Quantitative geometry, setup for microlocal \\
\hline
5. Microlocal & (4), App.~A & Band-limited parametrix, $O_M$ estimates \\
\hline
6. Trace Expansions & (3,5), Scattering & Localized trace identities \\
\hline
7. Local Weyl Law & (5,6), $\beta_\Gamma$ & Power-saving asymptotics \\
\hline
8. Error Budgets & (5--7), App.~J & Consolidated constants, dependency ledger \\
\hline
9. Applications & (7--8) & Extensions, numerics, open problems \\
\hline
Appendices & (all) & Proofs, audit trail, technical lemmas \\
\hline
\end{tabular}
\end{center}

% ----------------------------------------------------------------------
\subsection*{Audit Checklist for \texttt{00-readers-roadmap}}

\begin{itemize}
  \item \textbf{Completeness.} All chapters listed with explicit scope and dependencies. \emph{Status: sealed}.
  \item \textbf{Consistency.} Inputs and outputs match cross-references in main text. \emph{Status: sealed}.
  \item \textbf{Audit Trail.} Constants and error budgets traced to Appendix~J. \emph{Status: sealed}.
  \item \textbf{Navigation.} Forward/backward links explicitly stated. \emph{Status: sealed}.
\end{itemize}

\medskip
\noindent\textbf{Conclusion.}
This roadmap ensures that readers—from specialists to advanced students—
can follow the logical progression of the monograph. 
It provides a transparent audit trail, structural dependencies,
and explicit navigation, guaranteeing clarity, reproducibility, 
and efficient access to results.

% ======================================================================
% End of 00-readers-roadmap.tex
% ======================================================================
