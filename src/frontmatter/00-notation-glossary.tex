% --- 00-notation-glossary.tex ---
\section*{Notation and Glossary}

This section fixes all symbols, normalizations, and dependencies used throughout the monograph. 
All constants are explicit and their dependencies are stated. 
We organize the glossary into three layers: 
\emph{Basic Conventions}, \emph{Structural Framework}, and \emph{Global Normalizations}. 
This layered presentation ensures clarity and reproducibility.

% -------------------------------
\subsection*{Basic conventions}
\begin{itemize}
  \item Sets of numbers: $\mathbb{N}=\{1,2,\dots\}$, $\mathbb{Z}$, $\mathbb{Q}$, $\mathbb{R}$, $\mathbb{C}$.
  \item Asymptotic notation: 
    $A\lesssim B$ means $A\le C\,B$ for an absolute constant $C>0$; 
    $A\asymp B$ means $A\lesssim B$ and $B\lesssim A$.
  \item Parameter-dependent bounds: 
    $O_X(\cdot)$ indicates that the implied constant may depend on the parameter(s) $X$ only.
  \item Functions and sets: 
    $\supp f$ is the support of $f$; 
    $\mathbf{1}_S$ is the indicator of a set $S$.
  \item Fourier transform on $\mathbb{R}$ (non-unitary normalization): 
    \[
      \widehat{f}(\xi)=\int_{\mathbb{R}} f(t)\,e^{-i t \xi}\,dt.
    \]
    This choice aligns with Selberg’s convention.
\end{itemize}

% -------------------------------
\subsection*{Geometry and groups}
\begin{itemize}
  \item Hyperbolic plane: $\mathbb{H}=\{x+iy\in\mathbb{C}: y>0\}$ 
    with metric $ds^2=\frac{dx^2+dy^2}{y^2}$ 
    and area element $d\mu(z)=\frac{dx\,dy}{y^2}$.
  \item Fuchsian group: $\Gamma\subset \mathrm{PSL}_2(\mathbb{R})$ 
    cofinite with cusps. The surface is $M=\Gamma\backslash\mathbb{H}$ 
    of finite area $\vol(M)$.
  \item Cusps: a cusp is a $\Gamma$-equivalence class of parabolic fixed points. 
    Near a cusp we use standard horocyclic coordinates $(x,y)$.
  \item Height truncation: for $Y>0$, define $M(Y)$ as the subset obtained 
    by removing cusp regions $\{y>Y\}$ in standard coordinates.
  \item Injectivity radius: $\inj(x)$ for $x\in M$; define $\inj(M)=\inf_{x\in M}\inj(x)$.
  \item Closed geodesics: $\gamma$ denotes a primitive closed geodesic on $M$, 
    with hyperbolic length $\ell(\gamma)>0$. 
    Define $N(\gamma)=e^{\ell(\gamma)}$.
\end{itemize}

% -------------------------------
\subsection*{Laplace operator and spectrum}
\begin{itemize}
  \item Laplace--Beltrami operator: $\Delta$ is the nonnegative Laplacian on $M$.
  \item Discrete spectrum: eigenvalues $\{\lambda_j\}_{j\ge 0}$ with 
    $0=\lambda_0<\lambda_1\le \lambda_2\le\cdots$, 
    eigenfunctions $\{\varphi_j\}$ forming an orthonormal basis of $L^2_{\mathrm{disc}}(M)$.
  \item Spectral parameter: we parametrize $\lambda_j=\tfrac{1}{4}+r_j^2$ with $r_j\in[0,\infty)$.
  \item Continuous spectrum: spanned by Eisenstein series 
    $E_{\mathfrak{a}}(z,\tfrac{1}{2}+ir)$ attached to each cusp $\mathfrak{a}$, 
    normalized so that the spectral decomposition on $L^2(M)$ reads
    \[
      f=\sum_j \langle f,\varphi_j\rangle \varphi_j 
        \;+\; \sum_{\mathfrak{a}}\frac{1}{4\pi}\int_{-\infty}^{\infty} 
        \langle f,E_{\mathfrak{a}}(\cdot,\tfrac{1}{2}+ir)\rangle 
        E_{\mathfrak{a}}(\cdot,\tfrac{1}{2}+ir)\,dr.
    \]
  \item Spectral gap: denote by $\beta\in(0,\tfrac{1}{4}]$ a lower bound on the gap to $\tfrac{1}{4}$. 
    For instance, $\min\{r_j^2:\lambda_j\ne 0\}\ge \beta$. 
    All constants depending on $\beta$ are recorded as $O_{\Gamma,\beta}(\cdot)$.
\end{itemize}

% -------------------------------
\subsection*{Localization parameters and projectors}
\begin{itemize}
  \item Central frequency: $\lambda\ge 1$.
  \item Window width: $\eta=\eta(\lambda)$ satisfies 
    $\lambda^{-\theta}\le \eta\le 1$ for a fixed $0<\theta<\theta_0$, 
    where $\theta_0>0$ depends only on cusp geometry.
  \item Spectral projector: $P_{\lambda,\eta}$ is a smooth spectral projector 
    localizing to $[\lambda-\eta,\lambda+\eta]$. 
    Its precise construction is given in Chapter~4.
  \item Propagation time: $T\asymp \log \lambda$ arises naturally on the geometric side 
    of the trace formula, setting the time scale of wave propagation.
  \item Semiclassical parameter: we write $h=\lambda^{-1}$ throughout. 
    All oscillatory integrals are expanded in powers of $h$.
\end{itemize}

% -------------------------------
\subsection*{Kernels, transforms, and microlocal tools}
\begin{itemize}
  \item Radial kernel on $\mathbb{H}$: 
    for $d(z,w)$ the hyperbolic distance, 
    $k(d(z,w))$ is a radial kernel; 
    its Selberg/Harish--Chandra transform is denoted $h(r)$.
  \item Wave group: $U(t)=\cos\!\big(t\sqrt{\Delta}\big)$ acting unitarily on $L^2(M)$.
  \item Microlocal calculus: $\Op_h(\cdot)$ denotes semiclassical quantization; 
    symbol classes $S^m$ are defined relative to the hyperbolic metric. 
    Egorov’s theorem and stationary phase are applied in this framework.
  \item Parametrices: local Fourier integral operator representations 
    are constructed for the wave group $U(t)$ (see Chapter~5).
\end{itemize}

% -------------------------------
\subsection*{Counting functions and geometric data}
\begin{itemize}
  \item Localized counting function: $N(\lambda,\eta)$ counts Laplace eigenvalues 
    in $[\lambda-\eta,\lambda+\eta]$, with multiplicity.
  \item Geometric amplitudes: $A_\gamma(\lambda,\eta)$ are explicit weights 
    attached to closed geodesics $\gamma$, 
    computable from the geometry of $M$ and the chosen cutoff.
\end{itemize}

% -------------------------------
\subsection*{Norms and function spaces}
\begin{itemize}
  \item $L^2(M)$ inner product: 
    \[
      \langle f,g\rangle=\int_M f(z)\overline{g(z)}\,d\mu(z),
    \]
    with area element $d\mu(z)=y^{-2}dx\,dy$.
  \item Sobolev norms: $H^s(M)$ defined by 
    $\|f\|_{H^s}=\|(1+\Delta)^{s/2}f\|_{L^2}$.
  \item Schwartz space: $\mathcal{S}(\mathbb{R})$ functions used for cutoff and Fourier transforms; 
    Paley–Wiener bounds are invoked where required.
  \item Distributions: $\mathcal{D}'(M)$ denotes the space of distributions on $M$, 
    used for microlocal analysis and propagation of singularities.
\end{itemize}

% -------------------------------
\subsection*{Asymptotic notation and limits}
\begin{itemize}
  \item Unless otherwise stated, limits are taken as $\lambda\to\infty$.
  \item Error terms: $O(\lambda^{-\delta})$ indicate a power-saving estimate 
    with some $\delta>0$ explicit. 
    When dependence on parameters matters we write $O_{\Gamma,\beta}(\cdot)$.
  \item For windowed quantities such as $N(\lambda,\eta)$, 
    error terms $O(\lambda^{1-\delta})$ are measured relative to the main term $\lambda \eta$. 
    These carry full dependence on cusp geometry and spectral gap.
  \item Notation $o(1)$ refers to terms vanishing as $\lambda\to\infty$; 
    rates are always specified where needed.
\end{itemize}

% -------------------------------
\subsection*{Constants and dependency recording}
\begin{itemize}
  \item Explicit constants: 
    given in closed form in terms of $\vol(M)$, cusp widths, injectivity radius, and spectral gap $\beta$.
  \item Dependency notation: $C=C(\Gamma,\beta,\text{cusp data},\inj(M))$ 
    indicates that constants depend only on fixed geometric and spectral invariants, 
    never on $\lambda$ or $\eta$ unless explicitly declared.
  \item Polynomial control: all constants grow at most polynomially in cusp widths and related parameters.
\end{itemize}

% -------------------------------
\subsection*{Labeling and cross-references}
\begin{itemize}
  \item Structural labels: each section, lemma, theorem is labeled by descriptive tags, 
    e.g. \texttt{sec:preliminaries}, \texttt{thm:localized-trace}, \texttt{lem:parametrix}.
  \item Figures and tables: indexed by chapter number, e.g. Figure~5.1 or Table~6.2.
  \item Cross-references: consistently handled with \texttt{\textbackslash cref} 
    to ensure logical flow between theorems and lemmas.
\end{itemize}

% -------------------------------
\subsection*{Normalization choices (fixed once and for all)}
\begin{itemize}
  \item Laplacian sign: $\Delta\ge 0$, so spectrum lies in $[0,\infty)$.
  \item Eisenstein series normalization: continuous spectrum measure is $dr/(4\pi)$.
  \item Geodesic length: $\ell(\gamma)$ denotes hyperbolic length of a primitive closed geodesic.
  \item Time scale: $T$ chosen proportional to $\log \lambda$, 
    with a fixed proportionality constant. 
    The precise choice is immaterial for stated asymptotics.
\end{itemize}

% -------------------------------
\subsection*{Audit of 00-notation-glossary}
\begin{itemize}
  \item \textbf{Goal G0.1:} Fix notations for groups, operators, and kernels. \textbf{Verified.}
  \item \textbf{Goal G0.2:} Declare constants and their dependencies explicitly. \textbf{Verified.}
  \item \textbf{Goal G0.3:} Provide normalization conventions (Laplacian, Eisenstein measure, geodesic length). \textbf{Verified.}
  \item \textbf{Invariant I0.1:} All constants independent of $(\lambda,\eta)$ unless declared. \textbf{Maintained.}
  \item \textbf{Invariant I0.2:} Error terms always tracked with dependency subscripts. \textbf{Maintained.}
  \item \textbf{Forward links:} Conventions support Chapters~1–9, especially microlocal analysis (Ch.~5) and trace formula expansions (Ch.~6–7).
  \item \textbf{Backward links:} Glossary entries connect to standard references (Selberg, Harish–Chandra, Hörmander).
\end{itemize}
\medskip

\noindent\textbf{Conclusion.}  
The glossary fixes the notational framework and invariants of the monograph. 
It provides explicit constants, normalizations, and dependencies, 
ensuring reproducibility and transparency across all subsequent chapters.
