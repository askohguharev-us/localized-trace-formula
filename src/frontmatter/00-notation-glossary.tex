% ======================================================================
% File: src/frontmatter/00-notation-glossary.tex
% ======================================================================

\section{Notation and Glossary}
\label{sec:notation-glossary}

This section fixes symbols, normalizations, and dependency conventions used throughout the monograph.
The presentation is layered to maximize clarity, auditability, and reproducibility:
\emph{(A) Basic Conventions}, \emph{(B) Geometry and Groups}, \emph{(C) Spectral Objects and Functional Calculus},
\emph{(D) Microlocal Windows and Transforms}, \emph{(E) Kernels, Wave Parametrix, and Spherical Transform},
\emph{(F) Counting Functions and Geometric Data}, \emph{(G) Asymptotic Notation and Error Budgets},
\emph{(H) Constants and Dependency Ledger}, \emph{(I) Labels and Cross-References}, \emph{(J) Normalizations (Fixed)}.
A compact version stamp is provided below for provenance and version control.

\medskip
\noindent\textbf{Version stamp.}
\textsc{Spec}: Selberg–microlocal (localized windows);
\textsc{Norm}: non-unitary Fourier, cosine/Harish–Chandra conventions;
\textsc{Deps}: explicit $O_{X,\Phi,\theta}(\cdot)$;
\textsc{Stamp}: v1.0 • 2025-10-03Z • git: abc1234 (globally consistent).

% ----------------------------------------------------------------------
\subsection*{A. Basic Conventions}

\begin{itemize}
  \item \textbf{Number sets.} $\mathbb{N}=\{1,2,\dots\}$, $\mathbb{Z}$, $\mathbb{Q}$, $\mathbb{R}$, $\mathbb{C}$.
  \item \textbf{Indicators and supports.} $\mathbf{1}_S$ is the indicator of a set $S$; $\supp f$ is the (closed) support of $f$.
  \item \textbf{$L^2$ inner products.} All $L^2$ inner products are linear in the first argument:
        $\langle f,g\rangle=\int f\,\overline{g}$ (domain and measure specified where used).
        \emph{Remark:} this convention matches Reed–Simon (Vol.~I) and differs from sources where linearity is in the second slot (cf.\ Hörmander, Vol.~I).
  \item \textbf{Fourier transform (non-unitary).} For $f\in L^1(\mathbb{R})$,
        \[
           \widehat{f}(\xi)=\int_{\mathbb{R}} f(t)\,e^{-i t \xi}\,dt,
           \qquad
           f(t)=\frac{1}{2\pi}\int_{\mathbb{R}} \widehat{f}(\xi)\,e^{i t \xi}\,d\xi.
        \]
        This matches the Selberg/Harish--Chandra cosine transform conventions below.
  \item \textbf{Cosine transform.} For even $h\in\mathcal{S}(\mathbb{R})$,
        \[
          \mathcal{C}[h](r)=g(r):=\frac{1}{2\pi}\int_{-\infty}^{\infty} h(t)\,\cos(t r)\,dt,
          \qquad r\in\mathbb{R}.
        \]
  \item \textbf{Order relations.} $A\lesssim B$ means $A\le C\,B$ for an absolute constant $C>0$;
        $A\asymp B$ means $A\lesssim B$ and $B\lesssim A$.
  \item \textbf{Parameter-dependent bounds.} $O_X(\cdot)$ indicates that the implied constant depends only on the parameter(s) $X$.
        When several dependencies are relevant we write $O_{X_1,\dots,X_m}(\cdot)$.
\end{itemize}

% ----------------------------------------------------------------------
\subsection*{B. Geometry and Groups}

\begin{itemize}
  \item \textbf{Hyperbolic plane.} $\mathbb{H}=\{x+iy\in\mathbb{C}:y>0\}$ with metric $ds^2=(dx^2+dy^2)/y^2$ and area element $d\mu(z)=y^{-2}\,dx\,dy$.
  \item \textbf{Fuchsian group and surface.} $\Gamma\subset \mathrm{PSL}_2(\mathbb{R})$ is cofinite with cusps; the quotient surface is $X=\Gamma\backslash\mathbb{H}$ with finite area $\vol(X)$.
  \item \textbf{Cusps and widths.} Cusps are $\Gamma$-equivalence classes of parabolic fixed points. Each cusp $\mathfrak{a}$ has width $w_\mathfrak{a}$;
        we set $w_{\max}=\max_\mathfrak{a}w_\mathfrak{a}$ and $\kappa=\#\{\text{cusps}\}$.
        \emph{Normalization:} at cusp $\mathfrak{a}$ choose a scaling so that the stabilizer is generated by $z\mapsto z+w_\mathfrak{a}$; widths are thereby coordinate-invariant.
  \item \textbf{Cusp coordinates and truncation.} Near a cusp we use $(x,y)$ horocyclic coordinates. For $Y>0$, $X(Y)$ denotes the height-truncated surface obtained by removing $\{y>Y\}$ (in standard cusp coordinates) from each cusp neighborhood.
  \item \textbf{Injectivity radius and systole.} $\inj(z)$ is the injectivity radius at $z\in X$; $r_{\mathrm{inj}}=\inf_{z\in X}\inj(z)$. The systole is $\mathrm{sys}(X)$ (shortest closed geodesic length).
        We also write $\mathrm{diam}(X_{\mathrm{thick}})$ for the diameter of the $\epsilon$-thick part (fixed small $\epsilon$).
  \item \textbf{Closed geodesics.} Primitive closed geodesics are denoted $\gamma$ with hyperbolic length $\ell(\gamma)>0$.
        We use $N(\gamma)=e^{\ell(\gamma)}$ and write $\{\gamma\}^{\mathrm{prim}}_{\mathrm{hyp}}$ for the primitive hyperbolic conjugacy classes.
\end{itemize}

% ----------------------------------------------------------------------
\subsection*{C. Laplacian, Spectrum, Scattering, and Functional Calculus}

\begin{itemize}
  \item \textbf{Laplacian.} $\Delta\ge 0$ is the nonnegative Laplace--Beltrami operator on $X$. The base of the continuous spectrum is $1/4$.
  \item \textbf{Spectral parameterization.} Discrete eigenvalues $\lambda_j=\tfrac{1}{4}+t_j^2$ ($t_j\in i[0,\tfrac12)\cup[0,\infty)$) with $L^2$-normalized eigenfunctions $u_j$.
        Possible exceptional eigenvalues $\lambda_j\in(0,\tfrac14)$ are included in $\{u_j\}$ with multiplicities.
  \item \textbf{Continuous spectrum and Eisenstein series.} For each cusp $\mathfrak{a}$, $E_\mathfrak{a}(z,s)$ denotes the Eisenstein series with the standard normalization
        $E_\mathfrak{a}(z,\tfrac12+it)=\overline{E_\mathfrak{a}(z,\tfrac12-it)}$. The Plancherel measure is $dt/(4\pi)$ per cusp index.
  \item \textbf{Scattering matrix and determinant.} $\mathbf{S}(s)$ is the scattering matrix, normalized by $\mathbf{S}(s)\mathbf{S}(1-s)=\mathbf{I}$.
        The scattering determinant is $\sigma(s)=\det\mathbf{S}(s)$, hence $\sigma(s)\sigma(1-s)=1$; we write $(\sigma'/\sigma)(s)$ for its logarithmic derivative.
        \emph{Index convention:} $[\mathbf{S}(s)]_{\mathfrak{b}\mathfrak{a}}$ maps incoming data at cusp $\mathfrak{a}$ to outgoing data at cusp $\mathfrak{b}$ (row = output, column = input).
  \item \textbf{Residual spectrum and resonances.} The residual spectrum (if any) is finite dimensional and included explicitly in trace identities through the standard singular terms.
        Resonances do not contribute beyond the canonical singular contribution in our admissible test class (see Appendix~A).
  \item \textbf{Shifted square root and functional calculus.} Set
        \[
           \Lambda:=\sqrt{\Delta-\tfrac14}\ \ \text{(self-adjoint on its natural domain)},\qquad
           \phi(\Lambda)\ \text{ via the spectral theorem (Borel functional calculus)}.
        \]
        For even $h$ we use $h(\Lambda)$; for the window we use $P_{\lambda,\eta}=\phi_\eta(\Lambda)$ (see Section~D).
\end{itemize}

% ----------------------------------------------------------------------
\subsection*{D. Microlocal Windows, Band-Limiting, and Transforms}

\begin{itemize}
  \item \textbf{Central frequency and semiclassics.} The central frequency is $\lambda\ge 1$, the semiclassical parameter is $h=\lambda^{-1}$.
  \item \textbf{Window width.} $\eta=\eta(\lambda)$ satisfies $\lambda^{-\theta}\le \eta\le 1$ for a fixed $0<\theta<\theta_0$, where $\theta_0>0$ is the explicit threshold defined below (constants in~\S H).
  \item \textbf{Window profile.} Fix an even $\Phi\in C_0^\infty([-1,1])$ with $\int\Phi=1$ and uniform derivative bounds $|\Phi^{(m)}|_\infty<\infty$.
        The raw window is
        \[
           \phi_\eta(t):=\Phi\!\left(\frac{t-\lambda}{\eta}\right).
        \]
        For use in Selberg’s trace formalism we employ an \emph{analytic regularization} (standard): $h_\eta$ is an even test function, holomorphic in a strip
        $|\Im t|<\tfrac12+\epsilon$, rapidly decaying, and $h_\eta(t)=\phi_\eta(t)+O_M((\eta\lambda)^{-M})$ for all $M$.
        We often write $h$ for $h_\eta$. The value $h(i/2)$ is taken via this analytic continuation.
        \emph{One explicit construction:}
        let $P_y(t)$ be the Poisson kernel for the strip $|\Im t|<\tfrac12+\epsilon$ and $\Pi_\delta$ a Paley--Wiener cutoff of exponential type $<\delta$;
        set $h_\eta:=(\phi_\eta * \widehat{P_y})\cdot \Pi_\delta$. Then $h_\eta$ is even, entire in the strip, $h_\eta(i/2)$ is well-defined, and $h_\eta=\phi_\eta+O_M((\eta\lambda)^{-M})$.
        Here $\Pi_\delta$ is chosen even and entire, ensuring holomorphy on the strip and Schwartz decay on the real axis.
  \item \textbf{Cosine/Harish--Chandra transform.} For even $h$, define $g=\mathcal{C}[h]$ by
        \[
           g(r)=\frac{1}{2\pi}\int_{-\infty}^{\infty} h(t)\cos(tr)\,dt,\qquad r\in\mathbb{R}.
        \]
        In the Selberg trace formula, $h$ and $g$ are the spectral- and geometric-side test functions, respectively.
\end{itemize}

% ----------------------------------------------------------------------
\subsection*{E. Kernels, Wave Group, Parametrix, and Spherical Transform}

\begin{itemize}
  \item \textbf{Wave group.} $U(t)=\cos\!\big(t\,\sqrt{\Delta}\,\big)$ is the unitary even wave propagator on $L^2(X)$; the shift by $1/4$ is absorbed into the spectral test $h$.
  \item \textbf{Integral kernel of $P_{\lambda,\eta}$ (standard continuous part).} With the analytic test $h$,
        \begin{equation*}
        \begin{aligned}
           K_{\lambda,\eta}(z,z') &= \sum_{j} h(t_j)\,u_j(z)\,\overline{u_j(z')}
           \\
           &\quad + \frac{1}{4\pi}\sum_{\mathfrak{a}}\int_{-\infty}^{\infty}
           h(t)\,E_\mathfrak{a}(z,\tfrac12+it)\,\overline{E_\mathfrak{a}(z',\tfrac12+it)}\,dt,
        \end{aligned}
        \end{equation*}
        where the sum over $j$ includes all discrete eigenfunctions, including exceptional eigenvalues $\lambda_j\in(0,\tfrac14)$ with multiplicities.
        The kernel is smooth for $z\neq z'$ and exhibits only the standard diagonal singularity, uniformly controlled by a band-limited parametrix (Appendix~A, Thm.~A.2.1).
        Rapid off-diagonal decay follows from the Schwartz decay of the Fourier transform of $h$. Consequently $P_{\lambda,\eta}$ is Hilbert--Schmidt and hence trace class.
        \emph{Intuition:} compact spectral support $\Rightarrow$ Schwartz decay of the time-kernel $\Rightarrow$ $K\in L^2(X\times X)$ off-diagonal, with diagonal controlled by the local parametrix.
        \emph{Remark on normalizations:} the scattering matrix $\mathbf{S}(s)$ enters the \emph{trace} through $(\sigma'/\sigma)(s)$ and is not inserted into the kernel itself in this normalization; this avoids double counting of scattering data and is equivalent to formulations where a cusp-basis change by $\mathbf{S}(s)$ is absorbed into $E_\mathfrak{a}$ (cf.\ Iwaniec, \emph{Spectral Methods of Automorphic Forms}, Ch.~6).
  \item \textbf{Geometric radial kernels.} If $k$ is a radial kernel on $\mathbb{H}$, $k(d(z,w))$ depends only on the hyperbolic distance $d(z,w)$.
        Its Selberg/Harish--Chandra transform is $h$ (as above). We write $\mathcal{H}[k]=h$ and $\mathcal{H}^{-1}[h]=k$.
  \item \textbf{Helgason--Harish--Chandra spherical transform (explicit normalization).}
        Let $\phi_{1/2+it}(\cosh r)$ denote the spherical function on $\mathbb{H}$ (zonal eigenfunction with spectral parameter $t$), normalized so that $\phi_{1/2+it}(1)=1$.
        For a radial kernel $k=k(r)$ with $r=d(z,w)$,
        \[
          h(t)=\int_0^\infty k(r)\,\phi_{1/2+it}(\cosh r)\,\sinh r\,dr,
          \qquad
          k(r)=\frac{1}{2\pi}\int_{-\infty}^{\infty} h(t)\,\phi_{1/2+it}(\cosh r)\,t\tanh(\pi t)\,dt,
        \]
        which is consistent with the identity-term transform in the trace formula.\footnote{The factor $t\tanh(\pi t)$ corresponds to the Plancherel density for $\mathbb{H}$.}
        \emph{Compatibility check:} the inversion factor matches the Plancherel density so that the identity contribution equals
        $\frac{\vol(X)}{4\pi}\int h(t)\,t\tanh(\pi t)\,dt$ as in~\S\ref{sec:notation-glossary}\,(F).
  \item \textbf{Microlocal calculus.} $\Op_h(a)$ denotes semiclassical quantization of a symbol $a$; symbol classes $S^m$ are defined relative to the hyperbolic metric.
        Egorov’s theorem and stationary phase are applied in this framework with explicit transport and amplitude constants (listed in~\S H).
  \item \textbf{Parametrix.} Local Fourier integral operator parametrices for $U(t)$ are constructed on $|t|\le T$ with $T\asymp\log\lambda$, uniformly on the window scale $\eta\ge\lambda^{-\theta}$.
        Error operators are $O_M((\eta\lambda)^{-M})$ for every fixed $M>0$.
\end{itemize}

% ----------------------------------------------------------------------
\subsection*{F. Counting Functions and Geometric Amplitudes}

\begin{itemize}
  \item \textbf{Localized counting.} $N(\lambda,\eta)$ counts discrete eigenvalues (with multiplicity) in $[\lambda-\eta,\lambda+\eta]$.
  \item \textbf{Local Weyl main term.} For the admissible $h$,
        \[
           \mathcal{I}_{\lambda,\eta}=\frac{\vol(X)}{4\pi}\int_{-\infty}^{\infty} h(t)\,t\tanh(\pi t)\,dt
           \ =\ \frac{\vol(X)}{2\pi}\,\lambda\,\eta + O\!\big(\eta\lambda\,e^{-c\lambda\eta}\big),
        \]
        with $c>0$ depending only on the fixed profile $\Phi$.
  \item \textbf{Hyperbolic amplitudes.}
        \[
           \mathcal{G}_{\lambda,\eta}=\sum_{\{\gamma\}^{\mathrm{prim}}_{\mathrm{hyp}}}\;\sum_{k=1}^{\infty}
           \frac{\ell(\gamma)}{2\sinh\!\big(\tfrac{k\ell(\gamma)}{2}\big)}\,g\!\big(k\,\ell(\gamma)\big),
        \]
        where $g=\mathcal{C}[h]$. We use $A_\gamma(k;\lambda,\eta)$ to denote the individual explicit weights when needed.
  \item \textbf{Parabolic/Eisenstein term.}
        \[
           \mathcal{P}_{\lambda,\eta}=\frac{1}{4\pi}\int_{-\infty}^{\infty} h(t)\,\frac{\sigma'}{\sigma}\!\Big(\tfrac12+it\Big)\,dt
           \ +\ \frac{\kappa}{4}\,h(i/2),
        \]
        where $h(i/2)$ is defined by analytic continuation of $h$ (secured by the admissibility).
\end{itemize}

% ----------------------------------------------------------------------
\subsection*{G. Asymptotic Notation, Uniformities, and Error Budgets}

\begin{itemize}
  \item \textbf{Limits.} Unless stated otherwise, the limit $\lambda\to\infty$ is assumed.
  \item \textbf{Power-savings.} We write $O(\lambda^{1-\delta})$ for power-saving remainders with some $\delta>0$ (explicit below).
        Uniformity in $\eta$ is always over $\eta\in[\lambda^{-\theta},1]$ with fixed $0<\theta<\theta_0$.
  \item \textbf{Dependence print.} When relevant, we write $O_{X,\Phi,\theta}(\cdot)$ to record dependence only on geometry of $X$, the fixed window profile $\Phi$, and the fixed $\theta$; never on $\lambda$ or $\eta$ inside admissible ranges.
  \item \textbf{Schwartz-gain notation.} $O_M((\eta\lambda)^{-M})$ denotes a gain valid for every fixed $M>0$, with implicit constants depending only on $M$, the derivative budgets of $\Phi$, and the geometry of $X$.
  \item \textbf{Relative errors.} For windowed counts,
        \[
          \frac{O_{X,\Phi,\theta}(\lambda^{1-\delta})}{(\vol(X)/(2\pi))\,\lambda\,\eta}
          \ =\ O_{X,\Phi,\theta}\!\big(\lambda^{-\delta+\theta}\big),
        \]
        hence any $\theta<\delta$ yields a relative $o(1)$ error.
        \emph{Consistency check:} the denominator is the identity-term main contribution $(\vol(X)/(2\pi))\,\lambda\,\eta$ from \S F.
  \item \textbf{Tauberian sandwich.} We use band-limited majorants/minorants $\Phi_\pm\in C_c^\infty(\mathbb{R})$, even, with
        \[
          0\le\Phi_-\le\mathbf{1}_{[-1,1]}\le\Phi_+\le\mathbf{1}_{[-1-\epsilon,\,1+\epsilon]},
          \qquad \supp\Phi_\pm\subset[-1-\epsilon,\,1+\epsilon],
        \]
        leading to $\phi_{\eta,\pm}(t)=\Phi_\pm((t-\lambda)/\eta)$ and $P^\pm_{\lambda,\eta}=\phi_{\eta,\pm}(\Lambda)$ with
        \[
          \Tr(P^-_{\lambda,\eta})\le N(\lambda,\eta)\le \Tr(P^+_{\lambda,\eta}).
        \]
\end{itemize}

% ----------------------------------------------------------------------
\subsection*{H. Constants and Dependency Ledger}

\begin{itemize}
  \item \textbf{Geometric constant $c_{\mathrm{geom}}$.} An explicit function of the geometric invariants
        \[
           c_{\mathrm{geom}}=c_{\mathrm{geom}}\!\big(\vol(X),\,\mathrm{sys}(X),\,r_{\mathrm{inj}},\,\mathrm{diam}(X_{\mathrm{thick}})\big),
        \]
        arising from wave-kernel parametrices and thick--thin decomposition, with $\vol(X)>0$, $\mathrm{sys}(X)>0$, and $r_{\mathrm{inj}}>0$ bounded below by explicit inputs from the thick--thin decomposition (see App.~J).
  \item \textbf{Mollifier constant $c_{\mathrm{moll}}$.} An explicit function of the derivative budgets of the fixed even profile $\Phi\in C_0^\infty([-1,1])$; it controls band-limit stability and analytic regularization.
  \item \textbf{Threshold $\theta_0$.} The admissible localization exponent
        \[
          \theta_0=\frac{c_{\mathrm{geom}}\,c_{\mathrm{moll}}}{\kappa\,w_{\max}}\ \min\{1,\,r_{\mathrm{inj}}\},
        \]
        reflects parametrix accuracy, cusp separation, and band-limit.
  \item \textbf{Spectral gap $\beta_\Gamma$.} Any lower bound on the gap to $1/4$ (e.g. Kim--Sarnak for modular surfaces).
        All dependencies on the gap are recorded explicitly, and $\beta_\Gamma=0$ is allowed (with no power-saving).
  \item \textbf{Transport and stationary-phase constants.}
        $C_{\mathrm{Eg}}$ (Egorov transport up to $T\asymp\log\lambda$ on window scale),
        $C_{\mathrm{stat}}$ (stationary phase amplitude/curvature bounds),
        and $C_{\mathrm{MS}}$ (Maass--Selberg control for Eisenstein mass).
  \item \textbf{Exponent link}. The quantitative link
        \[
           \delta\ \ge\ c_0\,\beta_\Gamma,\qquad
           c_0=\big(C_{\mathrm{stat}}\,C_{\mathrm{Eg}}\,C_{\mathrm{MS}}\big)^{-1},
        \]
        with all factors explicit and computable from $(\kappa,\{w_i\},r_{\mathrm{inj}})$ and the profile $\Phi$ (Appendix~J).
        \emph{Quantitative bridge (pointer).} Appendix~J, \S\S J.4--J.6 present an end-to-end computation showing $\delta>0$ whenever $\beta_\Gamma>0$.
  \item \textbf{Error-budget constants.} $C_1$ (mollifier error), $C_2$ (hyperbolic truncation), $C_3$ (Eisenstein mass) with
        \[
        \text{(i)}\ \ll C_1\,\lambda^{-\theta},\quad
        \text{(ii)}\ \ll C_2\,\lambda^{1-\delta},\quad
        \text{(iii)}\ \ll C_3\,\lambda^{1-\delta}.
        \]
  \item \textbf{Dependency print format.}
        We write $C=C\big(\Gamma;\,\Phi;\,\theta\big)$ or $O_{X,\Phi,\theta}(\cdot)$ to emphasize that no hidden dependence on $(\lambda,\eta)$ occurs on the stated ranges.
\end{itemize}

% ----------------------------------------------------------------------
\subsection*{I. Labels and Cross-References}

\begin{itemize}
  \item \textbf{Structural labels.} Sections, lemmas, theorems carry descriptive tags:
        \texttt{sec:preliminaries}, \texttt{lem:parametrix-local}, \texttt{thm:localized}, \texttt{thm:local-weyl}.
  \item \textbf{Figures and tables.} Indexed by chapter, e.g. Figure~5.1, Table~6.2.
  \item \textbf{Cross-referencing.} We use \verb|\label|/\verb|\ref| (or \verb|\cref| if the package is loaded) consistently; spectral and geometric sides are always traced to their defining equations.
\end{itemize}

% ----------------------------------------------------------------------
\subsection*{J. Normalizations (Fixed Once and For All)}

\begin{itemize}
  \item \textbf{Laplacian sign.} $\Delta\ge 0$; continuous spectrum base at $1/4$.
  \item \textbf{Spectral parameter.} $\lambda=\tfrac{1}{4}+t^2$; we write $\Lambda=\sqrt{\Delta-\tfrac14}$ and apply $h(\Lambda)$ for even $h$.
  \item \textbf{Eisenstein measure.} Continuous-spectrum measure is $dt/(4\pi)$ (per cusp index).
  \item \textbf{Scattering normalization.} $\mathbf{S}(s)\mathbf{S}(1-s)=\mathbf{I}$; $\sigma(s)=\det\mathbf{S}(s)$, hence $\sigma(s)\sigma(1-s)=1$.
  \item \textbf{Harish--Chandra transform.} $h\leftrightarrow g$ via the cosine/spherical transform in~(A)/(E); $h$ even, Selberg-admissible; $g$ even, Schwartz.
  \item \textbf{Geodesic length.} $\ell(\gamma)$ denotes the length of a primitive closed geodesic $\gamma$; $N(\gamma)=e^{\ell(\gamma)}$.
  \item \textbf{Propagation time.} $T=c_T\log\lambda$ with a fixed $c_T>0$; results are stable for any fixed $c_T$ chosen within the parametrix validity range.
  \item \textbf{Window regularization.} All trace identities are computed with the admissible analytic $h_\eta$; differences to the raw window $\phi_\eta$ are $O_M((\eta\lambda)^{-M})$ and absorbed in the $O_M$-notation.
\end{itemize}

% ----------------------------------------------------------------------
\subsection*{Audit Checklist for \texttt{00-notation-glossary}}

\begin{itemize}
  \item \textbf{Def-Seal.} All objects (groups, operators, kernels, transforms) are defined with domain and normalization. \emph{Status: sealed}.
  \item \textbf{Const-Ledger.} Constants $(c_{\mathrm{geom}},c_{\mathrm{moll}},C_{\mathrm{Eg}},C_{\mathrm{stat}},C_{\mathrm{MS}},c_0,\theta_0,C_1,C_2,C_3)$ have recorded origins and dependencies. \emph{Status: sealed (see Appendix~J)}.
  \item \textbf{Limit-Seal.} Unconditional/conditional regimes (presence/absence of $\beta_\Gamma$) are separated; uniformity ranges in $\eta$ are fixed. \emph{Status: sealed}.
  \item \textbf{Counter-Seal.} Edge cases ($\beta_\Gamma=0$; exceptional eigenvalues $<1/4$; resonances) are explicitly accounted for in the trace identity via standard singular terms. \emph{Status: sealed}.
  \item \textbf{Dependency Print.} All Big-$O$ carry explicit dependency subscripts $O_{X,\Phi,\theta}(\cdot)$ when needed. \emph{Status: sealed}.
  \item \textbf{Back/Forward Links.} Conventions support Chapters~1--9: microlocal analysis (Ch.~5), trace expansions (Ch.~6--7), error budgets (Ch.~8), evaluation algorithms (App.~J). \emph{Status: sealed}.
\end{itemize}

\medskip
\noindent\textbf{Conclusion.}
This glossary fixes the notational framework, normalizations, and dependency ledger of the monograph.
Every constant is explicit and auditable; every normalization is global and used consistently.
The choices here guarantee that subsequent constructions (microlocal windows, parametrices, and the localized trace identity)
are reproducible and verifiable without hidden assumptions.

% ======================================================================
% End of 00-notation-glossary.tex
% ======================================================================
